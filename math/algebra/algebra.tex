\chapter{抽象代数}


\section{群论}

\subsection{幺半群}

\subsubsection{幺半群}

幺半群的定义是集合 $S$ 和二元运算 $\cdot$,二元运算 $\cdot$ 满足结合律,而且集合包含一个幺元 $e$,对任意 $x \in S$ 满足,
$x \cdot e = e \cdot x = x$。下文为了简洁会省略运算符号。

\subsubsection{广义结合律}

首先证明一个引理 

\[
    (x_1 x_2 \dotsb x_n) (y_1 y_2 \dotsb y_m) = (x_1 x_2 \dotsb x_n y_1 y_2 \dotsb y_m)
\]

证明的思路是对 $m$ 作归纳法,当 $m=1$ 时自然成立,当 $m=k+1$ 时

\begin{align*}
(x_1 x_2 \dotsb x_n) (y_1 y_2 \dotsb y_{k+1}) & = (x_1 x_2 \dotsb x_n) (y_1 y_2 \dotsb y_{k}) y_{k+1} \\
& = (x_1 x_2 \dotsb x_n y_1 y_2 \dotsb y_{k}) y_{k+1} \\
& = (x_1 x_2 \dotsb x_n y_1 y_2 \dotsb y_{k+1})
\end{align*}

矩阵乘法的结合律也可以用线性变换或者线性映射的结合律证明。


\subsubsection{幺元唯一}

这个证明只需要用到幺元的定义即可。

\[
e_1e_2 = e_1 = e_2
\]

\subsubsection{同态}

同态可以针对两个不同的幺半群,这两个幺半群会具有很多相似的性质。下面给出定义,幺半群 $(S, \cdot)$ 和 $(T, *)$ 同态当且仅单存在映射 $f: S \to T$
满足对 $x,y \in S $ 满足 $f(x \cdot x) = f(x) * f(y)$,且 $f(e) = e'$。这里要注意和同构的区别。


\subsection{循环群}

\subsubsection{群的定义}

群的定义可以概括为: 封闭的二元运算,结合律,单位元以及逆元。

如果一个群中的元素是有限多个,这个群中元素的数量定义为群的阶。


\subsubsection{循环群}

如果一个群中所有元素都可以用某一个元素的整数次幂表示,这个群就是循环群。例如整数集合,加法运算构成循环群。

循环群一定是阿贝尔群,循环群可能是有限群,也可能是无限群。

\subsubsection{元素的阶}

对于群 $G$ 中的元素 $a$,如果存在最小的正整数 $n$ 满足 $a^n = e$,那么 $n$ 就是这个元素的阶。如果不存在这样的 $n$,称这个 $a$ 是无限阶元素。


\subsubsection{有限群是循环群的等价条件}

有限群 $G$ 是循环群,当且仅当存在 $a \in G$,满足 $a$ 的阶等于 $G$ 的阶。下面给出证明。

充分性:如果 $a$ 的阶等于 $n$,那么集合 $X = \{ k \in \mathbb{Z} | a^k \}$ 中包含的元素数量一定是 $n$,由于二元运算的封闭性,必然有 $X \subseteq G$,
因为 $X$ 和 $G$ 包含的元素数量相等所以 $X = G$。

必要性:根据循环群定义就可以证明。


\subsubsection{$a^k$ 的阶}

已知 $a$ 的阶等于 $n$,则$a^k$ 的阶等于 $\frac{n}{(n,k)}$。下面给出证明。

假设 $a^{ks} = e$,则必然有 $n \vert ks$,因此有 $\frac{n}{(n,k)} \vert \frac{k}{(n,k)}s $,因为 

$\frac{n}{(n,k)}$ 和 $\frac{k}{(n,k)}$ 互素,所以有  $\frac{n}{(n,k)} \vert s$。

再考虑 $(a^k)^{\frac{n}{(n,k)}} = (a ^ {n})^{\frac{k}{(n,k)}} = e$,所以 $s \vert \frac{n}{(n,k)} $

所以 $s = \frac{n}{(n, k)}$


\subsubsection{$ab$ 的阶}

若满足 $ab = ba$ 且 $a$ 的阶 $n$ 和 $b$ 的阶 $m$ 互素,则 $ab$ 的阶等于 $nm$。

证明:假设$ab$ 的阶为 $s$, $(ab)^{nm} = a^{nm}b^{mn} = e$ 说明 $s \vert nm$。
又因为 $e = (ab)^{sn} = b^{sn}$ 所以 $m \vert sn$,因为 $m,n$ 互素可以得出 $m \vert s$,同理可得 $n \vert s$,再结合 $m,n $ 互素得到 $nm \vert s$。


\subsubsection{有限阿贝尔群}

有限阿贝群里面,存在一个元素,它的阶是其他所有元素阶的倍数。证明如下:


