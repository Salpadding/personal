\chapter{抽象代数}


\section{群论}

\subsection{幺半群}

\subsubsection{幺半群}

幺半群的定义是集合 $S$ 和二元运算 $\cdot$,二元运算 $\cdot$ 满足结合律,而且集合包含一个幺元 $e$,对任意 $x \in S$ 满足,
$x \cdot e = e \cdot x = x$。下文为了简洁会省略运算符号。

\subsubsection{广义结合律}

首先证明一个引理 

\[
    (x_1 x_2 \dotsb x_n) (y_1 y_2 \dotsb y_m) = (x_1 x_2 \dotsb x_n y_1 y_2 \dotsb y_m)
\]

证明的思路是对 $m$ 作归纳法,当 $m=1$ 时自然成立,当 $m=k+1$ 时

\begin{align*}
(x_1 x_2 \dotsb x_n) (y_1 y_2 \dotsb y_{k+1}) & = (x_1 x_2 \dotsb x_n) (y_1 y_2 \dotsb y_{k}) y_{k+1} \\
& = (x_1 x_2 \dotsb x_n y_1 y_2 \dotsb y_{k}) y_{k+1} \\
& = (x_1 x_2 \dotsb x_n y_1 y_2 \dotsb y_{k+1})
\end{align*}

矩阵乘法的结合律也可以用线性变换或者线性映射的结合律证明。


\subsubsection{幺元唯一}

这个证明只需要用到幺元的定义即可。

\[
e_1e_2 = e_1 = e_2
\]

\subsubsection{同态}

同态可以针对两个不同的幺半群,这两个幺半群会具有很多相似的性质。下面给出定义,幺半群 $(S, \cdot)$ 和 $(T, *)$ 同态当且仅单存在映射 $f: S \to T$
满足对 $x,y \in S $ 满足 $f(x \cdot x) = f(x) * f(y)$,且 $f(e) = e'$。这里要注意和同构的区别。

\subsection{群}

\subsubsection{群的定义}

群的定义可以概括为: 封闭的二元运算,结合律,单位元以及逆元。

如果一个群中的元素是有限多个,这个群中元素的数量定义为群的阶。

用符号语言描述就是 $abc = a(bc)$,$ae = ea = a$,$a^{-1}a = aa^{-1} = e$

\subsubsection{更小的充要条件}

群的定义可以简化为:满足结合律的封闭的二元运算,存在左单位元,存在左逆元。

下面给出证明:对于 $a \in G$,我们令它的左逆元为 $a'$,并且这个 $a'$ 的左逆元为 $a''$,于是有 $a'a = a''a'= e$。
我们分析下 $aa'$,得到了 $aa' = eaa' = a''a'aa' = a''a' = e$,所以左逆元同时也是右逆元。
继续分析左单位元 $ae = aa'a = ea = a$,所以左单位元也是右单位元。

\subsubsection{另一个等价条件}

这个等价条件没有那么显然,如果一个集合 $G$ 上有满足结合律的二元运算,
而且对任意 $a,\,b \in G$,$ax = b$ 和 $ya  = b$ 都有解。那么这个集合和二元运算构成群。

下面给出证明,我们证明的目标是找到左单位元和左逆元。首先对于任意 $b \in G$有 $yb = b$ 有解 $y = e_b$,
令 $bx = a$ 的解为 $c$,有 $e_ba = e_bbc = bc =  a$,所以 $e_b$ 是左单位元,我们可以记作 $e$。
同理对任意的 $a \in G$,$ya = e$ 有解 $a'$,所以我们找到了左逆元,根据之前的结论,$G$ 一定构成群。

\subsubsection{例题}
如果 $G$ 是一个非空有限集合,而且上面有满足结合律的二元运算,如果左消去律和右消去律成立,那么 $G$ 构成一个群。
也就是说 $ac = bc$ 可以得到 $a = b$,$ca= cb$ 可以得到 $a = b$。

下面给出证明:假设 $G = \{ a_1, a_2, .. a_n \}$,那么取任意 $a \in G$,得到 $aG = \{ aa_1, aa_2, .., aa_n\}$。
根据消去律 $aG$ 包含 $n$ 个元素,所以 $aG = G$,所以对任意 $b \in G$, $ax = b$ 一定有解。同理也可以证明 $ya = b$ 有解。
根据上面的定理 $G$ 构成一个群。



\subsection{循环群}


\subsubsection{循环群}

如果一个群中所有元素都可以用某一个元素的整数次幂表示,这个群就是循环群。例如整数集合,加法运算构成循环群。

循环群一定是阿贝尔群,循环群可能是有限群,也可能是无限群。

\subsubsection{元素的阶}

对于群 $G$ 中的元素 $a$,如果存在最小的正整数 $n$ 满足 $a^n = e$,那么 $n$ 就是这个元素的阶。如果不存在这样的 $n$,称这个 $a$ 是无限阶元素。


\subsubsection{有限群是循环群的等价条件}

有限群 $G$ 是循环群,当且仅当存在 $a \in G$,满足 $a$ 的阶等于 $G$ 的阶。下面给出证明。

充分性:如果 $a$ 的阶等于 $n$,那么集合 $X = \{ k \in \mathbb{Z} | a^k \}$ 中包含的元素数量一定是 $n$,由于二元运算的封闭性,必然有 $X \subseteq G$,
因为 $X$ 和 $G$ 包含的元素数量相等所以 $X = G$。

必要性:根据循环群定义就可以证明。


\subsubsection{$a^k$ 的阶}

已知 $a$ 的阶等于 $n$,记 $g = (n,k)$ 则$a^k$ 的阶等于 $n/g$。下面给出证明。

假设 $a^{ks} = e$,则必然有 $n \wvert ks$,因此有 $\frac{n}{g} \vert \frac{k}{g}s $,因为 

$\frac{n}{g}$ 和 $\frac{k}{g}$ 互素,所以有  $\frac{n}{g} \vert s$。

再考虑 $(a^k)^{\frac{n}{g}} = (a ^ {n})^{\frac{k}{g}} = e$,所以 $s \wvert \frac{n}{g} $

所以 $s = \frac{n}{g}$


\subsubsection{$ab$ 的阶}

若满足 $ab = ba$ 且 $a$ 的阶 $n$ 和 $b$ 的阶 $m$ 互素,则 $ab$ 的阶等于 $nm$。

证明:假设$ab$ 的阶为 $s$, $(ab)^{nm} = a^{nm}b^{mn} = e$ 说明 $s \wvert nm$。
又因为 $e = (ab)^{sn} = b^{sn}$ 所以 $m \wvert sn$,因为 $m,n$ 互素可以得出 $m \wvert s$,同理可得 $n \wvert s$,再结合 $m,n $ 互素得到 $nm \wvert s$。


\subsubsection{$\langle a \rangle$ = $\langle a^{-1} \rangle$}

这个显然成立。

\subsubsection{若 $G$ 是有限群且 $a \in G$, $G = \langle a \rangle$ 和 $\lvert G \rvert =  \mathrm{ord}\,a$ 等价}

结合 $\langle a \rangle \subseteq G$ 以及 $\lvert \langle a \rangle \rvert = \mathrm{ord}\, a$ 很容易证明。

\subsubsection{若 $\langle a \rangle$ 是无限循环群,则 $\langle a \rangle =  \{ a^{k} \wvert k \in \mathbb{Z}\}$,并且 $a^k = a^l$ 可以得到 $k=l$}

这个可以用反证法,假设有 $k > l$ 且 $a^k = a^l$ 可以得到 $a^{k-l} = e$,这与 $\langle  a \rangle$ 是无限循环群矛盾。

\subsubsection{若 $\langle a \rangle$ 是有限循环群,则 $\langle a \rangle =  \{ a^{k} \wvert 0 \le k < \mathrm{ord}\, \langle a \rangle \}$,并且 $a^k = a^l$ 可以得到 $n \wvert k -l $}

证明很容易 $a^k = a^l$ 可以得到 $a^{k-l} = e$,令 $k-l = s$,$\mathrm{ord}\, a = m$,令 $s = qm + r$,其中 $0 \le r < m$,得到 $a^r = a^{s}a^{-qm} = e$,根据阶的定义得到 $r = 0$,所以 $m \wvert s$。

\subsubsection{$a^k$ 是 $\langle a \rangle$ 生成元的充分必要条件是 $k$ 和 $a$ 的阶互素}

下面给出证明,先证明充分性,记 $\mathrm{ord}\, a = n$,若 $(n,k) = 1$,则有 $\mathrm{ord}\, a^k = n$。因为 $\langle a^k \rangle \subseteq \langle a \rangle$,所以若两个集合不相等,它们的阶一定不相等。
再证明必要性,若 $(n,k) > 1$ 则 $\mathrm{ord}\, a^k < n$,这两个集合就不可能相等。

\subsubsection{ $\langle a \rangle$ 的生成元数量等于 $\phi(n)$,其中 $\phi$ 是欧拉函数}

上面结论的推论。

\subsubsection{循环群的任意子群都是循环群}

假设 $H$ 是 $G =\langle a \rangle$ 的非空子群,注意这里 $G$ 可能是无限循环群, 若它包含一个不等于单位元的元素,记

$m = \min \{ k \wvert a^k \in H,\, k > 0,\, a^k \ne e \}$

下面证明 $H = \langle a^m \rangle $,假设 $a^s \in H$,我们对 $s$ 做带余除法得到 $s = qm + r,\, 0 \le r < m$。
于是得到 $a^{r} = a^{s}a^{-qm} \in H$,根据 $0 \le r < m$ 得到 $r = 0$,所以 $m \wvert s$,也就是 $H$ 的任意元素都可以通过 $a^m$ 生成得到。

\subsubsection{若$G = \langle a \rangle $,且 $\lvert G \rvert = \infty$,那么 $G$ 的所有子群是 $\{ \langle a ^d \rangle \wvert d = 0,1,2,..\}$ }

显然右边的每个元素都是子群,而且每个子群都是不同的子群。
我们希望证明所有的子群都可以用 $\langle a^d \rangle$ 表示。而上面我们证明过,只要取子群的最小正整数 $m$ 满足 $a^m \in H,\, a^m \ne e$即可。

\subsubsection{若$G = \langle a \rangle $,且 $\lvert G \rvert = n$,那么 $G$ 的所有子群是 $\{ \langle a ^d \rangle \wvert d > 0,\, d \wvert n\}$ }

这个也是同理,通过取最大正整数可以得到子群的生成元 $a^d$,上面的结论可以得到子群的阶等于 $\frac{n}{(n,d)}$。
如果子群的阶不相同,那么一定不是相同的子群,所以所有的子群可以通过枚举有限循环群阶的因子得到。

\subsubsection{无限循环群 $\langle a \rangle$ 同构于整数加法群}

我们可以令 $f(m) = a^m$,容易验证 $f$ 满足 $f(k + l) = f(k)f(l)$,而且 $f$ 是双射。

\subsubsection{有限循环群同构于模 $n$ 下的整数加法群}

我们可以令 $f(m) = a^m$,容易验证 $f$ 满足 $f(k+l) = f(k)f(l)$,而且 $f$ 是双射。

\subsection{子群}

$G$ 的一个非空子集 $H$ 构成子群 iff $H$ 关于 $G$ 的运算构成群,则称 $H$ 是 $G$ 的子群,记作 $H \leq G$。

\subsubsection{判定方法1}

如果群 $G$ 的一个非空子集 $H$ 满足 $\forall a,b \in H,\, ab \in H$,并且 $\forall a \in H,\, a^{-1} \in H$,那么 $H$ 是 $G$ 的子群。

证明如下,我们取任意 $a \in H$,因为 $a^{-1} \in H$,所以必然有 $ aa^{-1} = e \in H$,所以 $H$ 包含了单位元,并且对二元运算封闭。

\subsubsection{判定方法2}

如果$G$ 的非空子集 $H$ 满足对任意 $a,b \in H$ 有 $ab^{-1} \in H$,那么 $H$ 构成 $G$ 的子群。
这个定义没有要求 $a$ 和 $b$ 不同,所以我们可以取 $a = a,\, b = a$ 得到 $aa^{-1} = e \in H$。我们再取 $a = e$ 得到对任意 $b \in H$ 有 $b^{-1} \in H$。
然后我们对于任意 $a,b$ 可以先找到 $a, b^{-1}$ 然后代入条件得到 $a(b^{-1})^{-1} = ab \in H$。

\subsubsection{子群的交集}

子群的交集依然是子群,假设 $H_1$ 和 $H_2$ 都是 $G$ 的子群,因为 $e \in H_1$ 并且 $e \in H_2$ 
所以 $H_1 \cap H_2 \ne \emptyset$,我们取任意 $a, b \in H_1 \cap H_2$,因为 $H_1$ 和 $H_2$
都是子群,所以有 $a b^{-1} \in H_1 \cap H_2$,所以 $H_1 \cap H_2$ 是子群。

\subsubsection{生成子群}

若集合 $S$ 是群 $G$ 的子集,那么记 $S$ 的生成子群为

\[
\langle S \rangle = \bigcap_{S \subseteq H \leq G} H
\]

\subsubsection{生成子群是包含 $S$ 的最小子群}

这个根据定义就可以证明。

\subsubsection{生成子群的性质}

若 $S = \{a_1, a_2, .., a_n \}$,也就是说 $S$ 是大小为 $n$ 的有限集,并且 $S$ 是群 $G$ 的子集。
那么 $\langle S \rangle = \{ a_{i_1}^{l_1} a_{i_2}^{l_2} .. a_{i_k}^{l_k} \wvert l_j = \pm 1,\, 1 \le j \le n \}$

下面给出证明:记 $T = \{ a_{i_1}^{l_1} a_{i_2}^{l_2} .. a_{i_k}^{l_k} \wvert l_j = \pm 1,\, 1 \le j \le n \}$

首先,对于任意一个包含 $S$ 的子群 $H$,它一定包含了 $a_i,\,a_i^{-1}\, 1 \le i \le n$,因此 $T$ 是每个 $H$ 的子集,所以 $T \subseteq \langle S \rangle $。
我们继续分析集合 $T$,可以看到 $T$ 不是一个空集,于是我们试图证明 $T$ 是一个群。
首先有 $a_1a_1^{-1} = e \in T$,所以 $T$ 包含单位元,其次对于任意 $a, b \in T$,显然有 $ab \in T$,所以 $T$ 是一个群并且有 $S \subseteq T$,所以我们有 $\langle S \rangle \subseteq T$,
最终得到 $\langle S \rangle = T$。


\section{对称群}

\subsection{贝尔数}

贝尔数推导的思想如下,我可以吧 $x_{n+1}$ 单独拿出来,然后考虑 $B(n)$,再把 $\{ x_{n+1} \}$ 放到 $B(n)$ 的划分当中。
同理,我也可以从 $x_1, x_2, .. x_n$ 当中取一个 $x_i$ 和 $x_{n+1}$ 组成 $\{ x_i, x_{n+1} \}$,然后考虑 $B(n-1)$,再把 $\{ x_i, x_{n+1}\}$ 放到 $B(n-1)$ 当中。
显然上面的划分都是不相交的,而且对于 $B(n+1)$ 的每一个划分,其中必然包含一个包含 $x_{n+1}$ 的集合,这个集合也一定包含在我们的枚举过程中。
所以有

\[
B_{n+1} = \sum_{k=0}^{n} \binom{n}{k} B(k)
\]

\subsection{对称群定义}

对于集合 $X$,对称群 $S_X$ 指的是 $\{ f \wvert f: X \to X,\, f\text{ is bijective}\}$。

\subsection{ $n$ 次对称群}

如果 $X$ 是有限的,并且包含 $n$ 个元素,那么它上面的对称群 $S_X$ 就是 $n$ 次对称群,记作 $S_n$。

\subsection{置换}

$n$ 次对称群中的元素就是置换。

\subsection{置换群}

$n$ 次对称群的子群就是置换群。


\subsection{轮换}



