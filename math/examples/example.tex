\chapter{奇思妙想}

\section{布尔代数与模二剩余类}

\subsection{定义}

我想用数论的方法研究布尔代数,因为布尔代数只有真和假两个值,自然会想到模二剩余类。于是我们定义 $a, \: b \in \{ 0, 1\}$,其中 0 代表假,1 代表真,我们可以先定义异或运算和与运算。

\begin{align*}
    a \oplus b & = a + b \\
    a \land b &= a \cdot b \\
    \neg a &= a + 1 \\
\end{align*}

通过 $\neg(a \lor b) = \neg a \land \neg b $ 可以定义或运算

\begin{align*}
    a \lor b = (a+1)(b+1) + 1 = a + b + ab
\end{align*}


\subsection{性质}

通过上面的定义可以推导出集合运算的很多性质,例如分配律

\subsubsection{$(A \cup B) \cap C = (A \cap C) \cup (B \cap C)$}

左边对应的模二剩余类表达式为 $(a+b+ab)c = ac +bc + abc$,右边对应的表达式为 $ac + bc + abc^2$ 注意到模二剩余类中有 $c^2 = c$,所以左右相等。


\subsubsection{$(A \cap B) \cup C = (A \cup C) \cap (B \cup C)$}

左边对应的模二剩余类表达式为 $ab + c + abc = ab(c+1) +c$,右边对应的表达式为 

\begin{align*}
(a + c + ac)(b + c + bc) &= (a(c+1) +c)(b(c+1) +c) = ab(c+1)^2 + c + ac(c+1) + bc(c+1) \\
    & = ab(c+1)^2 +c + c(c+1)(a+b) \\
    & = ab(c+1) + c
\end{align*}

注意到 $c(c+1) = 0$

\subsubsection{$(A \setminus B) \cup (B \setminus A)$}

\begin{align*}
    & = a(b+1) + b(a+1) + ab(a+1)(b+1) \\
    & = a + b + ab(ab + 1 + a+b) \\
    & = a + b
\end{align*}

其实就是 $A \oplus B$

\subsubsection{判断子集关系}

如果 $\neg a \lor b = 1$ 也就是 $ (a+1) + b + b(a+1)$ 等于 $1$ 恒成立,那么 $a$ 是 $b$ 的子集,因为 $a$ 为真时 $b$ 一定为真。例如我们可以以此判断 $A \cap B$ 是否是 $A$ 的子集

\[
ab+1 + a + a(ab+1) = ab + a + ab + a  + 1= 1
\]

\section{经典的数学问题}

\subsection{集合论}

\subsection{自然常数}

根据指数函数的定义我们可以得到

\[
e = 1/0! + 1/1! + 1/2! + .. 
\]

下面我们将证明

\[
\lim_{n \to \infty}(1+1/n)^n = e
\]

首先我们把 $(1+1/n)^n$ 展开得到

\[
(1+1/n)^n = \sum_{k=0}^{n}\frac{1}{k!}\frac{n(n-1)..(n-k+1)}{n^k}
\]

我们令

\[
h(n,k) = \frac{n(n-1)..(n-k+1)}{n^k}
\]

显然有 $h(n,k) \le 1$ 且 $h(n,k) \le h(n+1,k)$,$h(n,k+1) \le h(n,k)$ 并且有

\[
\lim_{n \to \infty}h(n,k) = 1
\]

所以我们得到了 

\[
(1+1/n)^n \le \sum_{k=0}^{n}\frac{1}{k!} \le e
\]

所以 $(1+1/n)^n$ 有界,根据 $h(n+1,k) \ge h(n,k)$ 我们得到 $(1+1/n)^n$ 单调。所以 $(1+1/n)^n$ 收敛。

我们继续证明 $(1+1/n)^n$ 收敛到 $e$,假设

\[
e - \sum_{k=0}^{p}\frac{1}{k!} \le \epsilon
\]

我们计算 $(1+1/n)^n$ 的前 $p+1$ 项,得到

\[
(1+1/n)^n \ge \sum_{k=0}^{p}\frac{1}{k!}h(n,k)
\]

然后我们对 $n$ 取极限得到,我们之所以这么做是因为这时候 $p$ 已经固定了


\begin{align*}
& \lim_{n \to \infty}(1+1/n)^n \ge \sum_{k=0}^{p}\frac{1}{k!} \ge e - \epsilon \\
& 0 \ge \lim_{n \to \infty}(1+1/n)^n - e \ge -\epsilon 
\end{align*}

因为 $\epsilon$ 是任意的,所以有

\[
\lim_{n \to \infty}(1+1/n)^n = e
\]

\subsubsection{自然数集的子集满足良序}

这个要用到选择公理。

\subsection{实数集}


\subsubsection{有理数在实数集中的稠密性}

\subsubsection{上确界存在}

\subsubsection{实数集完备}

\subsubsection{单调有界必收敛}

\subsubsection{闭区间上的有限覆盖定理}

\subsubsection{实数集不可数}

\subsubsection{有界序列必有收敛子列}

\subsection{闭区间上连续函数的性质}

\subsubsection{有界}

\subsubsection{最大值和最小值}

\subsubsection{一致连续}

\subsubsection{象集连通}

\subsection{微分学}

\subsubsection{中值定理}

\subsubsection{洛必达法则}

\subsection{泰勒级数}

\begin{align*}
    & \lvert f(x) - L_0 - L_1(x-x_0) - \frac{L_2(x-x_0)^2}{2}- \frac{L_3(x-x_0)^3}{6} \rvert = \\
    & = \lvert\int_{x_0}^{x}f_1(t) - L_1 - L_2(t-x_0) - \frac{L_3(t-x_0)^2}{2} \rvert\\ 
    & = \lvert \int_{x_0}^{x} \int_{x_0}^{t} f_2(s) - L_2 - L_3(s-x_0) \rvert \\ 
    & \le \int_{x_0}^{x} \int_{x_0}^{x} \lvert f_2(s) - L_2 - L_3(s-x_0) \rvert \\
    & \le \int_{x_0}^{x} \int_{x_0}^{t} \epsilon \lvert s-x_0 \rvert \\
    & \le \int_{x_0}^{x} \epsilon \frac{(t-x_0)^2}{2} \\
    & \le \epsilon \frac{(x-x_0)^3}{6} \\
\end{align*}

\begin{align*}
    & \lvert f(x) - L_0 - L_1(x-x_0) - \frac{L_2(x-x_0)^2}{2}- \frac{L_3(x-x_0)^3}{6} \rvert = \\
    & = \lvert\int_{x_0}^{x}f_1(t) - L_1 - L_2(t-x_0) - \frac{L_3(t-x_0)^2}{2} \rvert\\ 
    & = \lvert \int_{x_0}^{x} \int_{x_0}^{t} f_2(s) - L_2 - L_3(s-x_0) \rvert \\ 
    & = \lvert \int_{x_0}^{x} \int_{x_0}^{t} \int_{x_0}^{s} f_3(u) - L_3 \rvert \\ 
    & = \lvert \int_{x_0}^{x} \int_{x_0}^{t} \int_{x_0}^{s} \int_{x_0}^{u} f_4(v) \rvert \\ 
\end{align*}