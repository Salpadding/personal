\chapter{黎曼积分}

\section{划分}

\subsection{有界区间}

回顾下有界区间的定义,$(a,b) \,\, (a,b] \,\, [a,b) \,\, [a,b]$ 都是区间。注意这里不要求 $a$ 和 $b$ 不相等,也没有要求 $a \le b$,所以空集也是区间,一个只包含一个实数的集合也是区间。

\subsection{连通集}

连通集 $A$ 定义为 $x, y \in A$ 则 $\forall x \le t \le y, \, t \in A$ 

\subsection{实数集上的有界连通集和区间等价}

下面证明实数集上的有界连通集是区间,我们分别取 $A$ 的上确界为 $b$,下确界为 $a$,下面证明 $(a,b) \subseteq A \subseteq [a,b]$。
假设存在 $c \in (a,b), c \notin A$,因为上确界和下确界都是附着点,
所以我们可以找到 $a' \in A $ 和 $b' \in A$ 满足 $a \le a' < c < b' \le b$,这就跟 $A$ 是连通集矛盾了。
右半部分易证明,所以 $A$ 是一个区间。

\subsection{有界区间的交集也是有界区间}

连通集合的交集依然是连通集,有界集合的交集依然是有界的。


\subsection{区间的长度}

对于空集,定义区间长度为 $0$,对于非空的区间 $(a,b) \subseteq I \subseteq [a,b]$ 定义 $I$ 的长度为 $b-a$,这里也包含了单点集。

\subsection{划分}

$\mathbf{P} = \{ J \vert J \subseteq I,\, J \in \mathbf{P} \}$ 是对有界区间 $I$ 的一个划分当且仅当 $I$ 中的每个元素 $x$ 恰好属于 $\mathbf{P}$ 中的一个有界区间 $J$。

\subsection{有限可加性}

我们上面之所以这样定义划分就是为了能够满足有限可加性,这对黎曼积分的定义有重要的意义。

\[
    \lvert I \rvert = \sum_{J \in \mathbf{P}} \lvert J \rvert
\]

下面给出证明,假设 $(a,b) \subseteq I \subseteq [a,b] $,并且 $\mathbf{P}$ 是对 $I$ 的一个划分。如果 $I$ 是空集,那么成立。如果 $I$ 不是空集,那么设 $\mathbf{P}$ 中元素的数量为 $n > 0$。令 $a_0=a, I_0 = I, \mathbf{P_0} = \mathbf{P}$ 我们讨论 $a_0$ 是否属于 $I_0$,
如果 $a_0 \in I_0$,那么必然有 $a_0 \in J_0 ,\, J_0 \in \mathbf{P_0}$ 令 $[a_0, b_0) \subseteq J_0 \subseteq [a_0, b_0]$ 同时有 $I_0 = \bigcup_{J \in \mathbf{P_0}} J $ 两边同时对 $J_0$ 作差集得到 $I_0 \setminus J_0 = \bigcup_{J \in \mathbf{P_0}, J \ne J_0}J$ 
注意到 $I_0 \setminus J_0$ 依然是一个区间,我们可以记它为 $I_1$ ,注意到此时 $J \in \mathbf{P_0}, J \ne J_0$ 组成了对 $I_1$ 的一个划分,我们记这个划分为 $\mathbf{P_1}$ 所以我们可以对 $I_1, \mathbf{P_1}$ 作重复的操作,同时 $I_1$ 的下确界一定是 $b_0$。同理如果 $a_0 \notin I_0$,那么 $\mathbf{P_0}$ 中一定存在一个 $J_0$ 满足 $J_0$ 的下确界等于 $a_0$,否则 $a_0$ 附近一定有点不会被覆盖。同样的可以做上面的操作得到 $I_1, \mathbf{P_1}$。
我们重复以上操作和得到了 $\{ J_0, J_1, .., J_{n-1} \} = \mathbf{P}$,注意到 $J_0$ 的上确界就是 $J_1$ 的下确界,以此类推。我们得到 $\lvert J_0 \rvert + \lvert J_1 \rvert  + .. + \lvert J_{n-1} \rvert = a_1 -a_0 + a_2 - a_1 + .. + a_n - a_{n-1} = a_n - a_0 = b - a$。
注意到 $I_{n-1} \setminus J_{n-1} = I_n = \emptyset$,所以 $J_{n-1}$ 的上确界一定是 $b$。

\subsection{更细划分}

假设 $I $ 有两个划分 $\mathbf{P}$ 和 $\mathbf{P'}$ 我们称 $\mathbf{P'}$ 是 $\mathbf{P}$ 的更细划分当且仅当对任意 $J \in \mathbf{P}$ 都有 $J' \in \mathbf{P'}$ 满足 $J' \subseteq J$ 。

\subsection{公共加细}

假设 $I $ 有两个划分 $\mathbf{P}$ 和 $\mathbf{P'}$,$\mathbf{P}$ 和 $\mathbf{P'}$ 的公共加细定义为 

\[
    \mathbf{P} \# \mathbf{P'} = \{ J \cap K \wvert J \in \mathbf{P}, K \in \mathbf{P'}\}
\]

\section{分段常数函数}

\subsection{定义}
设 $f: I \to \mathbb{R}$,其中 $I$ 是有界区间,如果存在一个划分 $\mathbf{P}$ 使得 $\forall J \in \mathbf{P}, J \ne \emptyset \, f(J)$ 是一个单点集,那么 $f$ 就是关于 $\mathbf{P}$ 的分段常数函数。

注意到如果 $\mathbf{P'}$ 比 $\mathbf{P}$ 更细,那么 $f$ 也是在 $\mathbf{P'}$ 上的分段常数函数。所以分段常数函数对加减乘,$\max$ 和 $\min$ 运算保持封闭,如果 $\forall x, g(x) \ne 0$ 那么 $f/g$ 也是分段常数函数。

\subsection{分段常值积分}

若 $f$ 是在划分 $\mathbf{P}$ 上的分段常数函数,定义 $f$ 的分段常值积分为 
\[
    \int_{[\mathbf{P}]} f = \sum_{J \in \mathbf{P}}c_J \lvert J \rvert
\]

分段常值积分是和 $\mathbf{P}$ 无关的,为了证明这个,我们需要证明一个引理,如果 $\mathbf{P'}$ 是比 $\mathbf{P}$ 更细的划分,那么

\[
    \int_{[\mathbf{P}]} f =\int_{[\mathbf{P'}]} f
\]

如果 $\mathbf{P'}$ 是比 $\mathbf{P}$ 更细的划分,那么 $J' \in \mathbf{P'}$ 可以分类成 $J'_{11}, J'_{12}, .. \subseteq J_1$, $J'_{21}, J'_{22}, .. \subseteq J_2$ 其中 $J_i \in \mathbf{P}$。
并且 $J' \subseteq J_i$ 也是对 $J_i$ 的划分。首先 $J'$ 是不相交的有界区间,其次 $x \in J_i$ 必然有 $x \in J'$ 且 $J' \subseteq J_i$ ,因为如果 $J'$ 不是 $J_i$ 的子集的话就会出现 $J_i \cap J_j \ne \emptyset$。
于是我们计算 $f$ 在 $\mathbf{P'}$ 上的分段常值积分可以分组计算再求和,按照 $J \in \mathbf{P}$ 进行分组,这样各组的结果相加得到了 $\sum_{J \in \mathbf{P}}c_J \lvert J \rvert$

证明如下假设 $\mathbf{P}$ 和 $\mathbf{P'}$ 都是 $f$ 的一个划分,而且 $f$ 在 $\mathbf{P}$ 和 $\mathbf{P'}$ 上都是分段常值函数。我们根据引理可以得到  

\[
    \int_{[\mathbf{P}]} f =\int_{[\mathbf{P \# P'}]} f = \int_{[\mathbf{P'}]} f
\]

所以分段常值函数的积分可以写成如下形式

\[
\int_{I} f
\]

\subsection{分段常值积分性质}

\begin{enumerate}
    \item $\int_{I} (f + g) = \int_{I} f + \int_{I} g$ 这个用公共加细很容易证明。
    \item $\int_{I} cf = c\int_{I} f$ 这个可以套定义。
    \item $\int_{I} (f - g) = \int_{I} f - \int_{I} g$ 用上面两个的组合可以证明
    \item 如果 $f(x) \ge 0$, $\int_{I} f \ge 0$,套定义。
    \item 如果 $f(x) \ge g(x)$ 那么 $\int_{I} f \ge \int_{I}g $,也是套定义。
    \item 如果 $f(x) = c$ 那么 $\int_{I} f  = c \lvert I \rvert $,也是套定义。
    \item 有限可加性,如果 $\{J,K \}$ 是对 $I$ 的一个划分,那么 $\int_{I}f = \int_{J}f + \int_{K} f$,因为我们可以从 $J$ 和 $K$ 各自的划分得到 $I$ 的一个划分。
\end{enumerate}

\section{黎曼积分}

\subsection{上方控制和下方控制}

$f: I \to \mathbb{R}$ 且 $g: I \to \mathbb{R}$,如果 $g(x) \ge f(x)$,那么 $g(x) $ 从上方控制 $f(x)$。
如果 $g(x) \le f(x)$ 那么 $g(x)$ 从下方控制 $f(x) $。

\subsection{上下黎曼积分的定义}

假设 $A = \{ g(x) \}$ 是所有上方控制 $f(x)$ 的分段常数函数的集合,有界函数 $f(x)$ 的上黎曼积分定义为:

\[
   \overline{\int}_I f =  \inf \{ \int_{I} g \wvert \forall x \in I ,\, g(x) \ge f(x), g(x) \mathrm{\; is\; p.c.} \}
\]

下黎曼积分定义为

\[
    \underline{\int}_I = f\sup \{ \int_{I} g \wvert \forall x \in I ,\, g(x) \le f(x), g(x) \mathrm{\; is\; p.c.} \}
\]

这里要注意的是,上黎曼积分是用下确界定义的,而下黎曼积分是用上确界定义的。因为上确界可以用最大附着点定义,下确界可以用最小附着点定义,后面做证明的时候可以利用数列极限的性质。


\subsection{黎曼可积的定义}

假设 $f: I \to \mathbb{R}$ 是有界的,如果 $f$ 的上黎曼积分等于 $f$ 的下黎曼积分,那么 $f$ 黎曼可积。


\subsection{分段常数函数是黎曼可积的}

这个很容易证明。


\subsection{黎曼和}

为了便于构造分段常数函数,我们定义上黎曼和 还有 下黎曼和。假设 $f: I \to \mathbb{R}$ 是有界的,而且 $\mathbf{P}$ 是 $I$ 的一个划分,那么定义 $f$ 的上黎曼和为

\[
    U(f, \mathbf{P}) = \sum_{J \in \mathbf{P}} (\sup_{x \in J}f(x)) \lvert J \rvert
\]

定义下黎曼和为

\[
    L(f, \mathbf{P}) = \sum_{J \in \mathbf{P}} (\inf_{x \in J}f(x)) \lvert J \rvert
\]

\subsection{上下黎曼和与上下黎曼积分}

上黎曼和的下确界就是上黎曼积分,首先上黎曼和的定义中隐含了一个上方控制的分段常数函数,所以上黎曼和这个集合的附着点也是上黎曼积分这个集合的附着点,所以

\[
    \overline{\int}_I f \le \inf \{ U(f, \mathbf{P}) \}
\]

假设有一个从上方控制的分段常数函数 $g(x)$ 还有划分 $\mathbf{P}$,我们取 $h(x)$ 为

\[
    h(x) = \sup \{ f(J) \}, \: x \in J
\]

于是我们得到了 $h(x) \le g(x)$ 所以通过上黎曼积分对应的附着点,可以构造出上黎曼和的附着点,而且上黎曼和的附着点小于等于上黎曼积分对应的附着点。所以有

\[
 \inf \{ U(f, \mathbf{P}) \}   \le  \overline{\int}_I f
\]

结合之前的结论得到

\[
 \inf \{ U(f, \mathbf{P}) \}  =  \overline{\int}_I f
\]

同理可以证明


\[
 \sup \{ L(f, \mathbf{P}) \}  =  \underline{\int}_I f
\]

\section{黎曼积分的性质}

\subsection{代数性质}

下面假设有界函数 $f: I \to \mathbb{R}$ 和 $g: I \to \mathbb{R}$ 都是黎曼可积的

\begin{enumerate}
    \item $\int(f+g) = \int f + \int g $ 这个很好证明,$f'$从上方控制 $f$,$g'$ 从上方控制 $g$ 可以得到 $f'+g'$ 从上方控制 $f+g$。利用附着点的性质得到
    \[
  \underline{\int}f + \underline{\int}g \le \underline{\int}(f+g)  \le \overline{\int}(f+g) \le \overline{\int}f + \overline{\int}g
    \]

    \item $\int cf = c \int f$ 这个要讨论 $c > 0$ 和 $c < 0$,$c > 0$ 时,$f'$ 上方控制 $f$ 可以得到 $cf'$ 上方控制 $cf$。
    \[
  c\underline{\int}f \le  \underline{\int}cf \le \overline{\int}cf \le c \overline{\int}f 
    \]

    如果 $c < 0$,如果 $f'$ 上方控制 $f$ 可以得到 $cf'$ 从下方控制 $cf$ 所以得到

    \[
      c\overline{\int}f   \le \underline{\int}cf \le \overline{\int} cf \le c \underline{\int}f
    \]

    \item $\int_{I}(f-g) = \int_{I}f - \int_{I}g $ 使用上面的结论可以证明

    \item $f(x) \ge 0$ 则 $\int_{I} f(x) \ge 0$,因为 $0$ 从下方控制了 $f(x)$ 所以有
    \[
        0 \le \underline{\int}_{I} f
    \]

    \item $f(x) \ge g(x)$ 则 $\int_{I}f \ge \int_{I} g$ 用上面的结论很容易证明。

    \item $f(x) = c$ 则 $\int_{I}f = c \lvert I \rvert$,因为 $f(x) =c$ 从上方控制也从下方控制。

    \item 如果 $\{ J, K \} $ 是 $I$ 的一个划分,并且 $f: J \to \mathbb{R}$ 和 $f: K \to \mathbb{R}$ 都是黎曼可积的,那么有

    \[
        \int_{J}f + \int_{K}f = \int_{I}f
    \]

    同理 $g$ 和 $h$ 如果分别在 $J$ 和 $K$ 上方控制 $f$ ,那么可以扩展得到 $g + h$ 也在上方控制 $f$,所以有

    \[
 \underline{\int}_J f + \underline{\int}_K f \le \underline{\int}_I f \le \overline{\int}_I f \le \overline{\int}_J f + \overline{\int}_K f
    \]

    \item 假设 $(a,b) \subseteq I \subseteq [a,b]$ 并且 $f: I \to \mathbb{R}$  是黎曼可积的,那么 

    \[
        \int_{I}f = \int_{[a,b]}f 
    \]

    证明也很简单,利用上方控制函数,还有上下极限是附着点,数列极限的性质。

    \[
 \underline{\int}_{[a,b]}f \le \underline{\int}_{I}f \le \overline{\int}_{I}f \le \overline{\int}_{[a,b]}f 
    \]

\end{enumerate}

\subsection{$\max$ 和 $\min$ 保持黎曼可积}

若 $f: I \to \mathbb{R}$ 和 $g: I \to \mathbb{R}$ 都是黎曼可积的,那么 $\max(f,g)$ 也是黎曼可积的。

下面给出证明,先证明一个引理,令 $h = \max(f,g)$,对任意区间 $J \subseteq I $

\begin{align*}
\underline{a} & \le \inf \{ f(x) \wvert x \in J \} \\
\underline{b} & \le \inf \{ g(x) \wvert x \in J \} \\
\overline{a} & \ge \sup \{ f(x) \wvert x \in J \} \\
\overline{b} & \ge \sup \{ g(x) \wvert x \in J \} \\
\overline{c} & = \sup \{ h(x) \wvert x \in J \} \\
\underline{c} & = \inf \{ h(x) \wvert x \in J \} \\
\end{align*}

有

\[
  \max (\underline{a},\underline{b})  \le \underline{c} \le \overline{c} \le \max (\overline{a},\overline{b})
\]

证明很容易,利用附着点和数列的性质即可。有了这个不等式后可以得到:

\begin{align*}
  \overline{c} - \underline{c} & \le \max(\overline{a}, \overline{b}) - \max(\underline{a}, \underline{b}) \\
  \overline{c} - \underline{c} & \le \max(\overline{a}, \overline{b}) + \min(-\underline{a}, -\underline{b}) \\
  \overline{c} - \underline{c} & \le \max(\overline{a} + \min(-\underline{a}, -\underline{b}), \overline{b} + \min(-\underline{a}, -\underline{b})) \\
  \overline{c} - \underline{c} & \le \max(\overline{a} -\underline{a}, \overline{b} -\underline{b}) \\
\end{align*}

若分段常数函数 $\overline{f}$ 和 $\overline{g}$ 分别从上方控制 $f$ 和 $g$,那么取他们的公共加细,然后在公共加细上对 $h$ 取上下确界,可以得到 $\overline{h}$ 和 $\underline{h}$,而且有

\begin{align*}
    \overline{h} - \underline{h} & \le \max(\overline{f} - \underline{f}, \overline{g} - \underline{g}) \\
    \overline{\int}_I h - \underline{\int}_I h \le \int_{I}(\overline{h} - \underline{h}) & \le \max (\int_{I}(\overline{f} - \underline{f}),\int_{I}(\overline{g} - \underline{g}) ) \le \epsilon
\end{align*}

$\min$ 可积也是同理。只不过把证明过程中的 $\max$ 跟 $\min$ 轮换一下。

\subsection{绝对值保持黎曼可积}

令 $f_{+}(x) = \max(f(x), 0) = (f(x) + \lvert f(x) \rvert) / 2$,$f_{-}(x) = \min(f(x), 0) = (f(x) - \lvert f(x) \rvert) / 2$
则有 $\lvert f(x) \rvert = f_{+}(x) - f_{-}(x)$

\subsection{乘积保持黎曼可积}

若 $f: I \to \mathbb{R}$ 和 $g: I \to \mathbb{R}$ 都是黎曼可积的,那么 $h = f \cdot g$ 也是黎曼可积的。

这里要把 $f$ 和 $g$ 各自拆成 $f = f_{+} + f_{-}$ 还有 $g = g_{+} + g_{-}$ 的形式,那么  $f \cdot g = f_{+}g_{+} + f_{+}g_{-} + f_{-}g_{+} +f_{-}g_{-} $

我们只需要证明 $f_{+}g_{+}$ 可积,下面的证明过程中我们暂时把$\overline{f_{+}}$ 简单记为 $\overline{f}$。只用到 $f_{+}$ 可积并且非负这个性质,$g_{+}$ 同理。其他项的证明类似。假设 $\overline{f}$ 和 $\overline{g}$ 分别从上方控制 $f_{+}$ 和 $g_{+}$, 
$\underline{f}$ 和 $\underline{g}$ 分别从下方控制 $f_{+}$ 和 $g_{+}$,
那么有 $ \underline{f} \underline{g} \le f_{+}g_{+} \le \overline{f} \overline{g}$ 注意到有下面不等式

\[
    \overline{f}\overline{g} - \underline{f} \, \underline{g} =  \overline{f}(\overline{g} - \underline{g}) + \underline{g}(\overline{f} - \underline{f}) \\
\]

\begin{align*}
    \int_{I}\overline{f}\overline{g} - \int_{I}\underline{f}\underline{g} &= \int_{I}\overline{f}\overline{g} - \underline{f}\underline{g} \\
    & \le \int_{I}\overline{f}(\overline{g} - \underline{g}) + \int_{I}\underline{g}(\overline{f} - \underline{f})  \\
    & \le \int_{I}M(\overline{g} - \underline{g}) + \int_{I}M(\overline{f} - \underline{f}) \le 2M\epsilon
\end{align*}

上面的 $M$ 是 $\max \{ f_{+}(x), g_{+}(x) \wvert x \in I \}$,$\epsilon$ 是利用 $f_{+}$ 和 $g_{+}$ 黎曼可积。

所以有 

\[
    \overline{\int}_{I}f_{+}g_{+} - \underline{\int}_{I}f_{+}g_{+} \le 2 M \epsilon
\]

因为上面的 $\epsilon$ 可以取任意小所以$f_{+}g_{+}$ 黎曼可积,同理可以得到 $(-f_{-})(-g_{-})$ 黎曼可积等。

这是一个经典等式,只要有乘法和加法这个等式可以用于如下形式:

\[
    x_1y_1 - x_2y_2 = x_1(y_1 - y_2) + y_2(x_1 - x_2)
\]