\chapter{黎曼积分}

\section{划分}

\subsection{有界区间}

回顾下有界区间的定义,$(a,b) \,\, (a,b] \,\, [a,b) \,\, [a,b]$ 都是有界区间,其中 $a, b \in \R$。注意这里不要求 $a$ 和 $b$ 不相等,也没有要求 $a \le b$,所以空集也是区间,一个只包含一个实数的集合也是区间。

\subsection{连通集}

连通集 $A$ 定义为对任意两个 $x, y \in A,\, x \le y$ 则 $\forall x \le t \le y, \, t \in A$ 

\subsection{实数集上的有界连通集和区间等价}

下面证明实数集上的有界连通集是区间,我们分别取 $A$ 的上确界为 $b$,下确界为 $a$,下面证明 $(a,b) \subseteq A,\, A \subseteq [a,b]$。
假设存在 $c \in (a,b)$,因为上确界和下确界都是附着点,我们可以找到 $a'$ 和 $b'$ 满足

\[
a \le a' < c < b' \le b,\quad a', b' \in A
\]

根据 $A$ 是连通集的定义,可以得出 $c \in A$,因为 $c$ 是任意的,所以 $(a,b) \subseteq A$,再根据上下确界的性质得到

\[
A \subseteq [a,b]
\]


\subsection{有界区间的交集也是有界区间}

连通集合的交集依然是连通集,有界集合的交集依然是有界的。

\subsection{有界区间的上下确界}

如果有界区间 $I$ 有 $a = \inf I,\, b = \sup I$ 那么 $(a,b) \subseteq I$。

\subsection{区间的 $\alpha$ 长度}
令 $\alpha: \R \to \R$。
对于空集,定义区间的 $\alpha$ 长度为 $0$,对于非空的区间 $I$ 定义 $I$ 的 $\alpha$ 长度 $\alpha(I)$ 为 $\alpha(b)-\alpha(a)$,
其中 $a = \inf I,\, b = \sup I$
这里也包含了单点集。

\subsection{划分}

$\mathbf{P} = \{ J \wvert J \subseteq I \}$ 是对有界区间 $I$ 的一个划分当且仅当 $I$ 中的每个元素 $x$ 恰好属于 $\mathbf{P}$ 中的一个有界区间 $J$。

\subsection{有限可加性}

我们上面之所以这样定义划分就是为了能够满足长度的有限可加性,这对黎曼积分的定义有重要的意义。
有限可加性的命题如下,若 $I$ 是有界区间,$\mathbf{P} = \{ J \wvert J \subseteq I \}$ 是对有界区间 $I$ 的一个划分,那么有

\[
    \alpha(I) = \sum_{J \in \mathbf{P}} \lvert \alpha(J) \rvert
\]

我们先证明两个引理:

\begin{enumerate}
    \item 若非空有界区间 $I_1,\, I_2$ 不相交,且 $I = I_1 \cup I_2$ 也是有界区间。那么不妨令 $\inf I_1 \le \inf I_2$,于是有 $\sup I_1 = \inf I_2,\, \inf I = \inf I_1,\, \sup I = \sup I_2$。

    证明 $\inf I = \inf I_1$ 比较容易,因为 $I_1 \subseteq I $ 所以 $\inf I \le \inf I_1$。又因为 $\forall x \in I,\, x \ge \inf I_1$ 所以 $\inf I \ge \inf I_1$。

    同理可以证明 $\sup I = \sup I_2$。

    若 $I_1$ 和 $I_2$ 其中有一个是单点集,为了保持连通性,必然有 $\sup I_1 = \inf I_2$。
    
    这里我们不讨论 $I_1,\, I_2$ 都是单点集的情况,因为两个不相交的单点集的并集显然不连通。

    如果 $I_1,\, I_2$ 都不是单点集,我们用反证法,先假设 $\sup I_1 < \inf I_2$,那么存在 $c$ 满足 $\sup I_1 < c < \inf I_2$,
    显然 $\inf I < c < \sup I$,根据$I$ 是有界区间,所以 $c \in I$,但这和 $c \notin I_1,\, c \notin I_2$ 矛盾。

    继续假设 $\sup I_1 > \inf I_2$ 的情况,这个显然和 $I_1,\, I_2$ 不相交矛盾了。

    \item 若 $I$ 是非空有界区间,$\mathbf{P}$ 是 $I$ 的一个划分,
    如果 $\inf I \in I$,那么存在唯一的 $J \in \mathbf{P}$ 满足 $\inf I \in J,\, \inf J = \inf I$。如果 $\inf I \notin I$,那么存在唯一的非空 $J \in \mathbf{P}$
    满足 $\inf J = \inf I$。以上两种情况找到的 $J$ 都有 $I \setminus J$ 要么是空集 要么是非空的有界区间。
    若 $I \setminus J$ 非空,那么 $\sup (I \setminus J) = \sup I,\, \inf (I \setminus J) = \sup J$

    如果 $\inf I \in I$,而 $\mathbf{P}$ 是一个划分,根据不相交的并的性质,必然存在唯一的 $J$ 满足 $\inf I \in J,\, \inf J \le \inf I$,而 $J$ 是 $I$ 的子集,
    所以有 $\inf I \le \inf J$,最终得到 $\inf J = \inf I$。然后我们可以把 $J$ 的补集拆成两个连通集的并集,
    其中一个是$(-\infty, \inf J)$ 另一个可能是 $(\sup J, \infty)$ 或者 $[\sup J, \infty)$
    然后有

    \begin{align*}
    I \setminus J &= I \cap ((-\infty, \inf J) \cup (\sup J, \infty)) \\
    & = (I \cap (-\infty, \inf J)) \cup (I \cap (\sup J, \infty)) \\ 
    & = I \cap (\sup J, \infty) \\
    \end{align*}

    所以 $I \setminus J$ 也是有界区间。所以 $J$ 和 $I \setminus J$ 构成了一个对 $I$ 的一个划分,因为 $\inf J \le \inf I \setminus J$,
    所以有 $\sup J = \inf I \setminus J,\, \sup(I \setminus J) = \sup I$

    如果 $\inf I \notin I$,我们取所有 $\{ a_J = \inf J \wvert J \in \mathbf{P} \}$,显然根据子集的性质有 $\forall J \in \mathbf{P},\, \inf I \le a_J$。
    我们用反证法,假设 $\forall J,\, a_J > \inf I$,因为 $(\inf I, \sup I) \subseteq I$,这样会有部分 $\inf I$ 附近的点不属于任何 $J$,与条件矛盾。
    所以存在 $J \in \mathbf{P}$ 满足 $a_J = \inf I$。

    还有继续证明这个 $J$ 是唯一的,假设 $\inf J_1 = \inf J_2 = \inf I$,因为 $\inf I \notin J_1,\, \inf I \notin J_2$,所以
    $J_1,\, J_2$ 都是左开的区间,为了保证 $J_1$ 和 $J_2$ 各自非空,它们各自的上确界一定大于下确界,这样它们的交集一定非空,矛盾。

    同理我们可以证明 $I \setminus J$ 是有界区间。$\sup J = \inf I \setminus J,\, \sup(I \setminus J) = \sup I$


\end{enumerate}

下面给出证明

我们不考虑 $\mathbf{P}$ 中的空集,因为空集的长度等于$0$,所以可以直接去掉。 

假设 $\inf I = a,\, \sup I = b $,并且 $\mathbf{P}$ 是对 $I$ 的一个划分。如果 $I$ 是空集,那么成立。如果 $I$ 不是空集,那么设 $\mathbf{P}$ 中元素的数量为 $n > 0$。令 $a_0=a, I_0 = I, \mathbf{P_0} = \mathbf{P}$,然后我们通过递归的方式进行证明。

我们先讨论 $ a_0$ 是否属于 $I_0$。

如果 $a_0 \in I_0$,根据引理,那么必然存在 $J_0 \in \mathbf{P_0}$ 满足 $a_0 \in J_0,\, \inf J_0 = a_0$

对 $I_0 = \bigcup_{J \in \mathbf{P_0}} J $ 两边  作 $J_0$ 差集得到,这里用到交并运算的分配律。

\[
I_0 \setminus J_0 = (\bigcup_{J \in \mathbf{P_0}}J) \setminus J_0 = \bigcup_{J \in \mathbf{P_0},\, J \ne J_0}J
\]

根据引理 $I_0 \setminus J_0$ 依然是一个有界区间,我们可以记它为 $I_1$ ,并且有 $\inf I_1 = \sup J_0,\, \sup I_1 = \sup I_0$ 注意到此时 $J \in \mathbf{P_0}, J \ne J_0$ 组成了对 $I_1$ 的一个划分,我们记这个划分为 $\mathbf{P_1}$ 所以我们可以对 $I_1, \mathbf{P_1}$ 作重复的操作。

同理如果 $a_0 \notin I_0$,那么 $\mathbf{P_0}$ 中一定存在一唯一个 $J_0$ 满足 $J_0$ 的下确界等于 $a_0$
所以同样可以做上面的操作得到 $I_1, \mathbf{P_1}$。

我们重复以上操作和得到了 $\{ J_0, J_1, .., J_{n-1} \} = \mathbf{P}$,注意到 $J_0$ 的上确界就是 $J_1$ 的下确界,以此类推。
我们得到 $\alpha(J_0) + \alpha( J_1) + .. + \alpha( J_{n-1})  = \alpha(a_1) - \alpha(a_0) + \alpha(a_2) - \alpha(a_1) + .. + \alpha(b_{n-1}) - \alpha(a_{n-1}) = \alpha(a_n) - \alpha(a_0) = \alpha(b) - \alpha(a)$。

注意到 $I_1, I_2, .. I_{n-1}$ 的上确界都没有发生变化,而 $I_n = \emptyset$,所以有 $J_{n-1} = I_{n-1}$,所以 $\sup J_{n-1} = b$

\subsection{更细划分}

假设 $I $ 有两个划分 $\mathbf{P}$ 和 $\mathbf{P'}$ 我们称 $\mathbf{P'}$ 是 $\mathbf{P}$ 的更细划分当且仅当对任意 $J \in \mathbf{P}$ 都有 $J' \in \mathbf{P'}$ 满足 $J' \subseteq J$ 。

\subsection{公共加细}

假设 $I $ 有两个划分 $\mathbf{P}$ 和 $\mathbf{P'}$,$\mathbf{P}$ 和 $\mathbf{P'}$ 的公共加细定义为 

\[
    \mathbf{P} \# \mathbf{P'} = \{ J \cap K \wvert J \in \mathbf{P}, K \in \mathbf{P'}\}
\]

容易证明公共加细是对 $I$ 的划分,而且既是 $ \mathbf{P}$ 的更细划分,也是 $\mathbf{P'}$ 的更细划分。

\section{分段常数函数}

\subsection{定义}
设 $f: I \to \mathbb{R}$,其中 $I$ 是有界区间,如果存在一个划分 $\mathbf{P}$ 使得 $\forall J \in \mathbf{P}, J \ne \emptyset \, f(J)$ 是一个单点集,那么 $f$ 就是关于 $\mathbf{P}$ 的分段常数函数。

注意到如果 $\mathbf{P'}$ 比 $\mathbf{P}$ 更细,那么 $f$ 也是在 $\mathbf{P'}$ 上的分段常数函数。

利用公共加细这个性质,可以证明分段常数函数对加减乘,$\max$ 和 $\min$ 运算保持封闭,如果 $\forall x, g(x) \ne 0$ 那么 $f/g$ 也是分段常数函数。

\subsection{分段常值积分}

若 $f$ 是在划分 $\mathbf{P}$ 上的分段常数函数,定义 $f$ 的分段常值积分为 
\[
    \int_{[\mathbf{P}]} f \mathrm{d}\alpha = \sum_{J \in \mathbf{P}}c_J \alpha(J)
\]

分段常值积分是和 $\mathbf{P}$ 无关的,为了证明这个,我们需要证明一个引理,如果 $\mathbf{P'}$ 是比 $\mathbf{P}$ 更细的划分,那么

\[
    \int_{[\mathbf{P}]} f \mathrm{d}\alpha =\int_{[\mathbf{P'}]} f\mathrm{d}\alpha
\]

如果 $\mathbf{P}'$ 是比 $\mathbf{P}$ 更细的划分,对于任意 $J \in \mathbf{P}$,
我们取所有的 $\mathbf{K_J} = \{ K \in \mathbf{P}' \wvert K \subseteq J \}$。我们先证明 $K_J$ 是 $J$ 的一个划分。

先证明任意 $K_1 ,\, K_2 \in \mathbf{K_J}$,不相交,这个可以通过 $\mathbf{P}'$时一个划分证明。再证明 

\[
J \subseteq \bigcup_{K \in \mathbf{K_J}} K
\]

我们用反证法,假设存在 $x \in J$,但对任意 $K \in \mathbf{K_J},\, x \notin K$,那么一定存在 $K'$ 不是 $J$ 的子集满足 $x \in K'$,
不妨设 $K'$ 是另一个 $\mathbf{P}$ 中的 $J'$ 的子集,得到 $x \in J \cap J'$,得到矛盾。

于是我们计算 $f$ 在 $\mathbf{P'}$ 上的分段常值积分可以分组计算再求和。

\begin{align*}
\sum_{K \in \mathbf{P'}}c_K \alpha(K) &= \sum_{J \in \mathbf{P}}\sum_{K \in \mathbf{K_J}}c_K \alpha(K)  \\
&= \sum_{J \in \mathbf{P}}\sum_{K \in \mathbf{K_J}}c_J \alpha(K) = \sum_{J \in \mathbf{P}}c_J\sum_{K \in \mathbf{K_J}} \alpha(K) \\
&= \sum_{J \in \mathbf{P}}c_J \alpha(J) = \int_{[\mathbf{P}']} f \mathrm{d}\alpha\\
\end{align*}

证明如下假设 $\mathbf{P}$ 和 $\mathbf{P'}$ 都是 $f$ 的一个划分,而且 $f$ 在 $\mathbf{P}$ 和 $\mathbf{P'}$ 上都是分段常值函数。我们根据引理可以得到  

\[
    \int_{[\mathbf{P}]} f \mathrm{d}\alpha =\int_{[\mathbf{P \# P'}]} f \mathrm{d}\alpha= \int_{[\mathbf{P'}]} f \mathrm{d}\alpha
\]

因为分段常数函数的积分和划分无关了,所以可以写成如下形式

\[
\int_{I} f \mathrm{d}\alpha
\]

\subsection{分段常值积分性质}

以下我们暂时省略 $\mathrm{d} \alpha$

\begin{enumerate}
    \item $\int_{I} (f + g) = \int_{I} f + \int_{I} g$ 这个用公共加细很容易证明。
    \item $\int_{I} cf = c\int_{I} f$ 这个可以套定义。
    \item $\int_{I} (f - g) = \int_{I} f - \int_{I} g$ 用上面两个的组合可以证明
    \item 如果 $f(x) \ge 0$, $\int_{I} f \ge 0$,套定义。
    \item 如果 $f(x) \ge g(x)$ 那么 $\int_{I} f \ge \int_{I}g $,也是套定义。
    \item 如果 $f(x) = c$ 那么 $\int_{I} f  = c \lvert I \rvert $,也是套定义。
    \item 有限可加性,如果 $\{J,K \}$ 是对 $I$ 的一个划分,那么 $\int_{I}f = \int_{J}f + \int_{K} f$,因为我们可以从 $J$ 和 $K$ 各自的划分取并集得到 $I$ 的一个划分。
\end{enumerate}

\section{黎曼积分}

\subsection{上方控制和下方控制}

$f: I \to \mathbb{R}$ 且 $g: I \to \mathbb{R}$,如果 $g(x) \ge f(x)$,那么 $g(x) $ 从上方控制 $f(x)$。
如果 $g(x) \le f(x)$ 那么 $g(x)$ 从下方控制 $f(x) $。

\subsection{上下黎曼积分的定义}

上下黎曼积分是定义在有界函数以及有界区间上的。

令 $I \subseteq \R$ 是有界区间, $\alpha$ 单调增,有界函数 $f: I \to \R$ 的上黎曼积分定义为:

\[
   \overline{\int}_I f =  \inf \{ \int_{I} g \mathrm{d}\alpha ,\quad g\text{\,是从上方控制 \:}f \text{的分段常数函数}\}
\]

下黎曼积分定义为

\[
   \underline{\int}_I f =  \sup \{ \int_{I} g \mathrm{d}\alpha,\quad g\text{\,是从下方控制 \:}f \text{的分段常数函数}\}
\]

这里要注意的是,上黎曼积分是用下确界定义的,而下黎曼积分是用上确界定义的。因为上确界可以用最大附着点定义,下确界可以用最小附着点定义,后面做证明的时候可以利用数列极限的性质。


\subsection{黎曼可积的定义}

假设 $f: I \to \mathbb{R}$ 是有界的,如果 $f$ 的上黎曼积分等于 $f$ 的下黎曼积分,那么 $f$ 黎曼可积。


\subsection{分段常数函数是黎曼可积的}

这个很容易证明。分段常数函数既从上方控制它自身,又从下方控制它自身。

\subsection{黎曼和}

为了便于构造分段常数函数,我们定义上黎曼和 还有 下黎曼和。假设 $f: I \to \R$ 是有界的,而且 $\mathbf{P}$ 是 $I$ 的一个划分,那么定义 $f$ 的上黎曼和为

\[
    U(f, \mathbf{P}) = \sum_{J \in \mathbf{P}} (\sup_{x \in J}f(x)) \alpha(J)
\]

定义下黎曼和为

\[
    L(f, \mathbf{P}) = \sum_{J \in \mathbf{P}} (\inf_{x \in J}f(x)) \alpha(J)
\]

\subsection{上下黎曼和与上下黎曼积分}

上黎曼和的下确界就是上黎曼积分,首先上黎曼和的定义中隐含了一个上方控制的分段常数函数,所以上黎曼和这个集合的附着点也是上黎曼积分这个集合的附着点,所以

\[
    \overline{\int}_I f \le \inf \{ U(f, \mathbf{P}) \}
\]

假设有一个从上方控制的分段常数函数 $g(x)$ 还有划分 $\mathbf{P}$,我们取 $h(x)$ 为

\[
    h(x) = \sup \{ f(J) \}, \: x \in J
\]

于是我们得到了 $h(x) \le g(x)$ 所以通过上黎曼积分对应的附着点,可以构造出上黎曼和的附着点,而且上黎曼和的附着点小于等于上黎曼积分对应的附着点。所以有

\[
 \inf \{ U(f, \mathbf{P}) \}   \le  \overline{\int}_I f
\]

结合之前的结论得到

\[
 \inf \{ U(f, \mathbf{P}) \}  =  \overline{\int}_I f
\]

同理可以证明


\[
 \sup \{ L(f, \mathbf{P}) \}  =  \underline{\int}_I f
\]

\section{黎曼积分的性质}

\subsection{代数性质}

下面假设有界函数 $f: I \to \mathbb{R}$ 和 $g: I \to \mathbb{R}$ 都是黎曼可积的

\begin{enumerate}
    \item $\int(f+g) = \int f + \int g $ 这个很好证明,$\overline{f}$从上方控制 $f$,$\overline{g}$ 从上方控制 $g$ 可以得到 $\overline{f}+\overline{g}$ 从上方控制 $f+g$。利用附着点的性质得到
    \[
  \underline{\int}f + \underline{\int}g \le \underline{\int}(f+g)  \le \overline{\int}(f+g) \le \overline{\int}f + \overline{\int}g
    \]

    \item $\int cf = c \int f$ 这个要讨论 $c > 0$ 和 $c < 0$,$c > 0$ 时,$\overline{f}$ 上方控制 $f$ 可以得到 $c\overline{f}$ 上方控制 $cf$。
    \[
  c\underline{\int}f \le  \underline{\int}cf \le \overline{\int}cf \le c \overline{\int}f 
    \]

    如果 $c < 0$,如果 $\overline{f}$ 上方控制 $f$ 可以得到 $c\overline{f}$ 从下方控制 $cf$ 所以得到

    \[
      c\overline{\int}f   \le \underline{\int}cf \le \overline{\int} cf \le c \underline{\int}f
    \]

    \item $\int_{I}(f-g) = \int_{I}f - \int_{I}g $ 使用上面的结论可以证明

    \item $f(x) \ge 0$ 则 $\int_{I} f(x) \ge 0$,因为 $0$ 从下方控制了 $f(x)$ 所以有
    \[
        0 \le \underline{\int}_{I} f \le \int_{I} f
    \]

    \item $f(x) \ge g(x)$ 则 $\int_{I}f \ge \int_{I} g$ 用上面的结论很容易证明。

    \item $f(x) = c$ 则 $\int_{I}f = c \alpha(I)$,因为 $f(x) =c$ 从上方控制也从下方控制。

    \[
        c\alpha(I) \le \underline{\int}_I f \le \overline{\int_I}f \le c \alpha(I)
    \]

    \item 若 $I \subseteq J$ 而且 $J$ 是有界区间,构造函数 $F$

    \[
    F(x) = \begin{cases}
        f(x),\, x \in I \\
        0,\, x \notin I 
    \end{cases}
    \]

    那么 $F$ 在 $J$ 上黎曼可积,而且有

    \[
        \int_{J} F = \int_{I} f = \int_{I} F
    \]

    这里只要对从上方控制 $f$ 的分段常数函数 $\overline{f}$ 以及从下方控制 $f$
    的分段常数函数 $\underline{f}$ 作定义域扩展得到分别从上方和下方控制 $F$ 的分段常数函数即可。
    注意这里麻烦的地方在于要对 $K = J \setminus I$ 是否是一个连通集作讨论,如果不连通的话,需要分解成两个不相交但各自是连通的子集,
    也就是 $K \cap (-\infty, \sup I),\: K \cap (\inf I, \infty)$

    \item 如果 $\{ J, K \} $ 是 $I$ 的一个划分,并且 $f: J \to \mathbb{R}$ 和 $f: K \to \mathbb{R}$ 都是黎曼可积的,那么有

    \[
        \int_{J}f + \int_{K}f = \int_{I}f
    \]

    我们先证明从左边到右边,利用上面的结论我们可以得到

    \begin{align*}
    \int_{J}f &= \int_{J}f \chi_{J} = \int_{I}f \chi_{J} \\
    \int_{K}f &= \int_{K}f \chi_{K} = \int_{I}f \chi_{K} \\
    \int_{J}f + \int_{K} f &= \int_{I}(\chi_{J} + \chi_{K})f  = \int_{I}f
    \end{align*}

    再证明从右边可积可以得到左边两项都可积即可,只要把从上方控制 $f$ 的分段常数函数
    $\overline{f}$ 拆成 $J$ 和 $K$ 的两部分即可。

\end{enumerate}

\subsection{$\max$ 和 $\min$ 保持黎曼可积}

若 $f: I \to \mathbb{R}$ 和 $g: I \to \mathbb{R}$ 都是黎曼可积的,那么 $h=\max(f,g)$ 也是黎曼可积的。

先证明一个引理,假设 $\overline{f}$ 和 $\overline{g}$ 分别从上方控制 $f$ 和 $g$ 那么 $\overline{h} = \max(\overline{f},\overline{g})$
从上方控制 $\max(f,g)$。若 $\underline{f}$ 和 $\underline{g}$ 分别从下方控制 $f$ 和 $g$ 那么
$\underline{h} = \max(\underline{f}, \underline{g})$ 从下方控制 $\max(f,g)$,并且有

\[
\overline{h} - \underline{h} \le \max(\overline{f} - \underline{f}, \overline{g} - \underline{g})
\]

证明如下:

\begin{align*}
    f \le \overline{f} \le & \max(\overline{f}, \overline{g}) \\
    g \le \overline{g} \le & \max(\overline{f}, \overline{g}) \\
    \max(f,g)\le & \max(\overline{f}, \overline{g}) \\
\end{align*}

同理有


\begin{align*}
    \underline{f} \le f \le & \max(\overline{f}, \overline{g}) \\
    \underline{g} \le g \le & \max(\overline{f}, \overline{g}) \\
    \max(\underline{f},\underline{g})\le & \max(f, g) \\
\end{align*}

最后得到

\[
\max(\underline{f},\underline{g})\le  \max(f, g) \le \max(\overline{f}, \overline{g})
\]

并且有

\begin{align*}
  \overline{h} - \underline{h} & = \max(\overline{f}, \overline{g}) - \max(\underline{f}, \underline{g}) \\
    & = \max(\overline{f}, \overline{g}) + \min(-\underline{f}, -\underline{g}) \\
    & = \max(\overline{f} + \min(-\underline{f}, -\underline{g}), \overline{g} + \min(-\underline{f}, -\underline{g})) \\
    & \le \max(\overline{f} -\underline{f}, \overline{g} -\underline{g}) \\
\end{align*}

若任意分段常数函数 $\overline{f}$ 和 $\overline{g}$ 分别从上方控制 $f$ 和 $g$,那么取他们的公共加细,然后在公共加细上取 $\overline{h} = \max(\overline{f},\, \overline{g})$
可以得到一个在上方控制 $\max(f,g)$ 的分段常数函数。同理在公共加细上取 $\underline{h} = \max(\underline{f}, \underline{g})$
可以得到在下方控制 $\max(f,g)$ 的分段常数函数。

我们取分段常数函数的函数列 $\overline{f}_n$ 和 $\overline{g}_n$ 满足

\begin{align*}
    \lim_{n \to \infty}\int_{I}\overline{f}_n = \overline{\int}_{I}f \\
    \lim_{n \to \infty}\int_{I}\overline{g}_n = \overline{\int}_{I}g \\
\end{align*}


同理分段常数函数的函数列 $\underline{f}_n$ 和 $\underline{g}_n$ 满足

\begin{align*}
    \lim_{n \to \infty}\int_{I}\underline{f_n} = \underline{\int}_{I}f \\
    \lim_{n \to \infty}\int_{I}\underline{g_n} = \underline{\int}_{I}g \\
\end{align*}

然后取 $\overline{h_n} = \max(\overline{f}_n, \overline{g}_n)$,$\underline{h_n} = \max(\underline{f_n}, \underline{g_n})$

根据引理的条件有

\begin{align*}
\int_{I}\overline{h}_n - \int_{I}\underline{h_n} & = \int_{I}\overline{h}_n - \underline{h_n} \\
& \le \int_{I} \max(\overline{f}_n -\underline{f_n}, \overline{g}_n -\underline{g_n}) \\
& \le \int_{I} \overline{f}_n -\underline{f_n} + \overline{g}_n - \underline{g_n} \\ 
&\le \int_{I} \overline{f}_n -\underline{f_n} + \int_{I} \overline{g}_n -\underline{g_n}
\end{align*}

所以有

\begin{align*}
0 \le \overline{\int}h - \underline{\int}h & \le \int_{I}\overline{h}_n - \int_{I}\underline{h_n} \\
& \le \int_{I} \overline{f}_n -\underline{f_n} + \int_{I} \overline{g}_n -\underline{g_n}
\end{align*}

然后对 $n$ 取极限得到 

\[
\overline{\int}h - \underline{\int}h = 0
\]

所以 $\max$ 保持黎曼可积,注意到 $\min(f,g) = - \max(-f,-g)$,所以 $\min$ 也是黎曼可积的。

\subsection{绝对值保持黎曼可积}

$\lvert f \rvert = \max(f, -f)$

\subsection{正部和负部保持黎曼可积}

注意到 $f_{+} = \max(f,0)$ 且 $f_{-} = \max(-f,0)$

\subsection{乘积保持黎曼可积}

若 $f: I \to \mathbb{R}$ 和 $g: I \to \mathbb{R}$ 都是黎曼可积的,那么 $h = f \cdot g$ 也是黎曼可积的。

这里要把 $f$ 和 $g$ 各自拆成 $f = f_{+} + f_{-}$ 还有 $g = g_{+} + g_{-}$ 的形式,那么  $f \cdot g = f_{+}g_{+} + f_{+}g_{-} + f_{-}g_{+} +f_{-}g_{-} $

我们只需要证明 $f_{+}g_{+}$ 可积,下面的证明过程中我们暂时把$\overline{f_{+}}$ 简单记为 $\overline{f}$。只用到 $f_{+}$ 可积并且非负这个性质,$g_{+}$ 同理。其他项的证明类似。假设 $\overline{f}$ 和 $\overline{g}$ 分别从上方控制 $f_{+}$ 和 $g_{+}$, 
$\underline{f}$ 和 $\underline{g}$ 分别从下方控制 $f_{+}$ 和 $g_{+}$,
那么有 $ \underline{f} \underline{g} \le f_{+}g_{+} \le \overline{f} \overline{g}$ 注意到有下面不等式

\[
    \overline{f}\overline{g} - \underline{f} \, \underline{g} =  \overline{f}(\overline{g} - \underline{g}) + \underline{g}(\overline{f} - \underline{f}) \\
\]

\begin{align*}
    \int_{I}\overline{f}\overline{g} - \int_{I}\underline{f}\underline{g} &= \int_{I}\overline{f}\overline{g} - \underline{f}\underline{g} \\
    & \le \int_{I}\overline{f}(\overline{g} - \underline{g}) + \int_{I}\underline{g}(\overline{f} - \underline{f})  \\
    & \le \int_{I}M(\overline{g} - \underline{g}) + \int_{I}M(\overline{f} - \underline{f}) \le 2M\epsilon
\end{align*}

上面的 $M$ 是 $\max \{ f_{+}(x), g_{+}(x) \wvert x \in I \}$,$\epsilon$ 是利用 $f_{+}$ 和 $g_{+}$ 黎曼可积。

所以有 

\[
    \overline{\int}_{I}f_{+}g_{+} - \underline{\int}_{I}f_{+}g_{+} \le 2 M \epsilon
\]

因为上面的 $\epsilon$ 可以取任意小所以$f_{+}g_{+}$ 黎曼可积,同理可以得到 $(-f_{-})(-g_{-})$ 黎曼可积等。

这是一个经典等式,只要有乘法和加法这个等式可以用于如下形式:

\[
    x_1y_1 - x_2y_2 = x_1(y_1 - y_2) + y_2(x_1 - x_2)
\]

\section{黎曼可积的函数}

\subsection{有界区间上一致连续的函数}

假设 $f: I \to \R$ 一致连续,根据一致连续的定义,那么 $f$ 一定是有界的。假设 $f$ 无界,那么可以构造出一个发散到无穷的序列$f(x_n)$ 满足 $f(x_n) \ge n$
,这个 $x_n$ 可以有一个收敛的子列$x_{g(n)}$,于是 $f(x_{g(n)})$ 是柯西列,因为一致连续会把柯西列映射到柯西列,这显然和 $f(x_{g(n)})$ 发散矛盾了。

既然 $f$ 有界,我们可以分析它的上下黎曼积分。令 $\inf I =a$ 且 $\sup I = b$,我们把 $I$ 分成 $n$ 个有界区间 $J_1, J_2, .. J_n$,每个区间长度为 $(b-a)/n$。然后在每个有界区间上取上下确界
得到分段常数函数 $\overline{f}_n$ 和 $\underline{f_n}$ 

我们可以把$I$ 分的充分小,让每个划分中的区间满足一致收敛的条件,也就是 $(b-a)/n \le \delta$ $\overline{f}_n - \underline{f_n} \le \epsilon$
取 

\[
\lim_{n \to \infty}f(x_n) = \sup_{x \in J}f(x)
\]

和

\[
\lim_{n \to \infty}f(y_n) = \inf_{x \in J}f(x)
\]

可以得到

\[
\sup_{x \in J}f(x) - \inf_{x \in J}f(x) = \lim_{n \to \infty}f(x_n) - f(y_n)
\]

因为有 $f(x_n) - f(y_n) \le \epsilon$,所以有

\[
\sup_{x \in J}f(x) - \inf_{x \in J}f(x) \le \epsilon
\]

也就是说有 $\overline{f}_n - \underline{f_n} \le \epsilon $

因此有

\[
\overline{\int_I}f - \underline{\int_I}f \le \int_{I}\overline{f}_n - \underline{f_n} \le (\alpha(b) - \alpha(a))\epsilon
\]

因为 $\epsilon$ 是任意的,所以

\[
\overline{\int_I}f = \underline{\int_I}f
\]

\subsection{闭区间上的连续函数}

闭区间上的连续函数是一致连续的。

\subsection{有界区间上有界的连续函数}

这里要求 $\alpha$ 在区间的上下确界处是连续的

这里我们用闭区间上的连续函数分别从上方和下方控制黎曼积分。假设 $f: I \to \R$ 有界而且连续,那么 $f$ 在 $I$ 的任意一个闭区间子集上是黎曼可积的。
令 $a = \inf I,\, b = \sup I$ 也是对任意充分小的 $\epsilon > 0$ 有 $f$ 在 $[a+\epsilon, b-\epsilon]$ 上可积。

我们根据 $\epsilon$ 生成在 $I$ 上从上方控制 $f$ 的分段常数函数。$\overline{f}$ 在 $[a+\epsilon, b- \epsilon]$ 上从上方控制 $f$,那么令
$I_{\epsilon} = [a+\epsilon, b- \epsilon]$

\[
\overline{h}(x) = \begin{cases}
    \overline{f}(x),\, x \in I_{\epsilon} \\
    M,\, x \notin I_{\epsilon} \\
\end{cases}
\]

其中 $M$ 是 $f$ 的上确界,容易证明 $\overline{h}$ 是分段常数函数,而且从上方控制 $f$。
同理可以构造$\underline{h}$


\[
\underline{h}(x) = \begin{cases}
    \underline{f}(x),\, x \in I_{\epsilon} \\
    m,\, x \notin I_{\epsilon} \\
\end{cases}
\]

接下来我们取 $\overline{f}_n$ 和 $\underline{f_n}$ 满足

\[
\lim_{n \to \infty}\int_{I_\epsilon}\overline{f}_n = \lim_{n \to \infty}\int_{I_\epsilon}\underline{f_n}
\]

根据 $\overline{f}_n: I_\epsilon \to \R$ 和 $\underline{f_n}: I_\epsilon \to \R$ 我们可以构造对应的 $\overline{h}_n: I \to \R$
和 $\underline{h_n}: I_\epsilon \to \R$,注意到有

\[
\int_{I}\overline{h}_n - \int_{I}\underline{h_n} = \int_{[a,a+\epsilon]}(M-m)\mathrm{d}\alpha + \int_{[b-\epsilon,b]}(M-m)\mathrm{d}\alpha + \int_{I_\epsilon}\overline{f}_n + \int_{I_\epsilon}\underline{f_n}
\]

令 

\[
o(\epsilon) = \int_{[a,a+\epsilon]}(M-m)\mathrm{d}\alpha + \int_{[b-\epsilon,b]}(M-m)\mathrm{d}\alpha
\]

所以有

\[
\overline{\int_{I}}f - \underline{\int_{I}}f \le \int_{I}\overline{h}_n - \int_{I}\underline{h_n} \le o(\epsilon)  + \int_{I_\epsilon}\overline{f}_n + \int_{I_\epsilon}\underline{f_n}
\]

对 $n$ 取极限得到

\[
\overline{\int_{I}}f - \underline{\int_{I}}f \le  o(\epsilon)
\]

因为 $\alpha$ 在 $a$ 和 $b$ 处是连续的,所以对 $\epsilon \to 0$ 取极限得到

\[
\overline{\int_{I}}f = \underline{\int_{I}}f
\]

\subsection{有界区间上分段连续的有界函数}

利用之前的结论,先证明 $f$ 在每个划分上是黎曼可积的,然后利用黎曼积分的性质,合并区间。

\subsection{闭区间上的单调函数}

这里同样要求 $\alpha(x) = x$
我们先假设 $f$ 是单调的,然后把 $[a,b]$ 分成 $n$ 份左闭右开的区间,在并上一个单点集 $\{b\}$。每份上取上下确界得到上方控制的分段常数函数 $\overline{f}_n$,和从下方控制的分段常数函数 $\underline{f_n}$。
于是得到

\begin{align*}
\int_{[a,b]}\overline{f}_n - \int_{[a,b]}\underline{f_n} & \le \sum_{i=1}^{n}\frac{b - a}{n}(f(b_i) - f(a_i)) \\
    & \le (f(b) - f(a))\frac{b-a}{n}
\end{align*}

对 $n$ 取极限得到

\[
\int_{[a,b]}\overline{f}_n = \int_{[a,b]}\underline{f_n}
\]


\subsection{区间上的有界单调函数}

这里同样要求 $\alpha(x) = x$
我们可以通过在 $[a+\epsilon, b-\epsilon]$ 这个闭区间上构造单调函数,然后因为它是黎曼可积的,
所以可以用来构造在 $I$ 上从上方控制 $f$ 的函数,以及在 $I$ 上从下方控制 $I$ 的函数。

具体证明如下

令 $I_\epsilon = [a+\epsilon, b-\epsilon]$,因为 $f$ 在 $I_\epsilon$ 上可积,
取 $\overline{f}_n$ 在上方控制 $f$,以及 $\underline{f_n}$ 在下方控制 $f$,并且有

\[
\lim_{n \to \infty}\int_{I_\epsilon}\overline{f}_n = \overline{\int_{I_\epsilon}}f = \lim_{n \to \infty}\int_{I_\epsilon}\underline{f_n} = \underline{\int}_{I_\epsilon}f
\]

对 $\overline{f}_n$ 的定义域作扩展,补充$f$ 的上界得到在 $I$ 从上方控制 $f$ 的函数 $\overline{f}_n$,同理可得 $\underline{f_n}$。
于是有

\[
\int_{I}\overline{f_n} - \int_{I}\underline{f_n} \le \int_{I_\epsilon}\overline{f_n} - \int_{I_\epsilon}\underline{f_n} + 2\epsilon(M-m)
\]

对 $n$ 取极限得到证明。

\subsection{积分判别法}

\section{微积分基本定理}

\subsection{微积分第一基本定理}

$a < b,\, a,b \in \R$ 令 $f: [a,b] \to \R$ 是黎曼可积的函数。那么构造函数

\[
F(x) = \int_{[a,x]}f \mathrm{d}x
\]

那么 $F$ 是连续的,此外若 $x_0 \in [a,b]$ 且 $f$ 在 $x_0$ 处连续,那么 $F$ 在 $x_0$ 处可微,而且 $F'(x_0) = f(x_0)$

下面给出证明:

先不妨设 $\delta > 0$ 计算右极限

\[
F(x_0 + \delta) - F(x_0) = \int_{[x_0, x_0 + \delta]}f \mathrm{d}x 
\]

因为 $f$ 有界,所以

\[
\delta m \le \int_{[x_0, x_0 + \delta]}f \mathrm{d}x \le \delta M
\]

对 $\delta \to 0$ 取极限得到

\[
\lim_{\delta \to 0} F(x_0 + \delta) - F(x_0) = \lim_{\delta \to 0}\int_{[x_0, x_0 + \delta]}f \mathrm{d}x = 0
\]

左极限也是同理

如果 $f$ 在 $x_0$ 处连续,那么对任意 $\epsilon > 0$ 存在 $\delta$,对任意 $\lvert x - x_0 \rvert \le \delta$ 有

\[
-\epsilon \le f(x) - f(x_0) \le \epsilon
\]

我们取 $x_0 < y < x_0 + \delta$,在上面同时取积分得到

\[
-\epsilon(y-x_0) \le \int_{[x_0,y]}f(x) - (y-x_0)f(x_0) \le \epsilon(y-x_0)
\]

化简得到

\[
\lvert \frac{F(y) - F(x_0)}{y-x_0} - f(x_0) \rvert \le \epsilon 
\]

然后对 $\delta \to 0^+$ 取极限得到

\[
F'(x_0^+) = f(x_0)
\]

同理我们可以证明 $F'(x_0^-) = f(x_0)$

\subsection{微积分第二基本定理}

若 $a<b,\, a,b \in \R$ 且 $f: [a,b] \to \R$ 黎曼可积,若 $F$ 是 $f$ 的原函数,那么有

\[
\int_{[a,b]}f \mathrm{d} x = F(b) - F(a)
\]

假设 $f$ 有一个上黎曼和的序列 $\overline{f}_n$ 满足

\[
\lim_{n \to \infty} \int_{[a,b]}\overline{f}_n \mathrm{d}x = \int_{[a,b]}f \mathrm{d}x
\]

我们先证明有 $\forall n$ 

\[
f(b) - f(a) \le \int_{[a,b]}\overline{f}_n \mathrm{d}x
\]

因为 $\overline{f}_n$ 是分段常数函数所以有

\[
\int_{[a,b]}\overline{f}_n \mathrm{d}x = \sum_{J \in \mathbf{P}}\sup f(J) \lvert J \rvert = \sum_{i}\sup f(J_i) (b_i -a_i)
\]

而 $F(b) - F(a)$ 使用中值定理可以展开成

\[
F(b) - F(a) = \sum_{i} F(b_i) - F(a_i) = \sum_{i} f(\alpha_i)(b_i - a_i)
\]

显然有 $\alpha_i \le \sup f(J_i)$

所以有 


\[
F(b) - F(a) \le \int_{[a,b]}\overline{f}_n \mathrm{d}x
\]

同理可以证明


\[
\int_{[a,b]}\underline{f_n} \mathrm{d}x \le F(b) - F(a) 
\]

然后对 $n$ 取极限得到

\[
F(b) - F(a) = \int_{[a,b]}f \mathrm{d}x
\]

\section{微积分基本定理的结论}

\subsection{分部积分公式}

令 $I = [a,b]$,令 $F: [a,b] \to \R$ 和 $G: [a,b] \to R$ 都是可微,并且 $F'$ 和 $G'$
在 $[a,b]$ 上黎曼可积
那么有

\[
\int_{[a,b]}FG' = F(b)G(b) - F(a)G(a) - \int_{[a,b]}F'G
\]

因为 $(FG)'= F'G + FG'$,所以

\begin{align*}
\int_{[a,b]}(FG)' &= \int_{[a,b]}F'G + \int_{[a,b]}FG' = F(b)G(b) - F(a)G(a) \\
\end{align*}

所以有

\[
\int_{[a,b]}FG' = F(b)G(b) - F(a)G(a) - \int_{[a,b]}F'G
\]

\subsection{$\alpha$ 长度 I}

若 $\alpha: [a,b] \to \R$ 单调增且 $\alpha$ 在 $[a,b]$ 上可微,其导函数为 $\alpha'$ 并且 $\alpha'$
在 $[a,b]$ 上黎曼可积。令 $f: [a,b ] \to \R$ 是分段常数函数,那么有

\[
\int_{I}f \mathrm{d} \alpha = \int_{I}f \alpha' \mathrm{d}x
\]

证明:

\begin{align*}
    \int_{I}f \mathrm{d} \alpha & = \sum_{J \in \mathbf{P}}c_J \int_{J} \alpha' = \sum_{J \in \mathbf{P}}\int_{J} c_J\alpha' \\
    &= \sum_{J \in \mathbf{P}}\int_{I}f\alpha'\chi_{J} = \int_{I}\sum_{J \in \mathbf{P}}f \alpha' \chi_j \\
    & = \int_{I}f\alpha'(\sum_{J \in \mathbf{P}}\chi_J) = \int_{I}f \alpha'
\end{align*}


\subsection{$\alpha$ 长度 II}

若 $\alpha: [a,b] \to \R$ 单调增且 $\alpha$ 在 $[a,b]$ 上可微,其导函数为 $\alpha'$ 并且 $\alpha'$
在 $[a,b]$ 上黎曼可积。若 $f: [a,b ] \to \R$ 黎曼可积,那么有

\[
\int_{I}f \mathrm{d} \alpha = \int_{I}f \alpha' \mathrm{d}x
\]

这个证明会用到之前的结论,令 $\overline{f}_n$ 从上方控制 $f$,并且有

\[
\lim_{n \to \infty}\int_{I}\overline{f}_n \mathrm{d} \alpha = \overline{\int_{I}}f \mathrm{d} \alpha
\]

注意对任意 $n$ 有

\[
\int_{I}\overline{f}_n \mathrm{d} \alpha = \int_{I}\overline{f}_n \alpha'
\]

因为 $\overline{f}_n \alpha'$ 从上方控制 $f\alpha'$,尽管它不是分段常数函数,但是有

\[
\overline{\int_I}f \alpha' \le \int_{I}\overline{f}_n \alpha'
\]

对 $n$ 取极限得到

\[
\overline{\int_I}f \alpha' \le \overline{\int_I} f \mathrm{d} \alpha
\]

同理我们可以得到

\[
\underline{\int_I} f \mathrm{d} \alpha \le \underline{\int_I}f \alpha' 
\]

所以

\[
\int_{I}f \mathrm{d} \alpha = \int_{I}f \alpha' \mathrm{d}x
\]

\subsection{变量代换公式 I}

$\phi: [a,b] \to [\phi(a), \phi(b)]$  是连续的单调增函数,$f: [\phi(a), \phi(b)] \to \R$ 是分段常数函数。
那么 记 $I: [a,b],\, \phi(I) = [\phi(a), \phi(b)]$,有$f \circ \phi$ 依然是分段常数函数,并且有

\[
\int_{I} f \circ \phi \mathrm{d} \phi = \int_{\phi(I)}f
\]

先证明 $f \circ \phi$ 是分段常数函数,因为 $f$ 是分段常数函数,所以 $\phi(I)$ 可以划分成若干个不相交的子区间。
记 $\mathbf{P} = \{J_1, J_2, .. J_n\}$。我们容易证明 $\mathbf{P'} = {\phi^{-1}(J_1),\phi^{-1}(J_2), ..\phi^{-1}(J_n)}$ 
是对 $[a,b]$ 的划分,单调且连续函数的原像运算保持连通性。

我们可以证明 $\phi^{-1}(J_1)$ 是连通的,不妨令 $x_1,\, x_2 \in \phi^{-1}(J_1)$ 且 $x_1 \le x_2$,
那么若 $x_1 \le x_3 \le x_2$,必然有 $\phi(x_1) \le \phi(x_3) \le \phi(x_2)$,因为 $J_1$ 是连通的,
所以一定有 $\phi(x_3) \in J_1$,所以 $x_3 \in \phi^{-1}(J_1)$

所以我们可以这样计算 

\begin{align*}
 \int_{I} f \circ \phi \mathrm{d} \phi &= \sum_{J \in \mathbf{P}} c_J \: \phi(\phi^{-1}(J)) \\
 &= \sum_{J \in \mathbf{P}} c_J \lvert J \rvert = \int_{\phi(I)}f
\end{align*}


\subsection{变量代换公式 II}

$\phi: [a,b] \to [\phi(a), \phi(b)]$  是连续的单调增函数,$f: [\phi(a), \phi(b)] \to \R$ 是黎曼可积的函数。
那么 记 $I: [a,b],\, \phi(I) = [\phi(a), \phi(b)]$,有$f \circ \phi$ 是相对 $\phi$ 在 $[a,b]$ 上黎曼可积,并且有

\[
\int_{I} f \circ \phi \mathrm{d} \phi = \int_{\phi(I)}f
\]

证明如下,我们构造在 $\phi(I)$ 上从上方控制 $f$ 的函数 $\overline{f_n}$ 并且有

\[
\lim_{n \to \infty}\int_{\phi(I)}\overline{f}_n = \overline{\int_{\phi(I)}}f
\]

利用之前的结论有

\[
\int_{\phi(I)}\overline{f}_n = \int_{I} \overline{f}_n \circ \phi \,\mathrm{d} \phi
\]

注意到有 $\overline{f_n} \circ \phi$ 在 $I$ 上从上方控制 $f \circ \phi$,所以有

\[
\overline{\int_{I}}f \circ \phi \, \mathrm{d} \phi \le \int_{I} \overline{f}_n \circ \phi \,\mathrm{d} \phi
\]

对 $n$ 取极限得到

\[
\overline{\int_{I}}f \circ \phi \, \mathrm{d} \phi \le \overline{\int_{\phi(I)}}f
\]

同理可以得到


\[
\underline{\int_{\phi(I)}}f \le \underline{\int_{I}}f \circ \phi \, \mathrm{d} \phi
\]

所以

\[
\int_{I}f \circ \phi \, \mathrm{d} \phi = \int_{\phi(I)}f
\]


\subsection{变量代换公式 III}

$\phi: [a,b] \to [\phi(a), \phi(b)]$  是可微的单调增函数,并且在 $[a,b]$ 上黎曼可积。
$f: [\phi(a), \phi(b)] \to \R$ 是黎曼可积的函数。
那么 记 $I: [a,b],\, \phi(I) = [\phi(a), \phi(b)]$,
有$(f \circ \phi) \phi'$ 是 在 $[a,b]$ 上黎曼可积,并且有

\[
\int_{I}(f \circ \phi) \phi' = \int_{\phi(I)}f
\]

证明如下,根据之前的结论有

\[
\int_{\phi(I)} f = \int_{I} f \circ \phi \: \mathrm{d} \phi
\]

再结合 $\phi'$ 可微,而且可积有

\[
\int_{I} f \circ \phi \: \mathrm{d} \phi = \int_{I} (f \circ \phi) \: \phi'
\]