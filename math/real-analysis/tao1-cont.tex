\chapter{$\R$ 上的连续函数}

\section{点集拓扑}

\subsection{区间}

\begin{enumerate}
    \item 闭区间

    闭区间定义为 $a,b \in \R^*,\, [a,b] = \{x \in \R^* \wvert a \le x \le b \}$

    \item 半开区间


    半开区间定义为 $a,b \in \R^*,\, (a,b] = \{x \in \R^* \wvert a < x \le b \},\, [a,b) = \{x \in \R^* \wvert a \le x < b \}$

    \item 开区间


    开区间定义为 $a,b \in \R^*,\, (a,b) = \{x \in \R^* \wvert a < x < b \}$
\end{enumerate}

\subsection{附着点}

$X \subseteq \R$,$x \in \R$ 是 $X$ 的附着点当且仅当 $\forall \epsilon > 0, \exists y \in X,\, \lvert y -x \rvert \le \epsilon$

\subsection{闭包}

$X \subseteq \R$ 的闭包 $\overline{X}$ 定义为所有 $X$ 的附着点的集合。

\subsection{闭包运算性质}

\begin{enumerate}
    \item $x \in \overline{X} \: \text{iff} \: \exists y_n \in X,\, \lim_{n \to \infty} y_n = x$

    选择公理

    \item $X \subseteq \overline{X}$

    \item $\overline{X \cup Y} = \overline{X} \cup \overline{Y}$

    利用1 很容易证明

    \item $\overline{X \cap Y} \subseteq \overline{X} \cap \overline{Y}$


    利用1 很容易证明
\end{enumerate}

\subsection{区间的闭包}

\begin{enumerate}
    \item $(a,b),\, (a,b],\, [a,b),\, [a,b]$ 的闭包是 $[a,b]$


    我们先证明 $[a,b]$ 的闭包就是 $[a,b]$,令 $X = [a,b]$,自然有 $[a,b] \subseteq \overline{X}$,设 $x_n \in [a,b]$ 根据极限的性质有

    \[
        a \le \lim_{n \to \infty} x_n \le b
    \]
    所以有 $\overline{X} \subseteq [a,b]$

    $X= (a,b)$ 也是同理,利用 $(a,b) \subseteq [a,b]$ 得到 $\overline{X} \subseteq [a,b]$,再根据

    \[
        \lim_{n \to \infty} a+ \frac{1}{n} = a
    \]

    以及

    \[
        \lim_{n \to \infty} b- \frac{1}{n} = b
    \]

    得到 $a,b \in \overline{X}$ 而且 $(a,b) \subseteq \overline{X}$,所以 $[a,b] \subseteq \overline{X}$

    \item $(a, \infty)$ 或 $[a,\infty)$ 的闭包是 $[a, \infty)$

    证明方法类似

    \item $(-\infty, a)$ 或 $[-\infty, a]$ 的闭包是 $[-\infty, a]$

    \item $(-\infty, \infty)$ 的闭包是 $(-\infty, \infty)$

\end{enumerate}

\subsection{闭集}

集合 $X \subseteq \R$ 是闭集 当且仅当 $\overline{X} = X$

\subsection{极限点}

定义 $x \in \R$ 是 $X \subseteq \R$ 的极限点当且仅当 $x$ 是 $X \setminus \{ x\} $ 的附着点。也就是存在 $x_n \in X,\, x_n \ne x$ 满足

\[
\lim_{n \to \infty} x_n =x
\]

\subsection{Heine-Borel 定理}

若 $X \subseteq \R$ 以下两个命题等价:

\begin{enumerate}
    \item $X$ 是有界闭集
    \item $a_n \in X$,则必然有 $a_n$ 的某个子列收敛到某个 $L \in X$ 
\end{enumerate}

利用实数数列的性质很容易证明。

\section{函数的极限}

\subsection{函数极限定义}

函数 $f: E \to \R$,设 $x_0$ 是 $E$ 的附着点,我们称 $f$ 在 $x=x_0$ 处取得极限当且仅当 对任意 $\epsilon > 0$ 存在 $\delta > 0$ 
对任意 $x \in E \cap (x_0 - \delta, x_0 + \delta)$ 有 $\lvert f(x) - L \rvert \le \epsilon $。

\subsection{函数极限等价定义}

以下两个命题等价。

\begin{enumerate}
    \item $f: E \to \R$ 在 $x = x_0$ 处取得极限 $L$
    \item 对任意 $\lim_{n \to \infty} a_n = x_0,\, a_n \in E$,有 $\lim_{n \to \infty} f(a_n) = L$
\end{enumerate}

先证明 $1 \to 2$,只要找到 $N$ 满足 $\forall n \ge N,\, \lvert a_n - L\rvert \le \epsilon$ 即可。再证明 $2 \to 1$,用反证法,假设存在 $\epsilon$,
存在无限多的 $x$ 满足 $\lvert f(x) - L\rvert > \epsilon $,根据选择公里可以构造出序列 $c_n$ 满足 $\lim_{n \to \infty} c_n = x_0$ 但 $c_n$ 不收敛到 $L$。

\subsection{函数在一个点只能有一个极限}

证明很容易,假设存在两个极限,然后用三角不等式证明这两个极限的距离小于任意的 $\epsilon$。

\subsection{函数极限运算法则}

假设 $f,g : E \to \R$ 在 $x = x_0$ 处分别取得极限 $L_1, L_2$

\begin{enumerate}
    \item $f+g$ 在 $x = x_0$ 处取得极限 $L_1 + L_2$

    用数列的极限法则易证

    \item $c \in \R, cf$ 在 $x = x_0$ 处取得极限 $cL_1$

    用数列的极限法则易证

    \item $fg$ 在 $x = x_0$ 处取得极限 $L_1 L_2$

    用数列的极限法则易证

    \item $\max(f,g)$ 在 $x = x_0$ 处取得极限 $\max(L_1, L_2)$

    用数列的极限法则易证

    \item $\min(f,g)$ 在 $x = x_0$ 处取得极限 $\min(L_1, L_2)$
    
    $\min(f,g) = -\max(-f,-g)$

    \item 若 $L_2 \ne 0$,$f/g$ 在 $x = x_0$ 处取得极限 $L_1/L_2$

    用数列的极限法则易证

\end{enumerate}

\subsection{复合函数的极限}

设 $f: X \to Y,\, g: Y \to \R\quad X,\, Y \subseteq \R$ 若有

\begin{align*}
& \lim_{x \to x_0} f(x) = y_0 \\
& \lim_{y \to y_0} g(y) = L \\
\end{align*}

那么有

\[
    \lim_{x \to x_0} g \circ f(x) = L 
\]

这个可以用数列来证明,下面假设有任意序列 $x_n \in X$ 满足

\[
\lim_{n \to \infty} x_n = x_0
\]

根据 $f$ 在 $x_0$ 处取得极限得到

\[
\lim_{n \to \infty} f(x_n) = y_0
\]

我们令 $y_n = f(x_n)$,注意到有 $y_n \in Y$ 利用 $g$ 在 $y_0$ 处的极限定义得到


\[
\lim_{n \to \infty} g(y_n) = \lim_{n \to \infty} g \circ f(x_n) = L
\]

\section{连续函数}

\subsection{连续的定义}

$f: X \to \R,\, X \subseteq \R$,若 $x_0 \in X$ 我们称 $f$ 在 $x = x_0$ 处连续如果满足

\[
\lim_{x \to x_0} f(x) = f(x_0)
\]

\subsection{连续的等价命题}

\begin{enumerate}
    \item 对任意点列 $x_n \in X, \lim_{n \to \infty} x_n = x_0$ 有 $\lim_{n \to \infty}f(x_n) = f(x_0)$
    \item 对任意 $\epsilon >0$ 存在 $\delta > 0$  满足对任意 $x \in X,\, \lvert x-x_0 \rvert \le \epsilon$
    \item 对任意 $\epsilon >0$ 存在 $\delta > 0$  满足对任意 $x \in X,\, \lvert x-x_0 \rvert < \epsilon$
\end{enumerate}


\subsection{保持连续的运算}

\begin{enumerate}
    \item $f+g$
    \item $cf$
    \item $\max(f,g)$
    \item $\min(f,g)$
    \item $fg$
    \item $f/g$ 如果 $g(x_0) \ne 0$
\end{enumerate}

以上这些可以套用数列极限证明

\subsection{指数函数连续}

\begin{enumerate}
    \item $a > 0,\, f: \R \to \R,\, f(x) = a^x$ 是连续函数

    很容易证明: $f(x+\delta) - f(x) = a^x(a^{\delta} - 1)$ 因为 $a^x$ 有界而且 $\lvert a^{\delta} -1 \rvert \le \epsilon$


    \item $p \in \R,\, f: (0,\infty) \to \R,\, f(x) = x^p$ 是连续函数

    首先我们证明 $p \in \N$ 的情况,这个用数学归纳法就可以证明。然后证明 $p \in \Z$ 的情况,这个用到复合函数的连续性。然后继续证明 $p \in \Q$ 的情况,这个会用到单调反函数的连续性以及复合函数的连续性。
    然后我们构造函数列 $f_n(x) = x^{p_n}$ 一致收敛到 $f(x) = x^p$,然后证明 $f(x)$ 的连续性。

\end{enumerate}

\subsection{绝对值函数是连续的}

$\lvert x\rvert = \max(x, -x)$

\subsection{连续函数的复合保持连续}

用复合函数极限的性质,很容易证明


\section{闭区间上连续函数的良好性质}

\subsection{有界}

$f: [a,b] \to \R$ 连续,那么 $f$ 在 $[a,b]$ 上有界。

假设 $f$ 在 $[a,b]$ 上无界,那么可以取无限个 $x_n \in [a,b]$ 满足 $\lvert f(x_n) \rvert \ge n$,根据有界序列必有收敛子列,我们可以取 $x_n$ 的一个收敛子列满足

\[
\lim_{n \to \infty}x_{h(n)} = x_0,\, x_0 \in [a,b]
\]

根据连续性得到 


\[
\lim_{n \to \infty}f(x_{h(n)}) = f(x_0),\, \lvert f(x_0) \rvert < \infty
\]

这显然和 $\lvert f(x_n) \rvert \ge n$ 矛盾了。


\subsection{有最大值和最小值}

取 $\{f(x) \wvert x \in [a,b]\}$ 的上确界 $L$,那么存在 $x_0 \in [a,b]$ 满足 $f(x_0) = L$。

证明:根据上确界的定义,那么一定有 $x_n \in [a,b]$ 满足 

\[
    \lim_{n \to \infty}f(x_n) = L,\, f(x_n) \le L
\]

因为$x_n$ 有界,那么一定有收敛子列 $x_{h(n)}$ 满足

\[
    \lim_{n \to \infty}x_{h(n)} = x_0,\, x_0 \in [a,b]
\]

根据 $f$ 的连续性得到

\[
f(x_0) = f(\lim_{n \to \infty}x_{h(n)}) = \lim_{n \to \infty}f(x_{h(n)}) = L
\]

\subsection{介值定理}

若 $f: [a,b] \to \R$ 且 $f(a) \le y \le f(b)$,那么存在 $x \in [a,b]$ 满足 $f(x) = y$

证明:若 $f(a) = f(b)$ 或者 $y = f(a)$ 时显然有。下面说明 $f(a) < y < f(b)$ 的情况。我们采用二分法,取 $l_1 = a,\, r_1 = b,\, m_1 = (a+b)/2$。然后我们按照 $l_n, \, r_n,\, m_n$ 不断
生成 $l_{n+1},\, r_{n+1},\, m_{n+1}$ 规则是这样的,如果 $f(m_n) \le y$,那么取 $l_{n+1} = m_n,\, r_{n+1} = r_n,\, m_{n+1} = (l_{n+1} + r_{n+1}) /2$,否则取
$l_{n+1} = l_n,\, r_{n+1} = m_n,\, m_{n+1} = (l_{n+1} + r_{n+1}) /2$。通过数学归纳法可以得到 $f(l_n) \le y$ 并且 $f(r_n) \ge y$。
我们还要证明 $l_n$ 和 $r_n$ 是等价的柯西列。注意到有 $\lvert r_n - l_n \rvert \le (b-a)/n$,并且固定 $N$ 后用数学归纳法容易证明有 $\forall k  \ge N,\, l_N \le l_k \le r_N$,于是有 $\forall k, p \ge N,\, \lvert l_k - l_p \rvert \le r_N - l_N \le 1/N $,所以有

\[
    y \le \lim_{n \to \infty} f(l_n) = \lim_{n \to \infty} f(r_n) \le y
\]

假设

\[
\lim_{n \to \infty} l_n = x_0,\, x_0 \in [a,b]
\]

根据连续函数的性质那么有

\[
f(x_0) = \lim_{n \to \infty}f(l_n) = y
\]

\subsection{像集封闭而且连通}

利用上面两条很容易证明。


\subsection{反函数连续}

假设 $f: [a,b] \to \R$ 可逆,那么必然有 $f(a) \ne f(b)$,不妨设 $f(a) < f(b)$ 那么根据介值定理,运用反证法很容易证明 $f$ 严格单调增。

根据上面的条件,若 $f$ 连续,不妨设 $f$ 单调增,对任意 $x_0 \in (a,b)$,我们可以取很小的 $\epsilon$ 满足 $I_{\epsilon}  = [x_0 - \epsilon, x_0 + \epsilon] \subseteq [a,b]$,然后
可以计算得到 $f(I_{\epsilon}) = [f(x_0 - \epsilon), f(x_0 + \epsilon)]$ 同样也是一个闭区间,注意到 $y_0 = f(x_0) \in f(I_{\epsilon})$ 是一个内点,所以存在 $\delta > 0$ 满足
$(y_0 - \delta, y_0 + \delta) \subseteq f(I_{\epsilon})$。我们两边同时计算$f^{-1}$的像集得到 $f((y_0 - \delta, y_0 + \delta)) \subseteq I_{\epsilon}$。如果 $x_0 = a$,也是同理
此时 $I_{\epsilon} = [a, a + \epsilon],\, f(I_{\epsilon}) = [f(a), f(a+\epsilon)]$,此时 $y_0$ 不是内点,但是可以找到 $\delta > 0$ 满足 $(y_0 - \delta, y_0 + \delta) \cap [f(a), f(b)] \subseteq f(I_{\epsilon})$。

对于单调减函数,我们可以利用 $-f^{-1}(x)= (-f)^{-1}(x)$,然后利用 $-f$ 连续并且单调增这一事实。


\section{一致连续}

\subsection{定义}
$f: X \to R$ 一致连续定义为 $\forall \epsilon > 0,\, \exists \delta >0,\, \forall x,y\, \in X,\, \lvert x - y \rvert \le \delta,\, \lvert f(x) -f(y) \rvert \le \epsilon$

\subsection{等价序列映射到等价序列}

一致连续可以理解为把等价序列映射到等价序列,证明如下。首先证明从左到右,假设 $\lim_{n \to \infty} x_n -y_n = 0,\quad x_n,\, y_n \in X$。
根据一致连续的定义,$\forall \epsilon >0$,只要 $\lvert x_n - y_n \rvert \le \delta$ ,就有 $\lvert f(x_n) - f(y_n)\rvert \le \epsilon $,所以我们只要取 $n$ 充分大就可以。

继续证明从右到左,这里我们用反证法,假设存在 $\lim_{n \to \infty} x_n -y_n = 0,\quad x_n,\, y_n \in X$ 并且存在 $\epsilon > 0$ 使得有无限多的 $\lvert f(x_n) - f(y_n)\rvert > \epsilon$。显然这个和一致连续的定义矛盾了,我们把 $\epsilon$ 代入一致连续
的定义就能得到矛盾。


\subsection{柯西列映射到柯西列}

只要利用 $k,p$ 充分大时有 $d(x_k, x_p) \le \delta$ 从而有 $d(f(x_k),f(x_p)) \le \epsilon$ 即可。


\subsection{闭区间上的连续函数一致连续}

我们可以利用反证法,假设存在 $\epsilon > 0$ 对任意 $\delta > 0$ 都有 $x,y \in X,\, d(x,y) \le \delta$ 满足 $d(f(x), f(y)) > \epsilon$。我们取 $\delta_n = 1/n$,于是得到了序列 $x_n ,y_n$。
注意到 $x_n$ 有收敛子列 $x_{h(n)}$,而 $ y_{h(n)}$ 也有收敛子列 $y_{g(h(n))}$,注意到 $x_{g(h(n))}$ 是 $x_{h(n)}$ 的子列。于是我们得到了

\begin{align*}
    & \lim_{n \to \infty}x_{g(h(n))} = x_0 \\
    & \lim_{n \to \infty}y_{g(h(n))} = y_0 \\
\end{align*}

因为 $d(x_{g(h(n))}, y_{g(h(n))}) \le 1/n$,所以 $x_0 = y_0$ 且 $x_0 \in X$,根据连续函数的定义有


\begin{align*}
     \lim_{n \to \infty}f(x_{g(h(n))}) = \lim_{n \to \infty}f(y_{g(h(n))}) = f(x_0)
\end{align*}

这显然和 $d(f(x_{g(h(n))}), f(y_{g(h(n))})) > \epsilon$ 矛盾了。