\chapter{勒贝格积分}

\section{简单函数}

\subsection{定义}

若$\Omega \subseteq \R^{n},\, f: \Omega \to \R$ 可测,则 $f$ 是简单函数当且仅当 $f(\Omega) \subseteq \R$ 是有限集。 

\subsection{保持简单函数的运算}

若 $f,g$ 都是简单可测函数那么
$f + g,\, cf$ 都是简单可测函数。用上一章的定理可以证明。


\subsection{表示成示性函数}

简单函数可以表示成示性函数的和,若 $f: \Omega \to \R$ 是简单函数,那么存在一组有限的不相交的示性函数 $\chi_{E_i},\, 1 \le i \le n$,和常数 $c_i$ 满足

\[
f(x) = \sum_{i=1}^{n}c_i \chi_{E_i}
\]

这个证明也很容易,既然 $f$ 简单可测,那么令 $f(\Omega) = \{ y_1, y_2, .. y_n\}$,然后令 $E_i = \{ x \in \Omega \wvert f(x) = y_i\}$。容易验证得到

\[
f(\Omega) = \bigcup_{i}E_i,\, \forall i \ne j,\, E_i \cap E_j = \emptyset
\]

并且对 $f(x)$ 的取值分类讨论后有 

\[
f(x) = \sum_{i=1}^{n}y_i \chi_{E_i}
\]


\subsection{简单函数逼近非负可测函数}

该命题如下,假设 $f: \Omega \to \R$ 非负,那么存在函数列 $f_n: \Omega \to \R$ 满足 $f_n$ 是简单函数,$f_n$ 非负,并且 $f_n$ 单调增,即

\[
\forall x \in \Omega,\, f_1(x) \le f_2(x) .. \le f_n(x) 
\]

并且 $f_n$ 逐点收敛到 $f$

证明如下,我们可以按照如下方式构造 $f_n$,我们把 $[0,n)$ 分割成 $n2^n$ 个区间 

\[
[0,n) = \bigcup_{i=0}^{n-1} \bigcup_{j=0}^{2^n-1} [i+\frac{j}{2^n}, i+\frac{j+1}{2^n}) = \bigcup_{i=0}^{n-1}\bigcup_{j=0}^{2^n-1}Y_{i,j}
\]

容易验证 $Y_{i,j}$ 不相交,令 $c_{i,j} = \inf \{ Y_{i,j}  \}$,令
$E_{i,j} = f^{-1}(Y_{i,j})$,容易得到 $E_{i,j}$  也不相交。

\[
f_n(x) = \begin{cases}
    \sum_{i,j}c_{i,j}\chi_{E_{i,j}} ,\, x \in \bigcup_{i,j}E_{i,j} \\
    n ,\, x \notin \bigcup_{i,j}E_{i,j}
\end{cases}
\]

显然 $f_n$ 的像集是一个有限集,而且 $f_n$ 像集的任意一个子集的原像都是可测集,所以 $f_n$ 是简单函数。根据$f_n$定义容易得到 $f_n(x) \ge 0$。

我们继续证明 $f_n$ 单调增,对此我们要分析 $f_n$ 和 $f_{n+1}$ 以及 $f(x)$ 所在范围,首先当 $0 \le f(x) < n$ 时,我们有

\[
i + \frac{j}{2^n} \le f(x) < i + \frac{j+1}{2^n}
\]

计算得到 $i = \lfloor f(x)\rfloor$,$j = \lfloor 2^n(f(x) - i) \rfloor $,同理把 $f_{n+1}$ 的定义代入得到

\begin{align*}
    f_n(x) & = \lfloor f(x)\rfloor + \frac{\lfloor 2^n(f(x) - \lfloor f(x) \rfloor) \rfloor}{2^n} \\
    f_{n+1}(x) &= \lfloor f(x)\rfloor + \frac{\lfloor 2^{n+1}(f(x) - \lfloor f(x)\rfloor) \rfloor}{2^{n+1}} \\
    \text{let}\: \alpha & = 2^{n+1}(f(x) - \lfloor f(x) \rfloor) \rfloor \\
    f_{n+1}(x) - f(x) &= \frac{1}{2^{n+1}}( \lfloor \alpha \rfloor -   2\lfloor \frac{\alpha}{2} \rfloor)
\end{align*}


注意到有 $\lfloor \frac{\alpha}{2} + \frac{\alpha}{2}  \rfloor -   \lfloor \frac{\alpha}{2} \rfloor - \lfloor \frac{\alpha}{2} \rfloor \ge 0$,所以当 $f(x) < n$ 时有 $f_n(x) \le f_{n+1}(x)$
而当 $n \le f(x) $ 时 $f_n(x) = n$,$f_{n+1}(x) \ge \lfloor f(x)\rfloor \ge n$,所以 $f_{n+1} \ge f_n$。

继续证明 $f_n$ 逐点收敛到 $f(x)$。我们任取 $x \in \Omega$ 得到 $f(x) = y$,假设 $\lfloor y \rfloor = m$,我们得到 $\forall n \ge m + 1$ 有

\[
f_n(x) = m + \frac{\lfloor 2^n(f(x) - m \rfloor)}{2^n} 
\]

注意到 $0 \le f(x) -m < 1$,令 $\alpha = f(x) -m$ 得到

\[
f_n(x) -f(x) =\frac{\lfloor 2^n \alpha \rfloor}{2^n} - \alpha  = \frac{1}{2^n} (\lfloor2^n \alpha \rfloor-2^n\alpha)
\]

所以 $n \ge m + 1$ 时有

\[
- \frac{1}{2^n}\le f_n(x) -f(x) \le 0
\]

根据两边夹法则得到 

\[
\lim_{n \to \infty}f_n(x) = f(x)
\]


\subsection{非负简单函数的勒贝格积分}
非负简单函数 $f: \Omega \to \R $ 的勒贝格积分定义为

\[
\int_{\Omega}f = \sum_{\lambda \in f(\Omega)} \lambda m(\{ x \wvert f(x) = \lambda \})
\]

\subsection{示性函数勒贝格积分}

下面证明,若简单函数表示成有限个不相交的示性函数的非负线性组合,那么可以按如下方式计算勒贝格积分,并且和表示方法无关。

若

\[
f = \sum_{i=1}^{n}c_i\chi_{E_i},\, c_i \ge 0
\]

则

\[
\int_{\Omega} f = \sum_{i}^{n}c_im(E_i)
\]

证明如下,我们不妨设 $\forall i,\, c_i > 0$,因为如果 $c_i = 0$ 我们可以去掉该项,根据定义得到

\begin{align*}
\int_{\Omega} f &= \sum_{\lambda \in \{ c_i \wvert 1 \le i \le n\}}\lambda m(f(x) = \lambda)  \\
    & = \sum_{\lambda \in \{ c_i \wvert 1 \le i \le n\}}\lambda m(\bigcup_{c_i = \lambda }E_i) \\
    & = \sum_{\lambda \in \{ c_i \wvert 1 \le i \le n\}}\lambda (\sum_{c_i = \lambda}m(E_i)) \\
    & = \sum_{\lambda \in \{ c_i \wvert 1 \le i \le n\}}\sum_{c_i = \lambda}\lambda m(E_i) \\
    & = \sum_{\lambda \in \{ c_i \wvert 1 \le i \le n\}}\sum_{c_i = \lambda}c_i m(E_i)  \\
    & = \sum_{1 \le i \le n}c_i m(E_i)  \\
\end{align*}


\subsection{简单函数勒贝格积分的性质}

下面假设 $f,\,g: \Omega \to \R$ 都是非负简单函数。

\begin{enumerate}
    \item $0 \le \int_{\Omega}f \le \infty$ 并且 $\int_{\Omega} f =0$ 当且仅当 $\{ x \wvert f(x) \ne 0 \}$ 是零测集

    下面给出证明,$0 \le \int_{\Omega}f \le \infty$ 根据定义很容易证明。

    先证明右到左,因为 $f(\Omega)$ 是有限集, 所以假设 $f(\Omega) = \{ c_1,c_2,.. c_n\},\, E_i = \{x \in \Omega \wvert f(x) = c_i\}$,不妨设 $c_k = 0$,根据定义有

    \[
        \int_{\Omega}f = \sum_{1 \le i \le n}c_i m(E_i) = 0 \cdot m(E_k) + \sum_{1 \le i \le n,\, c_i \ne 0}c_i \cdot 0  = 0
    \]

    如果 $0 \notin f(\Omega)$ 那么我们可以把 $E_k$ 移出去,所以上面依然成立。


    继续证明左到右,我们这里用反证法,假设 $X= \{ x \in \Omega \wvert f(x) \ne 0 \} $ 不是零测集,然后枚举 $f(\Omega) = \bigcup_{1 \le i \le n}E_i,\, E_i = f^{-1}([c_i, c_i])$,我们取 $c_i > 0$ 的最小值为 $c$,
    我们之所以能这么做是因为 $\{ c_i > 0\}$ 不是空集

    \[
    \int_{\Omega}f = \sum_{1 \le i \le n}c_i m(E_i) \ge c\sum_{1 \le i \le n} m(E_i) \ge c \cdot m(X) > 0
    \]

    于是我们得到矛盾。

    \item $\int_{\Omega}(f + g) = \int_{\Omega}f + \int_{\Omega}g$

    证明如下:

    令 $h = f + g$,我们尝试把 $h$ 表示成示性函数,假设 $f = \sum_{i} c_i \chi_{E_i},\, g = \sum_{j} d_j \chi_{F_j}$ 得到

    \begin{align*}
    h &= \sum_{i \in I} c_i \chi_{E_i} + \sum_{j \in J} d_j \chi_{F_j}  \\
    &=  \sum_{(i,j) \in I \times J} (c_i + d_j) \chi_{E_i \cap F_j}
    \end{align*}

    所以有

    \begin{align*}
        \int_{\Omega}h &= \sum_{(i,j) \in I \times J}(c_i + d_j) m(E_i \cap F_j) =\sum_{(i,j) \in I \times J}c_i m(E_i \cap F_j) + \sum_{(i,j) \in I \times J}d_j m(E_i \cap F_j) \\
            & = \sum_{i \in I} \sum_{j \in J}c_im(E_i \cap F_j) + \sum_{j \in J} \sum_{i \in I}d_jm(E_i \cap F_j)  \\
            & = \sum_{i \in I} c_i\sum_{j \in J}m(E_i \cap F_j) + \sum_{j \in J} d_j \sum_{i \in I}m(E_i \cap F_j)  \\
            & = \sum_{i \in I} c_i m(\bigcup_{j \in J}E_i \cap F_j) + \sum_{j \in J} d_j \sum_{i \in I}m(\bigcup_{i \in I}F_j \cap E_i) \\  
            & = \sum_{i \in I} c_i m(E_i \cap \bigcup_{j \in J} F_j) + \sum_{j \in J} d_j \sum_{i \in I}m(F_j \cap \bigcup_{i \in I}E_i) \\  
            & = \sum_{i \in I} c_i m(E_i \cap \Omega) + \sum_{j \in J} d_j \sum_{i \in I}m(F_j \cap \Omega) \\  
            & = \int_{\Omega} f + \int_{\Omega}g \\  
    \end{align*}

    \item $\forall c >0,\, \int_{\Omega}cf = c\int_{\Omega}f$

    直接利用定义就可以证明。

    \item 若 $\forall x \in \Omega,\, f(x) \le g(x)$ 那么 $\int_{\Omega}f(x) \le \int_{\Omega}g(x)$

    我们分别对 $E_i$ 和 $F_j$ 作笛卡尔积,然后用示性函数表示 $\int_{\Omega}f$ 和 $\int_{\Omega}g$,很容易证明。


\end{enumerate}

\section{非负可测函数的勒贝格积分}

\subsection{定义}

假设 $f: \Omega \to \R$ 非负并且可测,那么 $f$ 的勒贝格积分定义为

\[
\int_{\Omega}f = \sup \{ \int_{\Omega}g \wvert g \text{ is simple},\, 0 \le g \le f \}
\]

注意到对于非负简单函数,该定义仍然适用

\subsection{非负可测函数的勒贝格积分性质}

对于任意非负可测函数 $f: \Omega \to \R$,我们有

\begin{enumerate}
    \item $0 \le \int_{\Omega}f \le \infty$ 并且 $\int_{\Omega}f = 0$ 当且仅当 $f(x)$ 几乎处处为 0

    首先上确界是一个极限点,我们可以用序列的运算法则证明。

    后面一个定理,先证明充分性。假设 $g$ 是非负简单函数,并且 $g \le f$。首先我们可以枚举 $g(\Omega)$ 然后得到有限个不相交的像集以及原像集。设这些不相交原像集为 $E_i, \, i \in I$。
    注意到 $g(E_i) \subseteq (0, \infty)$ 可以推导出 $f(E_i) \subseteq (0, \infty)$ 所以有 $E_i \subseteq f^{-1}((0, \infty)),\, m(E_i) \le 0$。

    我们把 $g$ 的勒贝格积分按照示性函数展开后得到结果一定为 $0$,所以充分性成立。

    继续证明必要性,这里我们用反证法,假设 $m(f^{-1}((0, \infty))) > 0$。注意到有

    \begin{align*}
        (0, \infty) &= [1, \infty) \cup \bigcup_{i=1} [\frac{1}{i+1}, \frac{1}{i}) \\
        m(f^{-1}((0, \infty))) &= m(f^{-1}([1, \infty))) + \sum_{i=1}^{\infty}m(f^{-1}([\frac{1}{i+1},\frac{1}{i})))
    \end{align*}

    所以要么存在 $K_0 = f^{-1}([1, \infty))$ 满足 $m(K_0) > 0$,要么存在 $i \ge 1,\, K_i = m(f^{-1}([\frac{1}{i}, \frac{1}{i+1}))) $ 满足$m(K_i) > 0$,此时 $i = \lceil 1/x\rceil - 1$。
    我们记这个原像集为 $K$ ,然后按照如下方式构造非负简单函数 $g$

    \[
        g = \begin{cases}
            & \inf(f(K)),\, x \in K \\
            & 0,\, x \notin K \\
        \end{cases}
    \]

    容易验证 $f \le g$ 并且 $g$ 是非负可测函数,而且有 $\int_{\Omega} g  > 0$,所以得到矛盾。

    \item $\int_{\Omega} cf = c \int_{\Omega} f$

    令 $L = \int_{\Omega} f$,我们先证明 $\int_{\Omega} cf \le cL$。假设有非负简单函数列 $g_n$ 从下方控制 $cf$ 并且有

    \[
        \lim_{n \to \infty} \int_{\Omega} g_n = \int_{\Omega} cf
    \]

    注意到 $g_n \le cf$ 可以得到 $g_n /c \le f$,所以 $g_n/c$ 是从下方控制 $f$  的函数列,根据极限的运算法则以及 $L$ 最大的极限点这一性质有

    \begin{align*}
        \lim_{n \to \infty} \int_{\Omega} g_n/ c &= \frac{1}{c}\int_{\Omega} cf \le L \\
        & \int_{\Omega} cf \le cL
    \end{align*}

    我们继续证明 $\int_{\Omega} cf \ge cL$

    假设有非负简单函数列 $h_n$ 从下方控制 $f$ 并且有

    \[
        \lim_{n \to \infty} \int_{\Omega} h_n = \int_{\Omega} f
    \]

    注意到 $h_n \le f$ 可以得到 $ch_n \le cf$,所以 $c h_n$ 是从下方控制 $cf$ 的非负简单函数列,根据极限运算法则有

    \begin{align*}
        \lim_{n \to \infty} \int_{\Omega} ch_n &=  cL \\
        & \le \int_{\Omega} cf
    \end{align*}

    \item $f \le g$ 则 $\int_{\Omega} f \le \int_{\Omega} g$

    用序列极限的性质很容易证明。

    \item 若 $f$ 几乎处处等于 $g$ 则 $\int_{\Omega}f = \int_{\Omega}g $

    假设 $L_1 = \int_{\Omega} f,\, L_2 = \int_{\Omega} g$,我们先证明 $L_1 \ge L_2$,假设有非负简单函数 $f_n$ 且 $f_n$ 从下方控制 $f$

    \begin{align*}
        \lim_{n \to \infty}\int_{\Omega}f_n = L_1
    \end{align*}
    
    记 $E = \{x \in \Omega f(x) \ne g(x)\}$,我们可以构造 $g_n$

    \begin{align*}
        g_n = \begin{cases}
            & f_n(x),\, x \notin E \\
            & 0,\, x \in E \\
        \end{cases} 
    \end{align*}

    注意到 $g_n$ 也是非负的简单函数,而且 $g_n$从下方控制 $g$,并且把 $g_n$ 按照示性函数展开后可以得到

    \begin{align*}
        g_n &= \sum_{i}c_i \chi_{E_i \cap (\Omega \setminus E)} \\
        \int_{\Omega} g_n &= \sum_{i}c_i m(E_i \cap (\Omega \setminus E))
    \end{align*}

    注意到

    \begin{align*}
        E_i &= (E_i \cap E) \cup (E_i \cap (\Omega \setminus E)) \\
        m(E_i) &= m(E_i \cap E) + m(E_i \cap (\Omega \setminus E)) \\
    \end{align*}

    因为 $m(E) = 0$,所以

    \[
        m(E_i \cap (\Omega \setminus E)) = m(E_i)
    \]

    所以有

    \[
        \int_{\Omega} f_n = \int_{\Omega} g_n
    \]

    所以根据极限的性质有

    \[
        \lim_{n \to \infty}\int_{\Omega} f_n = \lim_{n \to \infty}\int_{\Omega} g_n = L_1 \le L_2
    \]

    同理可以证明 $L_2 \le L_1$,所以 $L_1 = L_2$

    \item 若 $\Omega' \subseteq \Omega$ 可测,那么 $\int_{\Omega'}f \le \int_{\Omega} f$

    令 $h = f: \Omega' \to \R,\, h(x) = f(x)$ 并且有非负简单函数列满足

    \[
       \lim_{n \to \infty}\int_{\Omega'}h_n = \int_{\Omega'}h = L
    \]

    我们对 $h_n$ 稍作修改就能得到从下方控制 $f$ 的函数列 $f_n: \Omega \to \R$

    \[
        f_n(x) = \begin{cases}
           & h_n(x),\, x \in \Omega' \\
           & 0,\, x \notin \Omega'
        \end{cases}\
    \]

    容易验证 $f_n $ 是非负简单函数,而且 $f_n$ 从下方控制 $f$,并且

    \[
        \int_{\Omega} f_n = \int_{\Omega'}h_n
    \]

    根据上确界的性质,我们得到

    \[
    \lim_{n \to \infty}\int_{\Omega} f_n = L \le \int_{\Omega} f
    \]

    \item 若 $\Omega' \subseteq \Omega$ 可测,那么 $\int_{\Omega'}f = \int_{\Omega} f\chi_{\Omega'}$

    注意到有

    \[
        \int_{\Omega'}f = \int_{\Omega'}f\chi_{\Omega'} \le \int_{\Omega}f\chi_{\Omega'}
    \]

    然后我们在 $\Omega$ 上构造非负简单函数列 $f_n$ 从下方控制  $f \chi_{\Omega'}$,并且有

    \[
        \lim_{n \to \infty}\int_{\Omega} f_n = \int_{\Omega}f \chi_{\Omega'}
    \]

    我们可以对 $f_n$ 稍加改动得到从下方控制 $f: \Omega' \to \R$ 的简单函数列 $g_n$

    \[
        g_n(x) = f_n(x) = \sum_{i \in I} c_i \chi_{E_i \cap \Omega'},\, \forall x \in \Omega'
    \]

    根据 $0 \le f_n(x) \le f(x)\chi_{\Omega'}(x) \le 0,\, \forall x \notin \Omega'$,我们有 $\forall c_i >0,\, E_i \subseteq \Omega'$

    \[
        \int_{\Omega} f_n = \sum_{i \in I} c_i m(E_i) = \sum_{i \in I,\, c_i > 0} c_i m(E_i \cap \Omega') = \int_{\Omega'}f_n = \int_{\Omega'}g_n
    \]

    所以有

    \[
        \lim_{n \to \infty}\int_{\Omega} f_n = \lim_{n \to \infty}\int_{\Omega'}g_n = \int_{\Omega}f \chi_{\Omega'} \le \int_{\Omega'}f 
    \]

\end{enumerate}