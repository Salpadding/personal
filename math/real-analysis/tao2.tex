\part{实分析 II}

\chapter{拓扑空间}

\section{拓扑空间定义}

拓扑空间的核心在于开集,因为开集必须满足对有限的交集封闭,同时对任意多个并集封闭。

\section{$ \sigma $ 代数}

$\sigma$ 代数经常和可测集联系到一起,$\sigma$ 代数是一个集合族,这个集合族内的集合要对可数的并集运算封闭,对补集运算封闭。
从这里可以轻易的证明可数的交集也封闭。

\section{点集拓扑}

\subsection{内点}
集合$S$ 包含内点 $x$ 是指有一个包含$x$ 的开集$A$,满足 $A \subseteq S$。

\subsection{外点}

补集的内点就是外点。 


\subsection{边界点}

既不是内点也不是外点,边界点有个很好的性质,它的任意邻域都既有集合内的点又有集合外的点,它既是当前集合的极限点,又是当前集合补集的极限点。

\subsection{极限点}
$x$ 是集合 $S$ 的极限点,是指对于任意一个包含 $x$ 的开集 $A$ 都有 $ (A \setminus \{ x \}) \cap S \ne \emptyset$

\subsection{附着点}

比极限点宽松,$x$ 是集合 $S$ 的附着点,是指对于任意一个包含 $x$ 的开集 $A$ 都有 $A \cap S \ne \emptyset$。

\subsection{闭包}
$S$ 的闭包 $\overline{S}$ 是指 $S$ 和所有极限点的 union

\subsection{闭集可以用闭包定义}

闭包一定是闭集,证明如下。

假设 $\overline{S}$ 不是闭集,那么 $ Y = X \setminus \overline{S}$ 就不是开集,所以 $Y$ 内会有一个点 $y$ 不是 $ Y$ 的内点。
所以 $y$ 附近的任意小的邻域都有 $ \overline{S}$ 内的点,于是 $y$ 成为了 $\overline{S}$ 的一个极限点,于是 $y \in \overline{S}$,这与 $y \in Y$ 矛盾。

Q.E.D

闭集的闭包就是自身,证明如下。

假设 $S$ 是一个闭集,于是 $Y = X \setminus S$ 就是一个开集,假设 $S$ 内存在一个点列收敛到 $y \in Y$,因为 $Y$ 是开集,所以可以取 $y$ 附近的一个邻域 $A$ 满足 
$ y \in A, A \subseteq Y $,再利用极限的定义可以在点列里面找到位于 $Y$ 内的点,这个显然矛盾。

如果一个集合的闭包就是自身,那么这个集合就是闭集。

假设 $S$ 不是一个闭集,那么 $ Y \setminus S$ 一定不是开集,那么 $ S $ 一定存在收敛到 $ y \in Y$ 的一个点列。 

\section{紧集}

\subsection{紧集定义}

度量空间 $(X,d)$ 紧致,当且仅当每个序列都至少有一个收敛子列。对于拓扑空间,紧集的集合更具有一般性,拓扑空间上的紧集定义为:任意开覆盖都有有限子覆盖。


\chapter{一致收敛}

\section{度量空间上函数极限的定义}

\subsection{定义}

假设有函数 $f: E \to Y$,其中 $E \subseteq X$,$(X, d_X)$ 和 $(Y, d_Y)$ 是两个度量空间。$x_0 \in X $ 是 $E$ 的附着点,注意这里没有要求 $x_0 \in E$,
如果对任意 $ \epsilon > 0$ 都存在 $ \delta > 0$,只要 $x \in E$ 满足 $d_X(x, x_0) \le \delta $ 就有 $d_Y(f(x), L) < \epsilon$,那么我们称 $x$ 沿着 $E$ 收敛于 $x_0$ 时 $f(x)$ 沿着 $Y$ 收敛于 $L$。

以下是一些等价命题,及其证明。

\begin{enumerate}
    \item 对任意 $x_n \in E, x_n$ 收敛到 $x_0$ 有 $f(x_n)$ 收敛到 $L$。这个要用反证法,假设存在一个不收敛到 $L$ 的 $f(x_n)$,那么存在一个 $\epsilon$,有无限个 $d(f(x_n), L) > \epsilon$,这个和函数极限的定义矛盾了。
    \item 对任意包含 $L$ 的开集 $V \subseteq Y$,都存在一个包含 $x_0$ 的开集 $U \subseteq X$ 使的 $f(U \cap E) \subseteq V$。同样用反证法,假设存在一个包含 $L$ 的开集 $V \subseteq Y$,对任意的包含 $x_0$ 的开集 $U \subseteq X$ 都存在 $v \in  f(U \cap E), v \notin V $。因为 $V$ 是一个包含 $L$ 的开集,所以我们可以找到 $L$ 附近的一个开球 $B(L)$,而且这个开球是 $V$ 的子集,
    再套上函数极限的定义,存在 $x_0$ 附近的开球 $B(x_0)$ 满足 $f(B(x_0) \cap E) \subseteq B(L)$,显然这个和 存在 $v \in  f(U \cap E), v \notin V $ 矛盾。
\end{enumerate}

\subsection{函数极限唯一}

假设 $f(x)$ 再 $x_0$ 处有极限 $L$ 和 $L'$。令 $\epsilon = d(L, L')$,根据极限定义有 $f(B(x_0, \delta) \cap E) \subseteq B(L, \epsilon/4)$ 而且 $f(B(x_0, \delta) \cap E) \subseteq B(L', \epsilon/4)$,这里没有用两个 $\delta$ 是因为我们可以取他们的最小值。我们在这个开球取一个点得到 $d(f(x_1), L) \le \epsilon/4 $ 和 $d(f(x_1), L') \le \epsilon/4$ 运用三角不等式得到 $d(L, L') \le  \epsilon/2 $ 这里和 $d(L, L') \ne 0 $ 矛盾。

\section{逐点收敛的定义}

\subsection{定义}

设有函数序列 $f^{(n)}(x): E \to Y$,如果对任意的 $x \in E$ 都有 $f^{(n)}(x)$ 收敛到 $f(x)$,那么称函数 $f(x)$ 是函数序列 $f^{(n)}(x)$ 的逐点极限。

\subsection{局限性}

逐点收敛不能保持连续性,导数运算,极限运算和积分运算。因为逐点收敛的函数列中,不同 $x$ 取值之间的收敛速度可能有很大的差异。一个经典的例子是 $f^{(n)}(x) = x^n, \, x \in [0,1]$。这个函数每一项都是连续的而且可微的,但是逐点极限不再连续,也不可微。

\section{一致收敛}

\subsection{定义}

$f^{(n)}(x): E \to Y$ 一致收敛到 $f(x)$ 定义为对任意 $\epsilon$ 都存在 $N$,只要 $n \ge N$,则 $\max \{ d( f^{(n)}(x), f(x)) \wvert x \in E \} \le \epsilon$。

\subsection{性质}

一致收敛具有很多如下的性质

\begin{enumerate}
    \item 保持连续性 \\
    如果 $f^{(n)}(x)$ 在 $x_0$ 处都连续,那么 $f(x)$ 在 $x_0$ 处连续。证明很容易,对任意 $\epsilon > 0$,我们可以选择一个 $N$ 满足 $\forall x \in E, d(f^{(N)}(x), f(x_0)) \le \epsilon$,因为
    $f^{(N)}(x)$ 连续,所以存在 $\delta_N$ 满足 $\forall x \in B(x_0, \delta_N) \cap E, d(f^{(N)}(x), f^{(N)}(x_0)) \le \epsilon $。于是在 $B(x_0, \delta_N)$ 内我们得到了 $d(f(x), f(x_0)) \le d(f(x), f^{(N)}(x)) + d(f^{(N)}(x), f^{(N)}(x_0)) + d(f^{(N)}(x_0), f(x_0)) \le 3\epsilon $
    因为 $\epsilon$ 可以取任意小,所以 $f(x)$ 连续。

    \item 极限运算可交换: 如果 $f^{(n)}(x): E \to Y$ 一致收敛到 $f(x)$,而且度量空间 $(X, d_X)$ 和度量空间 $(Y, d_Y)$ 都是完备的,而且$f^{(n)}(x)$ 在 $f^{(n)}(x_0)$ 处都收敛到 $y_n$,那么 $y_n$ 收敛到极限 $L$,而且函数
    $f(x)$ 在 $x_0$ 处有极限 $L$。证明:首先我们证明 $y_n$ 是一个柯西列,当 $n \ge N$ 时取 $p,q \ge N$  有 $d(y_p, y_q) \le d(y_p, y_s) + d(y_s, y_t) + d(y_t, y_q)$ 其中 $f^{(p)}(x_s) = y_s$,$f^{(q)}(x_t) = y_t$,$d(x_s, x_0) \le \delta$,$d(x_t, x_0) \le \delta $
    这就得到了  $d(y_p, y_q) \le 3\epsilon $ 因为 $\epsilon $ 可以任意小,所以 $y_n$ 是一个柯西列,因为 $Y$ 是完备的,所以有 $y_n$ 收敛到 $L$。 \\
    继续证明 $f(x)$ 收敛到 $L$,对任意 $\epsilon > 0$,有 $N$,只要 $n \ge N$ 同时满足 $d(y_n, L) \le \epsilon$,$\forall x \in E, \, d(f(x), f^{(n)}(x) ) \le \epsilon $,因为 $f^{(N)}$ 在 $x_0$ 有极限 $y_N$,存在 $\delta_N$ 满足 $f(B(x_0, \delta_N) \cap E) \subseteq B(y_N, \epsilon)$
    那么对于 $x \in B(x_0, \delta_N) \cap E$ 有 $d(f(x), L) \le d(f(x), f^{(N)}(x)) + d(f^{(N)}(x), y_n) + d(y_n, L) \le 3\epsilon$ 所以我们得到了下面的式子:

\[
    \lim_{x \to x_0}\lim_{n \to \infty}f^{(n)}(x) = \lim_{n \to \infty}\lim_{x \to x_0}f^{(n)}(x)
\]


    \item 如果 $f^{(n)}: E \to Y$ 是连续的,而且 $f^{(n)}(x)$ 一致收敛到 $f(x)$,$x_n$ 收敛到 $x_0$,那么 $f^{(n)}(x_n)$ 收敛到 $f(x_0)$ \\
    证明:根据上面的结论我们可以得出 $f$ 是连续的,所以我们可以确定 $f(x_n)$ 是收敛到 $f(x_0)$,如果我们能证明 $f(x_n)$ 和 $f^{(n)}(x_n)$ 等价那么就可以证明了。这个根据一致收敛的定义也很容易证明。

    \item 保持有界, $f^{(n)}(x): E \to Y$ 一致收敛到 $f(x): E \to Y$,而且 $\forall f^{(n)}(x), \, f^{(n)}(E) \subseteq B(y_n, Rn)$ 那么 $f(x)$ 也是有界的。我们可以随意取一个 $\epsilon$ 并且可以得到一个 $N$ 满足
    $\forall x \in E, d(f(x), f^{(N)}(x)) \le \epsilon$,因为对任意 $y \in f^{(N)}(E)$ 有 $d(y, y_N) \le R_N$ 所以对任意 $y' \in f(E)$ 有 $d(y', y_N) \le d(y', y) + d(y, y_N) \le R_N + \epsilon $ 所以 $f(x)$ 也是有界的。

\end{enumerate}


