\chapter{Lebesgue 测度}

\section{外测度}

\subsection{扩展实数集}

外测度中会用到非负的扩展实数集,这个集合上包含了 $+\infty$ 和全体非负实数。这个集合上可以定义加法,我们暂时先用符号 $\mathit{(+)}$ 表示扩展实数集上的加法,以此和实数集上的加法区分。

\[
    a \: (+) \: b = a + b,\, \text{if }a,b \ne +\infty
\]

\[
     \: a \:(+)\: b = +\infty,\, \text{if }a = +\infty \: \text{or} \: b = +\infty,
\]

容易验证 $(+)$ 满足交换律,结合律,所以我们后面可以用 $+$ 替代 $(+)$

同理也可以定义扩展实数集上的小于等于 $(\le)$:

\[
a \:(\le)\: b \: \text{iff} \: b = +\infty \: \text{or} \: a \le b
\]

容易验证 $a \:(\le)\: b$ 和 $b \:(\le)\: a$ 至少有一个成立,而且 $a \:(\le)\: b$ 且 $b \:(\le)\: a$ 可以得出 $a = b$\\
后面可以用 $\le$ 替代 $(\le)$

\subsection{Open Box 定义}

$\mathbb{R}^{n}$ 上的 Open box 定义为 

\[
B = \prod_{i=1}^{n}(a_i, b_i) = \{ (x_1, x_2, .., x_n) \in \Rn \wvert x_i \in (a_i, b_i) \}
\]

其中 $a_i \le b_i, \: a_i, b_i \in \mathbb{R}$ 同时 Open box 的提及 vol 定义为

\[
\text{vol}(B) = \prod_{i=1}^{n}(b_i - a_i)
\]

\subsection{Open Box 覆盖}

假设 $\Omega \subseteq \Rn$,我们称一个包含 Open box 的集族 $(B_j)_{j \in J}  $ 覆盖  $ \Omega $ 当且仅当 

\[
\Omega \subseteq \bigcup_{j \in J} B_j
\]

\subsection{外测度定义}

假设 $\Omega \subseteq \Rn$,$\Omega$ 的外测度$m^*(\Omega)$定义为

\[
m^*(\Omega) = \inf \{\: \sum_{j \in J}\mathrm{vol}(B_j) \wvert \Omega \subseteq \bigcup_{j \in J}B_j,\, J \: \mathrm{ is \: at \: most \: countable} \:\}
\]

注意这里覆盖的Open box 不要求是有限的,而是至多可数的,$m^*(\Omega)$ 的结果可能是扩展实数集中的 $+\infty$。


\section{外测度性质}

\subsection{空集外测度为 0}

显然 $\Rn$ 上可以构造 Open box 
\[
B = \prod_{i=1}^{n}(0, \epsilon)
\]

覆盖 $\emptyset$ 所以根据下确界的性质有 $ 0 \le m^*(\emptyset) \le \epsilon^n$,因为 $\epsilon$ 可以任意小,所以 $m^*(\emptyset) = 0$

\subsection{外测度非负}

根据下确界的性质可以得到。

\subsection{单调性}

如果 $A \subseteq B \subseteq \Rn$,那么有 $m^*(A) \le m^*(B)$

因为 $B$ 的 Open box 覆盖一定能覆盖 $A$,集合的下确界一定小于等于子集的下确界。

\subsection{次可加性}

$(A_i)_{i \in I},\, I \subseteq \N$ 是至多可数的集族,那么有

\[
m^*(\bigcup_{i \in I}A_i) \le \sum_{i \in I}m^*(A_i)
\]

证明:不妨设有如下条件,因为如果右边等于 $\infty$ 那么必定成立。

\[
\sum_{i \in I}m^*(A_i) < +\infty ,\: \forall i \in I,\: m^*(A_i) < \infty
\]


令

\[
A = \bigcup_{i \in I}A_i
\]

因为 $A_i$ 的外测度是有限的,所以每个 $A_i$ 都可以被一组至多可数的 Open box 覆盖,并且这些 Open box 的体积之和可以无限接近 $m^*(A_i)$
我们记这组开覆盖为

\[
A_i \subseteq \bigcup_{j \in J_i} B_{i, j}
\]

并且有

\[
\sum_{j \in J_i}\mathrm{vol}(B_{i, j}) \le  \epsilon \frac{1}{2^i} + m^*(A_i)
\]


然后对$A_j$ 各自的至多可数个 Open box 覆盖作并集。

\[
B = \bigcup_{i \in I} \bigcup_{j \in J_i} B_{i,j}
\]

显然 $B$ 是 $A$ 的至多可数覆盖,因为至多可数个 \: 至多可数的集合 \: 的并集一定是至多可数的,思考一下对角线构造法。所以根据下确界的性质

\begin{align*}
m^*(A) & \le \sum_{i \in I}\sum_{j \in J_i} \mathrm{vol}(B_{i, j})  \\
 & \le \sum_{i \in I}(m^*(A_i) + \epsilon \frac{1}{2^i}) \\
 & \le 2\epsilon + \sum_{i \in I}m^*(A_i)
\end{align*}

因为 $\epsilon$ 可以任意小所以

\[
m^*(\bigcup_{i \in I}A_i) \le \sum_{i \in I}m^*(A_i)
\]

\subsection{Closed box 的外测度}

\[
B = \prod_{i=1}^{n}[a_i, b_i] = \{ (x_1, x_2, .., x_n) \in \Rn \wvert x_i \in [a_i, b_i] \}
\]

那么

\[
m^*(B) = \prod_{i=1}^{n}(b_i - a_i)
\]

证明如下:首先我们证明

\[
m^*(B) \le \prod_{i=1}^{n}(b_i - a_i)
\]

对任意 $\epsilon > 0$ 

\[
B \subseteq \prod_{i=1}^{n}(a_i - \epsilon, b_i + \epsilon)
\]

所以有

\[
m^*(B) \le \prod_{i=1}^{n}(b_i - a_i + \epsilon)
\]

因为 $n$ 是有限的而且 $\epsilon$ 可以任意小(考虑下二项展开)所以


\[
m^*(B) \le \prod_{i=1}^{n}(b_i - a_i)
\]

然后要证明


\[
m^*(B) \ge \prod_{i=1}^{n}(b_i - a_i)
\]

就比较困难,因为用开区间覆盖可能会比较分散,我们从 $n = 1$ 开始,用数学归纳法证明上面的不等式。当 $n$ = 1时,因为紧集的开覆盖一定有\,有限的子覆盖,所以我们可以认为开覆盖对应的指标集 $J$ 是一个有限集。

\[
    [a,b] \subseteq \bigcup_{j \in J}(a_j, b_j)
\]

令 $f_j: \R \to [0,1],\, f_j := \chi_{[a_j, b_j]}$ 这里 $\chi$ 是示性函数。根据 $\chi_{[a_j, b_j]}$  的性质可以得到 $f_j$ 是一个紧支撑函数,并且黎曼可积,并且

\[
\int_{\R}f_j = \int_{[a_j, b_j]}f_j = b_j - a_j
\]

运用黎曼积分的有限项性质,得到

\[
\int_{\R} \sum_{j \in J} f_j = \sum_{j \in J}b_j - a_j
\]

因为 $h = \sum_{j \in J} f_j$ 对任意 $x \in [a,b]$ 都有 $h(x) \ge 1$,所以 

\[
\int_{\R} \sum_{j \in J} f_j = \sum_{j \in J}b_j - a_j \ge b -a
\]

所以 $n=1$ 时有

\[
\sum_{j \in J} \mathrm{vol}(B_j) \ge \prod_{i=1}^{n}(b_i - a_i)
\]

我们现在假设 


\[
\sum_{j \in J} \mathrm{vol}(B_j) \ge \prod_{i=1}^{n}(b_i - a_i)
\]

对 $n = k$ 成立,下面考虑 $n = k +1$ 的情况,注意到对开盒子 $B_j$ 有

\[
    B_j = \prod_{i=1}^{k+1}(a_i^{(j)}, b_i^{(j)}) = (a_{k+1}^{(j)}, b_{k+1}^{(j)}) \cdot \prod_{i=1}^{k}(a_i^{(j)}, b_i^{(j)})
\]

我们令 

\[
    A_j = \prod_{i=1}^{k}(a_i^{(j)}, b_i^{(j)})
\]

同理构造 $f_j: \R \to [0,1]$

\begin{align*}
    f_j &= \chi_{j}  \cdot \mathrm{vol}(A_j) \\
\chi_{j} &= \chi_{[a_{k+1}^{(j)}, b_{k+1}^{(j)}]}
\end{align*}

得到

\begin{align*}
    \int_{\R} f_j &= \mathrm{vol}(B_j) \\
    \sum_{j \in J} \int_{\R} f_j &=  \int_{\R}\sum_{j \in J} f_j = \sum_{j \in J} \mathrm{vol}(B_j)
\end{align*}


接下来我们要证明,$\forall x \in [a_{k+1}, b_{k+1}]$,有

\[
\sum_{j \in J} \chi_j \mathrm{vol}(A_j) \ge \mathrm{vol}(A)
\]

假设 $x_{k+1} \in [a_{k+1}, b_{k+1}]$,我们取任意 $(x_1, x_2, .. x_k) \in A$,显然有 $(x_1, x_2, .. x_{k+1}) \in B$。
所以必然存在至少一个 $B_j$ 满足 $(x_1, x_2, .. x_k) \in B_j$ ,所以有 $(x_1, x_2, .. x_k) \in A_j$,
所以对任意 $x_{k+1}$, 所有包含它的 $B_j$ 的 $A_j$ 形成了 $A$ 的开盒子覆盖。

根据数学归纳法的假设,有


\[
\sum_{j \in J} \chi_j \mathrm{vol}(A_j) \ge \mathrm{vol}(A)
\]

然后两边同时积分得到

\begin{align*}
    \int_{\R} \sum_{j \in J} \chi_j \mathrm{vol}(A_j) &= \sum_{j \in J}\int_{\R} \chi_j \mathrm{vol}(A_j) \\
    &= \sum_{j \in J}\mathrm{vol}(B_j) \ge \mathrm{vol}(B)
\end{align*}

\subsection{Open box的外测度}

因为 Open box 自身就是一个有限覆盖,所以根据外测度的单调性,结合闭盒子的外测度有


\[
m^*(I) \le \prod_{i=1}^{n}(b_i - a_i)
\]

然后我们取一个很小的 $\epsilon$,用这个很小的 $\epsilon$ 去构造当前开盒子的一个子集闭盒子,所以有

\[
\prod_{i=1}^{n}(b_i - a_i - 2\epsilon) \le m^*(I)
\]

对 $\epsilon \to 0$ 取极限得到

\[
m^*(I) = \prod_{i=1}^{n}(b_i - a_i)
\]

\section{常见集合的外测度}

\subsection{实数集的外测度}

$m^*(\R) = +\infty$,因为任意 $[-M, M]$ 都是 $\R$ 的子集,同理 $\R^n$ 的外测度也是 $+\infty$

\subsection{有理数集的外测度}

因为有理数集是可数集,之前我们证明了外测度具有次可加性。所以有

\[
0 \le m^*(\Q) \le \sum_{n \in \N} 0 \le 0
\]

\section{外测度不满足可加性}

\subsection{外测度不满足可数可加性}

下面给出一个反例,来证明外测度不满足可数可加性。令 $x \in \R, \: x + \Q = \{ x + q \wvert q \in \Q \}$。
这个 $x + \Q$ 有个性质,两个实数 $x,\,y$,两个集合 $x + \Q$ 和 $y + \Q$ 要么相等,要么不相交。
因为如果 $x - y \in \Q$ 那么有 $ y + q = y + (x-y) + (y-x) + q = x + q'$, $x + \Q \cap y + \Q$ 会产生 $x -y \in \Q$ 这样的结果。

再考虑集合 

\[
\R / \Q = \{ x + \Q \wvert x \in \R \}
\]

假设 $A \in \R/\Q$,我们利用选择公理,从中取一点 $x_A$,不妨令 $x_A \in [0,1]$,因为有 $x - [x] \in A, \, x - [x] \in [0,1]$。
然后令

\[
 E = \{ x_A \wvert A \in \R/\Q, \, x_A \in A \cap [0,1] \}
\]

这个 $E$ 满足 $E \in [0,1]$,并且对任意的 $q_1, \ne q_2, \, q_1, q_2 \in \Q$ 有 $(q_1 + E) \cap (q_2 + E) = \emptyset$,证明:
假设 $x_1, x_2 \in E$ 满足 $x_1 + q_1 = x_2 + q_2$ 得到 $x_1 - x_2 \in \Q$,因为 $E$ 当中任意两个不同元素 $x,y$ 都有 $x-y \notin \Q$ ,所以有 $x_1 = x_2$ ,这个就跟 $q_1 \ne q_2$ 矛盾了。

我们继续构造集合

\[
X = \bigcup_{q \in [-1,1]} q + E
\]

下面分析 $X$,首先我们证明 $[0,1] \subseteq X$,因为对于 $x \in [0,1]$ 那么必然有 $x + \Q \in \R/\Q$,假设我们上面选择公理对应的函数为 $f$ 并且 $f(x + \Q) = x_0, \, x_0 \in E,\, x - x_0  = q_0 \in \Q$ 
注意到 $-1 \le x - x_0 \le 1$,所以必然有 $x \in (x - x_0) + E$,所以得出结论 $[0,1] \subseteq X$,又因为 $E \subseteq [0,1]$ 而且 $-1 \le q \le 1$ ,所以 $X \subseteq [-1, 2]$

根据外测度的单调性我们得到

\[
1 \le m^*(X) \le 3
\]

如果我们认为外测度满足可数可加性,因为 $X$ 是可数个不相交集合的并集,再根据外测度的平移不变形,有

\[
m^*(X) = \sum_{q \in [-1,1]}m^*(q+E) =  \sum_{q \in [-1,1]}m^*(E)
\]

这里 $m^*(E)$ 要么是 $0$,要么是一个有限的正实数,不论哪种都和 $1 \le m^*(X) \le 3$ 矛盾,所以外测度不满足可数可加性。


\section{可测集}

\subsection{定义}

$\R^n $ 上的可测集 $E$ 定义为满足以下条件的集合

\[
\forall A \subseteq \R^n,\, m^*(A) = m^*(A \cap E) + m^*(A \setminus E)
\]

\subsection{开区间可测}

首先给个不完全的证明,也就是对于 $I=(a,b),\, a, b \in \R$ 这样的开区间,那么有

\[
m^*(I) = m^*(A \cap (0,\infty)) + m^*(A \setminus (0,\infty))
\]

这个对 $0 < a$,$a \le 0 < b$ 和 $b \le 0$ 分类讨论即可。

类似的结论对 $(-\infty, 0)$ 也成立。

\subsection{开盒子可测}

这也是个不完全证明,也就是对于 $I = (a_1, b_1) \times (a_2, b_2) .. \times (a_n,b_n)$
这样的开盒子,以及半空间 $E = \{ (x_1, x_2, .. x_n) \in \R^n \wvert x_k > 0\}$有

\[
m^*(I) = m^*(A \cap E) + m^*(A \setminus E)
\]

这个同理,对 $0 < a_k$,$a_k \le 0 < b_k$ 和 $b_k \le 0$ 分类讨论即可。
对于不完全开也不完全闭的盒子,可以利用外测度的单调性,用开盒子的外测度以及闭盒子的外测度去夹。

类似的结论对 $\{ (x_1, x_2, .. x_n) \in \R^n \wvert x_k < 0\}$ 也成立。

\subsection{半空间 是可测集} 

证明如下,假设有任意 $A \subseteq \R$,令 $ E_k = \{ (x_1, x_2, .. x_n) \in \R^n \wvert x_k > 0\}$并且把 $A$ 的外测度写成极限的形式。

\[
\lim_{n \to \infty} \sum_{i \in I_n}m^*(B_{(n,i)}) = m^*(A)
\]

这里 $B_{(n,i)}$ 是开盒子,利用之前的结论有

\[
m^*(B_{(n,i)}) = m^*(B_{(n,i)} \cap E) + m^*(B_{(n,i)} \cap E^C)
\]


因为每一项都是非负的,所以可以任意交换次序

\begin{align*}
\lim_{n \to \infty} \sum_{i \in I_n}m^*(B_{(n,i)}) = \lim_{n \to \infty}\sum_{i \in I_n}m^*(B_{(n,i)} \cap E) + \lim_{n \to \infty}\sum_{i \in I_n}m^*(B_{(n,i)} \cap E^C)
\end{align*}

注意到 $B_{(n,i)} \cap E$ 构成对 $A \cap E$ 的覆盖,所以有

\[
\sum_{i \in I_n}m^*(B_{(n,i)}) \ge m^*(A \cap E)
\]

对 $n$ 取极限得到

\[
\lim_{n \to \infty} \sum_{i \in I_n}m^*(B_{(n,i)}) \ge m^*(A \cap E)
\]

注意到

\[
A \cap E^C \subseteq \bigcup_{i \in I_n} B_{(n,i)} \cap E^C
\]

所以有

\[
m^*(A \cap E^C) \le m^*(\bigcup_{i \in I_n} B_{(n,i)} \cap E^C) \le \sum_{i \in I_n} B_{(n,i)} \cap E^C
\]


对 $n$ 取极限得到

\[
\lim_{n \to \infty}\sum_{i \in I_n} B_{(n,i)} \cap E^C \ge m^*(A \cap E^C)
\]


所以得到 

\[
m^*(A) \ge m^*(A\cap E) + m^*(A \cup E) 
\]

类似的结论对负的半空间也成立。

\subsection{可测集的补集也是可测集}

这个其实就是利用定义的对称性

\subsection{可测集平移后可测}

假设 $A $ 为任意集合,我们把 $A-x$ 带入 $E$ 可测的条件得到

\[
m^*(A-x) = m^*((A-x) \cap E) + m^*((A-x) \setminus E)
\]

下面我们将证明 $((A-x) \cap E) + x = A \cap (E +x)$,并且 $((A-x) \setminus E) + x = A \setminus (E + x)$

首先我们证明集合的交运算和平移运算满足分配律,也就是对 $I, J \subseteq \R^n$ 有
 
\[
(I \cap J) + x = (I + x) \cap (J + x)
\]

这个用定义很容易证明。


然后证明平移运算可逆,而且有

$I + x - x = A$

继续证明补集运算和平移运算可以交换次序

$I^C +x = (I+x)^C$

这个用定义也很容易证明,然后我们利用外测度的平移不变性得到

\begin{align*}
m^*(A) &= m^*(A-x) = m^*((A-x) \cap E) + m^*((A-x) \cap E^C) \\
&= m^*(A \cap (E+x)) + m^*(A \cap (E^C + x)) \\
&= m^*(A \cap (E+x)) + m^*(A \cap (E+x)^C) \\
\end{align*}

\subsection{可测集的交集可测}

这个证明需要一些技巧,下面假设 $E_1$ 和 $E_2$ 都是可测的,利用可测集的定义可以得到如下的等式:

\begin{align*}
    m^*(A) & = m^*(A \cap E_1) + m^*(A \cap E_1^C) \\ 
    m^*(A) & = m^*(A \cap E_2) + m^*(A \cap E_2^C) \\ 
    m^*(A \cap E_1) & = m^*(A \cap E_1 \cap E_2) + m^*(A \cap E_1 \cap E_2^{C}) \\ 
    m^*(A \cap E_2) & = m^*(A \cap E_1 \cap E_2) + m^*(A \cap E_1^{C} \cap E_2) \\ 
    m^*(A \cap E_1^C) & = m^*(A \cap E_1^C \cap E_2) + m^*(A \cap E_1^C \cap E_2^{C}) \\ 
    m^*(A \cap E_2^C) & = m^*(A \cap E_1 \cap E_2^C) + m^*(A \cap E_1^{C} \cap E_2^C) \\ 
\end{align*}

把上面几个式子全部加起来得到

\[
m^*(A) = m^*(A \cap E_1 \cap E_2) + m^*(A \cap E_1 \cap E_2^{C}) + m^*(A \cap E_1^{C} \cap E_2) + m^*(A \cap E_1^{C} \cap E_2^C)
\]

注意到有 

\begin{align*}
&A \cap E_1 \cap E_2^{C} \cup A \cap E_1^{C} \cap E_2 \cup A \cap E_1^{C} \cap E_2^C \\
&= A \cap E_1 \cap E_2^{C} \cup (A \cap E_1^{C} \cap (E_2 \cup E_2^C)) \\
&= A \cap E_1 \cap E_2^{C} \cup A \cap E_1 ^C \\
&=  A \cap (E_1 \cap E_2^C \cup E_1^C) \\
& = A \cap ((E_1 \cup E_1^C) \cap (E_2^C \cup E_1^C)) \\
& = A \cap (E_1 \cap E_2)^C
\end{align*}


所以根据外测度的次可加性,得到

\begin{align*}
 & m^*(A \cap E_1 \cap E_2) + m^*(A \cap (E_1 \cap E_2)^C) \le m^*(A) \\
 & m^*(A) \le  m^*(A \cap E_1 \cap E_2) + m^*(A \cap (E_1 \cap E_2)^C)
\end{align*}

\subsection{开盒子可测}

$\R^n$ 上的开盒子可以看作是 $2n$ 个平移后半空间的交集,所以开盒子可测。因为闭盒子可以看作是半空间补集的交集,所以闭盒子也是可测的。

\subsection{外测度等于0的集合可测}

根据外测度的单调性有,$m^*(A \cap E) = 0,\, m^*(A \cap E^C) \le m^*(A)$

\subsection{有限可加性}

命题: 如果 $E_j$ 是有限的而且不相交的可测集,$A$ 是任意集合,注意这里 $A$ 不一定可测,那么有

\[
m^*(A \cap \bigcup_{j \in J}E_j) = \sum_{j \in J}m^*(A \cap E_j)
\]

这里把 $A$ 替换成 $\R^n$ 就变成了我们想要的有限可加性。

证明如下,首先我们从两个集合 $E_1, \, E_2$ 开始,令$X = E_1 \cup E_2$,根据 $E_2$ 可测得到如下等式
注意到有 $X \cap E_2 = E_2$ 且 $X \cap E_2^C = (E_1 \cap E_2^C) \cup (E_2 \cap E_2^C) = E_1$

\begin{align*}
    m^*(A \cap X) &= m^*(A \cap X \cap E_2) + m^*(A \cap X \cap E_2 ^C) \\
    &= m^*(A \cap E_2) + m^*(A \cap E_1)
\end{align*}

现在我们证明了两个集合满足有限可加,然后用归纳法证明 $n$ 个集合满足,令 $X_n = X_{n-1} \cup E_n$,根据 $E_n$ 可测得到


\begin{align*}
    m^*(A \cap X_n) &= m^*(A \cap X_n \cap E_n) + m^*(A \cap X_n \cap E_n ^C) \\
    &= m^*(A \cap E_n) + m^*(A \cap X_{n-1}) \\
    &= m^*(A \cap E_n) + \sum_{i=1}^{n-1}m^*(A \cap E_i) \\
    &= \sum_{i=1}^{n}m^*(A \cap E_i)
\end{align*}

\subsection{可数可加性}

命题:当 $E_i,\, i \in \N$ 可测,并且两两不相交时,我们希望 $E_i$ 的并集 $E$ 可测, 并且有如下的等式:

\[
    m^*(A \cap \bigcup_{i=1}^{\infty}E_i) = \sum_{i=1}^{\infty}m^*(A \cap E_i)
\]

之前我们证明了有限个可测集的交并补封闭,但还没有证明可数个可测集也对交并补封闭。


证明如下,根据外测度次可加性,我们有

\[
m^*(A \cap ) = m^*(\bigcup_{i =1}^{\infty}A \cap E_i) \le \sum_{i=1}^{\infty}m^*(A \cap E_i)
\]

然后我们令

\[
F_N = \bigcup_{i=1}^{N}E_i
\]

于是有根据有限可加性得到

\[
\sum_{i=1}^{N}m^*(A \cap E_i) = m^*(A \cap F_N)
\]

所以有

\[
\sum_{i=1}^{\infty}m^*(A \cap E_i) = \lim_{N \to \infty}m^*(A \cap F_N) \ge m^*(A \cap \bigcup_{i=1}^{\infty}E_i) 
\]

我们继续考虑 $A \setminus F_N$,注意到有 $A \setminus E \subseteq A \setminus F_N$,所以有

\[
m^*(A \setminus E) \le m^*(A \setminus F_N)
\]

结合上面的式子得到

\[
m^*(A \cap E) + m^*(A \setminus E) \le \lim_{N \to \infty}m^*(A \cap F_N) + m^*(A \setminus F_N)
\]

注意到 $F_N$ 可测,所以有

\[
m^*(A \cap F_N) + m^*(A \setminus F_N) = m^*(A)
\]

所以得到了

\[
m^*(A) \le m^*(A \cap E) + m^*(A \setminus E) \le m^*(A)
\]

所以我们证明了 $E$ 可测,我们还需要继续证明 $m^*(E)$ 和 $\sum m^*(E_i)$ 相等。

首先根据外测度次可加性有:

\[
m^*(E) \le \sum_{i=1}^{\infty}m^*(E_i)
\]

因为 $F_N$ 不相交,所以我们可以使用外测度的次可加性,还有可测集的有限的可加性得到

\[
\sum_{i=1}^{N}E_i \le m^*(F_N) \le m^*(E)
\]

然后我们对 $N$ 取极限得到

\[
\sum_{i=1}^{\infty}E_i \le m^*(E)
\]

这就证明了

\[
\sum_{i=1}^{\infty}E_i = m^*(\bigcup_{i=1}^{\infty}E_i) ,\, E_i \cap E_j = \emptyset ,\, E_i \:\text{is measurable}
\]

\subsection{$\sigma$ 代数性质}

可数个可测集的并集是可测的,可数个可测集的交集也是可测的。

假设我们有这样的一个集合列 $A_1, A_2, .., A_n$,我们按照如下的方式构造 $B_i$

\begin{align*}
    B_1 &= A_1 \\
    B_2 &= A_2 \setminus A_1 \\
    B_3 &= A_3 \setminus A_2 \setminus A_1 \\
    .. & \\
    B_k &= A_k \setminus A_{k-1} .. \setminus A_1
\end{align*}

利用良序原理容易验证有,并且 $B_i$ 两两不相交,所以 $A_1, A_2, .. A_n$ 是可测的。

\[
\bigcup_{i=1}^{\infty}B_i = \bigcup_{i=1}^{\infty}A_i
\]

同理,根据补集的性质,还有集合的运算规则可以得出,可数个集合的交集也是可测的。

\renewcommand{\f}{f: \Omega \to \R^m,\, \Omega \subseteq \R^n }
\section{可测函数}

\subsection{定义}

$\f$ 可测定义为,对于任意一个开集 $V \subseteq \R^m$,它的原像 $f^{-1}(V)$ 是可测的。

\subsection{连续函数可测}

这个很自然就能想到,假设 $V$ 是开集,$\f$ 是连续的,那么 $f^{-1}(V)$ 一定是开集,所以可测。

\subsection{等价条件}

$\f$ 可测,当且仅当对任意 $\R^m$ 上的开盒子 $B$,$f^{-1}(B)$ 可测。

充分性:假设 $V \subseteq \R^m$ 是一个开集,因为开集可以表示可数个开盒子的并集,根据原像的同态性质得到:

\begin{align*}
& \bigcup_{i \in I}B_i = V \\
& \bigcup_{i \in I}f^{-1}(B_i) = f^{-1}(V)
\end{align*}

其中 $f^{-1}(B_i)$ 是可测的,而 $I$ 是可数集, 所以 $f^{-1}(V)$ 可测。 

必要性很容易证明,因为 $\R^m$  上的开盒子都是开集,根据可测函数的定义可知, $f^{-1}(B)$ 可测。


\subsection{分量}

我们可以利用内积运算把 $\f$ 写成分量的形式,也就是 $f_i(\vecx) = f(\vecx)^T \mathbf{e}_j$,于是得到了 
$f(\vecx) = [f_1(\vecx), f_2(\vecx), .., f_m(\vecx)]^T $

我们想证明 $f$ 可测当且仅当每个 $f_i$ 可测。

首先我们证明充分性,假设 $B=(a_1,b_1) \cdot (a_2, b_2),.., \cdot(a_m,b_m)$ 是 $\R^m$ 上的一个开盒子,
我们令 $I_j \subseteq \R^m,\, I_j = \{ \vecy \in \R^m \wvert \vecy^T \mathbf{e}_j \in (a_j, b_j)\}$。于是我们可以得到对于每个 $1 \le j \le m$,
$f^{-1}(I_j) = f_j^{-1}((a_j,b_j))$ 都是可测的,根据原像的同态性质得到

\[
f^{-1}(B) = f^{-1}(\bigcap_{j \le m}I_j) =\bigcap_{j \le m}f_j^{-1}((a_j, b_j))
\]

因为可测集的有限交集是可测集,所以 $ f^{-1}(B)$  是可测集。


然后我们证明必要性,假设 $f$ 可测,对于 $f_j$ 值域上的开盒子 $ (a_j, b_j)$,令
$I_j = \{ \vecy \in \R^m \wvert \vecy^T \mathbf{e}_j \in (a_j, b_j)\}$,于是我们得到 $f_j^{-1}((a_j, b_j)) = f^{-1}(I_j)$,
因为 $I_j$ 是一个开集,所以 $f_j^{-1}((a_j, b_j))$ 可测。

这个定理也可以理解为,两个可测函数的笛卡尔积是可测的,可测函数的某个分量也是可测的。


\subsection{连续函数复合可测函数}

假设 $\f$ 可测,$g: X \to \Omega, X \in \R^p$ 是一个连续函数,那么 $g \circ f$ 是可测的。证明很容易,假设 $V$ 是$\R^p$ 上的一个开集,所以
$g^{-1}(V)$ 依然是一个开集,因为 $f$ 可测,所以 $ (g \circ f)^{-1}(V) = f^{-1}(g^{-1}(V))$ 也是可测的。

同理,如果 $f: \Omega\to \R,\, \Omega \subseteq \R^n$是可测的,那么 $\lvert f \rvert,\, \max(f(x), 0), \, \min(f(x), 0)$ 都是可测的。

同理 $f, g: \Omega\to \R,\, \Omega \subseteq \R^n$ 都是可测的,那么 $f +g$,$f-g$,$\max(f,g)$,$\min(f,g)$,$fg$ 都是可测的。

我们只需要证明 $x+y$,$x-y$,$\max(x,y)$,$\min(x,y)$这些 $\R^2 \to \R$ 是连续的就可以。

对于 $x+y$ 有 

\begin{align*}
\lvert h(x+\delta, y+\epsilon) - h(x,y) \rvert &= \lvert \delta + \epsilon \rvert \le \lvert \delta \rvert + \lvert \epsilon \rvert \\
& \le \sqrt{2\delta^2 + 2\epsilon^2}
\end{align*}

$x-y$ 同理。

对于 $fg$ 有

\begin{align*}
\lvert h(x+\delta, y+\epsilon) - h(x,y) \rvert &= \lvert \epsilon\delta + \epsilon x  + \delta y\rvert \le M(\lvert \delta \rvert + \lvert \epsilon \rvert) \\
& \le M\sqrt{2\delta^2 + 2\epsilon^2}
\end{align*}


对于 $x/y,\, y \ne 0$ 有

\begin{align*}
\lvert h(x+\delta, y+\epsilon) - h(x,y) \rvert &= \lvert \frac{\delta y - \epsilon x }{y(y+\epsilon)}\rvert  \\
& \le M(\lvert \delta \rvert + \lvert \epsilon \rvert)
\end{align*}

对于 $\max(x,y)$ 有,我们不妨设 $x \ge y$

\begin{align*}
\lvert h(x+\delta, y+\epsilon) - h(x,y) \rvert &= \lvert \max(x+\delta, y+\epsilon) - x\rvert 
\end{align*}

如果 $x+\delta \ge y + \epsilon$ 有


\begin{align*}
\max(x+\delta, y+\epsilon) - x = \delta
\end{align*}

如果 $x+\delta < y + \epsilon$ 有

\begin{align*}
&\delta \le \max(x+\delta, y+\epsilon) - x = y + \epsilon - x \le \epsilon \\
& - \lvert \delta \rvert \le \max(x+\delta, y+\epsilon) - x = y + \epsilon - x \le \lvert \epsilon \rvert \\
& \lvert\max(x+\delta, y+\epsilon) - x = y + \epsilon - x \rvert \le \max(\lvert \delta \rvert, \lvert \epsilon \rvert) \le \sqrt{2\delta^2 + 2\epsilon^2}
\end{align*}


而 $\min(x,y) = - \max(-x, -y)$ 所以无需再证明。

\subsection{另一个等价条件}

假设 $f: \Omega \to \R,\, \Omega \subseteq \R^n$,那么 $f$ 可测当且仅当对任意 $a \in \R$,$f^{-1}((a, +\infty))$ 可测。

现在假设有开区间 $(a,b)$ 那么根据已知条件 $f^{-1}((a,+\infty))$ 可测。令 $I_b = (b, \infty)$ 那么 $f^{-1}(I_b)$ 也是可测的。

我们继续分析 $f^{-1}(\R \setminus I_b) = \Omega \setminus f^{-1}(I_b)$ 也是可测的。以此类推根据,原像的同态性质,以及可测集对交并,补集运算封闭,
我们可以得到 $f^{-1}((a,b])$ 是可测的。我们可以把 $b$ 缩小,构造可数个集合 $(a, b - \epsilon_n]$ 满足,$a < b - \epsilon_n < b,\, (\epsilon_n)^{(\infty)} = 0$,于是得到了

\[
(a,b) = \bigcup_{n=1}^{\infty}(a, b - \epsilon_n]
\]

根据原像的性质得到

\[
f^{-1}(a,b) = \bigcup_{n=1}^{\infty} f^{-1}((a, b - \epsilon_n])
\]

所以 $f^{-1}(a,b)$ 可测。


必要性很容易证明,因为 $(a, +\infty)$是开集,可以直接套可测函数的定义。

\subsection{扩展实数系上的可测函数}

假设 $f: \Omega \to \R^*$ 的值域是扩展实数系,我们定义 $f$ 可测当且仅当对任意实数 $a$,$f^{-1}((a, +\infty])$ 可测。


\subsection{可测函数的极限可测}

假设有函数序列 $f_n: \Omega \to \R^*$,注意这里 $\R^*$ 是扩展实数集,并且都满足可测,
那么函数 $\sup_{n \in \N} f_n$,$\inf_{n \in \N}f_n$,$\lim \sup_{n \to \infty}f_n$ 和 
$\lim \inf_{n \to \infty}f_n$ 
都是可测的。

令 $g(x) = \sup_{n \in N}f_n(x)$ 则有

\[
g^{-1}((a, +\infty]) = \bigcup_{n = 0}^{\infty}f_n^{-1}((a, +\infty])
\]

证明如下,若 $x$ 在左边集合内,那么有 $\sup_{n \in \N}f_n(x) > a$,所以一定存在 $n$ 满足 $f_n(x) > a$,所以有 $x \in f_n^{-1}((a, +\infty])$
所以 $x$ 在左边能得出 $x$ 在右边。

若$x$ 在右边集合内,则存在 $f_n(x) > a$,所以 $g(x)$ 必然也大于 $a$ 所以 $x$ 在左边集合内,所以上述式子成立,根据可测集合可数并后可测得出 $g$ 也是可测的。


令 $g(x) = \inf_{n \in N}f_n(x)$ 同理也有

\[
g^{-1}([a, +\infty]) = \bigcap_{n = 0}^{\infty}f_n^{-1}([a, +\infty])
\]

证明方法也是同理,利用下确界的性质。

令 $g = \lim \sup_{n \to \infty}f_n$
得到

\[
g^{-1}((a, +\infty)) = \bigcap_{n = 0}^{\infty}\bigcup_{i \ge n}f_n^{-1}((a, +\infty))
\]


令 $g = \lim \inf_{n \to \infty}f_n$
得到

\[
g^{-1}([a, +\infty]) = \bigcup_{n = 0}^{\infty}\bigcap_{i \ge n}f_n^{-1}([a, +\infty])
\]

假设 $f_n$ 逐点收敛到 $g$,那么有

\[
g^{-1}((a, +\infty]) = \bigcup_{n = 0}^{\infty}\bigcap_{i \ge n}f_n^{-1}((a, +\infty])
\]