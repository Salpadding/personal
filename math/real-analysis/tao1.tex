\part{实分析 I}
\chapter{实数}

\section{实数定义: 有理数柯西列}

我们要找到一种合理的方式定义实数,让实数可以做运算,可以相互比较。

\subsection{柯西列定义}

柯西列的定义可以不依赖于实数,所以可以用于定义实数。柯西列的可以定义为满足如下条件的有理数数列 $ q_n $。

\[ \forall \epsilon > 0, \exists N \in \mathbb{N}, 
    \forall j, k \ge N, \lvert q_j - q_k \rvert \le \epsilon \] 

\subsection{柯西列的等价条件}

两个柯西列 $q_n$ 和 $r_n$ 等价的定义如下:

\[ 
    \forall \epsilon > 0, \exists N \in \mathbb{N},
    \forall n \ge N, \lvert q_n - r_n \rvert \le \epsilon 
\]

\subsection{柯西列等价是良定义的}

首先是自反性,易证。其次是传递性,通过三角不等式也很容易证明。最后是对称性,也易证。

\subsection{柯西列有界}

我们令 $\epsilon = 1 $,可以找到 $N$ 满足 $\forall n \ge N , \lvert q_n - q_N \rvert \le 1 $ 于是 $q_n$ 在 $n \ge N$ 的那部分是有界的,
而 $n < N$ 的那部分是有限的自然也是有界的,所以柯西列有界。

\subsection{习题}

证明: $q_n$ 和 $r_n$ 是等价有理数序列,$q_n$ 是柯西列当且仅当 $r_n$ 是柯西列。 \\

$\forall j, k \ge N, \lvert q_j - q_k \rvert \le \epsilon/4 $, 且 $ \lvert r_j - q_j \rvert \le \epsilon / 4 $, $\lvert r_k - q_k \rvert \le \epsilon / 4$, 那么有
$\lvert r_j - r_k \rvert \le \lvert  r_j - q_j + q_j - q_k + q_k - r_k \rvert \le \frac{3}{4} \epsilon $

\subsection{实数定义}

实数被定义为有理数柯西列的形式极限 也就是 
\[ x = \lim_{n \to \infty} q_n,\, q_n \in \Q \]

\section{实数的运算}

\subsection{实数的加运算}

通过柯西列,我们可以很方便地定义实数的加减乘运算。 \\
首先我们可以定义加减运算。

\begin{align*} 
x & = \lim_{n \to \infty}q_n,\, y  = \lim_{n \to \infty} r_n \\
x + y & := \lim_{n \to \infty} q_n + r_n 
\end{align*} \\

这个定义看似很合理,实际上并没有被证明是合理的。我们要证明这个定义要满足:

\begin{enumerate}
    \item $x +y $ 是一个柯西列
    \item $x + y$ 是一个二元元算,与柯西列的等价定义不冲突
    \item 满足加法的交换律和结合律
\end{enumerate}

首先证明 $x +y $ 是一个柯西列,有理数相加还是有理数,对于 $\epsilon $只要取 $N_x $ 和 $N_y$ 之间较大的 $N$ 就可以了。\\

证明等价性: $ \lvert q'_n + r'_n - q_n - r_n \rvert \le 2\epsilon $。\\

最后证明交换律和结合律: $ \lvert q_n + r_n - (q_n + r_n) \rvert = 0 \le \epsilon $ 以及 $ \lvert q_n + r_n + s_n - (q_n + (r_n + s_n)) \rvert = 0 \le \epsilon $ \\

\subsection{实数的乘法运算}
我们继续定义乘法运算。 \\
\begin{align*} 
x & = \lim_{n \to \infty}q_n,\, y  = \lim_{n \to \infty} r_n \\
xy & := \lim_{n \to \infty} q_nr_n 
\end{align*} \\

同样我们要证明 $q_nr_n$ 是一个柯西列, 有理数相乘还是有理数,我们观察到

\begin{align*}
\lvert q_kr_k - q_jr_j \rvert &= \lvert (q_j - (q_j - q_k)) (r_j - (r_j - r_k)) - q_jr_j \rvert \\
& \le \lvert  (q_j - q_k)(r_j - r_k)  \rvert + \lvert  (q_j - q_k)r_j  \rvert + \lvert  (r_j - r_k)q_j  \rvert
\end{align*}

因为 $\lvert r_j \rvert$ 和 $\lvert q_j \rvert$ 有界,我们可以取 $M$ 满足 $\lvert q_n \rvert \le M$ 且 $\lvert r_n \rvert \le M$ 所以对于 $\epsilon$,我们可以取一个很大的 $N$, 对 $k,j \ge N $ 满足 $\lvert q_j - a_k \rvert \le \epsilon $,$ \lvert r_k - r_j \rvert \le \epsilon$,这样就能得到

\begin{align*}
\lvert  (q_j - q_k)(r_j - r_k)  \rvert + \lvert  (q_j - q_k)r_j  \rvert + \lvert  (r_j - r_k)q_j  \rvert \le \epsilon(\epsilon + 2M)
\end{align*}

因为 $\epsilon$ 可以任意小,所以 $q_nr_n$ 是柯西列。

同理可以继续证明等价性: $\lvert q'_nr'_n - q_nr_n\rvert \le \epsilon(\epsilon + 2M)$

交换律和结合律,以及分配律可以直接利用有理数的性质。

\subsection{实数的三歧性}
有理数具有三歧性,通过柯西列我们也可以证明实数的三岐性。我们先定义正实数和负实数和零。\\
零很好定义,$\forall \epsilon >0, \exists N, \forall n \ge N, \lvert a_n \rvert \le \epsilon$ \\
以此类推可以定义正数,$\exists \epsilon > 0, \exists N, \forall n \ge N, a_n \ge \epsilon $ \\
同理可以定义负数,$\exists \epsilon > 0, \exists N, \forall n \ge N, a_n \le -\epsilon $

这些定义的合理性也需要证明,首先证明每个实数都具有三岐性之一。\\
对于实数 $x = \lim_{n\to \infty} a_n$,假设 $x = 0$,那么易证。如果 $x \ne 0$,根据柯西等价性的定义,那么就有 
\[ \exists \epsilon > 0, \forall N , \exists n \ge N, \lvert a_n \rvert \ge \epsilon \]
我们取这个 $\epsilon / 4 $ 代入 柯西列的定义 那么 $\exists N_{\epsilon} \ge N , \forall j,k \ge N_{\epsilon}, \lvert a_k - a_j\rvert \le \epsilon / 4 $,把 $j$ 换成 $N$ 得到 
$\lvert a_k  \rvert \ge  \lvert a_N \rvert - \lvert a_k - a_N \rvert \ge \frac{3}{4} \epsilon $。所以 $x$ 如果不为 0 的时候,只能是负数或者正数,显然 $x$ 不可能同时是正数和负数,所以 $x$ 满足三岐性。

\subsection{倒数运算}

对于正数和负数,我们可以定义倒数运算 

\begin{align*}
x & = \lim_{n \to \infty} a_n,\, \exists m > 0,\, \exists N,\, \forall n \ge N,\, \lvert a_n \rvert \ge m \\
b_n & = 
\begin{cases}
1 & n < N \\
\frac{1}{a_n} & n \ge N \\
\end{cases} \\
x^{-1} & = \lim_{n \to \infty} b_n \\
\end{align*}
倒数运算是良定义的而且保持等价性。

\[
\lvert \frac{1}{a_k} - \frac{1}{a_j} \rvert = \frac{\lvert a_k - a_j \rvert}{\lvert a_ka_j \rvert} \le \frac{\epsilon}{m^2}
\]

而且 $xx^{-1} = 1$,这个很容易证明。于是我们可以定义除法 $x / y = xy^{-1}$

\subsection{实数的封闭性}

\begin{enumerate}
    \item 如果有理数柯西列 $q_n \ge 0$ 那么 $x = \lim_{n \to \infty} q_n \ge 0$

    证明: 假设 $x < 0$ ,那么存在 $\epsilon > 0 , N_{\epsilon}$,$\forall n \ge N, q_n \le -\epsilon $,显然矛盾了。\\

    \item 如果 $q_n \ge x$ 那么 $y = \lim_{n \to \infty} q_n, y \ge x$


证明:假设 $y < x, \lim_{n \to \infty} r_n = x$ 。
根据负实数定义,存在 $ \exists \epsilon > 0, \;N, \; \forall n \ge N,  q_n - r_n  \le -\epsilon $。注意到这里 $N$ 可以取任意大,
我们可以把 $ \epsilon / 2$ 代入 $r_n$ 的柯西列定义,让 $ \lvert r_n - r_N \rvert \le \epsilon /2$ 。于是当 $n \ge N$ 时,有 $q_N - r_n = q_N - r_N + r_N - r_n \le - \epsilon / 2$
这时候得到了矛盾 $q_N < x$
\end{enumerate}




\subsection{习题}

\begin{enumerate}
    \item 对于任意实数 $ x < y$,存在有理数 $q$ 满足 $x < q < y$

这里用反证法比较容易,假设所有的有理数都满足 $ q \le x \quad or \quad q \ge y $,我们分析一下 $c = (x+y)/2$,显然有 $x < c < y$。
假设 $ c = \lim_{n \to \infty}a_n$,因为 $a_n$ 是有理数,所以 $a_n$ 必然满足 $a_n \le x$ 或者 $a_n \ge y$,又因为 $a_n$ 是柯西列而且存在有理数 $\delta $ 满足 $0 < \delta \le y-x$。
所以存在 $N_{\delta}$ 使得 $\forall k,j \ge N_{\delta}$,$\lvert a_k - a_j \rvert \le \delta $。假设 $a_{N_{\delta}} \le x$,那么对于 $n \ge N_{\delta}$,必然有 $a_n \le x$。
因为如果存在 $a_k \ge y$ 的话那么 $\lvert a_k - a_{N_\delta} \rvert \ge y -x > \delta $,矛盾了。同理如果 $a_{N_{\delta}} \ge y$,那么对于 $n \ge N_{\delta}$ 多有 $a_n \ge y$。
那么我们得到了 $ c \le x$ 或者 $c \ge y$, 显然矛盾了。

    \item 对于任意实数,存在唯一的整数满足 $N \le x < N + 1$

存在性: 因为 $x$ 是有界的,所以存在 $M > 0, M \in \mathbb{Z}$ 满足 $ -M \le x \le M $。定义 $ N $ 为满足 $N \le x, \lvert N \rvert \le M$ 的最大整数。显然有 $ N \le x < N + 1 $\\

唯一性: 假设 $N_1 \le x < N_1 + 1$ 且 $N_2 \le x < N_2$,那么有 $N_2 < N_1 + 1$, $N_1 < N_2 + 1$,也就是 $\lvert N_2 - N_1 \rvert < 1$,所以 $N_2 = N_1$
\end{enumerate}






\subsection{上确界性质}

对于任意的非空实数集合,存在最小上界。\\

证明: 假设 $E \subset \mathbb{R}, E \ne \emptyset $ 而且 $ E $ 具有上界 $M, M \in \mathbb{Q}$。取 $x \in E$, 有理数 $q \le x$
对于 $n$,我们把 $q \le y \le M$,分成 $n$ 份,每一份的长度为 $c_n = (M -q) / n$。于是对于 $N$,我们构造了有限的点集 $q + kc_n, 0 \le k \le n $
可以用数学归纳法证明,这个点集中存在 $E$ 的上界,我们取 $a_n = k $ 满足 $q + kC_n$ 是 $E$ 的上界,而且 $q + (k-1)C_n $ 不是 $E$ 的上界。
于是得到了

\begin{align*}
\frac{a_{n} - 1}{n} - \frac{a_n}{n} \le \frac{a_{n+1}}{n+1} - \frac{a_n}{n} & \le \frac{a_{n+1}}{n+1} - \frac{a_{n+1} - 1}{n+1} \\
\lvert \frac{a_{n+1}}{n+1} - \frac{a_n}{n} \rvert & \le \frac{1}{n}
\end{align*}

注意到这里的 $n+1$ 换成 $n + k$ 可以得到同样的结论,所以 $q + \frac{a_n}{n}$ 是一个柯西列,而且 $a + \frac{a_n - 1}{n}$ 也是柯西列。

令 $y = \lim_{n \to \infty} q + \frac{a_n}{n}$,可以得到 $y$ 是 $E$ 的上界。而且 $y$ 是最小上界,假设 $y' < y$,而且 $y'$ 也是 $E$ 的上界,注意到有 $q + \frac{a_n-1}{n} \le y'$,于是得到 $y \le y'$,矛盾。

\section{实数的幂运算 I}

\subsection{整数次幂}
正整数和负整数的幂运算很容易定义这里不作说明,对应的良定义,保持等价性,以及性质用数学归纳法都很好证明。

\subsubsection{$n$ 次根}
定义非负实数 $y$ 的 $n (n > 0,\, n \in \N)$ 次根为集合 $ \{ x \in \R \wvert  \, x^n \le y,\, x \ge 0 \} $ 的最小上界。

对于这个定义,对 $y$ 分成 $y \le 1$ 和 $y > 1$ 讨论后可以的得出这个集合是有界的,所以存在性和等价性都很好证明。

\subsection{习题}

\begin{enumerate}
    \item $(x^{1/n})^n = x$

这个并没有看上去那么好证明,需要用到反证法。假设 $c = x^{1/n}$ 而且 $ c^n < x$。我们的思路是尝试找一个非常小的 $\epsilon \in \mathbb{R}$ 满足 $(c+\epsilon)^n < x$。首先 $n$ 目前是固定的,所以我们可以使用数学归纳法。

下面证明引理: $\forall \epsilon > 0, \exists \delta > 0 , \forall \lvert x \rvert \le \delta $, 满足 $ \lvert c^n - (c+x)^n \rvert \le \epsilon $。这个可以理解为幂函数的连续性。
我们对 $n $ 采用数学归纳法,首先 $n = 1$ 时很好证明,令 $\delta = \epsilon / 2$ 即可。
假设 $n = k$ 时结论成立,考虑 $n = k + 1$ 时,我们代入 $n = k$ 条件下的 $\delta$,取 $ \lvert x \rvert \le \delta $,而且 $\lvert x \rvert < 1$ 注意到 $\lvert c^{k+1} - (c+x)^{k+1} \rvert = \lvert c(c^k - (c+x)^k) - x(c+x)^{k} \rvert \le  \lvert c \epsilon \rvert  +  \lvert x(c+x)^{k} \rvert \le  c \epsilon + \lvert \delta (c+1)^{k+1} \rvert $
注意到 $n = k + 1$ 时,这里的 $\delta$ $\epsilon $ 都是可以取任意小的。对于 $n = k + 1$ 的 $\epsilon $ 我们可以针对 $n = k$ 取很小的 $\epsilon' $ 和很小的 $\delta $,满足 
$c \epsilon' + \lvert \delta (c+1)^{k} \rvert  \le \epsilon $ 于是证明成功。 \\

假设 $c ^n < x$,那么根据引理我们可以找到一个很小的 $\epsilon$ 满足 $(c+\epsilon)^n < x$,于是根据 $c$ 是上界得到 $c + \epsilon \le c$,矛盾。 \\
假设 $c ^n > x$,那么根据引理我们可以找到一个很小的 $\epsilon$ 满足 $(c-\epsilon)^n > x$,于是 $c - \epsilon$ 也是集合 $\{ a \vert \, a^n \le x \}$ 的上界,这里和 $c$ 是最小上界矛盾。\\

    \item $(x^n)^{1/n} = x$

根据定义 $(x^n)^{1/n}$ 是集合 $\{a \vert \, a^n \le x^n \}$ 的最小上界。显然这个集合包含了 $x$,而且对于任意的 $a$,都有 $a \le x$,所以 $x$ 就是该集合的最小上界。\\

    \item $f(x) = x^{1/n}$ 是严格单调增

原理很简单,子集的最小上界一定小于等于父集的最小上界。再假设 $x^{1/n} = y ^{1/n}$,得到 $x = y$。

    \item $a < 1$ 时 $a^{1/k}$ 是关于 $k$ 的递减函数。

还是那个原理,当 $a < 1 $ 时,$a^{k+1} < a^{k} $,所以也存在子集关系。$x < a^{k+1} $ 可以得到 $ x^{k+1} < a$ 是 $x ^k < a$ 的子集。
所以 $a^{1/(k+1)} \le a^{1/k} $,而且这个等号也不可能成立。 

    \item $a > 1$ 时 $a^{1/k}$ 是关于 $k$ 的递增函数。

    同理。

    \item $x ^{1/n}$ 连续

上面已经证明了 $ x^n$ 是连续的,因为 $ x ^{1/n}$ 是 $x^n$ 的反函数,所以 $x^{1/n}$ 连续。
可逆连续的函数在拓扑学也叫做同胚映射。$f(x) = x^{n},\, x \ge 0$ 是单调增的函数,所以它的反函数也一定单调增,对于 $f(x_0) = y_0$ 和 $\epsilon > 0$ 且, 我们可以取 $\delta$ 满足 $U(y_0, \delta) \subseteq f([x_0 - \epsilon, x_0 + \epsilon] \cap [0, +\infty))$
然后两边计算$f^{-1}$ 的象集得到 $f^{-1}(U(y_0, \delta)) \subseteq U(x_0, 2\epsilon)$,这里 $\epsilon$ 可以取任意小,$\delta$ 能取到是根据闭区间上连续函数的最值原理。
\end{enumerate}



\subsection{实数的有理数次幂}

假设有理数 $q = a/b $,其中 $a$ 和 $b$ 都是整数,那么定义 $x^q = (x^{1/b})^a $。 \\

首先要证明保持等价性,假设 $q = a/b$ 而且 $q = c/d$,我们要证明 $(x^{1/b})^a = (x^{1/d})^c$。\\
根据 $x^{1/(ad)} = x^{1/(bc)}$ 得到 $(x^{1/(ad)})^{ac} = (x^{1/(bc)})^{ac}$,根据定义得到 $x ^{ac/(ad)} = x^{ac/(bc)}$
再得到 $ x^{c/d} = x^{a/b}$

\subsection{有理数幂性质}

\begin{enumerate}
    \item $x^{q+r} = x^qx^r $
    把 $q$ 和 $r$ 分解成比例形式,然后两边同时作幂运算很容易证明。

    \item $x^{-q} = 1/x^q $
    同理,利用定义分解成比例形式
    
    \item $q > 0$ 时 $x^q < y^q$ 等价于 $x < y$
    同理,利用定义分解成比例形式

    \item $a > 1$ 时 $a^q$ 关于 $q$ 单调增
    同理,利用定义分解成比例形式

    \item $(xy)^q = x^q y ^ q $
    同理,利用定义分解成比例形式,左右同时做幂运算

    \item $x^{qr} = (x^q)^r $
    同理,利用定义分解成比例形式,左右同时做幂运算
\end{enumerate}


\section{序列的极限}

\subsection{每项为实数的柯西列}

这个定义也很容易理解。

\[ \forall \epsilon \in \mathbb{R}, \epsilon >0, \exists N, \forall n \ge N, \lvert a_k - a_j \rvert \le \epsilon \]

为了能够使用这个定义,我们要证明这个定义和之前的定义是一致的。这个定义和之前的定义的区别在于,前面的 $\epsilon $ 是有理数,现在的 $\epsilon$ 是实数。
因为有理数就是实数,而正实数和0之间一定存在有理数。

\subsection{实数序列的极限}

$x_n$ 收敛到 $L$ 定义为: $\forall \epsilon > 0, \exists N, \forall n \ge N, \lvert a_n -L \rvert \le \epsilon $。


\subsection{收敛序列是柯西列}

利用三角不等式,很容易证明。

\subsection{形式极限和极限的定义一致}

假设有形式极限 $(q_n)^{\infty} = x$,我们观察 $q_N - x$ 也就是 $\lim_{n \to \infty} q_N - q_n$ 。因为 $q_n$ 是有理数柯西列,所以当 $N$ 充分大的时候,有 $\lvert q_N - q_n \rvert \le \epsilon $。
所以 $N$ 充分大的时候有 $ \lvert q_N - x \rvert = \lvert (q_N - q_n)^{(\infty)} \rvert \le \epsilon $,所以 $\lim_{n \to \infty} q_n = x$。\\

\subsection{柯西列有界}

很容易证明

\subsection{极限点,上下极限,收敛子列}

\begin{enumerate}
    \item 极限点的定义

    $\forall \epsilon > 0, \forall N, \exists n \ge N, \lvert a_n - L\rvert \le \epsilon $。

    \item 上极限定义

\[
    \varlimsup_{n \to \infty} a_n = \lim_{n \to \infty} \sup \{ a_k \wvert k \ge n\}
\]


    \item 下极限定义

\[
    \varliminf_{n \to \infty} a_n = \lim_{n \to \infty} \inf \{ a_k \wvert k \ge n\}
\]

    \item 极限点和收敛子列等价
    通过极限点的定义可以构造出收敛子列,通过收敛子列可以构造出极限点,所以这两者是等价的。

    \item 上极限是最大的极限点,下极限是最小的极限点。

这个很好证明,首先我们要证明一个引理,子列的上极限不大于序列的上极限。然后我们假设存在更大的极限点,这个极限点可以转化成子列,
对这个子列去上极限得到了更大的上极限,这个与引理相矛盾。

    \item 如果上下极限都存在而且相等,那么这个序列收敛,而且收敛到上下极限。

    这个可以用三明治定理证明。
\end{enumerate}





\subsection{柯西列收敛}

这是一个非常重要的定理,柯西列是有界的所以存在上极限和下极限。柯西列有个特点,就是任意两个子列都是 $\epsilon$ 稳定的,这个用柯西列的定义很容易证明。考虑到柯西列有界,上下极限也是子列,所以柯西列的上下极限存在并且相等。
所以柯西列收敛。这个性质可以推广到度量空间,更一般的形式是有限覆盖。

\subsection{一些常见的极限}

$\lim_{n \to \infty}x^{1/n} = 1$ \\

这个需要分成 $x > 1$ 和 $x < 1 $ 讨论,当 $ x > 1$ 时,对于任意小的 $ \epsilon$ 我们可以找一个很大 $N$ 满足 $(1 + \epsilon) ^ N \ge x$,于是得到
$x^{1/N} - 1 \le  \epsilon$。当 $x < 1$ 时,我们先利用单调有界证明极限存在,再利用 $y=x^{-1}$ 的连续性可以证明。

\section{实数的幂运算 II}

有理数对于幂运算来说并不封闭,我们希望实数在幂运算下保持封闭,也就是说实数的实数次幂依然是实数。

\subsection{实数的实数次幂}

我们可以通过极限定义实数的实数次幂,对于非负实数 $x$,并且形式极限 $y = \lim_{n \to \infty }q_n, q_n \in \mathbb{Q}$ ,定义 $x^y = \lim_{n \to \infty} x^{q_n}$ 。
首先我们要证明这个定义有意义,也就是 $x^{q_n}$ 收敛,但目前我们不知道收敛到哪里,所以只能用柯西列的定义证明。
注意到 $\lvert x^{q_k} - x^{q_j} \rvert = \lvert x^{q_j} \rvert \lvert (x^{q_k-q_j}-1) \rvert$,因为 $q_n$ 有界所以 $x^{q_j}$ 有界,前面我们也证明过 $a^x, x \in \mathbb{Q} $ 在 $x = 0$
附近可以无限接近 $1$, 也就是 $\lim_{n \to \infty} a^{1/n} = 1$。所以 $x^y$ 是一个柯西列。

我们用到了形式极限来定义实数次幂,所以还需要证明幂运算的结果保持等价性,也就是 $q_n$ 和  $q'_n$ 如果等价那么 $x^y = x^{y'}$
这个证明和上面的很相似,注意到$ \lvert x^{q_n} -x ^{q'_n} \rvert = \lvert x^{q'_n} \rvert \lvert x^{q_n -q'_n}-1\rvert$。同理得证。

\subsection{实数的实数次幂的性质}

包含了所有有理数的性质

\begin{enumerate}
    \item $x^{s+t} = x^sx^t$ 

    把 $s$ 和 $t$ 表示成形式极限,然后用定义证明。

    \item $x^{st} = (x^s)^t $

这个证明没有看上去那么简单,要避免使用二重极限。 \\
我们先证明对于有理数 $q$ 和 实数 $\lim_{n \to \infty}a_n = s,\, a_n \in \Q$,有 $x^{sq} = (x^s)^q = (x^q)^s$

证明: $(x^q)^s = \lim_{n \to \infty} (x^q)^{a_n} = \lim_{n \to \infty} x^{qa_n} = x^{sq} $ \\
$(x^s)^q = (\lim_{n \to \infty}x^{a_n})^q$,利用幂函数的连续性,得到 $(x^s)^q = \lim_{n \to \infty}(x^{a_n})^q = x^{qs}$

对于实数 $s$ 和 $t$,可以利用上面的结论。$(x^{s})^{t} = \lim_{n \to \infty}(x^s)^{b_n} = \lim_{n \to \infty}x^{s b_n} = x^{st}$


\item $(xy)^s = x^sy^s$ 

很容易证明

\item 对于实数 $a \ge 0, x \in \mathbb{R}, y = a^x$ 连续

这个和之前证明 $\lim_{n \to \infty}x^{1/n} = 1$ 的方法是一样的。$\lvert a^{x + \delta} - a^x \rvert = \lvert a^x\rvert \lvert a^{\delta} - 1\rvert$

\item 对于实数 $c \in \mathbb{R}, x > 0, y = x^c$ 连续


可以先证明 $f_n = x^{c_n}$ 在任意闭区间 $[a, b],\, a > 0$ 上一致收敛到 $f = x^c$ 上,然后利用一致收敛的性质。
$\lvert f_n - f_m \rvert  = \lvert  x^{c_m}\rvert \lvert x^{c_n - c_m} - 1\rvert \le M \epsilon $,这里如果 $ -1 \le c_n - c_m <0$,那么
$b^{c_n-c_m} - 1 \le x^{c_n - c_m} - 1 \le a^{c_n-c_m} - 1$,$ 0 \le c_n - c_m $ 时同理有$a^{c_n-c_m} - 1 \le x^{c_n - c_m} - 1 \le b^{c_n-c_m} - 1$
所以有

\[
-\max (\lvert a^{c_n-c_m} - 1 \rvert, \lvert b^{c_n-c_m} - 1 \rvert) \le x^{c_n - c_m} - 1 \le \max (\lvert a^{c_n-c_m} - 1 \rvert, \lvert b^{c_n-c_m} - 1 \rvert)
\]

于是可以取 $N$ 足够大,同时满足 $\lvert a^{c_n-c_m} - 1 \rvert \le \epsilon$ 和$\lvert b^{c_n-c_m} - 1 \rvert \le \epsilon$ 就能得到 $f_n$ 一致收敛。

\end{enumerate}
6. $y = a^x$ ,在 $a > 1$ 时单调增,在 $a < 1$ 时单调减。利用定义展开成极限,很容易证明。

\chapter{级数}

\section{判别法}

\subsection{交错级数判别法}
$a_n \ge 0 $ 并且 $a_n$ 单调减,那么 $S_n = (-1)^na_n$ 收敛当且仅当 $\lim_{n \to \infty}a_n = 0$ 。
证明: 令 $T_n = \sum_{k=1}^{2n}(-1)^na_n$,并且 $T'_n = \sum_{k=1}^{2n+1}(-1)^na_n$ \\
注意到 $T_{n+1} = T_n  - a_{2n+1} + a_{2n+2} \le T_n$,$T'_{n+1} = T'_n  + a_{2n+2} - a_{2n+3} \ge T'_n$ \\
所以 $T_n $ 具有上界,而 $T'n$ 具有下界,同时 $T'_n = T_n + (-1)^na_{2n+1}$,因为 $a_n$ 是有界的,所以 $T_n$ 和 $T'_n$ 都有界,所以
$T_n$ 和 $T'_n$ 都收敛。\\
又因为 $T'_n - T_n = a_{2n+1}$,所以如果 $\lim_{n \to \infty} a_n = 0$,$\lim_{n \to \infty}T_n = \lim_{n \to \infty} T'_n$,所以 $S_n$ 收敛。
同理如果 $S_n$ 收敛 $T_n$ 和 $T'_n$ 必然等价,所以 $a_n$ 必然也收敛。

\subsection{调和级数发散}
针对调和级数这类,也就是 $a_n \ge 0$,而且$a_n$ 单调减,可以找到一种判别法。
可以根据 $T_n = \sum_{k = 0}^{n-1}2^ka_{2^k}$ 来判断 $S_n$ 是否收敛。
证明也很容易,利用放缩法证明,$S_{2^k-1} \le T_k$。这里 $T_n$ 的定义是为了保证 $T_n$ 恰好有 $n$ 项,
而且每项的系数之和等于 $2^{k-1}$。

下面给出具体证明:

\begin{align*}
\sum_{n=1}^{2^k - 1}a_n & = \sum_{i=1}^{k}\sum_{j=2^{i-1}}^{2^i-1}a_j \le \sum_{i=1}^{k}2^{i-1}a_{2^{i-1}} \\
& \le \sum_{i=0}^{k-1}2^ia_{2^i}
\end{align*} \\

同时又有 \\

\begin{align*}
\sum_{n=1}^{2^{k+1}-1}a_n & \ge a_1 + \sum_{n=2}^{2^k}a_n \ge a_1 + \sum_{n=1}^{2^k-1}a_{n+1} \\
 & \ge a_1 + \sum_{i=1}^{k}\sum_{j=2^{i-1}}^{2^i-1}a_{j+1} \ge a_1 + \sum_{i=1}^{k}2^{i-1}a_{2^i} \\
2\sum_{n=1}^{2^{k+1}-1}a_n & \ge 2a_1 + \sum_{i=1}^{k}2^ia_{2^i} \ge \sum_{n=0}^{k}2^na_{2^n}
\end{align*} \\

所以有  \\

\[ \sum_{n=1}^{2^k-1}a_n \le \sum_{n=0}^{k-1}2^na_{2^n} \le 2\sum_{n=1}^{2^k-1}a_n \]

\subsection{黎曼函数}

利用柯西判别法可以判断 $\sum_{n=1}^{\infty}1/n^s$ 是否收敛。这个数列是单调减而且非负的。

\[ \sum_{n=0}^{k-1}2^k\frac{1}{2^{ks}} = \sum_{n=0}^{k-1}(2^{1-s})^k \]

所以当且仅当 $1-s < 0$ 也就是 $s > 1$ 时收敛,这个 $s$ 级数可以看作是一个关于 $s$ 的函数。


\subsection{根值判别法}
这个判别法只需要考虑 $a_n$ 的性质,不需要考虑 $S_n$ 的性质。\\

1. $\lvert a_n \rvert ^{1/n}$ 的上极限如果小于 1,那么 $S_n$ 绝对收敛。 \\

2. $\lvert a_n \rvert ^{1/n}$ 的上极限如果大于 1,那么 $S_n$ 发散。 \\

3. $\lvert a_n \rvert ^{1/n}$ 的上极限如果等于 1,那么未知。 \\

1很容易证明,根据定义能找到一个 $ \alpha < 1 $ 和 $N$,满足 $\forall n \ge N, \lvert a_n \rvert ^{1/n} \le \alpha $。
这样就得到了 $\lvert a_n \rvert \le \alpha ^n$,因为 $\sum_{n=1}^{M}\alpha^n$ 有界,所以 $\sum_{n=1}^{M}\lvert a_n \rvert $ 收敛。 \\

2也很容易证明,根据定义对任意 $N$ 都能找到 $n \ge N$,满足 $\lvert a_n \rvert ^{1/n} > 1$ 从而有 $\lvert a_n \rvert > 1$,假如 $\sum_{n=1}^{\infty}a_n$ 收敛,必然有 $\lim_{n \to \infty}\lvert a_n \rvert = 0$


\subsection{比值判别法}

假设 $c_n > 0$,那么有

\begin{align*}
\varliminf_{n \to \infty}\frac{c_{n+1}}{c_n} \le \varliminf_{n \to \infty} c_n ^{1/n} \le \varlimsup_{n \to \infty} c_n ^{1/n} \le \varlimsup_{n \to \infty} \frac{c_{n+1}}{c_n}
\end{align*}

证明:令 $\varlimsup_{n \to \infty}\frac{c_{n+1}}{c_n} = \alpha$,那么取 $\epsilon > 0$ 就存在 $N$ 对 $n \ge N$ 都有 $c_{n+1}/c_n \le \alpha + \epsilon$,所以对 $k \ge 1$ 有 $c_{N+k} \le c_N (\alpha+\epsilon)^k$,
两边同时取 $N+k$ 次根得到 

\[ \sqrt[N+k]{c_{N+k}} \le \sqrt[N+k]{c_{N}} (\alpha+\epsilon)^{k/(N+k)} \]

两边对 $k$ 取上极限得到

\[ \varlimsup_{n \to \infty}c_n^{1/n} \le \alpha + \epsilon \]

另一边也是同理,令 $\varliminf_{n \to \infty}\frac{c_{n+1}}{c_n} = \beta$,那么取 $\epsilon > 0$ 就存在 $N$ 对 $n \ge N$ 都有 $c_{n+1}/c_n \ge \beta - \epsilon$,所以对 $k \ge 1$ 有 $c_{N+k} \ge c_N (\beta - \epsilon)^k$,
两边同时取 $N+k$ 次根得到 


\[ \sqrt[N+k]{c_{N+k}} \ge \sqrt[N+k]{c_{N}} {(\beta-\epsilon)}^{k/(N+k)} \]

两边对 $k$ 取下极限得到

\[ \varliminf_{n \to \infty}c_n^{1/n} \ge \beta - \epsilon \]
\section{级数重排}


\subsection{非负项级数重排保持收敛}

非负项级数经过重排后可以保持收敛,而且和原来的收敛保持一致。

这个结论看上去不平凡,但证明却相当容易,因为非负项级数可以看作单调的数列,只要证明有界就可以证明收敛。证明有界很容易:

\[ \sum_{n=1}^{m}a_{f(n)} \le \sum_{n=1}^{M}a_{n}, \; M = \max \{f^{-1}(n) \vert 1 \le n \le m \} \]

证明收敛到相同的值也很容易。

假设 $ L - S_n  \le \epsilon $ 对 $n \ge N$ 恒成立。我们取 $M = \max \{f^{-1}(n) \vert 1 \le n \le N \} $ 
于是必然有 $ \forall n \ge M, \sum_{n=1}^{N} \le \sum_{n=1}^{M}a_{f(n)} \le L $

\subsection{绝对收敛的级数重排后保持收敛}
这个结论比刚才的使用范围更广,因为绝对收敛的级数可能包含负项。首先重排后,我们可以用证明非负项级数收敛的方法证明收敛。注意到绝对收敛蕴涵了收敛。\\
证明重排后收敛到相同的值需要一定的技巧。

注意我们取 $M = \{ f^{-1}(n) \vert 1 \le n \le N \} $ 的时候 $ \{ f(m) \vert 1 \le m \le M \}$ 可能包含大于 $N$ 的数值,但好就好在这一点大于 $N$,
因为 $N$ 很大的时候,这部分的值就会很小。

\begin{align*}
\lvert \sum_{k=1}^{M}a_{f(k)} - L \rvert & =  \lvert \sum_{n=1}^{N}a_{n} + \sum_{k=P}^{Q}a_{k} - L \rvert \\
& \le \lvert \sum_{n=1}^{N}a_n - L \rvert + \sum_{k=N+1}^{M}\lvert a_k \rvert \\
& \le 2\epsilon
\end{align*}

上面的 $N$ 取的是 $\sum_{n=1}^{\infty}a_n$ 极限和 $a_n$ 绝对收敛对应的柯西列,而且这个小于号不取决于 $M$ 的增加。

\subsection{富比尼定理}

假设有 $f: \N \times \N \to \R$ 并且 $\sum\lvert f(n,m) \rvert$ 有界,那么有

\begin{align*}
\sum_{(n,m) \in \N \times \N}f(n,m) & = \sum_{n=0}^{\infty}\sum_{m=0}^{\infty}f(n,m) \\
&= \sum_{(m,n) \in \N \times \N}f(n,m) \\
&= \sum_{m=0}^{\infty}\sum_{n=0}^{\infty}f(n,m)
\end{align*}

我们只需要证明如下等式成立,其他同理

\[
\sum_{(n,m) \in \N \times \N}f(n,m) = \sum_{n=0}^{\infty}\sum_{m=0}^{\infty}f(n,m) 
\]

为了证明该等式成立,我们令 $f(n,m) = f_+(n,m) + f_-(n,m)$ 其中

\begin{align*}
2f_+(n,m) &= f(n,m) + \lvert f(n,m)\rvert \\
2f_-(n,m) &= f(n,m) - \lvert f(n,m)\rvert \\
\end{align*}

我们先证明


\[
\sum_{(n,m) \in \N \times \N}f_+(n,m) = \sum_{n=0}^{\infty}\sum_{m=0}^{\infty}f_+(n,m) 
\]

根据已知条件很容易证明如果 $h: \N \to \N \times \N$ 而且 $h$ 是双射,那么必然有 $(f_+ \circ h)(n)$ 绝对收敛。 假设有

\[
\sum_{(n,m) \in \N \times \N}f_+(n,m) = \sum_{k=0}^{\infty} (f_+ \circ h) (k) = \alpha
\]

令

\[
S(N,M) = \sum_{n=0}^{N}\sum_{m=0}^{M}f_+(n,m)
\]

我们先固定 $N$ ,再对 $h(n)$ 取有限项覆盖 $\{ (n,m) \wvert 0 \le n \le N,\, 0 \le m \le M\}$ 得到对任意 $M \ge 0$ 都有 $S(N,M) \le \alpha$,根据单调有界原理,于是对任意 $N$ 都有

\[
H(N) = \lim_{M \to \infty}S(N,M) \le \alpha
\]

逐项比较后注意到 $H(N+1) \ge H(N)$ 根据单调有界原理,于是我们又有

\[
\lim_{N \to \infty}H(N) \le \alpha
\]

所以有

\[
\lim_{N \to \infty}H(N) = \sum_{n=0}^{\infty}\sum_{m=0}^{\infty}f_+(n,m) \le \alpha
\]

我们继续证明 

\[
\sum_{n=0}^{\infty}\sum_{m=0}^{\infty}f_+(n,m) \ge \alpha
\]

我们取 $h(k)$ 的有限项得到

\[
0\le \alpha - \sum_{k=0}^{M} (f_+ \circ h) (k) \le \epsilon
\]

然后取 $N = \max \{ n \wvert (n,m) = h(k),\, 0 \le k \le M\}$,这样就得到了 

\begin{align*}
    & \sum_{k=0}^{M} (f_+ \circ h) (k)  \le H(N) \le \lim_{N \to \infty}H(N) \\
    & 0  \le \alpha - \lim_{N \to \infty}H(N) \le \alpha - \sum_{k=0}^{M} (f_+ \circ h) (k) \le \epsilon \\
    & \alpha -\epsilon \le \sum_{n=0}^{\infty}\sum_{m=0}^{\infty}f_+(n,m)
\end{align*}

因为 $\epsilon$ 可以任意小,所以有

\[
\sum_{n=0}^{\infty}\sum_{m=0}^{\infty}f_+(n,m) \ge \alpha
\]

最终我们证明了


\[
\sum_{(n,m) \in \N \times \N}f_+(n,m) = \sum_{n=0}^{\infty}\sum_{m=0}^{\infty}f_+(n,m) 
\]

同理我们可以证明


\[
\sum_{(n,m) \in \N \times \N}f_-(n,m) = \sum_{n=0}^{\infty}\sum_{m=0}^{\infty}f_-(n,m) 
\]

根据序列的极限运算法则,我们有

\begin{align*}
\sum_{(n,m) \in \N \times \N}f(n,m) &= \sum_{k=0}^{\infty} (f \circ h) (k) \\
&= \sum_{k=0}^{\infty} (f_+ \circ h) (k) + (f_- \circ h) (k) \\
&= \sum_{k=0}^{\infty} (f_+ \circ h) (k) + \sum_{k=0}^{\infty} (f_- \circ h) (k) \\
&= \sum_{(n,m) \in \N \times \N}f_+(n,m) + \sum_{(n,m) \in \N \times \N}f_-(n,m)
\end{align*}

并且


\begin{align*}
\sum_{n=0}^{\infty}\sum_{m=0}^{\infty}f(n,m) &= \sum_{n=0}^{\infty}\sum_{m=0}^{\infty}(f_+(n,m) + f_-(n,m)) \\
&= \sum_{n=0}^{\infty}(\sum_{m=0}^{\infty}(f_+(n,m) + \sum_{m=0}^{\infty}(f_-(n,m)) \\
&= \sum_{n=0}^{\infty}\sum_{m=0}^{\infty}f_+(n,m) + \sum_{n=0}^{\infty}\sum_{m=0}^{\infty}f_-(n,m)
\end{align*}

所以有 


\[
\sum_{(n,m) \in \N \times \N}f(n,m) = \sum_{n=0}^{\infty}\sum_{m=0}^{\infty}f(n,m) 
\]

我们可以构造一个双射 $g: \N \times \N \to \N \times \N,\, g(n,m) = (m,n)$ 证明富比尼定理的剩余部分。


\chapter{集合}

\section{可数集判别法}

\subsection{$\N \times \N$ 至多可数}

令 $f: \N \times \N \to \N$

\[
f(m,n) = \frac{(m+n)(m+n+1)}{2} + n
\]

我们先证明 $f$ 是单射,先不妨设 $m_1+ n_1 < m_2 + n2$ ,则有 $ m_1+ n_1 + 1 \le m_2 + n2$

\begin{align*}
f(m_2, n_2) - f(m_1, n_1) & \ge \frac{(m_1+n_1+1)(m_1+n_1 + 2)}{2} + n_2 - \frac{(m_1+n_1)(m_1+n_1+1)}{2} - n_1 \\
    & \ge m_1+n_1+1 + n_2 - n_1 \ge 1
\end{align*}

再考虑$m_1 + n_1 = m_2 + n_2$ 的情况,不妨令 $m_1 < m_2, \: n_1 > n_2$ ,则有

\[
f(m_2,n_2) - f(m_1, n_1) = n_2 - n_1 < 0
\]

接着我们证明 $f$ 是满射,首先 $f(0, 0) = 0$ ,假设 $f(m,n) = k$,我们找 $f(m',n') = k+1$,如果 $m > 0$ ,则有 

\[
f(m-1,n+1) = k +1
\]

如果 $m = 0$ ,则有

\[
f(n+1, 0) = \frac{(n+1)(n+2)}{2} = \frac{n(n+1)}{2} + n + 1 = k + 1
\]

所以 $f$ 是单射也是满射,所以 $\N \times \N$ 是可数集。


\subsection{有限个至多可数集合的笛卡尔积}

我们只要证明两个至多可数集合的笛卡尔积是至多可数的即可,然后运用数学归纳法。证明两个至多可数集合的笛卡尔积至多可数,和证明 $\N \times \N$ 可数类似,
运用对角线法则即可。


\subsection{可数个至多可数集合的并集}

我们只要把 $A_i$ 按行排列,然后运用对角线法则。

\section{像集和原像}

\subsection{像集对并运算保持同态}

\[
f(\bigcup_{i \in I} X_i) = \bigcup_{i \in I} f(X_i)
\]


证明: 左边是右边子集,$y$ 在左边的集合内, 则有 $f(x) = y$ 且至少存在一个 $X_i$ 满足 $x \in X_i$,所以我们找到了 $X_i$ 满足 $y \in X_i$。也就是

\begin{align*}
    & x \in \bigcup_{i \in I}X_i,\, x \in X_k \\
    & y \in f(X_k) \subseteq \bigcup_{i \in I}f(X_i)
\end{align*}

继续证明右边是左边子集,如果 $y$ 在右边集合内,则至少有一个 $x \in X_i$,满足 $f(x) = y$,所以我们找到了 $X_i$ 满足 $x \in X_i,\, f(x) = y$,所以 
$y$ 位于左边的集合呢。

\begin{align*}
    & x \in \bigcup_{i \in I}X_i,\, x \in X_k \\
    & X_k \subseteq \bigcup_{i \in I}X_i ,\, y \in f(X_k) \subseteq f(\bigcup_{i \in I}X_i)
\end{align*}

\subsection{像集对交运算不一定保持同态}

想像这样一个函数,它存在两个不同的定义域上的元素,有相同的值,即 $f(x_1) = f(x_2) = y$

显然 $X = \{ x_1 \}$ 和 $Y = \{ x_2 \}$ 有 $X \cap Y = \emptyset$,$f(X) \cap f(Y) = \{ y \} $

如果 $f$ 是单射,那么有 

\[
f(\bigcap_{i \in I}X_i) = \bigcap_{i \in I}f(X_i)
\]

证明: 左边是右边子集,$y$ 在左边的集合内,所以存在 $x \in X_i, \forall i \in I$ 且 $f(x) = y$,所以我们找到了一个 $x$ 满足 $\forall i \in I,\, f(x) \in f(X_i)$,所以左边是右边的子集。

\begin{align*}
    & x \in \bigcap_{i \in I}X_i,\, f(x) = y \\
    & \forall i \in I,\, y \in f(X_i)
\end{align*}

继续证明右边是左边的子集,$y$ 在右边的集合内,那么对每一个 $X_i$ 都有 $x_i \in X_i,\, f(x_i) = y$。因为 $f$ 是单射,所以必然有 $\forall i,j \in I,\, x_i = x_j = x$,所以我们找到了
$x$ 满足 $\forall i \in I,\, x \in X_i,\, f(x) = y$,所以 $y$ 在左边的集合内。


\begin{align*}
    & \forall i \in I,\, x_i \in X_i,\, f(x_i) = y \\
    & \forall i \in I,\, x = x_i,\, x \in \bigcap_{i \in I}X_i
\end{align*}

\subsection{原像对并运算保持同态}

\[
f^{-1}(\bigcup_{i \in I} Y_i) = \bigcup_{i \in I} f^{-1}(Y_i)
\]


证明: 左边是右边子集,$x$ 在左边的集合内, 则有 $f(x) = y$ 且至少存在一个 $Y_i$ 满足 $y \in Y_i$,所以我们找到了 $Y_i$ 满足 $x \in f^{-1}(Y_i)$。

继续证明右边是左边子集,如果 $x$ 在右边集合内,则至少有一个 $y \in Y_i$,满足 $f(x) = y$,所以我们找到了 $Y_i$ 满足 $f(x) \in Y_i$,所以 
$y$ 位于左边的集合。


\subsection{原像对交运算保持同态}


\[
f^{-1}(\bigcap_{i \in I}Y_i) = \bigcap_{i \in I}f^{-1}(Y_i)
\]

证明: 左边是右边子集,$x$ 在左边的集合内,所以存在 $y \in Y_i, \forall i \in I$ 且 $f(x) = y$,所以左边是右边的子集。

继续证明右边是左边的子集,$x$ 在右边集合内,则每个 $Y_i$ 都有 $f(x) = y_i$,显然 $y_i$ 之间相等,令 $y = y_i$,那么有

\[
f(x) = y \in \bigcap_{i \in I}Y_i
\]

\subsection{原像和补集运算}

假设 $f: \Omega \to Y, \, A \subseteq Y$

\[
f^{-1}(Y \setminus A) = \Omega \setminus (f^{-1}(A))
\]

左边到右边,假设 $x \in f^{-1}(Y \setminus A)$ 必然有 $f(x) = y,\, y \notin A$,所以左边是右边的子集。

右边到左边,假设 $x \in \Omega, x \notin f^{-1}(A) $则有 $f(x) \notin A$,也就是 $f(x) \in Y \setminus A$,所以得到 $x \in f^{-1}(Y \setminus A)$

\section{上下极限}


\subsection{序列的上下极限}

假设 $a_n \in \R$ 那么 $a_n$ 的上确界定义为 $\sup_{n \in \N} a_n$,$a_n$ 的上极限定义为

\[
\overline{\lim_{n \to \infty}}a_n = \lim_{n \to \infty} \sup_{k \ge n} \{ a_k \}
\]

同理有下确界 $\inf_{n \in \N} \{ a_n \}$ 以及下极限

\[
\lim_{\overline{n \to \infty}}a_n = \lim_{n \to \infty} \inf_{k \ge n} \{ a_k  \}
\]

因为上面的两个序列分别都是单调减和单调增的,所以也可以不使用序列极限,用确界的方式定义上下极限。

\[
\overline{\lim_{n \to \infty}}a_n = \inf_{n \in N} \sup_{k \ge n} \{ a_k \}
\]


\subsection{集合的上下极限}

集合的上极限是这样定义的

\[
\overline{\lim_{n \to \infty}}A_n  = \bigcap_{n=0}^{\infty}\bigcup_{i \ge n}A_i
\]

这个定义看次很难理解,其实就是 $x$ 属于 $A_n$ 的上极限,当且仅当对任意 $N \in 0$ 都存在 $n \ge N$ 满足 $x \in A_n$


集合的下极限是这样定义的

\[
\varliminf_{n \to \infty}A_n  = \bigcup_{n=0}^{\infty}\bigcap_{i \ge n}A_i
\]

其实就是 $x$ 属于 $A_n$ 的下极限,当且仅当存在 $N  \ge 0$ 满足对所有的 $n \ge N$,$x \in A_n$

\subsection{上下极限}

序列的上下极限和集合的上下极限有很大的关系。例如序列的上极限和其他实数比较时,有如下性质

\[
\overline{\lim_{n \to \infty}} a_n \le a ,\:\text{iff}\: a \in \bigcup_{n=1}^{\infty} \bigcap_{k \ge n} \{ a_i \wvert i \ge k, a_i \le a \} 
\]


\[
\overline{\lim_{n \to \infty}} a_n \ge a ,\:\text{iff}\: a \in \bigcap_{n=1}^{\infty} \bigcup_{k \ge n} \{ a_i \wvert i \ge k, a_i \ge a \} 
\]

同理,下极限又有类似的性质。

\[
\varliminf_{n \to \infty} a_n \le a ,\:\text{iff}\: a \in \bigcap_{n=1}^{\infty} \bigcup_{k \ge n} \{ a_i \wvert i \ge k, a_i \le a \} 
\]


\[
\varliminf_{n \to \infty} a_n \ge a ,\:\text{iff}\: a \in \bigcup_{n=1}^{\infty} \bigcap_{k \ge n} \{ a_i \wvert i \ge k, a_i \ge a \} 
\]

\chapter{微分学}

\section{基本定义}

\subsection{可微的定义}

函数 $f: X \to \R$ 在极限点 $x_0$ 处可微定义为

\[
    \lim_{x \to x_0, x \in X \setminus \{x_0\} }\frac{f(x) -f(x_0)}{x -x_0} = L
\]

注意这里对 $x_0$ 的要求是 $x_0 \in X$ 而且 $x_0$ 是 $X$ 的极限点,比函数极限的定义更严格。

\subsection{牛顿近似逼近}

可微的定义和以下命题等价

$\forall \epsilon > 0, \exists \delta >0 $ 对任意 $0 < \lvert x -x_0\rvert \le \delta $ 有

\[
    \lvert f(x) - f(x_0) - L(x-x_0) \rvert \le \epsilon \lvert x -x_0 \rvert
\]


\subsection{可微必然连续}

利用牛顿近似逼近很容易证明

$\lvert f(x) -f(x_0)\rvert = \lvert f(x) -f(x_0) - L(x-x_0) + L(x-x_0)\rvert $

\subsection{可微函数}

$f: X \to \R$ 可微定义为 $f$ 在 $X$ 的每一个极限点上可微。

\subsection{可微函数一定是连续函数}

证明很容易,假设 $x_0$ 是 $X$ 的极限点,因为 $f$ 在 $x_0$ 处可微,所以必然在 $x_0$ 处连续。

\subsection{微分学}

\begin{enumerate}
    \item $f: X \to \R,\, f(x) = c$ 则 $f'(x) = 0$

    根据定义可以证明

    \item $f: X \to \R,\, f(x) = x $ 则 $f'(x) = 1$

    根据定义可以证明

    \item $f,\,g: X \to R\, $ 且 $f,\,g$ 在 $x_0$ 处可微,则 $ (f+g)'(x_0) = f'(x_0) + g'(x_0)$

    根据极限运算法则可以证明

    \item $f,\,g: X \to R\, $ 且 $f,\,g$ 在 $x_0$ 处可微,则 $ (fg)'(x_0) = f'(x_0)g(x_0) + f(x_0)g'(x_0)$

    同样利用极限运算法则

\begin{align*}
    \lim_{x \to x_0}\frac{f(x)g(x) - f(x_0)g(x_0)}{x-x_0} &= \lim_{x \to x_0}\frac{f(x)g(x) - f(x_0)g(x)}{x-x_0} + \lim_{x \to x_0}\frac{f(x_0)g(x) - f(x_0)g(x_0)}{x-x_0} \\
    &= f'(x_0)g(x_0) + f(x_0)g'(x_0)
\end{align*}


    \item $f: X \to R$ 且 $f'(x_0) = L$ 则 $cf$ 在 $x_0$ 处可微,且 $(cf)'(x_0) = cL$
    利用极限运算法则

    \item $(f-g)' = f' - g'$
    同样用极限运算法则

    \item $f: X \to R$ 且 $f(x) \ne 0,\, f'(x_0) = L$ 则 $g=\frac{1}{f}$ 在 $x_0$ 处可微,而且 $g'(x_0) = -\frac{f'(x_0)}{f(x_0)^2}$

    利用极限运算法则

\begin{align*}
    \lim_{x \to x_0}\frac{1/f(x) - 1/f(x_0)}{x-x_0} = - \lim_{x \to x_0}\frac{f(x) - f(x_0)}{f(x)f(x_0)(x-x_0)} = - \frac{f'(x_0)}{f(x_0)^2}
\end{align*}

    \item 除法法则,这里要求 $g(x) \ne 0$ 而且 $f,\,g$ 在 $x_0$ 处可微

\[
    (\frac{f}{g})'(x_0) = \frac{f'(x_0)g(x_0) - f(x_0)g'(x_0)}{g(x_0)^2}
\]
\end{enumerate}

\subsection{链式法则}

$f: X \to Y,\, f(x_0) = y_0,\, f'(x_0) = L_1,\, g: Y \to \R,\, g'(y_0) = L_2$ 注意这里要求 $f(x_0)$ 是 $Y$ 的极限点
那么

\[
(f \circ g)'(x_0) = g'(y_0)f'(x_0)
\]

证明要用到极限运算法则,这里因为 $f$ 在 $x=x_0$ 处连续,所以我们作代换

\begin{align*}
    \lim_{x \to x_0}\frac{g \circ f(x) - g \circ f(x_0)}{x-x_0} &= \lim_{x \to x_0} \frac{g \circ f(x) - g \circ f(x_0)}{f(x) - f(x_0)} \lim_{x \to x_0}\frac{f(x) - f(x_0)}{x-x_0} \\
    &= f'(x_0)\lim_{y \to y_0} \frac{g(y) - g(y_0)}{y - y_0} = g'(y_0)f'(x_0)
\end{align*}

\section{极值}

\subsection{极值点定义}

$f: X \to R,\, x_0 \in X$ 若存在 $\delta > 0$ 使得 $f: X \cap (x_0 -\delta, x_0 + \delta)$ 在 $x_0$ 处取得最大值,那么 $x_0$ 为极大值点。
同理定义极小值点。

\subsection{极值点是驻点}

$f: (a,b) \to \R,\, a<b,\, x_0 \in (a,b),\, f'(x_0) = L$ 且 $x_0$ 是极值点那么 $L=0$ 

证明: 若 $x_0$ 为极小值点

\begin{align*}
L &= \lim_{n \to \infty} \frac{f(x_0+1/n) - f(x_0)}{1/n} \ge 0 \\
  &= \lim_{n \to \infty} \frac{f(x_0-1/n) - f(x_0)}{-1/n} \le 0
\end{align*}

所以 $L = 0$


\subsection{罗尔定理}

$a < b,\, f: [a,b] \to \R $ 连续并且在 $(a,b)$ 上可微。若 $f(a) = f(b)$ 那么存在 $x_0 \in (a,b)$ 满足 $f'(x_0) = 0$

因为闭区间上的函数一定能取得最小值 $m$ 和最大值 $M$,若 $m = M$ 那么 $f(x)$ 是一个常数函数,所以 $\forall x \in [a,b],\, f'(x) = 0$。若 $m < M$
那么 $m$ 和 $M$ 一定有一个不等于 $f(a),\, f(b)$,不妨设 $f(a) \ne m$。那么必然有 $c \in (a,b)$ 满足 $f(c) = m$,根据极值点的性质得到 $f'(c) = 0$。

\subsection{中值定理}

$a < b,\, f: [a,b] \to \R $ 连续并且在 $(a,b)$ 上可微,那么存在 $x_0 \in (a,b)$ 满足 

\[
f'(x_0) = \frac{f(b)- f(a)}{b-a}
\]

证明: 构造函数 $g: [a,b] \to \R$

\[
g(x) = f(x) - f(a) + \frac{f(b)-f(a)}{b-a}(x-a)
\]

容易验证 $g(b) = g(a) = 0$ 而且 $g(x)$ 在 $(a,b)$ 上可微,根据罗尔定理必然有 $c \in (a,b),\, g'(c) = 0$ 也就是

\[
f'(c) - \frac{f(b)-f(a)}{b-a} = 0
\]

\subsection{柯西中值定理}

$a < b,\, f,\,g: [a,b] \to \R,\, g(b) \ne g(a),\, g'(x) \ne 0 $ 连续并且在 $(a,b)$ 上可微,那么存在 $x_0 \in (a,b)$ 满足 

\[
\frac{f'(x_0)}{g'(x_0)} = \frac{f(b)- f(a)}{g(b)-g(a)}
\]

证明: 构造函数 $h: [a,b]\to \R$

\[
h(x) = f(x) - f(a) - \frac{f(b) - f(a)}{g(b) - g(a)}(g(x) - g(a))
\]

容易验证 $h(a) = h(b) = 0$ 且 $h(x)$ 在 $(a,b)$ 上可微,根据罗尔定理有 $c \in (a,b),\, h'(c) = 0$ 也就是

\[
f'(c) - \frac{f(b) - f(a)}{g(b) -  g(a)}g'(c) = 0
\]

\section{单调函数的微分}

\subsection{单调函数的微分 I}

$f: X \to \R$ 是单调增函数,并且 $x_0 \in X$ 是极限点,若 $f$ 在 $x_0$ 处可微,那么必然有 $f(x_0) \ge 0$

证明,根据极限运算法则有

\[
f'(x_0) = \lim_{n \to \infty}\frac{f(x_0 + 1/n) - f(x_0)}{1/n} \ge 0
\]

同理,
$f: X \to \R$ 是单调减函数,并且 $x_0 \in X$ 是极限点,若 $f$ 在 $x_0$ 处可微,那么必然有 $f(x_0) \le 0$

\subsection{单调函数的微分 II}

$a,b \, f: [a,b] \to \R$ 可微,若 $\forall x \in [a,b],\, f'(x) > 0$ 那么 $f$ 是严格单调增函数。

用中值定理很容易证明

\section{反函数的微分}

\subsection{反函数的微分 I}

$f: X \to Y,\, f(x_0) = y_0$ 是可逆函数,并且 $x_0$ 和 $y_0$ 各自是 $X$ 和 $Y$ 的极限点。令 $g(x) = f^{-1}(x)$,若
$f$ 在 $x_0$ 处可微, $g$ 在 $y_0$ 处连续,并且 $f'(x_0) = L \ne 0$,那么 $g$ 在 $y_0$ 处可微,而且 $g'(y_0) = \frac{1}{f'(x_0)}$

这个证明只需要用到函数极限运算法则

\[
\lim_{y \to y_0}\frac{g(y) - g(y_0)}{y -y_0} = \lim_{x \to x_0}\frac{x-x_0}{f(x) - f(x_0)} = 1/L
\]

\section{L'h\^{o}pital 法则}

\subsection{L'h\^{o}pital I}

若 $f,\,g : X \to \R$ 并且 $x_0 \in X$ 是极限点,假设 $f(x_0) = g(x_0) = 0$ 并且 $f,\,g$ 在 $x_0$ 处可微,并且 $g'(x_0) \ne 0 $,那么存在 $\delta > 0$ 满足
$X_{\delta} = \{ x \in X \wvert 0 < \lvert  x - x_0\rvert <\delta \},\, \forall x  \in X_{\delta},\, g(x) \ne 0$ 并且有

\[
\lim_{x \to x_0, x \in X_{\delta}} \frac{f(x)}{g(x)} = \frac{f'(x_0)}{g'(x_0)}
\]

我们先证明 $X_{\delta}$ 存在,不妨设 $g'(x_0) = L > 0$,根据牛顿近似逼近,取 $\epsilon = L/2$ 有 $\delta$ 满足 $\forall x \in X, 0 < \lvert x -x_0\rvert < \delta$ 有
$ \lvert f(x) - f(x_0) - L(x-x_0)\rvert \le L/2 \lvert x -x_0 \rvert$
于是有 

\[
0 < L/2 \lvert x -x_0 \rvert\le \lvert f(x) \rvert
\]

所以 $X_{\delta}$ 存在,根据极限运算法则有

\[
\lim_{x \to x_0, x \in X_{\delta}} \frac{f(x)}{g(x)} = \frac{f(x)/(x-x_0)}{g(x)/(x-x_0)} = \frac{f'(x_0)}{g'(x_0)}
\]


\subsection{L'h\^{o}pital II}

若 $f,\,g : [a,b] \to \R$ 在 $(a,b]$  可微在 $[a,b]$ 连续 并且 $f(a) = g(a) = 0,\, \forall x \in (a,b),\, g'(x) \ne 0$
若有

\[
\lim_{x \to a,\, x>a}\frac{f'(x)}{g'(x)} =L ,\, L \in \R^*
\]

那么

\[
\lim_{x \to a,\, x>a}\frac{f(x)}{g(x)} =L
\]

证明如下

我们要先证明 $\forall x \in (a,b),\, g(x) \ne 0$,假设 $g(c) = 0$,根据罗尔定理得到 $g'(\alpha) = 0$ 和命题中条件矛盾。有了 $g(x) \ne 0$ 后,
我们可以使用柯西中值定理,我们先证明 $- \infty < L < \infty$ 的情况,套用$f'(x)/g'(x)$ 的 $\delta$ 和 $\epsilon$ 得到

\[
\lvert \frac{f(x)}{g(x)} - L \rvert = \lvert \frac{f'(\alpha_x)}{g'(\alpha_x)} -L \rvert \le \epsilon
\]

若 $L = \infty$,套用 $\forall x \in (a,\delta),\, f'(x)/g'(x) \ge M$ 得到

\[
\frac{f(x)}{g(x)} = \frac{f'(\alpha_x)}{g'(\alpha_x)} \ge M
\]

若 $L = -\infty$,同理。

\subsection{L'h\^{o}pital III}

已知 $a < b,\, f : (a,b) \to \R,\, g : (a,b) \to \R $ 都可微,并且有 

\[
\lim_{x \to a,\, x > a}  f(x) = \lim_{x \to a,\, x > a}  g(x) = \pm \infty
\]


\[
\lim_{x \to a,\, x >a} \frac{f'(x)}{g'(x)} = L ,\, L \in \R^*
\]

那么有

\[
\lim_{x \to a,\, x >a} \frac{f(x)}{g(x)} = L
\]

证明: 我们先证明 $-\infty < L < \infty$ 的情况

根据已知条件存在 $\delta$ 满足 $\forall x \in (a, a+\delta)$

\[
    \lvert \frac{f'(x)}{g'(x)} - L \rvert \le \epsilon 
\]

令 $c = a + \delta$,取任意 $a < x < a + \delta$,根据柯西中值定理得到,其中 $x < \alpha_x < c$

\[
    \frac{f(x) - f(c)}{g(x) - g(c)} = \frac{f'(\alpha_x)}{g'(\alpha_x)}
\]

注意这里我们可以调整 $\delta$ 满足 $\lvert g(x) - g(c)\rvert \ge \lvert g(x)\rvert - \lvert g(c)\rvert >0$

因为 $g(a^+) = \pm \infty$,所以我们可以继续调整 $\delta$ 满足 $\lvert g(x) \rvert > 0$ 以及 $1- g(c)/g(x) > 0$ 所以得到


\begin{align*}
    & \frac{f(x) - f(c)}{g(x) - g(c)} = \frac{f(x)/g(x)- f(c)/g(x)}{1- g(c)/g(x)} = \frac{f'(\alpha_x)}{g'(\alpha_x)} \\
    & \frac{f'(\alpha_x)}{g'(\alpha_x)} - \frac{f(x)}{g(x)} = \frac{f'(\alpha_x)g(c)}{g'(\alpha_x) g(x)} -f(c)/g(x) \\
\end{align*}

令

\[
r(x) = \frac{f'(\alpha_x)g(c)}{g'(\alpha_x) g(x)} -f(c)/g(x)
\]

显然有

\[
\lim_{x \to a,\, x>a}r(x) = 0
\]

所以我们可以再继续调整 $\delta$ 满足

\begin{align*}
    & \lvert \frac{f'(\alpha_x)}{g'(\alpha_x)} - \frac{f(x)}{g(x)}  \rvert \le \epsilon \\
    & \lvert \frac{f(x)}{g(x)}  - L \rvert = \lvert \frac{f(x)}{g(x)}  - \frac{f'(\alpha_x)}{g'(\alpha_x)} + \frac{f'(\alpha_x)}{g'(\alpha_x)} - L \rvert \le 2\epsilon \\
\end{align*}

因为 $\epsilon$ 是任意的,所以证明完毕。

如果 $L = \infty$,同理我们可以取 $\delta$ 满足 

\begin{align*}
    & \lvert g(x) - g(c)\rvert \ge \lvert g(x)\rvert - \lvert g(c)\rvert >0 \\
    & \lvert g(x)\rvert > 0 \\
    & 1- \frac{g(c)}{g(x)} > 0 \\
    & \frac{f'(\alpha_x)}{g'(\alpha_x)} \ge M 
\end{align*}

利用柯西中值定理得到

\[
     \frac{f'(\alpha_x)}{g'(\alpha_x)}(1-\frac{g(c)}{g(x)}) + \frac{f(c)}{g(x)}  =  \frac{f(x)}{g(x)}
\]

利用极限运算法则我们可以调整 $\delta$ 满足 $f(x)/g(x) \ge M/2 - 1$,因为 $M$ 可以任意大,所以有

\[
\lim_{x \to a,\, x >a} \frac{f(x)}{g(x)} = \infty
\]

当 $L = -\infty$ 同理

\subsection{L'h\^{o}pital IV}

$\exists a \in \R,\,  a > 0,\, f,g: (a, +\infty) \to \R$ 可微并且 $\forall x > a,\, g'(x) \ne 0$。若

\begin{align*}
& \lim_{x \to \infty}f(x) = 0 \\
& \lim_{x \to \infty}g(x) = 0 \\
\end{align*}

或者


\begin{align*}
& \lim_{x \to \infty}f(x) = \pm \infty \\
& \lim_{x \to \infty}g(x) = \pm \infty \\
\end{align*}

并且

\[
 \lim_{x \to \infty}\frac{f'(x)}{g'(x)} =L,\, L \in \R^*
\]

那么有

\[
\lim_{x \to \infty} \frac{f(x)}{g(x)} = L
\]

证明如下

根据已知条件构造函数 $h_1: (0,1/a) \to \R,\, h_1(x) = f(1/x)$ 和 $h_2: (0,1/a) \to \R,\, h_2(x) = g(1/x)$ 

我们先证明 $\lim_{x \to \infty}f(x) = \lim_{x \to \infty}g(x) = 0$ 的情况, 于是我们得到

\begin{align*}
& \lim_{x \to 0, x >0} h_1(x) = 0  \\
& \lim_{x \to 0, x >0} h_2(x) = 0  
\end{align*}

如果我们定义 $h_1(0) = 0$ 和 $h_2(0) = 0$,那么 $h_1$ 和 $h_2$ 就在 $x=0$ 处连续。于是对 $h_1$ 和 $h_2$ 使用 L'h\^{o}pital rule 得到

\[
\lim_{x \to 0,\, x > 0}\frac{h_1(x)}{h_2(x)} = \lim_{x \to 0,\, x >0}\frac{f'(1/x)}{g'(1/x)} = \lim_{x \to \infty}\frac{f(x)}{g(x)} = L
\]

同理如果

\begin{align*}
& \lim_{x \to 0, x >0} h_1(x) = \pm \infty  \\
& \lim_{x \to 0, x >0} h_2(x) = \pm \infty
\end{align*}

也有

\[
\lim_{x \to 0,\, x > 0}\frac{h_1(x)}{h_2(x)} = \lim_{x \to 0,\, x >0}\frac{f'(1/x)}{g'(1/x)}= \lim_{x \to \infty}\frac{f(x)}{g(x)} = L
\]

\section{习题}

\subsection{$f: \R \to \R,\, f(x) = x^n,\, n \in \N$ 的导函数}

$f'(x) = nx^{n-1}$ 用数学归纳法,以及乘法法则很容易证明

\subsection{$f: (-\infty,0)\cup (0, \infty) \to \R,\, f(x) = x^{n},\, n \in \Z$ 的导函数}

当 $n \ge 0$ 时候可以用前面的结论,若 $n < 0$ 有 $f(x) = 1/x^{-n}$,利用除法法则有 $f'(x) = nx^{n-1}$

\subsection{$f: (0, \infty) \to (0, \infty),\, f(x) = x^{1/n},\, n \in \N^+$ 的导函数}

首先 $g: (0, \infty) \to (0, \infty),\, g(x) = x^n$ 是严格单调而且可微的,所以 $f = g^{-1}$ 一定连续,因此 
$f = g^{-1}$ 可微有 

\[
f'(y = x^n) = \frac{1}{nx^{n-1}} = (ny^{\frac{n-1}{n}})^{-1} = \frac{1}{n}y^{\frac{1}{n} - 1}
\]

\subsection{$f: (0, \infty) \to (0, \infty),\, f(x) = x^{q},\, q \in \Q$ 的导函数}

令 $q = n/m ,\, n,m \in \Z,\, m \ne 0$ 于是得到

$x^q = x^{\frac{n}{m}} = (x^n)^{\frac{1}{m}}$,根据复合函数的微分法则得到

\begin{align*}
    f'(x) = \frac{1}{m}(x^n)^{\frac{1}{m} - 1}n x^{n-1} = \frac{n}{m}x^{\frac{n}{m}-1} = qx^{q-1}
\end{align*}


\subsection{$f: (0, \infty) \to (0, \infty),\, f(x) = x^{\alpha},\, \alpha \in \R$ 的导函数}

我们构造 $f_n = x^{q_n}$ 并且在任意闭区间上一致收敛到 $x^\alpha$,根据一致收敛的性质,注意到 $f'_n = q_nx^{q_n -1}$ 也一致收敛到 $\alpha x^{\alpha - 1}$,
并且 $f_n'(1)$ 极限存在,
所以有 $f'(x) = \alpha x^{\alpha - 1}$,。
\chapter{黎曼积分}

\section{划分}

\subsection{有界区间}

回顾下有界区间的定义,$(a,b) \,\, (a,b] \,\, [a,b) \,\, [a,b]$ 都是区间。注意这里不要求 $a$ 和 $b$ 不相等,也没有要求 $a \le b$,所以空集也是区间,一个只包含一个实数的集合也是区间。

\subsection{连通集}

连通集 $A$ 定义为 $x, y \in A$ 则 $\forall x \le t \le y, \, t \in A$ 

\subsection{实数集上的有界连通集和区间等价}

下面证明实数集上的有界连通集是区间,我们分别取 $A$ 的上确界为 $b$,下确界为 $a$,下面证明 $(a,b) \subseteq A \subseteq [a,b]$。
假设存在 $c \in (a,b), c \notin A$,因为上确界和下确界都是附着点,
所以我们可以找到 $a' \in A $ 和 $b' \in A$ 满足 $a \le a' < c < b' \le b$,这就跟 $A$ 是连通集矛盾了。
右半部分易证明,所以 $A$ 是一个区间。

\subsection{有界区间的交集也是有界区间}

连通集合的交集依然是连通集,有界集合的交集依然是有界的。


\subsection{区间的长度}

对于空集,定义区间长度为 $0$,对于非空的区间 $(a,b) \subseteq I \subseteq [a,b]$ 定义 $I$ 的长度为 $b-a$,这里也包含了单点集。

\subsection{划分}

$\mathbf{P} = \{ J \vert J \subseteq I,\, J \in \mathbf{P} \}$ 是对有界区间 $I$ 的一个划分当且仅当 $I$ 中的每个元素 $x$ 恰好属于 $\mathbf{P}$ 中的一个有界区间 $J$。

\subsection{有限可加性}

我们上面之所以这样定义划分就是为了能够满足有限可加性,这对黎曼积分的定义有重要的意义。

\[
    \lvert I \rvert = \sum_{J \in \mathbf{P}} \lvert J \rvert
\]

下面给出证明,假设 $(a,b) \subseteq I \subseteq [a,b] $,并且 $\mathbf{P}$ 是对 $I$ 的一个划分。如果 $I$ 是空集,那么成立。如果 $I$ 不是空集,那么设 $\mathbf{P}$ 中元素的数量为 $n > 0$。令 $a_0=a, I_0 = I, \mathbf{P_0} = \mathbf{P}$ 我们讨论 $a_0$ 是否属于 $I_0$,
如果 $a_0 \in I_0$,那么必然有 $a_0 \in J_0 ,\, J_0 \in \mathbf{P_0}$ 令 $[a_0, b_0) \subseteq J_0 \subseteq [a_0, b_0]$ 同时有 $I_0 = \bigcup_{J \in \mathbf{P_0}} J $ 两边同时对 $J_0$ 作差集得到 $I_0 \setminus J_0 = \bigcup_{J \in \mathbf{P_0}, J \ne J_0}J$ 
注意到 $I_0 \setminus J_0$ 依然是一个区间,我们可以记它为 $I_1$ ,注意到此时 $J \in \mathbf{P_0}, J \ne J_0$ 组成了对 $I_1$ 的一个划分,我们记这个划分为 $\mathbf{P_1}$ 所以我们可以对 $I_1, \mathbf{P_1}$ 作重复的操作,同时 $I_1$ 的下确界一定是 $b_0$。同理如果 $a_0 \notin I_0$,那么 $\mathbf{P_0}$ 中一定存在一个 $J_0$ 满足 $J_0$ 的下确界等于 $a_0$,否则 $a_0$ 附近一定有点不会被覆盖。同样的可以做上面的操作得到 $I_1, \mathbf{P_1}$。
我们重复以上操作和得到了 $\{ J_0, J_1, .., J_{n-1} \} = \mathbf{P}$,注意到 $J_0$ 的上确界就是 $J_1$ 的下确界,以此类推。我们得到 $\lvert J_0 \rvert + \lvert J_1 \rvert  + .. + \lvert J_{n-1} \rvert = a_1 -a_0 + a_2 - a_1 + .. + a_n - a_{n-1} = a_n - a_0 = b - a$。
注意到 $I_{n-1} \setminus J_{n-1} = I_n = \emptyset$,所以 $J_{n-1}$ 的上确界一定是 $b$。

\subsection{更细划分}

假设 $I $ 有两个划分 $\mathbf{P}$ 和 $\mathbf{P'}$ 我们称 $\mathbf{P'}$ 是 $\mathbf{P}$ 的更细划分当且仅当对任意 $J \in \mathbf{P}$ 都有 $J' \in \mathbf{P'}$ 满足 $J' \subseteq J$ 。

\subsection{公共加细}

假设 $I $ 有两个划分 $\mathbf{P}$ 和 $\mathbf{P'}$,$\mathbf{P}$ 和 $\mathbf{P'}$ 的公共加细定义为 

\[
    \mathbf{P} \# \mathbf{P'} = \{ J \cap K \wvert J \in \mathbf{P}, K \in \mathbf{P'}\}
\]

\section{分段常数函数}

\subsection{定义}
设 $f: I \to \mathbb{R}$,其中 $I$ 是有界区间,如果存在一个划分 $\mathbf{P}$ 使得 $\forall J \in \mathbf{P}, J \ne \emptyset \, f(J)$ 是一个单点集,那么 $f$ 就是关于 $\mathbf{P}$ 的分段常数函数。

注意到如果 $\mathbf{P'}$ 比 $\mathbf{P}$ 更细,那么 $f$ 也是在 $\mathbf{P'}$ 上的分段常数函数。所以分段常数函数对加减乘,$\max$ 和 $\min$ 运算保持封闭,如果 $\forall x, g(x) \ne 0$ 那么 $f/g$ 也是分段常数函数。

\subsection{分段常值积分}

若 $f$ 是在划分 $\mathbf{P}$ 上的分段常数函数,定义 $f$ 的分段常值积分为 
\[
    \int_{[\mathbf{P}]} f = \sum_{J \in \mathbf{P}}c_J \lvert J \rvert
\]

分段常值积分是和 $\mathbf{P}$ 无关的,为了证明这个,我们需要证明一个引理,如果 $\mathbf{P'}$ 是比 $\mathbf{P}$ 更细的划分,那么

\[
    \int_{[\mathbf{P}]} f =\int_{[\mathbf{P'}]} f
\]

如果 $\mathbf{P'}$ 是比 $\mathbf{P}$ 更细的划分,那么 $J' \in \mathbf{P'}$ 可以分类成 $J'_{11}, J'_{12}, .. \subseteq J_1$, $J'_{21}, J'_{22}, .. \subseteq J_2$ 其中 $J_i \in \mathbf{P}$。
并且 $J' \subseteq J_i$ 也是对 $J_i$ 的划分。首先 $J'$ 是不相交的有界区间,其次 $x \in J_i$ 必然有 $x \in J'$ 且 $J' \subseteq J_i$ ,因为如果 $J'$ 不是 $J_i$ 的子集的话就会出现 $J_i \cap J_j \ne \emptyset$。
于是我们计算 $f$ 在 $\mathbf{P'}$ 上的分段常值积分可以分组计算再求和,按照 $J \in \mathbf{P}$ 进行分组,这样各组的结果相加得到了 $\sum_{J \in \mathbf{P}}c_J \lvert J \rvert$

证明如下假设 $\mathbf{P}$ 和 $\mathbf{P'}$ 都是 $f$ 的一个划分,而且 $f$ 在 $\mathbf{P}$ 和 $\mathbf{P'}$ 上都是分段常值函数。我们根据引理可以得到  

\[
    \int_{[\mathbf{P}]} f =\int_{[\mathbf{P \# P'}]} f = \int_{[\mathbf{P'}]} f
\]

所以分段常值函数的积分可以写成如下形式

\[
\int_{I} f
\]

\subsection{分段常值积分性质}

\begin{enumerate}
    \item $\int_{I} (f + g) = \int_{I} f + \int_{I} g$ 这个用公共加细很容易证明。
    \item $\int_{I} cf = c\int_{I} f$ 这个可以套定义。
    \item $\int_{I} (f - g) = \int_{I} f - \int_{I} g$ 用上面两个的组合可以证明
    \item 如果 $f(x) \ge 0$, $\int_{I} f \ge 0$,套定义。
    \item 如果 $f(x) \ge g(x)$ 那么 $\int_{I} f \ge \int_{I}g $,也是套定义。
    \item 如果 $f(x) = c$ 那么 $\int_{I} f  = c \lvert I \rvert $,也是套定义。
    \item 有限可加性,如果 $\{J,K \}$ 是对 $I$ 的一个划分,那么 $\int_{I}f = \int_{J}f + \int_{K} f$,因为我们可以从 $J$ 和 $K$ 各自的划分得到 $I$ 的一个划分。
\end{enumerate}

\section{黎曼积分}

\subsection{上方控制和下方控制}

$f: I \to \mathbb{R}$ 且 $g: I \to \mathbb{R}$,如果 $g(x) \ge f(x)$,那么 $g(x) $ 从上方控制 $f(x)$。
如果 $g(x) \le f(x)$ 那么 $g(x)$ 从下方控制 $f(x) $。

\subsection{上下黎曼积分的定义}

假设 $A = \{ g(x) \}$ 是所有上方控制 $f(x)$ 的分段常数函数的集合,有界函数 $f(x)$ 的上黎曼积分定义为:

\[
   \overline{\int}_I f =  \inf \{ \int_{I} g \wvert \forall x \in I ,\, g(x) \ge f(x), g(x) \mathrm{\; is\; p.c.} \}
\]

下黎曼积分定义为

\[
    \underline{\int}_I = f\sup \{ \int_{I} g \wvert \forall x \in I ,\, g(x) \le f(x), g(x) \mathrm{\; is\; p.c.} \}
\]

这里要注意的是,上黎曼积分是用下确界定义的,而下黎曼积分是用上确界定义的。因为上确界可以用最大附着点定义,下确界可以用最小附着点定义,后面做证明的时候可以利用数列极限的性质。


\subsection{黎曼可积的定义}

假设 $f: I \to \mathbb{R}$ 是有界的,如果 $f$ 的上黎曼积分等于 $f$ 的下黎曼积分,那么 $f$ 黎曼可积。


\subsection{分段常数函数是黎曼可积的}

这个很容易证明。


\subsection{黎曼和}

为了便于构造分段常数函数,我们定义上黎曼和 还有 下黎曼和。假设 $f: I \to \mathbb{R}$ 是有界的,而且 $\mathbf{P}$ 是 $I$ 的一个划分,那么定义 $f$ 的上黎曼和为

\[
    U(f, \mathbf{P}) = \sum_{J \in \mathbf{P}} (\sup_{x \in J}f(x)) \lvert J \rvert
\]

定义下黎曼和为

\[
    L(f, \mathbf{P}) = \sum_{J \in \mathbf{P}} (\inf_{x \in J}f(x)) \lvert J \rvert
\]

\subsection{上下黎曼和与上下黎曼积分}

上黎曼和的下确界就是上黎曼积分,首先上黎曼和的定义中隐含了一个上方控制的分段常数函数,所以上黎曼和这个集合的附着点也是上黎曼积分这个集合的附着点,所以

\[
    \overline{\int}_I f \le \inf \{ U(f, \mathbf{P}) \}
\]

假设有一个从上方控制的分段常数函数 $g(x)$ 还有划分 $\mathbf{P}$,我们取 $h(x)$ 为

\[
    h(x) = \sup \{ f(J) \}, \: x \in J
\]

于是我们得到了 $h(x) \le g(x)$ 所以通过上黎曼积分对应的附着点,可以构造出上黎曼和的附着点,而且上黎曼和的附着点小于等于上黎曼积分对应的附着点。所以有

\[
 \inf \{ U(f, \mathbf{P}) \}   \le  \overline{\int}_I f
\]

结合之前的结论得到

\[
 \inf \{ U(f, \mathbf{P}) \}  =  \overline{\int}_I f
\]

同理可以证明


\[
 \sup \{ L(f, \mathbf{P}) \}  =  \underline{\int}_I f
\]

\section{黎曼积分的性质}

\subsection{代数性质}

下面假设有界函数 $f: I \to \mathbb{R}$ 和 $g: I \to \mathbb{R}$ 都是黎曼可积的

\begin{enumerate}
    \item $\int(f+g) = \int f + \int g $ 这个很好证明,$f'$从上方控制 $f$,$g'$ 从上方控制 $g$ 可以得到 $f'+g'$ 从上方控制 $f+g$。利用附着点的性质得到
    \[
  \underline{\int}f + \underline{\int}g \le \underline{\int}(f+g)  \le \overline{\int}(f+g) \le \overline{\int}f + \overline{\int}g
    \]

    \item $\int cf = c \int f$ 这个要讨论 $c > 0$ 和 $c < 0$,$c > 0$ 时,$f'$ 上方控制 $f$ 可以得到 $cf'$ 上方控制 $cf$。
    \[
  c\underline{\int}f \le  \underline{\int}cf \le \overline{\int}cf \le c \overline{\int}f 
    \]

    如果 $c < 0$,如果 $f'$ 上方控制 $f$ 可以得到 $cf'$ 从下方控制 $cf$ 所以得到

    \[
      c\overline{\int}f   \le \underline{\int}cf \le \overline{\int} cf \le c \underline{\int}f
    \]

    \item $\int_{I}(f-g) = \int_{I}f - \int_{I}g $ 使用上面的结论可以证明

    \item $f(x) \ge 0$ 则 $\int_{I} f(x) \ge 0$,因为 $0$ 从下方控制了 $f(x)$ 所以有
    \[
        0 \le \underline{\int}_{I} f
    \]

    \item $f(x) \ge g(x)$ 则 $\int_{I}f \ge \int_{I} g$ 用上面的结论很容易证明。

    \item $f(x) = c$ 则 $\int_{I}f = c \lvert I \rvert$,因为 $f(x) =c$ 从上方控制也从下方控制。

    \item 如果 $\{ J, K \} $ 是 $I$ 的一个划分,并且 $f: J \to \mathbb{R}$ 和 $f: K \to \mathbb{R}$ 都是黎曼可积的,那么有

    \[
        \int_{J}f + \int_{K}f = \int_{I}f
    \]

    同理 $g$ 和 $h$ 如果分别在 $J$ 和 $K$ 上方控制 $f$ ,那么可以扩展得到 $g + h$ 也在上方控制 $f$,所以有

    \[
 \underline{\int}_J f + \underline{\int}_K f \le \underline{\int}_I f \le \overline{\int}_I f \le \overline{\int}_J f + \overline{\int}_K f
    \]

    \item 假设 $(a,b) \subseteq I \subseteq [a,b]$ 并且 $f: I \to \mathbb{R}$  是黎曼可积的,那么 

    \[
        \int_{I}f = \int_{[a,b]}f 
    \]

    证明也很简单,利用上方控制函数,还有上下极限是附着点,数列极限的性质。

    \[
 \underline{\int}_{[a,b]}f \le \underline{\int}_{I}f \le \overline{\int}_{I}f \le \overline{\int}_{[a,b]}f 
    \]

\end{enumerate}

\subsection{$\max$ 和 $\min$ 保持黎曼可积}

若 $f: I \to \mathbb{R}$ 和 $g: I \to \mathbb{R}$ 都是黎曼可积的,那么 $\max(f,g)$ 也是黎曼可积的。

下面给出证明,先证明一个引理,令 $h = \max(f,g)$,对任意区间 $J \subseteq I $

\begin{align*}
\underline{a} & \le \inf \{ f(x) \wvert x \in J \} \\
\underline{b} & \le \inf \{ g(x) \wvert x \in J \} \\
\overline{a} & \ge \sup \{ f(x) \wvert x \in J \} \\
\overline{b} & \ge \sup \{ g(x) \wvert x \in J \} \\
\overline{c} & = \sup \{ h(x) \wvert x \in J \} \\
\underline{c} & = \inf \{ h(x) \wvert x \in J \} \\
\end{align*}

有

\[
  \max (\underline{a},\underline{b})  \le \underline{c} \le \overline{c} \le \max (\overline{a},\overline{b})
\]

证明很容易,利用附着点和数列的性质即可。有了这个不等式后可以得到:

\begin{align*}
  \overline{c} - \underline{c} & \le \max(\overline{a}, \overline{b}) - \max(\underline{a}, \underline{b}) \\
  \overline{c} - \underline{c} & \le \max(\overline{a}, \overline{b}) + \min(-\underline{a}, -\underline{b}) \\
  \overline{c} - \underline{c} & \le \max(\overline{a} + \min(-\underline{a}, -\underline{b}), \overline{b} + \min(-\underline{a}, -\underline{b})) \\
  \overline{c} - \underline{c} & \le \max(\overline{a} -\underline{a}, \overline{b} -\underline{b}) \\
\end{align*}

若分段常数函数 $\overline{f}$ 和 $\overline{g}$ 分别从上方控制 $f$ 和 $g$,那么取他们的公共加细,然后在公共加细上对 $h$ 取上下确界,可以得到 $\overline{h}$ 和 $\underline{h}$,而且有

\begin{align*}
    \overline{h} - \underline{h} & \le \max(\overline{f} - \underline{f}, \overline{g} - \underline{g}) \\
    \overline{\int}_I h - \underline{\int}_I h \le \int_{I}(\overline{h} - \underline{h}) & \le \max (\int_{I}(\overline{f} - \underline{f}),\int_{I}(\overline{g} - \underline{g}) ) \le \epsilon
\end{align*}

$\min$ 可积也是同理。只不过把证明过程中的 $\max$ 跟 $\min$ 轮换一下。

\subsection{绝对值保持黎曼可积}

令 $f_{+}(x) = \max(f(x), 0) = (f(x) + \lvert f(x) \rvert) / 2$,$f_{-}(x) = \min(f(x), 0) = (f(x) - \lvert f(x) \rvert) / 2$
则有 $\lvert f(x) \rvert = f_{+}(x) - f_{-}(x)$

\subsection{乘积保持黎曼可积}

若 $f: I \to \mathbb{R}$ 和 $g: I \to \mathbb{R}$ 都是黎曼可积的,那么 $h = f \cdot g$ 也是黎曼可积的。

这里要把 $f$ 和 $g$ 各自拆成 $f = f_{+} + f_{-}$ 还有 $g = g_{+} + g_{-}$ 的形式,那么  $f \cdot g = f_{+}g_{+} + f_{+}g_{-} + f_{-}g_{+} +f_{-}g_{-} $

我们只需要证明 $f_{+}g_{+}$ 可积,下面的证明过程中我们暂时把$\overline{f_{+}}$ 简单记为 $\overline{f}$。只用到 $f_{+}$ 可积并且非负这个性质,$g_{+}$ 同理。其他项的证明类似。假设 $\overline{f}$ 和 $\overline{g}$ 分别从上方控制 $f_{+}$ 和 $g_{+}$, 
$\underline{f}$ 和 $\underline{g}$ 分别从下方控制 $f_{+}$ 和 $g_{+}$,
那么有 $ \underline{f} \underline{g} \le f_{+}g_{+} \le \overline{f} \overline{g}$ 注意到有下面不等式

\[
    \overline{f}\overline{g} - \underline{f} \, \underline{g} =  \overline{f}(\overline{g} - \underline{g}) + \underline{g}(\overline{f} - \underline{f}) \\
\]

\begin{align*}
    \int_{I}\overline{f}\overline{g} - \int_{I}\underline{f}\underline{g} &= \int_{I}\overline{f}\overline{g} - \underline{f}\underline{g} \\
    & \le \int_{I}\overline{f}(\overline{g} - \underline{g}) + \int_{I}\underline{g}(\overline{f} - \underline{f})  \\
    & \le \int_{I}M(\overline{g} - \underline{g}) + \int_{I}M(\overline{f} - \underline{f}) \le 2M\epsilon
\end{align*}

上面的 $M$ 是 $\max \{ f_{+}(x), g_{+}(x) \wvert x \in I \}$,$\epsilon$ 是利用 $f_{+}$ 和 $g_{+}$ 黎曼可积。

所以有 

\[
    \overline{\int}_{I}f_{+}g_{+} - \underline{\int}_{I}f_{+}g_{+} \le 2 M \epsilon
\]

因为上面的 $\epsilon$ 可以取任意小所以$f_{+}g_{+}$ 黎曼可积,同理可以得到 $(-f_{-})(-g_{-})$ 黎曼可积等。

这是一个经典等式,只要有乘法和加法这个等式可以用于如下形式:

\[
    x_1y_1 - x_2y_2 = x_1(y_1 - y_2) + y_2(x_1 - x_2)
\]