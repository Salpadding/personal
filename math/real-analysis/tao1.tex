\chapter{实分析 I}

\section{实数}

\subsection{实数定义: 柯西列}

我们要找到一种合理的方式定义实数,让实数可以做运算,可以相互比较。

\subsubsection{柯西列定义}

柯西列的定义可以不依赖于实数,所以可以用于定义实数。柯西列的可以定义为满足如下条件的有理数数列 $ a_n $。

\[ \forall \epsilon > 0, \exists N \in \mathbb{N}, 
    \forall j, k \ge N, \lvert a_j - a_k \rvert \le \epsilon \] 

\subsubsection{柯西列的等价条件}

两个柯西列 $a_n$ 和 $b_n$ 等价的定义如下:

\[ 
    \forall \epsilon > 0, \exists N \in \mathbb{N},
    \forall n \ge N, \lvert a_n - b_n \rvert \le \epsilon 
\]

\subsubsection{柯西列等价是良定义的}

首先是自反性,易证。其次是传递性,通过三角不等式也很容易证明。最后是对称性,也易证。

\subsubsection{柯西列有界}

我们令 $\epsilon = 1 $,可以找到 $N$ 满足 $\forall n \ge N , \lvert a_n - a_N \rvert \le 1 $ 于是 $a_n$ 在 $n \ge N$ 的那部分是有界的,
而 $n < N$ 的那部分是有限的自然也是有界的,所以柯西列有界。

\subsubsection{习题5.2.1}

证明: $a_n$ 和 $b_n$ 是等价有理数序列,$a_n$ 是柯西列当且仅当 $b_n$ 是柯西列。 \\

$\forall j, k \ge N, \lvert a_j - a_k \rvert \le \epsilon/4 $, 且 $ \lvert b_j - a_j \rvert \le \epsilon / 4 $, $\lvert b_k - a_k \rvert \le \epsilon / 4$, 那么有
$\lvert b_j - b_k \rvert \le \lvert  b_j - a_j + a_j - a_k + a_k - b_k \rvert \le \frac{3}{4} \epsilon $

\subsubsection{实数定义}

实数被定义为有理数柯西列的形式极限 也就是 
\[ x = \lim_{n \to \infty} a_n \]

\subsection{实数的运算}

\subsubsection{实数的加减乘运算}

通过柯西列,我们可以很方便地定义实数的加减乘运算。 \\
首先我们可以定义加减运算。

\begin{align*} 
x & = \lim_{n \to \infty}a_n \quad y  = \lim_{n \to \infty} b_n \\
x + y & := \lim_{n \to \infty} a_n + b_n 
\end{align*} \\

这个定义看似很合理,实际上并没有被证明是合理的。我们要证明这个定义要满足 $x +y $ 是一个柯西列,并且与柯西列的等价定义不冲突,最后还要满足加法的交换律和结合律,这才是真正意义上的加法。 \\
首先证明 $x +y $ 是一个柯西列,这个对于 $\epsilon $只要取 $N_x $ 和 $N_y$ 之间较大的 $N$ 就可以了。\\

证明等价性: $ \lvert a'_n + b'_n - a_n - b_n \rvert \le 2\epsilon $。\\

最后证明交换律和结合律: $ \lvert a_n + b_n - (b_n + a_n) \rvert = 0 \le \epsilon $ 以及 $ \lvert a_n + b_n + c_n - (a_n + (b_n + c_n)) \rvert = 0 \le \epsilon $ \\

我们继续定义乘法运算。 \\
\begin{align*} 
x & = \lim_{n \to \infty}a_n \quad y  = \lim_{n \to \infty} b_n \\
xy & := \lim_{n \to \infty} a_nb_n 
\end{align*} \\

同样我们要证明 $a_nb_n$ 是一个柯西列, 我们观察下 $\lvert (a_k + \epsilon)(b_k + \epsilon ) - a_jb_j \rvert = \lvert a_kb_k - a_jb_j + \epsilon^2 + \epsilon(a_k + b_k) \rvert \le \epsilon + \epsilon^2 + \epsilon \lvert a_k + b_k\rvert$。
因为 $\lvert ak+b_k \rvert$ 有界,所以对于 $\delta$,我们可以取一个很大的 $N$, 满足 $\lvert a_k - a_j \rvert \le \epsilon $,$ \lvert b_k - b_j \rvert \le \epsilon $ 且$\epsilon + \epsilon^2 + \epsilon M \le \delta $ 其中 $\lvert a_k + b_k\rvert \le M$,
这样就能得到 $\lvert a_kb_k - a_jb_j\rvert \le \delta $ \\

同理可以继续证明等价性: $\lvert a'_nb'+n - a_nb_n\rvert \le \epsilon + \epsilon^2 + \epsilon \lvert a_n + b_n \rvert$

\subsubsection{实数的三歧性}
有理数具有三歧性,通过柯西列我们也可以证明实数的三岐性。我们先定义正实数和负实数和零。\\
零很好定义,$\forall \epsilon >0, \exists N, \forall n \ge N, \lvert a_n \rvert \le \epsilon$ \\
以此类推可以定义正数,$\exists \epsilon > 0, \exists N, \forall n \ge N, a_n \ge \epsilon $ \\
同理可以定义负数,$\exists \epsilon > 0, \exists N, \forall n \ge N, a_n \le -\epsilon $

这些定义的合理性也需要证明,首先证明每个实数都具有三岐性之一。\\
对于实数 $x = \lim_{n\to \infty} a_n$,假设 $x = 0$,那么易证。如果 $x \ne 0$,根据柯西等价性的定义,那么就有 
\[ \exists \epsilon > 0, \forall N , \exists n \ge N, \lvert a_n \rvert \ge \epsilon \]
我们取这个 $\epsilon / 4 $ 代入 柯西列的定义 那么 $\exists N_{\epsilon} \ge N , \forall j,k \ge N_{\epsilon}, \lvert a_k - a_j\rvert \le \epsilon / 4 $,把 $j$ 换成 $N$ 得到 
$\lvert a_k  \rvert \ge  \lvert a_N \rvert - \lvert a_k - a_N \rvert \ge \frac{3}{4} \epsilon $。所以 $x$ 如果不为 0 的时候,只能是负数或者正数,显然 $x$ 不可能同时是正数和负数,所以 $x$ 满足三岐性。

\subsubsection{倒数}

对于正数和负数,我们可以定义倒数运算 $x^{-1} = \lim_{n \to \infty} a_n^{-1}$。
倒数运算是良定义的而且保持等价性。

\[
\lvert \frac{1}{a_k} - \frac{1}{a_j} \rvert = \frac{\lvert a_k - a_j \rvert}{\lvert a_ka_j \rvert} \le \frac{\epsilon}{M}
\]

而且 $xx^{-1} = 1$,这个很容易证明。于是我们可以定义除法 $x / y = xy^{-1}$

\subsubsection{实数的封闭性}
如果 $a_n \ge 0$ 那么 $x = \lim_{n \to \infty} a_n \ge 0$
证明: 假设 $x < 0$ ,那么存在 $\epsilon > 0 , N_{\epsilon}$,$\forall n \ge N, a_n \le -\epsilon $,显然矛盾了。\\

如果 $a_n \ge x$ 那么 $y = \lim_{n \to \infty} a_n, y \ge x$,假设 $y < x, \lim_{n \to \infty} b_n = x$ 。
根据负实数定义,存在 $ \exists \epsilon > 0, N \; \forall n \ge N,  a_n - b_n  \le -\epsilon $ 。注意到这里 $N$ 可以取任意大,
我们可以把 $ \epsilon / 2$ 代入 $b_n$ 的柯西列定义,让 $ \lvert b_n - b_N \rvert \le \epsilon /2$ 。于是当 $n \ge N$ 时,有 $a_N - b_n = a_N - b_N + b_N - b_n \le - \epsilon / 2$
这时候得到了矛盾 $a_N < x$


\subsubsection{习题5.4.5}

对于任意实数 $ x < y$,存在有理数 $q$ 满足 $x < q < y$

这里用反证法比较容易,假设所有的有理数都满足 $ q \le x \quad or \quad q \ge y $,我们分析一下 $c = (x+y)/2$,显然有 $x < c < y$。
假设 $ c = \lim_{n \to \infty}a_n$,因为 $a_n$ 是有理数,所以 $a_n$ 必然满足 $a_n \le x$ 或者 $a_n \ge y$,又因为 $a_n$ 是柯西列而且存在有理数 $\delta $ 满足 $\delta \le y-x$。
所有存在 $N_{\delta}$ 使得 $\forall k,j \ge N_{\delta}$,$\lvert a_k - a_j \rvert \le \delta $。假设 $a_{N_{\delta}} \le x$,那么对于 $n \ge N_{\delta}$,必然有 $a_n \le x$。
因为如果存在 $a_k \ge y$ 的话那么 $\lvert a_k - a_{N_\delta} \rvert \ge y -x > \delta $,矛盾了。同理如果 $a_{N_{\delta}} \ge y$,那么对于 $n \ge N_{\delta}$ 多有 $a_n \ge y$。
那么我们得到了 $ c \le x$ 或者 $c \ge y$, 显然矛盾了。

\subsubsection{习题5.4.3}

对于任意实数,存在唯一的整数满足 $N \le x < N + 1$

存在性: 因为 $x$ 是有界的,所以存在 $M > 0, M \in \mathbb{Z}$ 满足 $ -M \le x \le M $。定义 $ N $ 为满足 $N \le x, \lvert N \rvert \le M$ 的最大整数。显然有 $ N \le x < N + 1 $\\

唯一性: 假设 $N_1 \le x < N_1 + 1$ 且 $N_2 \le x < N_2$,那么有 $N_2 < N_1 + 1$, $N_1 < N_2 + 1$,也就是 $\lvert N_2 - N_1 \rvert < 1$,所以 $N_2 = N_1$


\subsubsection{习题5.4.8}
如果 $a_n \in \mathbb{Q}, x \in \mathbb{R},  a_n \ge x $,那么 $\lim_{n \to \infty} a_n \ge x$。 \\
证明: 假设 $y = \lim_{n \to \infty}a_n, y < x$, 令 $q \in \mathbb{Q}, y < q < x$。注意到 $y - q = \lim_{n \to \infty} (a_n -q ) < 0$
所以存在 $N$, 当 $n \ge N$ 时 $a_n \le q$, 这个和 $a_n \ge x$ 矛盾了。

\subsection{上确界性质}

对于任意的非空实数集合,存在最小上界。\\

证明: 假设 $E \subset \mathbb{R}, E \ne \emptyset $ 而且 $ E $ 具有上界 $M, M \in \mathbb{Q}$。取 $x \in E$, 有理数 $q \le x$
对于 $n$,我们把 $q \le y \le M$,分成 $n$ 份,每一份的长度为 $c_n = (M -q) / n$。于是对于 $N$,我们构造了有限的点集 $q + kC_n, 0 \le k \le n $
可以用数学归纳法证明,这个点集中存在 $E$ 的上界,我们取 $a_n = k $ 满足 $q + kC_n$ 是 $E$ 的上界,而且 $q + (k-1)C_n $ 不是 $E$ 的上界。
于是得到了

\begin{align*}
\frac{a_{n} - 1}{n} - \frac{a_n}{n} \le \frac{a_{n+1}}{n+1} - \frac{a_n}{n} & \le \frac{a_{n+1}}{n+1} - \frac{a_{n+1} - 1}{n+1} \\
\lvert \frac{a_{n+1}}{n+1} - \frac{a_n}{n} \rvert & \le \frac{1}{n}
\end{align*}

注意到这里的 $n+1$ 换成 $n + k$ 可以得到同样的结论,所以 $q + \frac{a_n}{n}$ 是一个柯西列,而且 $a + \frac{a_n - 1}{n}$ 也是柯西列。

令 $y = \lim_{n \to \infty} q + \frac{a_n}{n}$,可以得到 $y$ 是 $E$ 的上界。而且 $y$ 是最小上界,假设 $y' < y$,而且 $y'$ 也是 $E$ 的上界,注意到有 $q + \frac{a_n-1}{n} \le y'$,于是得到 $y \le y'$,矛盾。

\subsection{实数的幂运算(1)}

\subsubsection{整数次幂}
正整数和负整数的幂运算很容易定义这里不作说明,对应的良定义,保持等价性,以及性质用数学归纳法都很好证明。

\subsubsection{n次根}
定义非负实数 $y$ 的 $n$ 次根为集合 $ \{ x \vert \, x^n \le y \} $ 的最小上界。

对于这个定义,因为这个集合是有界的,所以存在性和等价性都很好证明。

\subsubsection{习题5.6.1}

1. $(x^{1/n})^n = x$\\

这个并没有看上去那么好证明,需要用到反证法。假设 $c = x^{1/n}$ 而且 $ c^n < x$。我们的思路是尝试找一个非常小的 $\epsilon \in \mathbb{R}$ 满足 $(c+\epsilon)^n < x$。首先 $n$ 目前是固定的,所以我们可以使用数学归纳法。

下面证明引理: $\forall \delta > 0, \exists \epsilon > 0 , \forall \lvert o \rvert \le \epsilon $, 满足 $ \lvert c^n - (c+o)^n \rvert \le \delta $。这个可以理解为幂函数的连续性。
我们对 $n $ 采用数学归纳法,首先 $n = 1$ 时很好证明,令 $\epsilon = \delta / 2$ 即可。
假设 $n = k$ 时结论成立,考虑 $n = k + 1$ 时,我们代入 $n = k$ 条件下的 $\epsilon$,取 $ \lvert o \rvert \le \epsilon $,而且 $\lvert o \rvert < 1$ 注意到 $\lvert c^{k+1} - (c+o)^{k+1} \rvert = \lvert c(c^k - (c+o)^k) - o(c+o)^{k+1} \rvert \le  \lvert c \delta \rvert  + o \lvert (c+o)^{k+1} \rvert \le  c \delta + \lvert \epsilon (c+1)^{k+1} \rvert $
注意到 $n = k + 1$ 时,这里的 $\delta$ $\epsilon $ 都是可以取任意小的。对于 $n = k + 1$ 的 $\delta $ 我们可以针对 $n = k$ 取很小的 $\delta' $ 和很小的 $\epsilon $,满足 
$c \delta' + \lvert \epsilon (c+1)^{k+1} \rvert  \le \delta $ 于是证明成功。 \\

假设 $c ^n < x$,那么根据引理我们可以找到一个很小的 $\epsilon$ 满足 $(c+\epsilon)^n < x$,于是根据 $c$ 是上界得到 $c + \epsilon \le c$,矛盾。 \\
假设 $c ^n > x$,那么根据引理我们可以找到一个很小的 $\epsilon$ 满足 $(c-\epsilon)^n > x$,于是 $c - \epsilon$ 也是集合 $\{ a \vert \, a^n \le x \}$ 的上界,这里和 $c$ 是最小上界矛盾。\\

2. $(x^n)^{1/n} = x$ \\

根据定义 $(x^n)^{1/n}$ 是集合 $\{a \vert \, a^n \le x^n \}$ 的最小上界。显然这个集合包含了 $x$,而且对于任意的 $a$,都有 $a \le x$,所以 $x$ 就是该集合的最小上界。\\

3. $x^{1/n}$ 是严格单调增 \\

原理很简单,子集的最小上界一定小于等于父集的最小上界。再假设 $x^{1/n} = y ^{1/n}$,得到 $x = y$。

4. $a < 1$ 时 $a^{1/k}$ 是关于 $k$ 的递减函数。\\
还是那个原理,当 $a < 1 $ 时,$a^{k+1} < a^{k} $,所以也存在子集关系。$x < a^{k+1} $ 可以得到 $ x^{k+1} < a$ 是 $x ^k < a$ 的子集。
所以 $a^{1/(k+1)} \le a^{1/k} $,而且这个等号也不可能成立。 \\

5. $a > 1$ 时 $a^{1/k}$ 是关于 $k$ 的递增函数。\\
同理。\\

6. $x ^{1/n}$ 连续 。\\
上面已经证明了 $ x^n$ 是连续的,因为 $ x ^{1/n}$ 是 $x^n$ 的反函数,所以 $x^{1/n}$ 连续。
可逆连续的函数在拓扑学也叫做同胚映射,如果 $f$ 在 $x_0$ 附近连续,考虑 $\delta$ 领域的象集,这个象集同样是 $f(x_0)$ 附近的 $\epsilon$ 领域。
因为可逆映射可以把象集映射回原象,所以找反函数的 $\epsilon$ 和 $\delta$ 只要找原函数的 $\delta$ 和 $\epsilon$ 即可。

\subsection{实数的有理数次幂}

假设有理数 $q = a/b $,其中 $a$ 和 $b$ 都是整数,那么定义 $x^q = (x^{1/b})^a $。 \\

首先要证明保持等价性,假设 $q = a/b$ 而且 $q = c/d$,我们要证明 $(x^{1/b})^a = (x^{1/d})^c$。\\
根据 $x^{1/(ad)} = x^{1/(bc)}$ 得到 $(x^{1/(ad)})^{ac} = (x^{1/(bc)})^{ac}$,根据定义得到 $x ^{ac/(ad)} = x^{ac/(bc)}$
再得到 $ x^{c/d} = x^{a/b}$

其他基本性质:

$x^{q+r} = x^qx^r $ 把 $q$ 和 $r$ 分解成比例形式很容易证明。\\

$x^{-q} = 1/x^q $ 同理 \\

$q > 0$ 时 $x^q < y^q$ 等价于 $x < y$ 同理 \\

$a > 1$ 时 $a^q$ 关于 $q$ 单调增,也是同理 \\

$(xy)^q = x^q y ^ q $,同理,左右同时做幂运算

\subsection{序列的极限}

\subsubsection{实数的柯西列}

这个定义也很容易理解。

\[ \forall \epsilon \in \mathbb{R}, \epsilon >0, \exists N, \forall n \ge N, \lvert a_k - a_j \rvert \le \epsilon \]。

为了能够使用这个定义,我们要证明这个定义和之前的定义是一致的。这个定义和之前的定义的区别在于,前面的 $\epsilon $ 是有理数,现在的 $\epsilon$ 是实数。
因为有理数就是实数,而正实数和0之间一定存在有理数。

\subsubsection{实数序列的极限}

$x_n$ 收敛到 $L$ 定义为: $\forall \epsilon > 0, \exists N, \forall n \ge N, \lvert a_n -L \rvert \le \epsilon $。


\subsubsection{收敛序列是柯西列}

利用三角不等式,很容易证明。

\subsubsection{形式极限和极限的定义一致}

假设 $\mathrm{LIM}_{n \to \infty} a_n = x$,我们观察 $a_N - x$ 也就是 $\mathrm{LIM}_{n \to \infty} a_N - a_n$ 。因为 $a_n$ 是柯西列,所以当 $N$ 充分大的时候,有 $\lvert a_N - a_n \rvert \le \epsilon $。
所以 $N$ 充分大的时候有 $ \lvert a_N - x \rvert \le \epsilon $,所以 $\lim_{n \to \infty} a_n = x$。\\

\subsubsection{柯西列有界}

很容易证明

\subsubsection{极限点,上下极限,收敛子列}

极限点的定义: $\forall \epsilon > 0, \forall N, \exists n \ge N, \lvert a_n - L\rvert \le \epsilon $。
通过极限点的定义可以构造收敛的子列,通过收敛子列可以构造出极限点,所以这两者是等价的。 \\

上下极限定义比较复杂,这里不给出,请牢记。同时上极限和下极限也都是极限点,上极限和下极限的定义依赖于单调有界原理。 \\

上极限是最大的极限点,下极限是最小的极限点。这个很好证明,首先我们要证明一个引理,子列的上极限不大于序列的上极限。然后我们假设存在更大的极限点,这个极限点可以转化成子列,
对这个子列去上极限得到了更大的上极限,这个与引理相矛盾。

如果上下极限都存在而且相等,那么这个序列收敛,而且收敛到上下极限,这个可以用三明治定理证明。

\subsubsection{柯西列收敛}

这是一个非常重要的定理,柯西列是有界的所以存在上极限和下极限。柯西列有个特点,就是任意两个子列都是 $\epsilon$ 稳定的,这个用柯西列的定义很容易证明。考虑到柯西列有界,上下极限也是子列,所以柯西列的上下极限存在并且相等。
所以柯西列收敛。这个性质可以推广到度量空间,更一般的形式是有限覆盖。

\subsubsection{一些常见的极限}

$\lim_{n \to \infty}x^{1/n} = 1$ \\

这个需要分成 $x > 1$ 和 $x < 1 $ 讨论,当 $ x > 1$ 时,对于任意小的 $ \epsilon$ 我们可以找一个很大 $N$ 满足 $(1 + \epsilon) ^ N \ge x$,于是得到
$x^{1/N} - 1 \le  \epsilon$。当 $x < 1$ 时,我们先利用单调有界证明极限存在,再利用 $y=x^{-1}$ 的连续性可以证明。

\subsection{指数的幂运算(2)}

有理数对于幂运算来说并不封闭,我们希望实数在幂运算下保持封闭,也就是说实数的实数次幂依然是实数。

\subsubsection{实数的实数次幂}

我们可以通过极限定义实数的实数次幂,对于非负实数 $x$,并且形式极限 $y = \lim_{n \to \infty }q_n, q_n \in \mathbb{Q}$ ,定义 $x^y = \lim_{n \to \infty} x^{q_n}$ 。
首先我们要证明这个定义有意义,也就是 $x^{q_n}$ 收敛,但目前我们不知道收敛到哪里,所以只能用柯西列的定义证明。
注意到 $\lvert x^{q_k} - x^{q_j} \rvert = \lvert x^{q_j} \rvert \lvert (x^{q_k-q_j}-1) \rvert$,因为 $q_n$ 有界所以 $x^{q_j}$ 有界,前面我们也证明过 $a^x, x \in \mathbb{Q} $ 在 $x = 0$
附近可以无限接近 $1$, 也就是 $\lim_{n \to \infty} a^{1/n} = 1$。所以 $x^y$ 是一个柯西列。

我们用到了形式极限来定义实数次幂,所以还需要证明幂运算的结果保持等价性,也就是 $q_n$ 和  $q'_n$ 如果等价那么 $x^y = x^{y'}$
这个证明和上面的很相似,注意到$ \lvert x^{q_n} -x ^{q'_n} \rvert = \lvert x^{q'_n} \rvert \lvert x^{q_n -q'_n}-1\rvert$。同理得证。

\subsubsection{实数的实数次幂的性质}

包含了所有有理数的性质

1. $x^{s+t} = x^sx^t$,把 $s$ 和 $t$ 表示成形式极限,然后用定义证明。\\

2. $x^{st} = (x^s)^t $ 这个证明没有看上去那么简单,要避免使用二重极限。 \\
我们先证明对于有理数 $q$ 和 实数 $s$,有 $x^{sq} = (x^s)^q = (x^q)^s$
证明: $(x^q)^s = \lim_{n \to \infty} (x^q)^a_n = \lim_{n \to \infty} x^{qa_n} = x^{sq} $ \\
$(x^s)^q = (\lim_{n \to \infty}x^{a_n})^q$,利用幂函数的连续性,得到 $(x^s)^q = \lim_{n \to \infty}(x^{a_n})^q = x^{qs}$

对于实数 $s$ 和 $t$,可以利用上面的结论。$(x^{s})^{t} = \lim_{n \to \infty}(x^s)^{b_n} = \lim_{n \to \infty}x^{s b_n} = x^{st}$

3. $(xy)^s = x^sy^s$ 很容易证明

4. 对于实数 $a \ge 0, x \in \mathbb{R}, y = a^x$ 连续。 \\

这个和之前证明 $\lim_{n \to \infty}x^{1/n} = 1$ 的方法是一样的。\\

5. 对于实数 $a \in \mathbb{R}, x \ge 0, y = x^a$ 连续。 \\
这个要用到三明治定理,先假定 $a >0$,先证明 $y = x^a$ 在 $x = 1$ 处连续。把 $a $ 框定在 $1/N \le a \le N $ 之间。利用 $x^N$ 和 $x^{1/N}$ 连续进行证明。
$(x+\epsilon)^a - x^a = x^a((1+\epsilon/x)^a-1)$

6. $y = a^x$ ,在 $a > 1$ 时单调增,在 $a < 1$ 时单调减。利用定义展开成极限,很容易证明。

\section{级数}

\subsection{判别法}

\subsubsection{交错级数判别法}
$a_n \ge 0 $ 并且 $a_n$ 单调减,那么 $S_n = (-1)^na_n$ 收敛当且仅当 $\lim_{n \to \infty}a_n = 0$ 。
证明: 令 $T_n = \sum_{k=1}^{2n}(-1)^na_n$,并且 $T'_n = \sum_{k=1}^{2n+1}(-1)^na_n$ \\
注意到 $T_{n+1} = T_n  - a_{2n+1} + a_{2n+2} \le T_n$,$T'_{n+1} = T'_n  + a_{2n+2} - a_{2n+3} \ge T'_n$ \\
所以 $T_n $ 具有上界,而 $T'n$ 具有下界,同时 $T'_n = T_n + (-1)^na_{2n+1}$,因为 $a_n$ 是有界的,所以 $T_n$ 和 $T'_n$ 都有界,所以
$T_n$ 和 $T'_n$ 都收敛。\\
又因为 $T'_n - T_n = a_{2n+1}$,所以如果 $\lim_{n \to \infty} a_n = 0$,$\lim_{n \to \infty}T_n = \lim_{n \to \infty} T'_n$,所以 $S_n$ 收敛。
同理如果 $S_n$ 收敛 $T_n$ 和 $T'_n$ 必然等价,所以 $a_n$ 必然也收敛。

\subsubsection{调和级数发散}
针对调和级数这类,也就是 $a_n \ge 0$,而且$a_n$ 单调减,可以找到一种判别法。
可以根据 $T_n = \sum_{k = 0}^{n-1}2^ka_{2^k}$ 来判断 $S_n$ 是否收敛。
证明也很容易,利用放缩法证明,$S_{2^k-1} \le T_k$。这里 $T_n$ 的定义是为了保证 $T_n$ 恰好有 $n$ 项,
而且每项的系数之和等于 $2^{k-1}$。

下面给出具体证明:

\begin{align*}
\sum_{n=1}^{2^k - 1}a_n & = \sum_{i=1}^{k}\sum_{j=2^{i-1}}^{2^i-1}a_j \le \sum_{i=1}^{k}2^{i-1}a_{2^{i-1}} \\
& \le \sum_{i=0}^{k-1}2^ia_{2^i}
\end{align*} \\

同时又有 \\

\begin{align*}
\sum_{n=1}^{2^{k+1}-1}a_n & \ge a_1 + \sum_{n=2}^{2^k}a_n \ge a_1 + \sum_{n=1}^{2^k-1}a_{n+1} \\
 & \ge a_1 + \sum_{i=1}^{k}\sum_{j=2^{i-1}}^{2^i-1}a_{j+1} \ge a_1 + \sum_{i=1}^{k}2^{i-1}a_{2^i} \\
2\sum_{n=1}^{2^{k+1}-1}a_n & \ge 2a_1 + \sum_{i=1}^{k}2^ia_{2^i} \ge \sum_{n=0}^{k}2^na_{2^n}
\end{align*} \\

所以有  \\

\[ \sum_{n=1}^{2^k-1}a_n \le \sum_{n=0}^{k-1}2^na_{2^n} \le 2\sum_{n=1}^{2^k-1}a_n \]

\subsubsection{黎曼函数}

利用柯西判别法可以判断 $\sum_{n=1}^{\infty}1/n^s$ 是否收敛。这个数列是单调减而且非负的。

\[ \sum_{n=0}^{k-1}2^k\frac{1}{2^{ks}} = \sum_{n=0}^{k-1}(2^{1-s})^k \]

所以当且仅当 $1-s < 0$ 也就是 $s > 1$ 时收敛,这个 $s$ 级数可以看作是一个关于 $s$ 的函数。

\subsection{级数重排}


\subsubsection{非负项级数重排保持收敛}

非负项级数经过重排后可以保持收敛,而且和原来的收敛保持一致。

这个结论看上去不平凡,但证明却相当容易,因为非负项级数可以看作单调的数列,只要证明有界就可以证明收敛。证明有界很容易:

\[ \sum_{n=1}^{m}a_{f(n)} \le \sum_{n=1}^{M}a_{n}, \; M = \max \{f^{-1}(n) \vert 1 \le n \le m \} \]

证明收敛到相同的值也很容易。

假设 $ L - S_n  \le \epsilon $ 对 $n \ge N$ 恒成立。我们取 $M = \max \{f^{-1}(n) \vert 1 \le n \le N \} $ 
于是必然有 $ \forall n \ge M, \sum_{n=1}^{N} \le \sum_{n=1}^{M}a_{f(n)} \le L $

\subsubsection{绝对收敛的级数重排后保持收敛}
这个结论比刚才的使用范围更广,因为绝对收敛的级数可能包含负项。首先重排后,我们可以用证明非负项级数收敛的方法证明收敛。注意到绝对收敛蕴涵了收敛。\\
证明重排后收敛到相同的值需要一定的技巧。

注意我们取 $M = \{ f^{-1}(n) \vert 1 \le n \le N \} $ 的时候 $ \{ f(m) \vert 1 \le m \le M \}$ 可能包含大于 $N$ 的数值,但好就好在这一点大于 $N$,
因为 $N$ 很大的时候,这部分的值就会很小。

\begin{align*}
\lvert \sum_{k=1}^{M}a_{f(k)} - L \rvert & =  \lvert \sum_{n=1}^{N}a_{n} + \sum_{k=P}^{Q}a_{k} - L \rvert \\
& \le \lvert \sum_{n=1}^{N}a_n - L \rvert + \sum_{k=N+1}^{M}\lvert a_k \rvert \\
& \le 2\epsilon
\end{align*}

上面的 $N$ 取的是 $\sum_{n=1}^{\infty}a_n$ 极限和 $a_n$ 绝对收敛对应的柯西列,而且这个小于号不取决于 $M$ 的增加。


\subsection{判别法}

\subsubsection{根值判别法}
这个判别法只需要考虑 $a_n$ 的性质,不需要考虑 $S_n$ 的性质。\\

1. $\lvert a_n \rvert ^{1/n}$ 的上极限如果小于 1,那么 $S_n$ 绝对收敛。 \\

2. $\lvert a_n \rvert ^{1/n}$ 的上极限如果大于 1,那么 $S_n$ 发散。 \\

3. $\lvert a_n \rvert ^{1/n}$ 的上极限如果等于 1,那么未知。 \\

1很容易证明,根据定义能找到一个 $ \alpha < 1 $ 和 $N$,满足 $\forall n \ge N, \lvert a_n \rvert ^{1/n} \le \alpha $。
这样就得到了 $\lvert a_n \rvert \le \alpha ^n$,因为 $\sum_{n=1}^{M}\alpha^n$ 有界,所以 $\sum_{n=1}^{M}\lvert a_n \rvert $ 收敛。 \\

2也很容易证明,根据定义对任意 $N$ 都能找到 $n \ge N$,满足 $\lvert a_n \rvert ^{1/n} > 1$ 从而有 $\lvert a_n \rvert > 1$,假如 $\sum_{n=1}^{\infty}a_n$ 收敛,必然有 $\lim_{n \to \infty}\lvert a_n \rvert = 0$


\subsubsection{比值判别法}

\begin{align*}
\varliminf_{n \to \infty}\frac{c_{n+1}}{c_n} \le \varliminf_{n \to \infty} c_n ^{1/n} \le \overline{\lim_{n \to \infty}} c_n ^{1/n} \le \overline{\lim_{n \to \infty}} \frac{c_{n+1}}{c_n}
\end{align*}

\section{黎曼积分}

\subsection{划分}

\subsubsection{有界区间}

回顾下有界区间的定义,$(a,b) \,\, (a,b] \,\, [a,b) \,\, [a,b]$ 都是区间。注意这里不要求 $a$ 和 $b$ 不相等,也没有要求 $a \le b$,所以空集也是区间,一个只包含一个实数的集合也是区间。

\subsubsection{连通集}

连通集 $A$ 定义为 $x, y \in A$ 则 $\forall x \le t \le y, \, t \in A$ 

\subsubsection{实数集上的有界连通集和区间等价}

下面证明实数集上的有界连通集是区间,我们分别取 $A$ 的上确界为 $b$,下确界为 $a$,下面证明 $(a,b) \subseteq A \subseteq [a,b]$。
假设存在 $c \in (a,b), c \notin A$,因为上确界和下确界都是附着点,
所以我们可以找到 $a' \in A $ 和 $b' \in A$ 满足 $a \le a' < c < b' \le b$,这就跟 $A$ 是连通集矛盾了。
右半部分易证明,所以 $A$ 是一个区间。

\subsubsection{有界区间的交集也是有界区间}

连通集合的交集依然是连通集,有界集合的交集依然是有界的。


\subsubsection{区间的长度}

对于空集,定义区间长度为 $0$,对于非空的区间 $(a,b) \subseteq I \subseteq [a,b]$ 定义 $I$ 的长度为 $b-a$,这里也包含了单点集。

\subsubsection{划分}

$\mathbf{P} = \{ J \vert J \subseteq I,\, J \in \mathbf{P} \}$ 是对有界区间 $I$ 的一个划分当且仅当 $I$ 中的每个元素 $x$ 恰好属于 $\mathbf{P}$ 中的一个有界区间 $J$。

\subsubsection{有限可加性}

我们上面之所以这样定义划分就是为了能够满足有限可加性,这对黎曼积分的定义有重要的意义。

\[
    \lvert I \rvert = \sum_{J \in \mathbf{P}} \lvert J \rvert
\]

下面给出证明,假设 $(a,b) \subseteq I \subseteq [a,b] $,并且 $\mathbf{P}$ 是对 $I$ 的一个划分。如果 $I$ 是空集,那么成立。如果 $I$ 不是空集,那么设 $\mathbf{P}$ 中元素的数量为 $n > 0$。令 $a_0=a, I_0 = I, \mathbf{P_0} = \mathbf{P}$ 我们讨论 $a_0$ 是否属于 $I_0$,
如果 $a_0 \in I_0$,那么必然有 $a_0 \in J_0 ,\, J_0 \in \mathbf{P_0}$ 令 $[a_0, b_0) \subseteq J_0 \subseteq [a_0, b_0]$ 同时有 $I_0 = \bigcup_{J \in \mathbf{P_0}} J $ 两边同时对 $J_0$ 作差集得到 $I_0 \setminus J_0 = \bigcup_{J \in \mathbf{P_0}, J \ne J_0}J$ 
注意到 $I_0 \setminus J_0$ 依然是一个区间,我们可以记它为 $I_1$ ,注意到此时 $J \in \mathbf{P_0}, J \ne J_0$ 组成了对 $I_1$ 的一个划分,我们记这个划分为 $\mathbf{P_1}$ 所以我们可以对 $I_1, \mathbf{P_1}$ 作重复的操作,同时 $I_1$ 的下确界一定是 $b_0$。同理如果 $a_0 \notin I_0$,那么 $\mathbf{P_0}$ 中一定存在一个 $J_0$ 满足 $J_0$ 的下确界等于 $a_0$,否则 $a_0$ 附近一定有点不会被覆盖。同样的可以做上面的操作得到 $I_1, \mathbf{P_1}$。
我们重复以上操作和得到了 $\{ J_0, J_1, .., J_{n-1} \} = \mathbf{P}$,注意到 $J_0$ 的上确界就是 $J_1$ 的下确界,以此类推。我们得到 $\lvert J_0 \rvert + \lvert J_1 \rvert  + .. + \lvert J_{n-1} \rvert = a_1 -a_0 + a_2 - a_1 + .. + a_n - a_{n-1} = a_n - a_0 = b - a$。
注意到 $I_{n-1} \setminus J_{n-1} = I_n = \emptyset$,所以 $J_{n-1}$ 的上确界一定是 $b$。

\subsubsection{更细划分}

假设 $I $ 有两个划分 $\mathbf{P}$ 和 $\mathbf{P'}$ 我们称 $\mathbf{P'}$ 是 $\mathbf{P}$ 的更细划分当且仅当对任意 $J \in \mathbf{P}$ 都有 $J' \in \mathbf{P'}$ 满足 $J' \subseteq J$ 。

\subsubsection{公共加细}

假设 $I $ 有两个划分 $\mathbf{P}$ 和 $\mathbf{P'}$,$\mathbf{P}$ 和 $\mathbf{P'}$ 的公共加细定义为 

\[
    \mathbf{P} \# \mathbf{P'} = \{ J \cap K \wvert J \in \mathbf{P}, K \in \mathbf{P'}\}
\]

\subsection{分段常数函数}

\subsubsection{定义}
设 $f: I \to \mathbb{R}$,其中 $I$ 是有界区间,如果存在一个划分 $\mathbf{P}$ 使得 $\forall J \in \mathbf{P}, J \ne \emptyset \, f(J)$ 是一个单点集,那么 $f$ 就是关于 $\mathbf{P}$ 的分段常数函数。

注意到如果 $\mathbf{P'}$ 比 $\mathbf{P}$ 更细,那么 $f$ 也是在 $\mathbf{P'}$ 上的分段常数函数。所以分段常数函数对加减乘,$\max$ 和 $\min$ 运算保持封闭,如果 $\forall x, g(x) \ne 0$ 那么 $f/g$ 也是分段常数函数。

\subsubsection{分段常值积分}

若 $f$ 是在划分 $\mathbf{P}$ 上的分段常数函数,定义 $f$ 的分段常值积分为 
\[
    \int_{[\mathbf{P}]} f = \sum_{J \in \mathbf{P}}c_J \lvert J \rvert
\]

分段常值积分是和 $\mathbf{P}$ 无关的,为了证明这个,我们需要证明一个引理,如果 $\mathbf{P'}$ 是比 $\mathbf{P}$ 更细的划分,那么

\[
    \int_{[\mathbf{P}]} f =\int_{[\mathbf{P'}]} f
\]

如果 $\mathbf{P'}$ 是比 $\mathbf{P}$ 更细的划分,那么 $J' \in \mathbf{P'}$ 可以分类成 $J'_{11}, J'_{12}, .. \subseteq J_1$, $J'_{21}, J'_{22}, .. \subseteq J_2$ 其中 $J_i \in \mathbf{P}$。
并且 $J' \subseteq J_i$ 也是对 $J_i$ 的划分。首先 $J'$ 是不相交的有界区间,其次 $x \in J_i$ 必然有 $x \in J'$ 且 $J' \subseteq J_i$ ,因为如果 $J'$ 不是 $J_i$ 的子集的话就会出现 $J_i \cap J_j \ne \emptyset$。
于是我们计算 $f$ 在 $\mathbf{P'}$ 上的分段常值积分可以分组计算再求和,按照 $J \in \mathbf{P}$ 进行分组,这样各组的结果相加得到了 $\sum_{J \in \mathbf{P}}c_J \lvert J \rvert$

证明如下假设 $\mathbf{P}$ 和 $\mathbf{P'}$ 都是 $f$ 的一个划分,而且 $f$ 在 $\mathbf{P}$ 和 $\mathbf{P'}$ 上都是分段常值函数。我们根据引理可以得到  

\[
    \int_{[\mathbf{P}]} f =\int_{[\mathbf{P \# P'}]} f = \int_{[\mathbf{P'}]} f
\]

所以分段常值函数的积分可以写成如下形式

\[
\int_{I} f
\]

\subsubsection{分段常值积分性质}

\begin{enumerate}
    \item $\int_{I} (f + g) = \int_{I} f + \int_{I} g$ 这个用公共加细很容易证明。
    \item $\int_{I} cf = c\int_{I} f$ 这个可以套定义。
    \item $\int_{I} (f - g) = \int_{I} f - \int_{I} g$ 用上面两个的组合可以证明
    \item 如果 $f(x) \ge 0$, $\int_{I} f \ge 0$,套定义。
    \item 如果 $f(x) \ge g(x)$ 那么 $\int_{I} f \ge \int_{I}g $,也是套定义。
    \item 如果 $f(x) = c$ 那么 $\int_{I} f  = c \lvert I \rvert $,也是套定义。
    \item 有限可加性,如果 $\{J,K \}$ 是对 $I$ 的一个划分,那么 $\int_{I}f = \int_{J}f + \int_{K} f$,因为我们可以从 $J$ 和 $K$ 各自的划分得到 $I$ 的一个划分。
\end{enumerate}

\subsection{黎曼积分}

\subsubsection{上方控制和下方控制}

$f: I \to \mathbb{R}$ 且 $g: I \to \mathbb{R}$,如果 $g(x) \ge f(x)$,那么 $g(x) $ 从上方控制 $f(x)$。
如果 $g(x) \le f(x)$ 那么 $g(x)$ 从下方控制 $f(x) $。

\subsubsection{上下黎曼积分的定义}

假设 $A = \{ g(x) \}$ 是所有上方控制 $f(x)$ 的分段常数函数的集合,有界函数 $f(x)$ 的上黎曼积分定义为:

\[
   \overline{\int}_I f =  \inf \{ \int_{I} g \wvert \forall x \in I ,\, g(x) \ge f(x), g(x) \mathrm{\; is\; p.c.} \}
\]

下黎曼积分定义为

\[
    \underline{\int}_I = f\sup \{ \int_{I} g \wvert \forall x \in I ,\, g(x) \le f(x), g(x) \mathrm{\; is\; p.c.} \}
\]

这里要注意的是,上黎曼积分是用下确界定义的,而下黎曼积分是用上确界定义的。因为上确界可以用最大附着点定义,下确界可以用最小附着点定义,后面做证明的时候可以利用数列极限的性质。


\subsubsection{黎曼可积的定义}

假设 $f: I \to \mathbb{R}$ 是有界的,如果 $f$ 的上黎曼积分等于 $f$ 的下黎曼积分,那么 $f$ 黎曼可积。


\subsubsection{分段常数函数是黎曼可积的}

这个很容易证明。


\subsubsection{黎曼和}

为了便于构造分段常数函数,我们定义上黎曼和 还有 下黎曼和。假设 $f: I \to \mathbb{R}$ 是有界的,而且 $\mathbf{P}$ 是 $I$ 的一个划分,那么定义 $f$ 的上黎曼和为

\[
    U(f, \mathbf{P}) = \sum_{J \in \mathbf{P}} (\sup_{x \in J}f(x)) \lvert J \rvert
\]

定义下黎曼和为

\[
    L(f, \mathbf{P}) = \sum_{J \in \mathbf{P}} (\inf_{x \in J}f(x)) \lvert J \rvert
\]

\subsubsection{上下黎曼和与上下黎曼积分}

上黎曼和的下确界就是上黎曼积分,首先上黎曼和的定义中隐含了一个上方控制的分段常数函数,所以上黎曼和这个集合的附着点也是上黎曼积分这个集合的附着点,所以

\[
    \overline{\int}_I f \le \inf \{ U(f, \mathbf{P}) \}
\]

假设有一个从上方控制的分段常数函数 $g(x)$ 还有划分 $\mathbf{P}$,我们取 $h(x)$ 为

\[
    h(x) = \sup \{ f(J) \}, \: x \in J
\]

于是我们得到了 $h(x) \le g(x)$ 所以通过上黎曼积分对应的附着点,可以构造出上黎曼和的附着点,而且上黎曼和的附着点小于等于上黎曼积分对应的附着点。所以有

\[
 \inf \{ U(f, \mathbf{P}) \}   \le  \overline{\int}_I f
\]

结合之前的结论得到

\[
 \inf \{ U(f, \mathbf{P}) \}  =  \overline{\int}_I f
\]

同理可以证明


\[
 \sup \{ L(f, \mathbf{P}) \}  =  \underline{\int}_I f
\]

\subsection{黎曼积分的性质}

下面假设有界函数 $f: I \to \mathbb{R}$ 和 $g: I \to \mathbb{R}$ 都是黎曼可积的

\begin{enumerate}
    \item $\int(f+g) = \int f + \int g $ 这个很好证明,$f'$从上方控制 $f$,$g'$ 从上方控制 $g$ 可以得到 $f'+g'$ 从上方控制 $f+g$。利用附着点的性质得到
    \[
  \underline{\int}f + \underline{\int}g \le \underline{\int}(f+g)  \le \overline{\int}(f+g) \le \overline{\int}f + \overline{\int}g
    \]

    \item $\int cf = c \int f$ 这个要讨论 $c > 0$ 和 $c < 0$,$c > 0$ 时,$f'$ 上方控制 $f$ 可以得到 $cf'$ 上方控制 $cf$。
    \[
  c\underline{\int}f \le  \underline{\int}cf \le \overline{\int}cf \le c \overline{\int}f 
    \]

    如果 $c < 0$,如果 $f'$ 上方控制 $f$ 可以得到 $cf'$ 从下方控制 $cf$ 所以得到

    \[
      c\overline{\int}f   \le \underline{\int}cf \le \overline{\int} cf \le c \underline{\int}f
    \]

    \item $\int_{I}(f-g) = \int_{I}f - \int_{I}g $ 使用上面的结论可以证明

    \item $f(x) \ge 0$ 则 $\int_{I} f(x) \ge 0$,因为 $0$ 从下方控制了 $f(x)$ 所以有
    \[
        0 \le \underline{\int}_{I} f
    \]

    \item $f(x) \ge g(x)$ 则 $\int_{I}f \ge \int_{I} g$ 用上面的结论很容易证明。

    \item $f(x) = c$ 则 $\int_{I}f = c \lvert I \rvert$,因为 $f(x) =c$ 从上方控制也从下方控制。


\end{enumerate}