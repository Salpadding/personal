%!LW recipe=latexmk (xelatex)
% a4 页面 字体大小12像素
\documentclass[12pt,a4paper]{ctexart}
% 数学公式
\usepackage{amsmath}
\usepackage{amsfonts}
\usepackage{amssymb}
\usepackage[hidelinks]{hyperref}
\usepackage{enumitem}
\usepackage{geometry}


\geometry{left=1.5cm, right=1.5cm, top=2cm, bottom=2cm}

\title{学习记录}
\author{朱英杰}
\date{\today}

\begin{document}
\zihao{-4}
\maketitle
\tableofcontents

\section{多元函数微积分}

\subsection{偏导连续可以推出函数可微}

\subsubsection{$f$ 为实值函数}

若 $f: \mathbb{R}^n \to \mathbb{R}$ 满足 $\frac{\partial f}{\partial x_i}$ 连续,那么 $f$ 连续并且可微。

证明如下,令 $\mathbf{x'} = \mathbf{x} + \Delta = \mathbf{x} + [\delta_1, \delta_2, .., \delta_n]^T$

\begin{align*}
    & f(\mathbf{x}')  - f(\mathbf{x}) - \sum_{i=1}^{n}f_i(\mathbf{x})\delta_i = \int_{x_1}^{x_1 + \delta_1}f_1(t, x_2, .., x_n) \text{d}t  + \\ 
    & \int_{x_2}^{x_2 + \delta_1}f_2(x_1 + \delta_1, t, .., x_n) \text{d}t + .. + \int_{x_2}^{x_n + \delta_n}f_n(x_1 + \delta_1, x_2 + \delta_2, .., t) \text{d}t - \sum_{i=1}^{n}f_i(\mathbf{x}) \delta_i
\end{align*}

因为 $f_1, f_2, .. f_n$ 都是连续的,所以我们可以让 $\delta_1, \delta_2, .. \delta_n$ 都充分小,然后代入 $f_1, f_2, .., f_n$ 连续的性质,得到

\begin{align*}
\lvert f_1(t,x_2,..,x_n) - f_1(x_1, x_2, .. ,x_n)\rvert & \le \epsilon \\
\lvert f_2(x_1 + \delta_1,t,..,x_n) - f_2(x_1, x_2, .. ,x_n)\rvert & \le \epsilon \\
.. \\
\lvert f_n(x_1 + \delta_1,x_2+\delta_2,..,x_n+\delta_n) - f_n(x_1, x_2, .. ,x_n)\rvert & \le \epsilon \\
\end{align*}

再代入上面得到

\begin{align*}
    \lvert f(\mathbf{x}')  - f(\mathbf{x}) - \sum_{i=1}^{n}f_i(\mathbf{x})\delta_i \rvert \le \lvert \sum_{i=1}^{n} \delta_i \epsilon \rvert \le \epsilon \sum_{i=1}^{n} \lvert \delta_i \rvert
\end{align*}

两边同时除以 $\Delta$ 的二范数得到 


\begin{align*}
    \frac{\lvert f(\mathbf{x} + [\delta_i]^T)  - f(\mathbf{x}) - \sum_{i=1}^{n}f_i(\mathbf{x})\delta_i \rvert}{\sqrt{\sum_{i=1}^{n}\delta_i^2}}   \le \epsilon \frac{\sum_{i=1}^{n} \lvert \delta_i \rvert}{\sqrt{\sum_{i=1}^{n}\delta_i^2}}
\end{align*}

根据柯西不等式有

\[
\sum_{i=1}^{n} \lvert \delta_i \rvert \le \sqrt{n} \sqrt{\sum_{i=1}^{n}\delta_i^2}
\]

所以有

\[
\frac{\lvert f(\mathbf{x} + [\delta_i]^T)  - f(\mathbf{x}) - \sum_{i=1}^{n}f_i(\mathbf{x})\delta_i \rvert}{\sqrt{\sum_{i=1}^{n}\delta_i^2}} \le \epsilon \sqrt{n}
\]

根据定义得到了 $f$ 在 $\mathbf{x}$ 处可微,而且有

$\nabla f = [f_1, f_2, .. , f_n]$

\subsubsection{$f \in \mathbb{R}^n \to \mathbb{R}^m$ }

我们可以令 $g_i(\mathbf{x}) = \mathbf{e}_i^T f(\mathbf{x})$,也就是第 $i$ 个分量,然后我们可以得到 $g_i$ 都是可微的。

令 $\mathbf{x'} = \mathbf{x} + \Delta = \mathbf{x} + [\delta_1, \delta_2, .., \delta_n]^T$

当 $\| \Delta \|_2$ 充分小的时候,我们有

\begin{align*}
f(\mathbf{x} + \Delta) - f(\mathbf{x}) - \mathbf{J}_f(\mathbf{x}) \Delta &= \\
f(\mathbf{x} + \Delta) - f(\mathbf{x}) - \begin{bmatrix}
    (\nabla g_1)^T \\
    (\nabla g_2)^T \\
    .. \\
    (\nabla g_m)^T \\
\end{bmatrix}  \Delta
\end{align*}

因为 $g_i$ 都是可微的,所以有

\[
\alpha_i = \| g_i(\mathbf{x} + \Delta) - g_i(\mathbf{x}) - (\nabla g_i)^T\Delta \|_2 \le \epsilon \| \Delta \|_2
\]

因为

\begin{align*}
\| f(\mathbf{x} + \Delta) - f(\mathbf{x}) - \mathbf{J}_f(\mathbf{x}) \Delta \|_2 &= \sqrt{\sum_{i=1}^{m}\alpha_i^2} \le \sqrt{m \cdot \max_{i \le m} \alpha_i^2} \\
& \le \sqrt{m}(\sqrt{\max_{i \le m} \alpha_i^2}) \le \sqrt{m}\sum_{i=1}^{m} \alpha_i \\
&\le m \sqrt{m} \epsilon \| \Delta \|_2
\end{align*}

所以 $f$ 可微

\subsubsection{其他结论}

通过这个定理,可以验证得到下面这些函数都是可微的。

\begin{enumerate}
    \item $f(x,y) = x + y$
    \item $f(x,y) = xy$
    \item $f(x,y) = x^2 + y^2$
\end{enumerate}

\subsection{Clairaut's Theorem}

若 $f: \mathbb{R}^n \to \mathbb{R}$ 二阶连续可微,n阶连续可微是指满足n阶偏导连续,那么有

\[
\frac{\partial^2 f}{\partial x_j \partial x_i} = 
\frac{\partial^2 f}{\partial x_i \partial x_j}
\]

下面给出证明,我们只要证明在 $\mathbf{x} = \mathbf{0}$ 处即可,其他情况可以转化成 $f(\mathbf{x} + \mathbf{x}_0)$。令 

\[
X(\delta) = f(\delta \mathbf{e}_i + \delta \mathbf{e}_j) - f(\delta \mathbf{e}_i) - f(\delta \mathbf{e}_j) + f(\mathbf{0})
\]

我们对 $x_j$ 展开成积分

\begin{align*}
X(\delta)  &= \int_{0}^{\delta}\frac{\partial f}{\partial x_j}(\delta \mathbf{e}_i + x_j \mathbf{e}_j) \text{d}x_j - \int_{0}^{\delta}\frac{\partial f}{\partial x_j}(x_j \mathbf{e}_j) \text{d}x_j \\
& = \int_{0}^{\delta}\frac{\partial f}{\partial x_j}(\delta \mathbf{e}_i + x_j \mathbf{e}_j) - \frac{\partial f}{\partial x_j}(x_j \mathbf{e}_j) \text{d}x_j \\
&= \int_{0}^{\delta}\int_{0}^{\delta} \frac{\partial^2 f}{\partial x_i \partial x_j}(x_i \mathbf{e}_i + x_j \mathbf{e}_j) \text{d}x_j \text{d}x_i
\end{align*}

我们可以让 $\delta$ 充分小,足以满足二阶偏导数连续的性质,所以有

\[
\lvert \frac{\partial^2 f}{\partial x_i \partial x_j}(x_i \mathbf{e}_i + x_j \mathbf{e}_j) - \frac{\partial^2 f}{\partial x_i \partial x_j}(\mathbf{0}) \rvert \le \epsilon
\]

令

\[
\frac{\partial^2 f}{\partial x_i \partial x_j}(\mathbf{0}) = a
\]

得到

\begin{align*}
\lvert X(\delta) - \delta^2 a \rvert &= \lvert X - \int_{0}^{\delta}\int_{0}^{\delta}a \text{d}x_i \text{d}x_j \rvert \\
& \le  \int_{0}^{\delta}\int_{0}^{\delta} \lvert \frac{\partial^2 f}{\partial x_i \partial x_j}(x_i \mathbf{e}_i + x_j \mathbf{e}_j) - a\rvert \text{d}x_i \text{d}x_j \\
& \le \epsilon \delta^2
\end{align*}

同理我们可以把 $X(\delta)$ 对 $x_i$ 展开成积分


\begin{align*}
X(\delta)  &= \int_{0}^{\delta}\frac{\partial f}{\partial x_i}(x_i \mathbf{e}_i + \delta \mathbf{e}_j) \text{d}x_i - \int_{0}^{\delta}\frac{\partial f}{\partial x_i}(x_i \mathbf{e}_i) \text{d}x_i \\
& = \int_{0}^{\delta}\frac{\partial f}{\partial x_i}(\delta x_i\mathbf{e}_i + \delta \mathbf{e}_j) - \frac{\partial f}{\partial x_i}(x_i \mathbf{e}_i) \text{d}x_i \\
&= \int_{0}^{\delta}\int_{0}^{\delta} \frac{\partial^2 f}{\partial x_j \partial x_i}(x_i \mathbf{e}_i + x_j \mathbf{e}_j) \text{d}x_i \text{d}x_j
\end{align*}

令

\[
a' = \frac{\partial^2 f}{\partial x_j \partial x_i}(\mathbf{0})
\]

我们得到

\[
\lvert X(\delta) - \delta^2 a' \rvert \le \epsilon \delta^2
\]

运用三角不等式得到

\[
\lvert \delta^2 a - \delta^2 a' \rvert \le2\epsilon \delta^2
\]

因为 $\delta > 0$ 而 $\epsilon$ 可以任意小,所以

\[
a= a'
\]

\section{行列式计算}

\subsection{降阶法}

\subsubsection{组合数}

\begin{align*}
    X_n &= \begin{vmatrix}
        1 & 1 & 1 & .. & 1 \\
        1 & \binom{2}{1} & \binom{3}{1} & .. & \binom{n}{1} \\
        1 & \binom{3}{2} & \binom{4}{2} & .. & \binom{n+1}{2} \\
        .. & .. & .. & .. & .. \\
        1 & \binom{n}{n-1} & \binom{n+1}{n-1} & .. & \binom{2n-2}{n-1}
    \end{vmatrix} =     \begin{vmatrix}
        1 & 1 & .. & 1 & 1\\
        0 & \binom{1}{1} &  \binom{2}{1} & .. & \binom{n-1}{1} \\
        0 & \binom{2}{2} & \binom{3}{2} & .. & \binom{n}{2}  \\
        .. & .. & .. & .. & ..\\
        0 & \binom{n-1}{n-1} & \binom{n}{n-1} & .. & \binom{2n-3}{n-1}
    \end{vmatrix} \\
    &=\begin{vmatrix}
          1 &  \binom{2}{1} & .. & \binom{n-1}{1} \\
          1 & \binom{3}{2} & .. & \binom{n}{2}  \\
          .. & .. & .. & ..\\
          1 & \binom{n}{n-1} & .. & \binom{2n-3}{n-1}
    \end{vmatrix} = \begin{vmatrix}
          1 &  \binom{1}{0} & .. & \binom{n-2}{0} \\
          1 & \binom{2}{1} & .. & \binom{n-1}{1}  \\
          .. & .. & .. & ..\\
          1 & \binom{n-1}{n-2} & .. & \binom{2n-4}{n-2}
    \end{vmatrix} \\
    & = \begin{vmatrix}
          1 &  1 & .. & 1 \\
          1 & \binom{2}{1} & .. & \binom{n-1}{1}  \\
          .. & .. & .. & ..\\
          1 & \binom{n-1}{n-2} & .. & \binom{2n-4}{n-2}
    \end{vmatrix} = X_{n-1}
\end{align*}

所以有 $X_n= X_1 = 1$

\subsubsection{$a_i b_j$}

\begin{align*}
    \begin{vmatrix}
        a_1b_1 & a_1b_2 & a_1b_3 & .. & a_1b_n \\
        a_1b_2 & a_2b_2 & a_2b_3 & .. & a_2b_n \\
        a_1b_3 & a_2b_3 & a_3b_3 & .. & a_3b_n \\
        .. & .. & .. & .. & .. \\
        a_1b_n & a_2b_n & a_3b_n & .. & a_nb_n \\
    \end{vmatrix} &= 
    a_1 \begin{vmatrix}
        b_1 & b_2 & b_3 & .. & b_n \\
        a_1b_2 & a_2b_2 & a_2b_3 & .. & a_2b_n \\
        a_1b_3 & a_2b_3 & a_3b_3 & .. & a_3b_n \\
        .. & .. & .. & .. & .. \\
        a_1b_n & a_2b_n & a_3b_n & .. & a_nb_n \\
    \end{vmatrix} 
    \\
   &= a_1\begin{vmatrix}
        b_1 & b_2 & b_3 & .. & b_n \\
        a_1b_2 - a_2b_1 & 0 & 0 & .. & 0 \\
        a_1b_3 - a_3b_1 & a_2b_3 - a_3b_2 & 0 & .. & 0 \\
        .. & .. & .. & .. & .. \\
        a_1b_n - b_1a_n & a_2b_n - b_2a_n & a_3b_n -a_nb_3 & .. & 0 \\
   \end{vmatrix} \\
   &= a_1b_n \prod_{i=1}^{n-1}(a_ib_{i+1} - a_{i+1}b_i)
\end{align*}

\subsubsection{爪型行列式}


\begin{enumerate}
    \item 第一种情况 $a_2a_3..a_n \ne 0$

\begin{align*}
    \begin{vmatrix}
        a_1 & b_2 & b_3 & .. & b_n \\
        c_2 & a_2 & 0 & .. & 0 \\
        c_3 & 0 & a_3 & .. & 0 \\
        .. & .. & .. & .. & ..  \\
        c_n & 0 & 0 & .. & a_n 
    \end{vmatrix} &=   {a_2a_3..a_n}  \begin{vmatrix}
        a_1 & \frac{b_2}{a_2} & \frac{b_3}{a_3} & .. & \frac{b_n}{a_n} \\
        c_2 & 1 & 0 & .. & 0 \\
        c_3 & 0 & 1 & .. & 0 \\
        .. & .. & .. & .. & ..  \\
        c_n & 0 & 0 & .. & 1 
    \end{vmatrix} \\
    &= a_2a_3..a_n \begin{vmatrix}
        a_1 - \sum_{i=2}^{n}\frac{b_ic_i}{a_i} & \frac{b_2}{a_2} & \frac{b_3}{a_3} & .. & \frac{b_n}{a_n} \\
        0 & 1 & 0 & .. & 0 \\
        0 & 0 & 1 & .. & 0 \\
        .. & .. & .. & .. & ..  \\
        0 & 0 & 0 & .. & 1 
    \end{vmatrix} \\
    &= a_2a_3..a_n(a_1 - \sum_{i=1}^{n}\frac{b_ic_i}{a_i})
\end{align*}

    \item 如果存在 $a_i = 0$ 我们可以用 Laplace 定理对 $a_i = 0$ 的行进行展开。

    下面不妨设 $a_2 = 0$,那么有

    \begin{align*}
     \begin{vmatrix}
        a_1 & b_2 & b_3 & .. & b_n \\
        c_2 & 0 & 0 & .. & 0 \\
        c_3 & 0 & a_3 & .. & 0 \\
        .. & .. & .. & .. & ..  \\
        c_n & 0 & 0 & .. & a_n 
    \end{vmatrix}  =     -c_2     \begin{vmatrix}
        b_2 & b_3 & .. & b_n \\
        0 & a_3 & .. & 0 \\
        .. & .. & .. & ..  \\
        0 & 0 & .. & a_n 
    \end{vmatrix} = -c_2b_2 \prod_{i=3}^{n}a_i
    \end{align*}

    同理,对于更一般的 $a_k = 0,\, k \ge 2$ 有

    \begin{align*}
     \begin{vmatrix}
        a_1 & b_2 & b_3 & .. & b_n \\
        c_2 & a_2 & 0 & .. & 0 \\
        c_3 & 0 & a_3 & .. & 0 \\
        .. & .. & .. & .. & ..  \\
        c_k & .. & a_k & .. & 0 \\
        .. & .. & .. & .. & ..  \\
        c_n & 0 & 0 & .. & a_n 
    \end{vmatrix}  =     (-1)^{k+1}c_k     \begin{vmatrix}
          b_2 & b_3 & .. & b_n \\
          a_2 & 0 & .. & 0 \\
          0 & a_3 & .. & 0 \\
         .. & .. & .. & ..  \\
          0 & 0 & .. & a_n 
    \end{vmatrix} 
    \end{align*}

    注意对于 $i < k$ 的行,余子式的对角线元素变成了 $i,j-1$ 对于 $i > k$  的行,余子式的对角线元素依然是 $i,j$
    注意到余子式的第 $k-1$ 列可以按照 $b_k$ 展开得到

    \[
        b_k(-1)^{k}a_2a_3..a_{k-1}a_{k+1}..a_n
    \]

    最终得到

    \[
        -c_kb_k a_2a_3..a_{k-1}a_{k+1}..a_n
    \]

    注意到我们第一次展开的时候有

    \begin{align*}
        i' &= \begin{cases}
            i & i < k \\
            i - 1 & i > k \\
        \end{cases} \\
        j' &= j-1 \\
    \end{align*}

    第二次展开的时候有


    \begin{align*}
        i' &= i-1 \\
        j' &= \begin{cases}
            j & j < k-1 \\
            j - 1 & j > k-1 \\
        \end{cases}
    \end{align*}

    把这两个变换作复合得到

    \begin{align*}
        i' &= \begin{cases}
            i-1 & i < k \\
            i - 2 & i > k \\
        \end{cases} \\
        j' &= \begin{cases}
            j - 1& j < k \\
            j - 2 & j > k \\
        \end{cases}
    \end{align*}

    可以看到两次变换后 $i = j$ 可以推断出 $i'=j'$ 也就是对角线元素保持不变。

    \item 转化成爪型行列式

    \begin{align*}
        \begin{vmatrix}
            x_1 - a_1 & x_2 & x_3 & .. & x_n \\
            x_1 & x_2 -a_2 & x_3 & .. & x_n \\
            x_1 & x_2 & x_3 - a_3& .. & x_n \\
            ..& ..&..&..&.. \\
            x_1 & x_2 & .. & .. & x_n -a_n 
        \end{vmatrix} &=         \begin{vmatrix}
            x_1 - a_1 & x_2 & x_3 & .. & x_n \\
            a_1 & -a_2 & 0 & .. & 0 \\
            a_1 & 0 &  - a_3& .. & 0 \\
            ..& ..&..&..&.. \\
            a_1 & 0 & .. & .. &  -a_n 
        \end{vmatrix} \\
        &= (-1)^{n-1}a_2a_3..a_n(x_1 - a_1 + a_1\sum_{i=2}^{n}\frac{x_i}{a_i})
    \end{align*}

\end{enumerate}

\subsubsection{直接展开}

如下,按照第一行展开

\begin{align*}
    \begin{vmatrix}
        a & 0 & .. & 0 & 1 \\
        0 & a & .. & 0 & 0 \\
        .. & .. & .. & .. & .. \\
        0 & 0 & .. & a & 0 \\
        1 & 0 & .. & 0 & a \\
    \end{vmatrix} = a^n + (-1)^{n+1}(-1)^{n}a^{n-2} = a^n - a^{n-2}
\end{align*}

另一个例子

假设 $\lvert A \rvert = \lvert a_{ij}\rvert$ 是一个 $n$ 阶行列式,$A_{ij}$ 是它的第 $i,j$ 元素的代数余子式,那么有

\[
\begin{vmatrix}
    a_{11} & a_{12} & .. & a_{1n} & x_1 \\ 
    a_{21} & a_{22} & .. & a_{2n} & x_2 \\ 
    .. & .. &.. &.. &.. \\
    a_{n1} & a_{n2} & .. & a_{nn} & x_n \\ 
    y_1 & y_2 & .. & y_n & z \\ 
\end{vmatrix} = z \cdot \text{det} (A) - \sum_{i=1}^{n} \sum_{j=1}^{n}A_{ij}x_i y_j
\]

先按最后一列展开,然后再按最后一行展开得到

\begin{align*}
& = z \text{det}(A) + \sum_{i=1}^{n}(-1)^{n+i+1}x_i\sum_{j=1}^{n}(-1)^{n+j}y_jM_{ij} \\
& = z \text{det}(A) + \sum_{i=1}^{n}(-1)^{i+j+1}\sum_{j=1}^{n}x_iy_iM_{ij} \\
& = z \text{det}(A) - \sum_{i=1}^{n}\sum_{j=1}^{n}x_iy_iA_{ij} \\
\end{align*}

\section{实变函数习题}

\subsection{映射与基数}

\subsubsection{一一映射(1)}

$f: \mathbb{R} \to \mathbb{R}$ 记 $f_1 = f$,$f_{n}(x) = f(f_{n-1}(x))$,若存在 $n_0$ 满足 $f_{n_0}(x) = x$ ,
那么$f$ 是 $\mathbb{R}$ 到 $f(\mathbf{R})$ 上的一一映射。

先证明 $f$ 是满射,那么存在 $c \in \mathbb{R}$ 满足 $c \notin f(\mathbb{R})$,以此类推可以得到 $f_{n+1}(\mathbb{R}) = f(f_{n}(\mathbb{R})) \subseteq f(\mathbb{R})$,
但这个显然和 $f_{n_0}(c) = c$ 矛盾。

继续证明 $f$ 是单射,假设存在 $x_1 \ne x_2$ 满足 $x_1 = x_2$,用数学归纳可以证明 $f_{n_0}(x_1)  = f_{n_0}(x_2)$,这个显然和 $f_{n_0}(x) = x$ 矛盾。


\subsubsection{一一映射(2)}

不存在 $\mathbb{R}$ 上的连续函数,它在无理数 $\mathbb{R} \setminus \mathbb{Q}$ 上是一一映射,而在 $\mathbb{Q}$ 上不是一一映射。
假设存在 $q_1 < q_2,\, q_1,\, q_2 \in \mathbb{Q}$ 满足 $f(q_1) = f(q_2)$

因为 $f$ 在 $[q_1, q_2]$ 上连续,我们可以用最值原理,至少能找到一个不等于 $f(q_1)$ 的最值。因为如果最大和最小值都等于 $f(q_1)$ 就和 $f$ 在无理数集上单射矛盾。

不妨设 $f$ 在 $[q_1, q_2]$ 上取到最大值 $f(c) = M > f(q_1)$,我们分析 $(f(q_1), M)$ 之间的数,根据介值定理,对任意 $d \in (f(q_1), M)$ 都存在
$x_1 \in [q_1, c]$ 和 $x_2 \in [c, q_2]$ 满足 $f(x_1) = f(x_2) = d$

根据我们的条件 $x_1$ 和 $x_2$ 都是有理数,这样 $f$ 就能够把一个有理数集的子集$A$映射到一个区间上 $(f(q_1), M) \subseteq f(A)$,
这显然和 $f(A)$ 是可数集矛盾了

\end{document}