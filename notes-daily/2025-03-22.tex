%!LW recipe=latexmk (xelatex)
% a4 页面 字体大小12像素
\documentclass[12pt,a4paper]{ctexart}
% math 
\usepackage{amsmath,amsfonts,amssymb,amsthm}
% cross reference, use \autoref instead of \ref
\usepackage{aliascnt}
\usepackage[hidelinks]{hyperref}
\usepackage{enumitem}
\usepackage{geometry}


\geometry{left=1.5cm, right=1.5cm, top=2cm, bottom=2cm}

\newtheorem{thm}{Theorem}[section]

\newaliascnt{lem}{thm}
\newaliascnt{prop}{thm}
\newaliascnt{definition}{thm}
\newaliascnt{exercise}{thm}
\newaliascnt{corollary}{thm}

\theoremstyle{definition}
\newtheorem{lem}[lem]{Lemma}
\newtheorem{prop}[prop]{Proposition}
\newtheorem{definition}[definition]{Definition}
\newtheorem{exercise}[exercise]{Exercise}
\newtheorem{corollary}[corollary]{Corollary}

\def\lemautorefname{Lemma}
\def\thmautorefname{Theorm}
\aliascntresetthe{lem}
\aliascntresetthe{prop}
\aliascntresetthe{definition}
\aliascntresetthe{exercise}
\aliascntresetthe{corollary}


\title{泰勒级数分析}
\author{朱英杰}
\date{\today}


\begin{document}
\zihao{-4}

\maketitle

\section{Definitions}

\subsection{$n$ 阶可微}

数学上通常用 $C^n$ 表示 $n$ 阶连续可微,例如 $C^0$ 表示连续,$C^1$ 表示可微并且导函数连续,$C^2$ 表示二阶可微并且二阶导函数连续。

\subsection{拉格朗日余项}

若函数 $f: [a,b] \to \mathbb{R}$ 满足 $n+1$ 阶连续可微,那么取 $x_0 \in [a,b]$,$f(x)$ 可以近似表示为

\[
f(x) = \sum_{k=0}^{n}\frac{1}{k!}f^{(k)}(x_0)(x-x_0)^k + \frac{1}{(n+1)!}f^{(n+1)}(\xi)(x-x_0)^{k},\, \lvert \xi - x_0 \rvert \le \lvert x - x_0 \rvert
\]

其中

\[
\frac{1}{k!}f^{(k)}(\xi)(x-x_0)^{k}
\]

被称作拉格朗日余项,这个定理可以把 $n$ 阶连续可微的函数表示成多项式加上一个关于 $\xi$ 的函数,
而 $\xi$ 的取值和 $x$ 有关。

\section{证明}

记 
\[
R(t) = f(t) - \sum_{k=0}^{n}\frac{1}{k!}f^{(k)}(x_0)(t-x_0)^k
\]

对 $R(t)$ 和 $(t-x_0)^{n+1}$ 在 $x_0$ 和 $x$ 用中值定理,注意这里有

\[
\lvert(t-x_0)^n \rvert > 0,\, \forall t \ne x_0
\]

这保证了我们可以安全地用中值定理。

\[
\frac{R(x)-R(x_0)}{(x-x_0)^{n+1}} = \frac{R'(\xi_1)}{(n+1)(\xi_1-x_0)^n}
\]

注意到 $R'(x_0) = 0$
继续对 $R'(t)$ 和 $(t-x_0)^{n}$ 在 $x_0$ 和 $\xi_1$ 用中值定理得到

\[
\frac{R'(\xi_1)}{(n+1)(\xi_1-x_0)^n} = \frac{R^{(2)}(\xi_2)}{(n+1)n(\xi_2-x_0)^{n-1}}
\]

以此类推得到

\[
\frac{R(x)}{(x-x_0)^{n+1}} = \frac{R^{(n)}(\xi_n)}{(n+1)!(\xi_n - x_0)} =\frac{f^{(n)}(\xi_n)- f^{(n)}(x_0)}{(n+1)!(\xi_n - x_0)}
\]

再用一次
得到


\[
\frac{R(x)}{(x-x_0)^{n+1}}  =\frac{f^{(n+1)}(\xi_{n+1})}{(n+1)!}
\]


\end{document}
