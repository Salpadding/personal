%!LW recipe=latexmk
\documentclass[11pt,a4paper]{article}
\input% 数学公式
\usepackage{amsmath}
\usepackage{amsfonts}
\usepackage{amssymb}
\usepackage[hidelinks]{hyperref}
\usepackage{enumitem}
\usepackage{geometry}


\geometry{left=1.5cm, right=1.5cm, top=2cm, bottom=2cm}

\title{Learning Notes}
\author{Yingjie Zhu}
\date{\today}


\begin{document}
\maketitle

\section{Theory of Real Variable Function}

\subsection{Exercise}

\begin{lem}
    $E$ is a perfect set iff it is closed and contains no isolated points.
\end{lem}

\begin{proof}
    let $E$ be a perfect set such that $E = E'$, then we have

    \[
        \overline{E} = E \cup E' = E
    \]

    so $E$ is closed, and isolated points have

    \[
        E \setminus E' = E \setminus E = \emptyset
    \]

    On the other hand, if $E$ is closed and has no isolated points, then we have

    \begin{align*}
        E \setminus E' &= \emptyset \\
        E & \subseteq E' \\ 
        E \cup E' &= E \\
        E' & \subseteq E
    \end{align*}
\end{proof}


\begin{lem}
    let $x$ be a limit point of $E$, $E \subseteq \mathbb{R}^n$, and $y \in E$, then $x$ is also a limit point of $E \setminus y$ 
\end{lem}

\begin{proof}
    since $x$ is a limit point of $E$, there should exists a distinct sequence $x_n$ and

    \[
        \lim_{n \to \infty}x_n = x
    \]

    then we can construct a sequence of $E \setminus \{ y \}$ by remove $y$ of $x_n$. Since $x_n$ 
    is distinct, so $y$ occurs at most once. So $x$ is a limit point of $E \setminus \{ y \}$
\end{proof}

\begin{exercise}
    Perfect Set on $\mathbb{R}^n$ is not countable. 
\end{exercise}

\begin{proof}
    let $E$ be a non empty perfect set and assume it is countable so that

    \[
        E = \{ x_1, x_2, .. \}
    \]

    since every finite set under $\mathbb{R}^n$ has isolated points, so $E$ cannot be finite.

    we first pick $y_1 \in E \setminus \{ x_1 \}$, since $y_1$ is a limit point of $E \setminus \{ x_1 \}$. 
    there should exists an open qube $Q_1$ conatins $y_1$, and $x_1 \notin \overline{Q_1}$

    let's consider $E \setminus \{ x_2 \}$, since $y_1$ is limit point of $E \setminus \{ x_2 \}$, and $Q_1$ is a openset contains $y_1$ so
    we have

    \[
       \text{int}(Q_1) \cap \left( E \setminus \{ x_2 \} \right) \ne \emptyset
    \]

    now we can select $y_2 \in \text{int}(Q_1) \cap E \setminus \{ x_2 \}$, and we can create open cube $Q_2$ contains 
    $y_2$, so that $Q_2 \subseteq Q_1$, and $\{ x_1, x_2 \} \cap \overline{Q_2} = \emptyset$

    let's consider $E \setminus \{ x_3 \}$, since $y_2$ is a limit point of $E \setminus \{ x_3 \}$, and $Q_2$ is an
    open set contains $y_2$, so we have

    \[
       \text{int}(Q_2) \cap \left( E \setminus \{ x_3 \} \right) \ne \emptyset
    \]

    now we can select $y_3 \in \text{int}(Q_2) \cap \left( E \setminus \{ x_3 \} \right)$
    as $Q_2$ is open, and $\{ x_1, x_2 \} \cap Q_2 = \emptyset$. so we can create 
    a smaller qube $Q_3$ contains $y_3$ so that $\{x_1 ,x_2, x_3 \} \cap \overline{Q_3} = \emptyset$

    so we can create a sequence $y_n$ and open qube $Q_n$ contains $y_n$, and also a 
    sequence of closed cube $\overline{Q_n}$

    let's consider $\overline{Q_n}$, since $\overline{Q_n}$ is closed and non empty, by cantor's theorm,
    we have

    \[
        \bigcap_{n=1}^{\infty}\overline{Q_n} \ne \emptyset
    \]

    howevery, for any $x_n \in E$, the qube $\overline{Q_{n}}$ not contains $x_n$,
    which is contradict.
\end{proof}

\end{document}









