%!LW recipe=latexmk (xelatex)
% a4 页面 字体大小12像素
\documentclass[12pt,a4paper]{ctexart}
\input% math 
\usepackage{amsmath,amsfonts,amssymb,amsthm}
% cross reference, use \autoref instead of \ref
\usepackage{aliascnt}
\usepackage[hidelinks]{hyperref}
\usepackage{enumitem}
\usepackage{geometry}


\geometry{left=1.5cm, right=1.5cm, top=2cm, bottom=2cm}

\newtheorem{thm}{Theorem}[section]

\newaliascnt{lem}{thm}
\newaliascnt{prop}{thm}
\newaliascnt{definition}{thm}
\newaliascnt{exercise}{thm}
\newaliascnt{corollary}{thm}

\theoremstyle{definition}
\newtheorem{lem}[lem]{Lemma}
\newtheorem{prop}[prop]{Proposition}
\newtheorem{definition}[definition]{Definition}
\newtheorem{exercise}[exercise]{Exercise}
\newtheorem{corollary}[corollary]{Corollary}

\def\lemautorefname{Lemma}
\def\thmautorefname{Theorm}
\aliascntresetthe{lem}
\aliascntresetthe{prop}
\aliascntresetthe{definition}
\aliascntresetthe{exercise}
\aliascntresetthe{corollary}


\title{学习记录}
\author{朱英杰}
\date{\today}

\begin{document}
\zihao{-4}
\maketitle
\tableofcontents

\section{初等数论}

\subsection{不定方程}

\subsubsection{定理}

不定方程 

\[
a_1x_1 + a_2x_2 + .. a_nx_n = b
\]

有解的充分必要条件是

\[
(a_1,a_2,..,a_n) \vert b
\]

进而,令 $d = (a_1,a_2,..a_n)$ 那么该方程和

\[
\frac{a_1}{d}x_1 + \frac{a_2}{d}x_1 + .. + \frac{a_n}{d}x_n = \frac{b}{d}
\]

同解。

证明:

\begin{enumerate}
    \item 充分性

    因为最大公约数是线性组合,所以存在 $x_1', x_2', .. x_n'$ 满足

    \[
    a_1x_1' + a_2x_2' + .. a_n x_n' = d
    \]

    若 $d \vert b$ 因此有

    \[
    \frac{d}{b}a_1x_1' + \frac{d}{b}a_2x_2' + .. \frac{d}{b}a_n x_n' = b
    \]

    即有解 $x_i = dx_i'/b$

    \item 必要性
    
    可以看到 $b$ 是 $a_1,a_2,..,a_n$ 的线性组合,根据公约数的性质有

    \[
        d \vert b
    \]

\subsubsection{通解推导}

若 $(a_1,a_2) = 1$,而且不定方程有一个特解 $x_1', x_2'$

\[
a_1x_1 + a_2x_2 = b
\]

首先我们可以观察到左边,令 $f(x_1,x_2) = a_1x_1 + a_2x_2$,
显然有 $f(x_1-a_2, x_2 + a_1) = f(x_1, x_2)$,所以我们可以猜测通解可以表示为一个特解加上一项的形式

\begin{align*}
x_1 = x_1' - a_2 t \\
x_2 = x_2' + a_1 t \\
\end{align*}

其中 $t \in \mathbb{Z}$

下面给出证明

\begin{enumerate}
    \item 充分性

    易证

    \item 必要性

    若有 $a_1x_1 + a_2x_2 = b$,我们得到 $a_1(x_1 - x_1') = -a_2(x_2 - x_2')$,所以有 $a_1 \vert x_2-x_2'$,
    于是令 $x_2 - x_2' = t a_1$,得到 $a_1(x_1 -x_1') = -ta_1a_2$,利用消去律得到 $x_1 - x_1' = -ta_2$
\end{enumerate}

对于更一般的情况,即 $a_1, a_2$ 不互素时,可以转化成互素的方程。



\end{enumerate}

\subsubsection{辗转相除解不定方程}

\begin{align*}
    15x_1 + 10x_2 + 6x_3 &= 61 \\
    6x_3 &= 61 - 15x_1 -10x_2 \\
    x_3 &= 10 - 2x_1 -2x_2 + \frac{1}{6}\left(1 - 3x_1 + 2x_2\right) \\
\end{align*}

于是得到不定方程

\[
6x_4 = 1 - 3x_1 + 2x_2
\]

用同样的方法得到

\[
x_2 = 3x_4 + x_1 + \frac{1}{2}\left(x_1-1\right)
\]

转化为方程

\[
2x_5 - x_1 = -1
\]

易得到解为

\begin{align*}
    x_1 &= 1 + 2x_5  \\
    x_2 &= 3x_4 + 1 + 2x_5 + x_5 = 3x_4 + 3x_5 + 1 \\
    x_3 &= 10 -2(3x_4 + 5x_5 + 2) + x_4 \\
\end{align*}

\section{实变函数论}

\subsection{点集拓扑}

构建一个在有理数点上间断,在无理数点上连续的函数。

构造 $f: (a,b) \to \mathbb{R}$,记 $(a,b)$ 上所有的有理数为 $q_1,q_2,..,q_n,..$,定义 $f$ 为

\[
f(x) = \sum_{q_n < x} \frac{1}{n^2}
\]

我们先证明 $f$ 在有理点上间断,取有理数 $q = q_N$,则

下面开始用 $f(x+0)$ 表示在$x$ 处的右极限,用 $f(x-0)$ 表示在$x$处的左极限

\begin{align*}
f(q + 0) &\ge f(q) + \frac{1}{N^2}  \\
f(q - 0) & \le f(q) \\
f(q+0) - f(q-0) &\ge \frac{1}{N^2}
\end{align*}

所以 $f$ 在有理点间断。

我们继续证明 $f$ 在无理点连续,为此我们要先证明一个引理,取任意实数 $r$,令

\[
\alpha(n) = \min \{ l \,\vert\, r < q_l < r + \frac{1}{n}\}
\]

显然 $\alpha(n+1) \ge \alpha(n)$,我们将用反证法证明

\[
\lim_{n \to \infty} \alpha(n) = \infty
\]

假设 $\alpha(n)$ 有界,那么存在正整数 $M$,满足对任意的 $n$ 都有

\[
\{ q_m \,\vert\, m \le M \} \cap (r, r+\frac{1}{n}) \ne \emptyset
\]

这显然是不可能的,我们可以对所有 $1 \le m \le M$ 计算 $\lvert q_m - r\rvert$ 的最小值,然后取一个足够大的 $n$ 即可让它们交集是空集。

利用 $\alpha(n)$ 可以计算 $f(r+0)$,注意到 

\[
f(r+\frac{1}{n}) - f(r) \le \sum_{j=\alpha(n)}^{\infty}\frac{1}{j^2}
\]

两边同时取极限得到

\[
\lim_{n \to \infty}f(r + \frac{1}{n}) \le f(r)
\]

利用 $f$ 的单调性,和两边夹法则得到

\[
f(r+0) = f(r)
\]

同理我们可以证明 $f(r-0) = f(r)$

\section{高等代数}

\subsection{组合定义}

行列式的组合定义为,若  $|A|$ 是 $n$ 阶行列式

\[
|A| = \sum_{\sigma \in S_n} (-1)^{\sigma} a_{1\sigma(1)}a_{2\sigma(2)}..a_{n\sigma(n)}
\]

其中 $S_n$ 是 $n$ 阶对称群

\[
(-1)^{\sigma} = \begin{cases}
    -1 & \sigma \text{\quad is odd} \\
    1 & \sigma \text{\quad is even} \\
\end{cases}
\]

\subsection{行列式函数求导}

\begin{align*}
    F(t) = \begin{vmatrix}
        f_{11}(t) & f_{12}(t) & .. & f_{1n}(t) \\
        f_{21}(t) & f_{22}(t) & .. & f_{2n}(t) \\
        .. & .. &.. &..  \\
        f_{n1}(t) & f_{n2}(t) & .. & f_{nn}(t) \\
    \end{vmatrix}
\end{align*}

于是有

\begin{align*}
    F(t) &= \sum_{\sigma \in S_n} (-1)^{\sigma} \prod_{j=1}^{n}f_{\sigma(j)j} \\
    F'(t) &= \sum_{\sigma \in S_n} (-1)^{\sigma} \sum_{j=1}^{n}f'_{\sigma(j)j}\prod_{k \ne j}f_{ \sigma(k)k} \\
    &= \sum_{j=1}^{n}\sum_{\sigma \in S_n} (-1)^{\sigma} f'_{\sigma(j)j}\prod_{k \ne j}f_{\sigma(k)k} \\
    &= \sum_{j=1}^{n}F_j
\end{align*}

其中 

\[
F_j = \begin{vmatrix}
    f_{11}(t) & f_{12}(t) & .. & f'_{1j}(t) & .. & f_{1n}(t) \\
    f_{21}(t) & f_{22}(t) & .. & f'_{2j}(t) &.. & f_{2n}(t) \\
    .. & .. &.. &.. & .. \\
    f_{n1}(t) & f_{n2}(t) & .. & f'_{nj}(t) & .. & f_{nn}(t) \\
\end{vmatrix}
\]

\subsection{计算多项式系数}

计算下列多项式的最高次项和 $n-1$ 次项的系数

\[
f(x) = \begin{vmatrix}
    x-a_{11} & -a_{12} & .. & -a_{1n} \\
    -a_{21} & x-a_{22} & .. & -a_{2n} \\
    .. & .. & .. & .. \\
    -a_{n1} & -a_{n2} & .. & x-a_{nn} \\
\end{vmatrix}
\]

最高次项系数为 $1$,$n-1$ 次项系数为

\[
-(a_{11} + a_{22} + .. + a_{nn})
\]

我们分析那些组合可以生成至少 $n-1$ 次的多项式

考虑对称群 $S_n$ 上的置换 $\sigma$,如有至少  $n-1$ 个不同的整数 $i_1,i_2,..,i_{n-1}$满足 $\sigma(i_1) = i_1, \sigma(i_2) = i_2, .., \sigma(i_{n-1}) = i_{n-1}$
根据抽屉原理,那么 $\sigma$ 必然是恒等映射,所以只要对角线上会出现 $x^{n-1}$ 项。

\subsection{反对称行列式}

证明奇数阶反对称行列式等于 $0$,反对称行列式有 $a_{ij}= -a_{ji}$

根据反对称行列式的性质有

\begin{align*}
    |A| &= \sum_{\sigma \in S_n}(-1)^{\sigma}\prod_{i=1}^{n}a_{i \sigma(i)} \\
    &= \sum_{\sigma \in S_n}(-1)^{\sigma}(-1)^n\prod_{i=1}^{n}a_{\sigma(i)i} \quad \text{hint:\:} n \:\text{is odd} \\
    &= (-1)\sum_{\sigma \in S_n}(-1)^{\sigma}\prod_{i=1}^{n}a_{\sigma(i)\sigma^{-1}(\sigma(i))} \\
    &= (-1)\sum_{\sigma \in S_n}(-1)^{\sigma}\prod_{i=1}^{n}a_{i\sigma^{-1}(i)} \\
    &= (-1)\sum_{\sigma \in S_n}(-1)^{\sigma^{-1}}\prod_{i=1}^{n}a_{i\sigma^{-1}(i)} \\ 
    &= - |A|
\end{align*}

注意到 $\sigma$ 是置换,所以对任意  $f: \mathbb{N} \to R$,注意这里 $R$ 代表一个交换环, 有 $f(\sigma(1)) + f(\sigma(2)) + .. + f(\sigma(n)) = f(1) + f(2) + .. + f(n)$
和 $f(\sigma(1))f(\sigma(2))..f(\sigma(n)) = f(1)f(2)..f(n)$

\subsection{整数行列式}

$|A|$ 每个元素都是整数且 $|A|$ 能整除它的任意元素,证明 $|A|^2 =1 $

\begin{align*}
    |A| = \sum_{\sigma \in S_n}(-1)^{\sigma}\prod_{i=1}^{n}a_{i \sigma(i)}
\end{align*}

可以看到展开后的每一项都是元素的线性组合,所有 $|A|^n$ 整除$ |A|$,所以 $|A|^2$ 只能是 $1$

\end{document}


