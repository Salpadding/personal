%!LW recipe=latexmk (xelatex)
% a4 页面 字体大小12像素
\documentclass[12pt,a4paper]{ctexart}
\input% math 
\usepackage{amsmath,amsfonts,amssymb,amsthm}
% cross reference, use \autoref instead of \ref
\usepackage{aliascnt}
\usepackage[hidelinks]{hyperref}
\usepackage{enumitem}
\usepackage{geometry}


\geometry{left=1.5cm, right=1.5cm, top=2cm, bottom=2cm}

\newtheorem{thm}{Theorem}[section]

\newaliascnt{lem}{thm}
\newaliascnt{prop}{thm}
\newaliascnt{definition}{thm}
\newaliascnt{exercise}{thm}
\newaliascnt{corollary}{thm}

\theoremstyle{definition}
\newtheorem{lem}[lem]{Lemma}
\newtheorem{prop}[prop]{Proposition}
\newtheorem{definition}[definition]{Definition}
\newtheorem{exercise}[exercise]{Exercise}
\newtheorem{corollary}[corollary]{Corollary}

\def\lemautorefname{Lemma}
\def\thmautorefname{Theorm}
\aliascntresetthe{lem}
\aliascntresetthe{prop}
\aliascntresetthe{definition}
\aliascntresetthe{exercise}
\aliascntresetthe{corollary}


\title{学习记录}
\author{朱英杰}
\date{\today}

\begin{document}
\zihao{-4}
\maketitle
\tableofcontents

\section{组合数学}

\subsection{线性递推方程}

\[
h_n = a_1h_{n-1} + a_2h_{n-2} + .. +a_kh_{n-k}
\]

不难得到,若 $x_n$ 满足方程,那么 $c x_n$ 也满足,同理,若 $x_n,\, y_n$ 满足,那么 $x_n + y_n$ 也满足,
所以该方程的解可以用几个特解的线性组合表示。

\subsection{特征方程}

特解可以用如下方程求解

\[
x^k = a_1x^{k-1} + a_2x^{k-2} + .. + a_k
\]

若以上方程有实根 $q$,那么 $x = q^n$ 就是递推方程的特解。若以上方程有根 $q_1,q_2,..,q_m$,
那么通解可以表示为

\[
c_1q_1^n + c_2q_2^n + .. + c_mq_m^n
\]

若有 $k$ 个不同实根,那么给定初值条件 $h_0, h_1, .., h_{k-1}$,如下的方程一定有解

\begin{align*}
    h_0 &= x_1 + x_2 + .. + x_k \\
    h_1 &= x_1q_1 + x_2q_2 + .. + x_kq_k \\
    .. & .. \\
    h_{k-1} &= x_1q_1^{k-1} + x_2q_2^{k-1} + .. + x_kq_k^{k-1} \\
\end{align*}

因为上面系数对应的行列式是 Vandermonde 行列式

\section{实变函数}

\subsection{点集拓扑}

\subsubsection{极限点}

若 $E \subseteq \mathbb{R}^n$,且 $x \in \mathbb{R}^n$。 若存在 $E$ 中互异的点列 $\{ x_n \}$ 满足

\[
\lim_{n \to \infty} \lvert x_n - x \rvert = 0
\]

那么称 $x$ 是 $E$ 的极限点

注意给定集合 $E$,它的极限点不一定包含于 $E$ 中。

\subsubsection{定理}

若 $E \subseteq \mathbb{R}^n$ 那么 $x$ 是 $E$ 的极限点等价于如下命题

\[
\bigcap_{n = 1}^{\infty} \left( B(x, \frac{1}{n}) \setminus \{ x \} \right) \cap E \ne \emptyset \quad B(x,\frac{1}{n}) = \{ y \, \vert \, \lvert y-x\rvert < \frac{1}{n} \}
\]


证明:

\begin{enumerate}
    \item 充分性

    我们将递归地构造一个互异的点列,首先取 $y_0 \in E,\, y_0 \ne x$ 满足 $\lvert y_0 -x \rvert < 1$。根据命题,这个 $y_0$ 一定是可以取到的,
    只需要将 $n=1$ 代入即可。因为 $y_0 \ne x$,那么一定存在一个足够大的自然数 $N_1$ 满足  

    \[
        \frac{1}{N_1} \le \lvert y_0 - x\rvert < \frac{1}{1}
    \]

    我们把这个 $N_1$ 代入命题,可以去到一个离 $x$ 更近的点 $y_1$。然后以此类推下去可以得到 $y_2,\, y_3, ..$,
    注意到有

    \[
    \lvert y_{m} -x \rvert <\lvert y_{m-1} -x \rvert < .. < \lvert y_{1} -x \rvert < \lvert y_{0} -x \rvert
    \]

    所以 $y_0,y_1, ..$ 都是互异的点,而且根据数学归纳法可以证明 $\lvert y_{m} - x\rvert < \frac{1}{m+1}$:

    \[
        \lvert y_{m+1} - x\rvert < \frac{1}{N_{m+1}} \le \lvert y_m -x \rvert < \frac{1}{m+1}
    \]

    因为 $N_{m+1}$ 是整数,所以有

    \[
        \frac{1}{N_{m+1}} \le \frac{1}{(m+1) + 1}
    \]

    所以

    \[
        \lvert y_{m+1} - x\rvert < \frac{1}{(m+1) + 1}
    \]

    所以我们构造了一个收敛到 $x$ 的互异的点列 $y_n$

    \item 必要性

    用极限的性质即可证明,互异的点列至多只包含一项等于 $x$,我们可以去掉这一项得到一个每项都不为 $x$
    且收敛到 $x$ 的点列。

\end{enumerate}

\subsubsection{度量}

$\mathbb{R}^n$ 上的度量有如下关系

\begin{align*}
    l_1 & \le l_2 \le \sqrt{n} l_1 \\
    l_{\infty} & \le l_1 \le n l_{\infty}
\end{align*}

\subsubsection{Bolzano-Weierstrass Theorem}

$\mathbb{R}^n$ 中任意一个无限且有界的点集至少有一个极限点。

$\mathbb{R}$ 上有界序列一定有收敛的子列,$\mathbb{R}^n$ 同样有这样的定理。
根据 $l_{\infty}$ 和 $l_2$ 的等价性,集合 $E \subseteq \mathbb{R}^n$ 如果有界,那么它的每个点 $x = (x_1,x_2,..,x_n)$,每个点各自的 $x_1,x_2,..,x_n$ 也都是有界的。

我们取 $E$ 中的任意一个互异的点列 $x_n$,因为 $E$ 是无限集,所以我们可以这么做。

我们先考虑 $x_1$,因为 $x_1$ 有界,我们可以利用实数集上的 Bolzano-Weierstrass Theorem 得到子列 $x_{f(n)}$,
对这个子列有 $\langle x_{f(n)}, \mathbf{e}_1 \rangle$ 收敛。以此类推我们可以找到子列 $g(f(n))$ 满足
$\langle x_{g(f(n))}, \mathbf{e}_1 \rangle$ 和 $\langle x_{g(f(n))}, \mathbf{e}_2 \rangle$ 都收敛。

重复 $n$ 次后我们就得到了收敛的子列 $x_{h(n)}$,它对 $1 \le i \le n$ 都有 $\langle x_{h(n)}, \mathbf{e}_i\rangle$ 收敛。

\subsubsection{闭集的定义}

闭集通常有两种定义,第一种将闭集定义为开集的补集,第二种定义为包含所有极限点的集合。如果用邻域
定义开集和极限点,那么这两种定义是等价的,下面给出证明。

\begin{enumerate}
    \item 开集的补集包含了所有极限点

    假设某个开集 $X$ 的补集 $X^C$,它有不在集合内的极限点 $x,\, x \notin X^C$,根据补集的性质,那么有 $x \in X$。
    因为开集的所有点都是内点,所以存在邻域 $B(x, r)$ 满足 $B(x,r) \cap X^C = \emptyset$,这个和 $x$ 是 $X^C$ 极限点矛盾了。

    \item 包含了所有极限点的集合,它的补集是开集

    令 $X$ 包含了所有极限点,令 $x \in X^C$。显然 $x$ 不是 $X$ 的极限点,所以存在 $x$ 去心邻域 $B(x, r) \setminus \{ x \}$,满足
    $ \left( B(x, r) \setminus \{ x \} \right) \cap X = \emptyset$,所以有 $\left( B(x, r) \setminus \{ x \} \right) \in X^C$。
\end{enumerate}

\section{高等代数}

\subsection{矩阵乘法}

矩阵乘法的行向量视角 $A: \mathbb{R}^{m \times n},\, B: \mathbb{R}^{n \times k}$, $AB$ 可以理解为

\begin{align*}
    A &= \begin{bmatrix}
        \mathbf{x}_1^T \\
        \mathbf{x}_2^T \\
        .. \\
        \mathbf{x}_m^T \\
    \end{bmatrix}\quad
    B = \begin{bmatrix}
        \mathbf{y}_1^T \\
        \mathbf{y}_2^T \\
        .. \\
        \mathbf{y}_n^T \\
    \end{bmatrix} \\
    AB &= \begin{bmatrix}
        c_{11}\mathbf{y}_1 + c_{12}\mathbf{y}_2 + .. +  c_{1n}\mathbf{y}_n \\
        c_{21}\mathbf{y}_1 + c_{22}\mathbf{y}_2 + .. +  c_{2n}\mathbf{y}_n \\
         .. \\
        c_{m1}\mathbf{y}_1 + c_{m2}\mathbf{y}_2 + .. +  c_{mn}\mathbf{y}_n \\
    \end{bmatrix}
\end{align*}

\subsection{矩阵乘法与行列式}

若 $A,B \in \mathbb{R}^{n \times n}$,那么有

\[
\det(AB) = \det(A) \det(B)
\]

证明:

\begin{align*}
    |A||B| = \begin{vmatrix}
        A & O \\
        I_n & B \\
    \end{vmatrix} =     \begin{vmatrix}
        O & -AB \\
        I_n & B \\
    \end{vmatrix}  =  (-1)^{n^2} (-1)^n|AB| = |AB|
\end{align*}

因为 $n^2+n = n(n+1)$ 必然是一个偶数

以上行变化可以理解为,将第 $n+1$ 到 $2n$ 行进行线性组合后加到第 $1$ 到 $n$ 行。线性组合后得到的矩阵为

\begin{align*}
    B &= \begin{bmatrix}
        \beta_1^T \\
        \beta_2^T \\
        .. \\
        \beta_n^T \\
    \end{bmatrix} \\
    -AB &= -\begin{bmatrix}
        a_{11}\beta_1^T + a_{12}\beta_2^T + .. + a_{1n}\beta_n^T \\
        a_{21}\beta_1^T + a_{22}\beta_2^T + .. + a_{2n}\beta_n^T \\
        .. \\
        a_{n1}\beta_1^T + a_{n2}\beta_2^T + .. + a_{nn}\beta_n^T \\
    \end{bmatrix}
\end{align*}

\section{Cauchy-Binet Formula}

Proof: let $m < n,\, A \in \mathbb{R}^{m \times n},\, B \in \mathbb{R}^{n \times m}$, by Laplace theorm, we have:

let 

\begin{align*}
    \alpha &= (n+1) + (n+2) +  .. + (n+m) + 1 + 2 + .. + m  \\
    &= 2\binom{m+1}{2} + nm \\
    &= nm 
\end{align*}

只考虑奇偶性可以去掉偶数项

\begin{align*}
    C = \begin{vmatrix}
        A_{m \times n} & O_{m}  \\
       I_n & B_{n \times m} \\
    \end{vmatrix}  &= \begin{vmatrix}
        O_{m \times n} & -AB  \\
       I_n & B_{n \times m}  \\
    \end{vmatrix}= (-1)^{\alpha + m} |AB| =(-1)^{nm + m} |AB|
\end{align*}

we also expand $C$ by first m rows.

let 

\[
\beta=1+2+.. +m + j_1+j_2 + .. +j_m = \binom{m+1}{2} + j_1 + j_2 + .. + j_m
\],

also let

\[
\gamma = 1+2+.. + (n-m) + i_1 + i_2 + .. + i_{n-m}
\]

\begin{align*}
    |C| &= \sum_{1 \le j_1 < j_2 .. < j_m \le n}A \begin{bmatrix}
        1 & 2 & .. & m \\
        j_1 & j_2 & .. & j_m \\
    \end{bmatrix}(-1)^{\beta} \begin{vmatrix}
       \mathbf{e}_{i_1} & \mathbf{e}_{i_2} & .. & \mathbf{e}_{i_{n-m}} & B
    \end{vmatrix} \\
    &= \sum_{1 \le j_1 < j_2 .. < j_m \le n}A \begin{bmatrix}
        1 & 2 & .. & m \\
        j_1 & j_2 & .. & j_m \\
    \end{bmatrix}(-1)^{\beta}(-1)^{\gamma}B \begin{bmatrix}
       j_1 & j_2 & .. j_m \\ 
       1 & 2 & .. & m
    \end{bmatrix}
\end{align*}

\begin{align*}
\alpha + m + \beta + \gamma &= nm + m + \binom{n+1}{2} + \binom{m+1}{2} + \binom{n-m+1}{2} \\
&= nm + m + \frac{n^2 + n + m^2 + m + n^2 -2mn + m^2 + n-m }{2} \\
&= nm + m + n^2  + m^2 -mn + n \\
&= 0
\end{align*}

所以得到

\begin{align*}
    |AB| = \sum_{1 \le j_1 < j_2 .. < j_m \le n}A \begin{bmatrix}
        1 & 2 & .. & m \\
        j_1 & j_2 & .. & j_m \\
    \end{bmatrix}B \begin{bmatrix}
       j_1 & j_2 & .. j_m \\ 
       1 & 2 & .. & m
    \end{bmatrix}
\end{align*}

\end{document}



