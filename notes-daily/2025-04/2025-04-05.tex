
%!LW recipe=latexmk (xelatex)
% a4 页面 字体大小12像素
\documentclass[12pt,a4paper]{ctexart}
\input% 数学公式
\usepackage{amsmath}
\usepackage{amsfonts}
\usepackage{amssymb}
\usepackage[hidelinks]{hyperref}
\usepackage{enumitem}
\usepackage{geometry}


\geometry{left=1.5cm, right=1.5cm, top=2cm, bottom=2cm}

\title{学习记录}
\author{朱英杰}
\date{\today}

\begin{document}
\zihao{-4}
\maketitle
\tableofcontents

\section{数学分析}


\subsection{多项式定理}

\subsubsection{二项式定理}


\[
(x+y)^n = \sum_{i=0}^{n} \binom{n}{i}x^i y^{n-i}
\]

用数学归纳法很容易证明

\begin{align*}
(x+y)^{n+1} &= (x+y)^n(x+y) = (x+y)\sum_{k=0}^{n}\binom{n}{k}x^ky^{n-k} \\
&= \sum_{k=0}^{n}\binom{n}{k}x^{k+1}y^{n-k} + \sum_{k=0}^{n}\binom{n}{k}x^{k}y^{n+1-k} \\
&= \sum_{k=1}^{n+1}\binom{n}{k-1}x^{k}y^{n+1-k} + \sum_{k=1}^{n}\binom{n}{k}x^{k}y^{n+1-k}  + y^{n+1}\\ 
&= x^{n+1} + y^{n+1} + \sum_{k=1}^{n} \binom{n+1}{k}x^ky^{n+1-k} \\
&= \sum_{k=0}^{n+1} \binom{n+1}{k}x^ky^{n+1-k}
\end{align*}

\subsubsection{计算 $(fg)^{(n)}$}
用于计算导数

\[
(fg)^{(n)} = \sum_{k=0}^{n}\binom{n}{k}f^{(k)}g^{(n-k)}
\]

证明

\begin{align*}
    (fg)^{(n+1)} &= \text{d} \left( \sum_{k=0}^{k}\binom{n}{k}f^{(k)}g^{(n-k)} \right) \\
    &= \sum_{k=0}^{n}\binom{n}{k}f^{(k+1)}g^{(n-k)}  + \sum_{k=0}^{n}\binom{n}{k}f^{(k)}g^{(n+1-k)} \\
    &= f^{(n+1)} + g^{(n+1)} + \sum_{k=1}^{n}\binom{n}{k-1}f^{(k)}g^{(n+1-k)}  + \sum_{k=1}^{n}\binom{n}{k}f^{(k)}g^{(n+1-k)} \\
    &= f^{(n+1)} + g^{(n+1)} + \sum_{k=1}^{n}\binom{n}{k}f^{(k)}g^{(n+1-k)} \\
    &= \sum_{k=0}^{n+1}\binom{n+1}{k}f^{(k)}g^{(n-k)}
\end{align*}

\subsubsection{二元函数的 $n$ 阶导}

$h(t) = f(b_1 + t(x-a_1), b_2 + t(y-a_2))$

为了方便起见,我们采用如下递归定义的简写

\[
\frac{\text{d}^{n+1} f}{\text{d}t^{n+1}} = \frac{\text{d}}{\text{d} t}\left( \frac{\text{d}^n f}{\text{d}t^n} \right)
\]

对 partial derive 也是同理

\[
\frac{\partial^{n+1} f}{\partial x^{n+1}} = \frac{\partial}{\partial x}\left( \frac{\partial^n f}{\partial x^n} \right)
\]

这里 $\partial x^n$ 可以理解为 $\partial x\partial x .. \partial x$,根据克莱罗定理,对于$m$ 次连续可微的函数,它的$k,\, k \le m$ 次偏导和每次计算偏导的次序无关。
所以以二元函数为例 $k$ 次可以写成如下的形式

\[
\frac{\partial^k f}{\partial x^{l} \partial y^{k-l}}
\]


令 $\alpha = b_1 + t(x-a_1),\, \beta = b_2 + t(y-a_2)$

那么可以计算一阶导

\begin{align*}
    \frac{\text{d} h}{\text{d} t} = x\frac{\partial f}{\partial x}(\alpha, \beta) + y \frac{\partial f}{\partial y}(\alpha, \beta)
\end{align*}

再计算二阶导

\begin{align*}
    \frac{\text{d}^2 h}{\text{d} t^2} = x^2\frac{\partial^2 f}{\partial x^2}(\alpha, \beta) + 2xy \frac{\partial^2 f}{\partial x \partial y}(\alpha, \beta) + y^2 \frac{\partial^2 f}{\partial y^2}(\alpha, \beta)
\end{align*}

猜测有结论

\[
\frac{\text{d}^n h}{\text{d} t^n} = \sum_{k=0}^{n} \binom{n}{k} x^{k}y^{n-k}\frac{\partial^n f}{\partial x^{k} \partial y^{n-k}}(\alpha, \beta)
\]

用数学归纳法证明:

\begin{align*}
\frac{\text{d}^{n+1} h}{\text{d} t^{n+1}} & = \frac{\text{d}}{\text{d} t} \left( \frac{\text{d}^n h}{\text{d} t^n} \right) = \frac{\text{d}}{\text{d} t}\left( \sum_{k=0}^{n} \binom{n}{k} x^{k}y^{n-k}\frac{\partial^n f}{\partial x^{k} \partial y^{n-k}}(\alpha, \beta) \right) \\
&= \sum_{k=0}^{n} \binom{n}{k} x^{k+1}y^{n-k}\frac{\partial^{n+1} f}{\partial x^{k+1} \partial y^{n-k}}(\alpha, \beta) + \sum_{k=0}^{n} \binom{n}{k} x^{k}y^{n+1-k}\frac{\partial^n f}{\partial x^{k} \partial y^{n+1-k}}(\alpha, \beta) \\
&= x^{n+1} \frac{\partial^{n+1} f}{\partial x^{n+1}}(\alpha, \beta) + \sum_{k=1}^{n} \binom{n}{k-1} x^{k}y^{n+1-k}\frac{\partial^{n+1} f}{\partial x^{k} \partial y^{n+1-k}}(\alpha, \beta) \\
& + y^{n+1} \frac{\partial^{n+1} f}{\partial y^{n+1}}(\alpha, \beta) + \sum_{k=1}^{n} \binom{n}{k} x^{k}y^{n+1-k}\frac{\partial^{n+1} f}{\partial x^{k} \partial y^{n+1-k}}(\alpha, \beta) \\
&= x^{n+1} \frac{\partial^{n+1} f}{\partial x^{n+1}}(\alpha, \beta) + y^{n+1} \frac{\partial^{n+1} f}{\partial y^{n+1}}(\alpha, \beta) + \sum_{k=1}^{n} \binom{n+1}{k} x^{k}y^{n+1-k}\frac{\partial^{n+1} f}{\partial x^{k} \partial y^{n+1-k}}(\alpha, \beta) \\
&= \sum_{k=0}^{n} \binom{n+1}{k} x^{k}y^{n+1-k}\frac{\partial^{n+1} f}{\partial x^{k} \partial y^{n+1-k}}(\alpha, \beta)
\end{align*}

\subsection{多项式定理}

\subsubsection{多重组合数}

针对更一般的多项式,例如

\[
(x_1 + x_2 + .. + x_m)^n
\]

那么有

\[
(x_1 + x_2 + .. + x_m)^n = \sum_{k_1 + k_2 + .. k_m = n}\binom{n}{k_1\,k_2..k_m}x_1^{k_1}x_2^{k_2}..x_m^{k_m}
\]

其中

\[
\binom{n}{k_1\,k_2..k_m} = \frac{n!}{k_1!k_2!..k_m!}
\]

也被称作多重组合数

\subsubsection{证明}

我们用数学归纳法证明

\begin{align*}
(x_1 + x_2 + .. + x_m + x_{m+1})^n &= \sum_{k=0}^{n}\binom{n}{k}(x_1 + x_2 + .. x_m)^{k}x_{m+1}^{n-k} \\
&= \sum_{k=0}^{n}\binom{n}{k}\left( \sum_{l_1+l_2+..+l_m = k} \binom{k}{l_1\,l_2..l_m}x_1^{l_1}x_2^{l_2}..x_m^{l_m}\right)x_{m+1}^{n-k} \\
&= \sum_{k=0}^{n}\left( \sum_{l_1+l_2+..+l_m = k,\, l_{m+1} = n-k} \binom{n}{l_1\,l_2..l_ml_{m+1}}x_1^{l_1}x_2^{l_2}..x_m^{l_m}x_{m+1}^{l_{m+1}}\right) \\ 
&= \sum_{l_1+l_2+..+l_m + l_{m+1} = n} \binom{n}{l_1\,l_2..l_ml_{m+1}}x_1^{l_1}x_2^{l_2}..x_m^{l_m}x_{m+1}^{l_{m+1}}
\end{align*}

\subsubsection{组合数学问题}

把 $n$ 个球分到 $m$ 个不同的盒子里,有多少种分法,允许空盒子,球必须全部分配,不能剩下。

这个问题等于是回答了 $(x_1 + x_2 + .. + x_m)^n$ 有多少个不同类的项。

\subsubsection{多元函数求导}

令 $f: \mathbb{R}^m \to \mathbb{R} \in C^{n+1}$, $h(t) = f(b_1 + t(x_1-a_1), b_2 + t(x_2-a_2), .., b_m + t(x_m - a_m))$

尝试计算 

\[
\frac{\text{d}^n h}{\text{d} t^n}
\]

根据多项式定理,我们可以先写出结论

\begin{align*}
\frac{\text{d}^n h}{\text{d} t^n} = \sum_{k_1 + k_2 + .. k_m = n}\frac{n!}{k_1!k_2!..k_m!}\frac{\partial ^n f}{\partial x_1^{k_1}\partial x_2^{k_2}..\partial x_m^{k_m}}(\mathbf{b} + t \mathbf{x-a})(x_1-a_1)^{k_1}(x_2-a_2)^{k_2}..(x_m-a_m)^{k_m}
\end{align*}


\subsection{多元函数的泰勒公式}

为了便于表示,下面引入多重指标记号 $\alpha \in \mathbb{R}^n,\, \alpha = (\alpha_1, \alpha_2, .., \alpha_n),\, \alpha_i \in \mathbb{N}$,并定义 

\[
| \alpha | = \alpha_1 + \alpha_2 + .. + \alpha_n
\]


并定义 $\alpha !$  

\[
\alpha ! = \alpha_1!\alpha_2! .. \alpha_n!
\]

定义运算符 $D^{\alpha} \in  (\mathbb{R}^n \to \mathbb{R}) \to \mathbb{R}^n \to \mathbb{R}$

\[
D^{\alpha} f(x) = \frac{\partial^{| \alpha |}f}{\partial^{\alpha_1}_{x_1}\partial^{\alpha_2}_{x_2}..\partial^{\alpha_n}_{x_n}}(x)
\]

定义向量的 $\alpha$ 幂运算 $\mathbb{R}^n \to \mathbb{R}$

\[
(x_1,x_2,..,x_n)^{\alpha} = x_1^{\alpha_1}  x_2^{\alpha_2} ..  x_n^{\alpha_n}
\]

定理:设 $X \subseteq \mathbb{R}^n$ 是开的凸集,且 $f \in C^{m+1}$。$a = (a_1,a_2,..,a_n) \in X$。
那么对任意 $x \in X$,存在 $\xi \in (0,1)$ 满足


\[
f(x) = \sum_{k=0}^{m} \sum_{ | \alpha | = k } \frac{D^{\alpha}f(a)}{\alpha !} \cdot (x-a)^{\alpha} + \sum_{|\alpha| = m+1} \frac{D^{\alpha} f(a+ \theta(x-a))}{\alpha!} \cdot (x-a)^{\alpha}
\]

证明:

令 $h: [0,1] \to \mathbb{R}$ 

\[
h(t) =  f(a + t(x-a))
\]

\begin{enumerate}
    \item 计算 $h^{k}(t)$

    在多项式定理中,我们可以计算得到 $h(t)$ 的 $k$ 阶导数为

    \[
        h^{(k)}(t) =  \sum_{ | \alpha | = k } \frac{k!}{\alpha !} D^{\alpha}f(a+t(x-a))\cdot (x-a)^{\alpha}
    \]

    \item 对 $h(t)$ 运用一元函数的泰勒展开,以及拉格朗日余项

    \[
    h(1) = h(0) + h'(0) + \frac{1}{2!}h^{(2)}(0) + .. +\frac{1}{m!}h^{(m)}(0) + \frac{1}{(m+1)!}h^{(m+1)}(\theta)
    \]

    其中有 $0 < \theta < 1$

    \item 展开 $h(1) = f(x),\, h(0) = f(a)$ 得到

    \begin{align*}
        f(x) &= \sum_{k=0}^{m}\frac{1}{k!}h^{(k)}(0) + \frac{1}{(m+1)!}\sum_{|\alpha| = m+1}\frac{1}{\alpha!}D^{\alpha}f(a + \theta(x-a))(x-a)^\alpha \\
            &= \sum_{k=0}^{m}\sum_{|\alpha| = m} \frac{1}{\alpha !}D^{\alpha}f(a)(x-a)^\alpha + \frac{1}{(m+1)!}\sum_{|\alpha| = m+1}\frac{1}{\alpha!}D^{\alpha}f(a + \theta(x-a))(x-a)^\alpha
    \end{align*}

\end{enumerate}


\end{document}


