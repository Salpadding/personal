%!LW recipe=latexmk
\documentclass[11pt,a4paper]{article}
\input% math 
\usepackage{amsmath,amsfonts,amssymb,amsthm}
% cross reference, use \autoref instead of \ref
\usepackage{aliascnt}
\usepackage[hidelinks]{hyperref}
\usepackage{enumitem}
\usepackage{geometry}


\geometry{left=1.5cm, right=1.5cm, top=2cm, bottom=2cm}

\newtheorem{thm}{Theorem}[section]

\newaliascnt{lem}{thm}
\newaliascnt{prop}{thm}
\newaliascnt{definition}{thm}
\newaliascnt{exercise}{thm}
\newaliascnt{corollary}{thm}

\theoremstyle{definition}
\newtheorem{lem}[lem]{Lemma}
\newtheorem{prop}[prop]{Proposition}
\newtheorem{definition}[definition]{Definition}
\newtheorem{exercise}[exercise]{Exercise}
\newtheorem{corollary}[corollary]{Corollary}

\def\lemautorefname{Lemma}
\def\thmautorefname{Theorm}
\aliascntresetthe{lem}
\aliascntresetthe{prop}
\aliascntresetthe{definition}
\aliascntresetthe{exercise}
\aliascntresetthe{corollary}


\title{Learning Notes}
\author{Yingjie Zhu}
\date{\today}


\begin{document}
\maketitle

\section{Theory of Real Variable Function}

\subsection{Borel Sets}

\begin{prop}
    The rational set $\mathbb{Q}$ is not $G_{\delta}$. It cannot be expressed as countable intersection of open sets.
\end{prop}

\begin{proof}
    For sake of contradiction, assume we have

    \[
        \mathbb{Q} = \bigcap_{n=1}^{\infty}V_n\quad V_n \, \text{is open}
    \]

    then we have

    \begin{align*}
        \mathbb{R} &= \left( \mathbb{R} \setminus \mathbb{Q} \right) \cup \mathbb{Q} \\
        &= \left( \bigcup_{n=1}^{\infty}V_n^C \right) \cup \bigcup_{q \in \mathbb{Q}} \{ q \}
    \end{align*}

    Since $\mathbb{Q} \subseteq V_n$, then we have $\overline{\mathbb{Q}} \subseteq \overline{V_n} $, which means $\overline{V_n} = \mathbb{R}$.
    so that $\text{int}\left({V_n^C} \right) = \emptyset$. as $V_n$ and $\{ q \}$ is closed and contains no interior point, 
    by Baire Theorem, their countable union also contains no interior point, which is contradict with $\mathbb{R}$ contains interior point.
\end{proof}

\begin{definition}
    no where dense set $X \subseteq \mathbb{R}^n$ is defined as 

    \[
        \mathring{\overline{X}} = \emptyset
    \]
\end{definition}

\begin{prop}
    let $G_n,\, n=1,2,..$ be a family of dense and open set where $\overline{G_n} = \mathbb{R}^n$. then 

    \[
        G= \bigcap_{n=1}^{\infty}G_n
    \]

    $G$ is also a dense set.
\end{prop}

\begin{proof}
    we will prove that $\overline{G} = \mathbb{R}^n$, since for any set $X$, we have $(\mathring{X})^C = \overline{(X^C)}$, then we have

    \begin{align*}
        \overline{G}  &= \overline{(G^C)^C} = (\text{int} (G^C))^C \\
        &= \left(\text{int} \left( \bigcup_{n=1}^{\infty} G_n^C \right)\right)^C
    \end{align*}

    since $G_n^C$ is closed, and $\text{int}(G_n^C) = \left( \overline{G_n} \right)^C = \emptyset$. so $G_n^C$ is closed and contains no interior point. 
    by baire theorem, countable union of such closed set also contains no interior point, so

    \[
        \text{int} \left( \bigcup_{n=1}^{\infty} G_n^C \right) = \emptyset
    \]

    and complement of empty set is $\mathbb{R}^n$
\end{proof}

\begin{exercise}
let $f_n \in C([a,b])$  and $f_n \to f$ pointwise, prove that $\forall t$

\[
\{ x \,\vert\, f(x) < t\}
\]

is $F_{\sigma}$ set


\end{exercise}


\begin{proof}
we first prove that

\[
\{ x \,\vert\, f(x) < t\} = \bigcup_{k=1}^{\infty}\bigcap_{N=1}^{\infty}\bigcup_{n=N}^{\infty} \{ x \,\vert\, f_n(x) < t - \frac{1}{k} \}
\]

let $f(x) < t$, then there should exists $K \in \mathbb{N}$ so that $f(x) < t - 1/K < t$. since $f(x) < t - 1/K$ and

\[
\lim_{n \to \infty}f_n(x) < t - 1/K
\]

there should exists infinite $n$ so that  $f_n(x) < t - 1/K$, otherwise we will have $f(x) \ge t - 1/K$

on the other hand, if we have $k$ and infinte $n$ so that $f_n(x) < t - 1/k$, by limit law, we have $f(x) \le t - 1/k$, so that 
$f(x) < t$

since $f_n$ is continuous, so 

\[
\bigcap_{N=1}^{\infty}\bigcup_{n=N}^{\infty} \{ x \,\vert\, f_n(x) < t - \frac{1}{k} \}
\]

should be $F_{\sigma}$
\end{proof}

\end{document}







