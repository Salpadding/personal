%!LW recipe=latexmk (xelatex)
% a4 页面 字体大小12像素
\documentclass[12pt,a4paper]{ctexart}
\input% 数学公式
\usepackage{amsmath}
\usepackage{amsfonts}
\usepackage{amssymb}
\usepackage[hidelinks]{hyperref}
\usepackage{enumitem}
\usepackage{geometry}


\geometry{left=1.5cm, right=1.5cm, top=2cm, bottom=2cm}

\title{学习记录}
\author{朱英杰}
\date{\today}

\begin{document}
\zihao{-4}
\maketitle
\tableofcontents

\section{实变函数论}

\subsection{点集拓扑}

\subsubsection{等势}

若 $A$ 是无限集,$B$ 是至多可数集,那么 $A$ 和 $A \cup B$ 等势。

证明: 令 $B = (b_0, b_1, ..)$ 且 $A_1 = (a_0, a_1, .. ),\, A_1 \subseteq A,\, A_1 \cup A_2 = A,\, A_1 \cap A_2 = \emptyset$。
不妨令  $A \cap B = \emptyset$,

这里可以不妨令,是因为如果不满足不妨令的条件,我们可以令 $B' = B \setminus A$,这样我们可以对 $B'$ 进行分解 $B' = (b_0,b_1,..)$ 而且 $B'$ 也是至多可数集。
而且有 $B \cup A = B' \cup A$

然后我们构造函数  $f: A \cup B \to A$

\begin{align*}
    f(a_i) &= f(a_{2i}),\, a_i \in A_1 \\
    f(b_i) &= f(a_{2i+1}),\, b_i \in B \\
    f(x) &= x,\, x \in A_2 \\
\end{align*}

显然 $f$ 是双射。

如果 $B$ 是一个有限集,不妨令 $B=(b_0,b_1,..,b_{n-1})$ 且 $A \cap B = \emptyset$,

\begin{align*}
    f(a_i) &= f(a_{i+n}),\, a_i \in A_1 \\
    f(b_i) &= f(a_{i}),\, b_i \in B \\
    f(x) &= x,\, x \in A_2 \\
\end{align*}


显然 $f$ 是双射。

\subsubsection{无限集和它的真子集等势}

无限集的充分必要条件就是它和它的某个真子集等势。

证明:

\begin{enumerate}
    \item 充分性

    用反证法证明,假如它是有限集,那么它一定不可以和它的真子集构造双射

    \item 必要性

    如果某个集合$A$是无限集,然后我取它其中任意一个元素 $a$,令 $B = A \setminus \{ a \}$,利用上面的条件可以得到,$B$ 和 $B \cup A = A$ 等势。
\end{enumerate}

\subsubsection{凸函数几乎处处可微}

凸函数不可微的点是一个至多可数的集合。

证明:

\begin{enumerate}
    \item 构造一个二元函数

    定义 

    \[
        L(x,y)= \frac{f(y) - f(x)}{y-x}
    \]

    我们将证明对任意 $x_1 < x_2 < x_3$  有

    \[
        L(x_1, x_2) \le L(x_1,x_3) \le L(x_2, x_3)
    \]

    我们把 $x_1, x_3$ 代入凸函数的定义

    \[
    f((1-t)x_1 + tx_3) \le (1-t)f(x_1) + tf(x_3)
    \]

    然后令 $x_2 = (1-t)x_1 + tx_3$ 得到

    \begin{align*}
        f(x_2) - f(x_1) & \le \frac{x_2 - x_1}{x_3 - x_1}\left(f(x_3) - f(x_1)\right) \\
        \frac{f(x_2) - f(x_1)}{x_2 - x_1} & \le \frac{f(x_3) - f(x_1)}{x_3 - x_1} \\
        L(x_1, x_2) &\le L(x_1, x_3) \\
    \end{align*}

    同理,变形后可以得到

    \begin{align*}
        f(x_2) - f(x_3) &\le (1-t)(f(x_1) - f(x_3)) \\   
        \frac{f(x_2 - f(x_3))}{x_2 - x_3} & \ge - \frac{f(x_1) -f(x_3)}{x_3 - x_1} \\
        L(x_2, x_3) & \ge L(x_1, x_3) 
    \end{align*}

    最终得到

    \[
        L(x_1, x_2) \le L(x_1, x_3) \le L(x_2, x_3)
    \]

\item 构造单调函数

固定点 $x_0$ 后,我们令 

\[
\alpha(t) = L(x_0 + t, x_0)
\]

当 $t_1 < t_2$ 时,根据上面的不等式有

\[
L(x_0, x_0 + t_1) \le L(x_0,x_0 + t_2) \le L(x_0 + t_1, x_0 + t_2)
\]

也就是

\[
\alpha(t_1) \le \alpha(t_2)
\]

所以 $\alpha(t)$ 是单调增函数

\item 证明刚才构造的单调增函数有下界

任取一点 $x_0 - \epsilon < x_0$,对任意 $t > 0$ 根据刚才证明的不等式有

\[
L(x_0 - \epsilon, x_0) \le L(x_0 - \epsilon, x_0 + t) \le L(x_0, x_0 + t) \le \alpha(t)
\]

所以 $\alpha(t)$ 有下界

\item 证明凸函数右导数存在

我们只要证明 $\alpha(t)$ 有右极限即可,令

\[
c = \inf \{ \alpha(t) \,\vert\, t > 0 \}
\]

我们将证明

\[
\lim_{t \to 0, t > 0} \alpha(t) = c
\]

根据下确界的性质,配合选择公理,一定存在 $t_n $ 满足 $t_n > 0$ 且 $\alpha(t_n) - c \le 1/n$。
这样我们构造了一个收敛到 $c$ 的序列。

于是,对于任意 $\epsilon > 0$,一定有 $N$ 满足 $\alpha(t_N) - c \le \epsilon$,根据 $\alpha$ 的单调性, 只要 $0 < t < t_n$,则必然有 
$0 \le \alpha(t) - L \le \alpha(t_N) \le \epsilon$。所以凸函数存在右导数。

用相同的方法可以证明左导数也存在。

\item 证明左导数小于等于右导数

先固定 $x_0$,令左导数为 $f'(x_0 - 0)$,右导数为 $f'(x_0 + 0)$,得到

\begin{align*}
f'(x_0 - 0) &= \lim_{t \to 0,\, t > 0}L(x_0 - t, x_0) \\
f'(x_0 + 0) &= \lim_{s \to 0,\, s > 0}L(x_0, x_0 + s) \\
\end{align*}

根据之前证明的不等式,我们有

\[
L(x_0 - t, x_0) \le L(x_0-t, x_0 + s) \le L(x_0, x_0+s)
\]

我们先对 $t \to 0$ 取极限得到

\[
f'(x_0-0) \le L(x_0, x_0+s)
\]

再对 $s \to 0$ 取极限得到


\[
f'(x_0-0) \le f'(x_0 + 0)
\]

\item 证明对 $x_1 < x_2$ 有 $f'(x_1 +0) \le f'(x_2-0)$

\begin{align*}
f'(x_1 + 0) &= \lim_{t \to 0,\, t > 0}L(x_1, x_1+t) \\
f'(x_2 - 0) &= \lim_{s \to 0,\, s > 0}L(x_2-s, x_2) \\
\end{align*}


我们可以让 $t$ 和 $s$ 都足够小,满足 $x_1 +t < x_2,\, x_1 < x_2 -s$ 根据之前证明的不等式,我们有

\begin{align*}
    L(x_1, x_1 + t) &\le L(x_1, x_2) \le L(x_1+t, x_2) \\
    L(x_1, x_2-s) &\le L(x_1,x_2) \le L(x_2 -s, x_2)
\end{align*}

取极限得到

\[
\lim_{t \to 0,\,t>0}L(x_1, x_1 + t)\le L(x_1,x_2) \le \lim_{s \to 0, s>0}L(x_2 -s,x_2)
\]

所以有 $f'(x_1+0) \le f'(x_2 - 0)$

\item 证明不可微的点可以映射到不相交的开区间,而且是单射。

对于凸函数的不可微的点,因为它的左右导数都存在,而且左导数小于等于右导数,加上它是不可微的,所以它的左导数一定小于右导数。

任取凸函数上不可微的两个点 $x_1 < x_2$ ,因为 $x_1$ 的右导数小于等于 $x_2$ 的左导数,所以它们各自左导数和右导数构成的开区间一定不相交。

\item  证明不可微的点至多可数

因为不相交的开区间一定是至多可数的,所以不可微的点至多可数。

\end{enumerate}


\section{高等代数}

\subsection{Laplace 定理}

Laplace 定理允许行列式按多行 或者按多列展开,例如我们想按第 $i_1,i_2,..,i_m,\, i_1 < i_2 < .. < i_m$ 行展开有

用数学归纳法证明

\begin{align*}
    |A| &= \sum_{j=1}^{n}a_{i_r j}(-1)^{i_r + j} D \begin{bmatrix}
       1 & 2 & .. & i_r - 1 & i_r + 1 & .. & n \\  
       1 & 2 & .. & j - 1 & j + 1 & .. & n \\  
    \end{bmatrix} \\
    &= \sum_{j=1}^{n}a_{i_r j}(-1)^{i_r + j} 
    \sum_{1 \le k_1 < .. < k_{r-1} \le n,\, k_{*} \ne j}
    (-1)^{I_{r-1} + k_1 + k_2+ .. + k_{r-1}  - (n-j)} 
    D \begin{bmatrix}
        i_1 & i_2 & .. & i_{r-1} \\
        k_1 & k_2 & .. & k_{r-1} \\
    \end{bmatrix} \overline{D} \begin{bmatrix}
        i_1 & i_2 & .. & i_{r-1} & i_r \\
        j & k_1 & k_2 & .. & k_{r-1} \\
    \end{bmatrix}
\end{align*}

\end{document}


