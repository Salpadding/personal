%!LW recipe=latexmk (xelatex)
% a4 页面 字体大小12像素
\documentclass[12pt,a4paper]{ctexart}
\input% math 
\usepackage{amsmath,amsfonts,amssymb,amsthm}
% cross reference, use \autoref instead of \ref
\usepackage{aliascnt}
\usepackage[hidelinks]{hyperref}
\usepackage{enumitem}
\usepackage{geometry}


\geometry{left=1.5cm, right=1.5cm, top=2cm, bottom=2cm}

\newtheorem{thm}{Theorem}[section]

\newaliascnt{lem}{thm}
\newaliascnt{prop}{thm}
\newaliascnt{definition}{thm}
\newaliascnt{exercise}{thm}
\newaliascnt{corollary}{thm}

\theoremstyle{definition}
\newtheorem{lem}[lem]{Lemma}
\newtheorem{prop}[prop]{Proposition}
\newtheorem{definition}[definition]{Definition}
\newtheorem{exercise}[exercise]{Exercise}
\newtheorem{corollary}[corollary]{Corollary}

\def\lemautorefname{Lemma}
\def\thmautorefname{Theorm}
\aliascntresetthe{lem}
\aliascntresetthe{prop}
\aliascntresetthe{definition}
\aliascntresetthe{exercise}
\aliascntresetthe{corollary}


\title{学习记录}
\author{朱英杰}
\date{\today}

\begin{document}
\zihao{-4}
\maketitle
\tableofcontents

\section{实变函数论}

\subsection{点集拓扑}

\subsubsection{闭集}

$f: \mathbb{R}^n \to \mathbb{R}$,那么 $f \in C(\mathbb{R})$ 的充分必要条件是:

\[
\forall t \in \mathbb{R},\, f^{-1}((-\infty, t]) \quad \text{and} \quad f^{-1}([t, \infty)) \quad \text{is closed}
\]

证明:

\begin{enumerate}
    \item 必要性

    let $E = f^{-1}((-\infty, t])$, and $c$ is a limit point of $E$, then there exists a sequence $c_n$ meets

    \[
        \lim_{n \to \infty} c_n = c \quad c_i \ne c_j\quad c_n \in E
    \]

    then we have

    \[
        f(c_n) \le t
    \]

    since $f$ is continuous, then we have

    \[
        f(c) = \lim_{n \to \infty}f(c_n) \le t
    \]

    so $c \in E$, then $E$ is closed

    \item 充分性

    fix $f(x) = y,\, x \in \mathbb{R}^n,\, y \in \mathbb{R},\, \epsilon > 0$, then we have

    \[
        B(y, \epsilon)^C = (y-\epsilon, y+\epsilon)^C = (-\infty, y-\epsilon] \cup [y+\epsilon, \infty)
    \]

    根据已知条件有

    \begin{align*}
        f^{-1}(B(y, \epsilon)^C) &= f^{-1}((-\infty, y-\epsilon] \cup [y+\epsilon, \infty))  \\
        &= f^{-1}((-\infty, y-\epsilon]) \cup f^{-1}([y+\epsilon, \infty))
    \end{align*}

    所以 $f^{-1}(B(y, \epsilon)^C)$ 是一个闭集,而 $f^{-1}(B(y, \epsilon))^C = f^{-1}(B(y, \epsilon)^C)$,所以 $f^{-1}(B(y, \epsilon))$ 是一个闭集的补集。

    根据闭集的性质,闭集的补集包含的点一定是内点,所以一定存在开邻域 $B(x, \delta)$ 满足

    \[
        B(x, \delta) \subseteq f^{-1}(B(y, \epsilon))
    \]

    所以 $f$ 在 $x$ 处连续
\end{enumerate}

\subsubsection{闭包计算}

开球 $B(x_0, r)$ 的闭包是闭球 $C(x_0, r)$

\begin{enumerate}
    \item $C(x_0, r)$ 是闭集 且 $B(x_0, r) \subseteq C(x_0, r)$
    
    任取 $C(x_0, r)$ 的极限点 $x$, 有

    \[
        \lim_{n \to \infty} \lvert x_n -x\rvert =0\quad x_n \in C(x_0, r)
    \]

    根据三角不等式有

    \[
        \lvert x -x_0\rvert \le \lvert x-x_n\rvert + \lvert x_n -x_0\rvert \le  \lvert x-x_n\rvert + r
    \]

    对 $n$ 取极限得到

    \[
        \lvert x -x_0 \rvert \le r
    \]

    所以 $x \in C(x_0, r)$,所以 $C(x_0,r)$ 是闭集

    \item $C(x_0, r)$ 中任意一点都是 $B(x_0, r)$ 的极限点

    取 $x \in C(x_0, r)$,构造

    \[
        x_n = x_0 + (1-\frac{1}{n})(x-x_0)
    \]

    显然有

    \[
        \lvert x_n -x_0 \rvert = \lvert (1-\frac{1}{n})(x-x_0) \rvert < r
    \]

    所以有 $x_n \in B(x_0, r)$,而且有

    \[
        \lim_{n \to \infty} \lvert x -x_n \rvert = 0
    \]

\end{enumerate}

\subsection{思考题}

\subsubsection{不可数集必有极限点}

$E \subseteq \mathbb{R}$ 且 $E$ 不可数,那么 $E$ 必然有极限点。

证明: $E^+ = E \cap [0,\infty)$ 和 $E^- = \cap (-\infty, 0)$ 中间一定有一个是不可数的,
不妨设 $E^+$ 是不可数集,然后对 $E^+$ 进行分解

\[
E^+ = \bigcup_{n=0}^{\infty}E^+ \cap [n,n+1)
\]

令 $E_n = E^+ \cap [n,n+1)$,那么一定存在 $n$ 满足 $E_n$ 是无限集,否则每个 $E_n$ 都是有限集能得出 $E_n^+$ 是可数集。
根据 Bolzano-Weierstrass 定理 $E_n$ 一定有极限点,

\subsubsection{极限点构成的集合是可数集反推原集合是可数集}

令 $E'$ 是 $E$ 所有的极限点构成的集合,若 $E'$ 是可数集,那么 $E $ 也一定是可数集。

证明:

因为 $E \setminus E'$ 中所有的点是孤立点,而孤立点集是可数的。
所以 $E = (E \setminus E') \cup E'$ 是可数集。

\subsubsection{三进制表示法}

let $0 \le c < 1$, 按如下方式构造 $c_n$

\begin{align*}
    c_0 &= c \\
    d_1 &= \lfloor 3c_0 \rfloor \quad c_1 = 3c_0 - d_1 \\
    d_2 &= \lfloor 3c_1 \rfloor \quad c_2 = 3c_1 - d_2 \\
    .. \\
    d_n &= \lfloor 3c_{n-1} \rfloor \quad c_n = 3c_{n-1} - d_n \\
\end{align*}

then we have 

\begin{enumerate}
    \item $d_n \in \{ 0,1,2 \},\, 0 \le c_n < 1$

    用数学归纳法证明,if $0 \le c_n < 1$, then $0 \le 3c_n <3$, so $ \lfloor 3c_n \rfloor \in \{ 0,1,2 \}$, also we have
    $c_{n+1} = 3c_{n} - \lfloor 3c_n\rfloor$, so $0 \le c_{n+1} < 1$


    \item limit

    \[
        \sum_{n=1}^{\infty} \frac{d_n}{3^n} = c
    \]

    by induction, we have

    \[
        c_n = 3^n c - (3^{n-1}d_1 + 3^{n-2}d_2 + .. + 3^{0}d_n)
    \]

    equal to

    \begin{align*}
        \frac{c_n}{3^n} = c - (3^{-1}d_1 + 3^{-2}d_2 + .. + 3^{-n}d_n)
    \end{align*}

    take limitation on both side:

    \[
        0 = c - \sum_{n=1}^{\infty}\frac{1}{3^n} d_n
    \]
\end{enumerate}

\subsubsection{并集和极限点运算可以交换次序}

暂时定义 $L(A)$ 为取集合 $A$ 的所有极限点得到的集合,那么有 $L(A \cup B) = L(A) \cup L(B)$

证明:

\begin{enumerate}
    \item $L(A) \cup L(B) \subseteq L(A \cup B)$

    without loss of generality, assume $x \in L(A)$, then we have a sequence of distinct points $x_n \in A$ and

    \[
        \lim_{n \to \infty}x_n =x
    \]

    by definition, $x$ become a limit point of $A \cup B$, since $x_n \in A \cup B$

    \item $L(A \cup B) \subseteq L(A) \cup L(B)$

    assume $x \in L(A \cup B)$, then we have a sequence of distinct points $x_n \in A \cup B$

    \[
        \lim_{n \to \infty}x_n =x
    \]

    if $x_n$ contains infinite points of $A$, then $x$ become a limit point of $A$, otherwise it should 
    be a limit point of $B$. Because $x_n$ is infinite, so at least one of $\{ x_n \} \cap A,\, \{ x_n \} \cap B$ 
    should be infinite.
\end{enumerate}

\subsubsection{闭包和并集运算可以交换次序}

proof: let $C(A)$ become operator of take closure of $A$.

\begin{align}
    C(A \cup B) &=  A \cup B \cup L(A \cup B)  = A \cup B \cup L(A)  \cup L(B) \\
    &=  (A \cup L(A)) \cup (B \cup L(B)) = C(A) \cup C(B)
\end{align}

\subsection{思考题}

\subsubsection{找反例}

$\overline{A \cap B} \ne \overline{A} \cap \overline{B}$

let $A = (0,1),\, B=(1,2)$, then $\overline{A \cap B} = \emptyset$, $\overline{A} \cap \overline{B} = \{ 1 \}$

\subsubsection{find contradiction}

let $E_n \subseteq \mathbb{R},\quad E = \bigcup_{n=1}^{\infty}E_n$ if $x$ is a limit point of $E$, could we
make sure that $x$ is a limit point of any $E_n$

the anwser is not: 

let $E_n = (\frac{1}{n}, 1]$, then $E = (0,1)$. $x = 0$ is a limit point of $E$. however, for any $n$, $x$ is not limit point of $E_n$

\subsubsection{Cantor's Theorem}

let $E_n$ be bounded and closed set of $\mathbf{R}^n$, and $E_n$ is monotone: $E_1 \supseteq E_2 .. \supseteq E_n ..$

then 

\[
E = \bigcap _{n=1}^{\infty} E_n \ne \emptyset
\]

proof:

we will discuss on two cases:

\begin{enumerate}
    \item case 1

    in case 1, the strictly decrease pairs $E_n \supset E_{n+1},\, E_n \ne E_{n+1}$ is finite. More precisely, there exists $N$,
    and $\forall n \ge N$, we have $E_{n} = E_N$. The result is obviously. $E = E_N,\, E_N \ne \emptyset$

    \item case 2

    in this case, the strictly decrease pairs $E_n \supset E_{n+1},\, E_n \ne E_{n+1}$ is infinite. We will focus on those 
    decrease pairs. it is obviously that

    \[
        E = \bigcap_{n=1,\, E_n \ne E_{n+1}}^{\infty}E_{n+1}
    \]

    we will construct a distinct sequence, for every $E_n \ne E_{n+1}$, select $x_n \in E_n \setminus E_{n+1}$.
    then $x_n$ is distinct. when $x_n \in E_n$ fixed, we have $\forall m > n, x_m \in E_{n+1}$. $x_m = x_n$ leads to 
    $x_n \in E_{n+1}$, which is contradict. because $E_n$ is bounded and closed, there should exists a subsequence $x_{f(n)}$

    \[
        \lim_{n \to \infty}x_{f(n)} = x 
    \]

    for every $n$, $x$ is a limit point of  $E_n$. since when $m$ is fixed, $ \{ x_{f(n)} \} \cap E_m$ is infinite. so $\forall n,\, x \in E_n$.
    because $E_n$ contains its limit points. So

    \[
        x \in E
    \]

\end{enumerate}

\subsubsection{Open set}

let $f: \mathbb{R}^n \to \mathbb{R}$, define $h(x_0,\delta)$ as:

\[
h(x_0, \delta) = \sup \{ \lvert f(x) - f(y) \rvert \,\vert\, x,y \in B(x_0, \delta)\}
\]

when $x_0$ is fixed, then $f$ is monotone decrease on $\delta$, so we can define $\alpha(x_0)$ as:

\[
\alpha(x_0) = \lim_{\delta \to 0} h(x_0, \delta)
\]

then we prove that $\forall t,\, \{ x \vert \alpha(x) < t\}$ is open.

proof:

fix $t$, let $E = \{ x \vert \alpha(x) < t\}$, assume $E \ne \emptyset$. if $E = \emptyset$ then $E$ is open.

select $x_0 \in E$, if $\alpha(x_0) < t$, then we can find $\epsilon > 0$ so that $\alpha(x_0) < t - \epsilon < t$.
by definition of $\alpha$, we have

\[
\lim_{n \to \infty}h(x_0, \frac{1}{n}) = \alpha(x_0) < t - \epsilon
\]

so we can find $N$, meets:

\[
h(x_0, \frac{1}{N}) < t-\epsilon
\]

let $A = B(x_0, \frac{1}{N})$, we then prove $A \subseteq E$

select $x' \in A$, then we can find $M$ so that $B(x', 1/M) \subseteq B(x_0, 1/N)$. because $B(x_0, 1/N)$ is open, its element is interior point.

then we have

\[
h(x', 1/M) \le h(x_0, 1/N) < t - \epsilon
\]

and 

\[
\alpha(x') \le h(x', 1/M) < t-\epsilon
\]

so $x' \in E$, so we find a open ball $B(x_0, 1/N) \subseteq E$. so $E$ is open

\end{document}




