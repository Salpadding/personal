%!LW recipe=latexmk
\documentclass[11pt,a4paper]{article}
\input% math 
\usepackage{amsmath,amsfonts,amssymb,amsthm}
% cross reference, use \autoref instead of \ref
\usepackage{aliascnt}
\usepackage[hidelinks]{hyperref}
\usepackage{enumitem}
\usepackage{geometry}


\geometry{left=1.5cm, right=1.5cm, top=2cm, bottom=2cm}

\newtheorem{thm}{Theorem}[section]

\newaliascnt{lem}{thm}
\newaliascnt{prop}{thm}
\newaliascnt{definition}{thm}
\newaliascnt{exercise}{thm}
\newaliascnt{corollary}{thm}

\theoremstyle{definition}
\newtheorem{lem}[lem]{Lemma}
\newtheorem{prop}[prop]{Proposition}
\newtheorem{definition}[definition]{Definition}
\newtheorem{exercise}[exercise]{Exercise}
\newtheorem{corollary}[corollary]{Corollary}

\def\lemautorefname{Lemma}
\def\thmautorefname{Theorm}
\aliascntresetthe{lem}
\aliascntresetthe{prop}
\aliascntresetthe{definition}
\aliascntresetthe{exercise}
\aliascntresetthe{corollary}


\title{Learning Notes}
\author{Yingjie Zhu}
\date{\today}


\begin{document}
\maketitle

\section{Theory of Real Variable Function}

\begin{exercise}
    let $E \subseteq [0,1]$ be a measurable set. And $m(E) = 1$, prove: $\overline{E} = [0,1]$
\end{exercise}

\begin{proof}
    assuming exists $x \in [0,1]$ and $x \notin \overline{E}$. since $\left(\overline{E}\right)^C$ is open, there 
    should exists a open ball $B(x,r) \subseteq \left(\overline{E}\right)^C$. and we have

    \begin{align*}
        \overline{E} & \subseteq B(x,r)^C \\
        [0,1] \cap \overline{E} & \subseteq [0,1] \cap B(x,r)^C \\
        \overline{E} & \subseteq [0,1] \cap B(x,r)^C \\
        m(\overline{E}) & \le m([0,1] \cap B(x,r)^C) < 1
    \end{align*}

    which is contradict with $m(\overline{E}) \ge m(E) \ge 1$
\end{proof}

\begin{exercise}
    let $A_n$ be disjoint measurable sets. And $B_n \subseteq A_n$. prove that

    \[
        m^*(\bigcup_{n=1}^{\infty}B_n) = \sum_{n=1}^{\infty}m^*(B_n)
    \]

\end{exercise}

\begin{proof}
    we will proof at first that, for any $N \ge 1$, we have

    \[
        m^*(\bigcup_{n=1}^{N}B_n) = \sum_{n=1}^{N}m^*(B_n)
    \]

    since $A_1$ is measurable, we take the definition of measurable sets as below

    \begin{align*}
        m(\bigcup_{n=1}^{N}B_n) &= m(\bigcup_{n=1}^{N}B_n \cap A_1)  + m(\bigcup_{n=1}^{N}B_n \cap A_1^C) \\
        &= m(\bigcup_{n=1}^{N}\left(B_n \cap A_1 \right)) + m(\bigcup_{n=1}^{N}\left( B_n \setminus A_1 \right))\\
        &= m(B_1) + m(\bigcup_{n=2}^{N}B_n)
    \end{align*}

    if you are doubt about deduction above consider that:

    \begin{align*}
        B_1 & \cap A_1 = A_1 \\
        B_2 & \cap A_1 \subseteq A_2 \cap A_1 \subseteq \emptyset \\
        B_1 & \setminus A_1 \subseteq A_1 \setminus A_1 \subseteq \emptyset \\
        B_2 & \setminus A_1 = A_2 \cap B_2 \cap A_1^C = B_2 \cap A_2  \cap A_1^C = B_2 \cap A_2 = B_2
    \end{align*}

    then we can take measurable definition of $A_2$ on $B_2 \cup B_3 ... \cup B_N$, and so on, finally we got

    \[
        m^*(\bigcup_{n=1}^{N}B_n) = \sum_{n=1}^{N}m^*(B_n)
    \]

    then we take $N \to \infty$, where the equation holds on because both side is monotone and convergence
    under extended real number set.
    
\end{proof}

\end{document}










