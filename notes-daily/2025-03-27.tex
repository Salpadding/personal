%!LW recipe=latexmk (xelatex)
% a4 页面 字体大小12像素
\documentclass[12pt,a4paper]{ctexart}
% math 
\usepackage{amsmath,amsfonts,amssymb,amsthm}
% cross reference, use \autoref instead of \ref
\usepackage{aliascnt}
\usepackage[hidelinks]{hyperref}
\usepackage{enumitem}
\usepackage{geometry}


\geometry{left=1.5cm, right=1.5cm, top=2cm, bottom=2cm}

\newtheorem{thm}{Theorem}[section]

\newaliascnt{lem}{thm}
\newaliascnt{prop}{thm}
\newaliascnt{definition}{thm}
\newaliascnt{exercise}{thm}
\newaliascnt{corollary}{thm}

\theoremstyle{definition}
\newtheorem{lem}[lem]{Lemma}
\newtheorem{prop}[prop]{Proposition}
\newtheorem{definition}[definition]{Definition}
\newtheorem{exercise}[exercise]{Exercise}
\newtheorem{corollary}[corollary]{Corollary}

\def\lemautorefname{Lemma}
\def\thmautorefname{Theorm}
\aliascntresetthe{lem}
\aliascntresetthe{prop}
\aliascntresetthe{definition}
\aliascntresetthe{exercise}
\aliascntresetthe{corollary}


\title{学习记录}
\author{朱英杰}
\date{\today}

\begin{document}
\zihao{-4}
\maketitle
\tableofcontents

\section{指数函数分析}

\subsection{$\exp(x+y)$ 分析}

对两个绝对收敛的序列作积得到

\[
\lim_{n \to \infty}A_n\lim_{n \to \infty}B_n = \lim_{n \to \infty} A_nB_n
\]

展开 $A_nB_n$ 得到

\begin{align*}
A_nB_n &=(\sum_{i=0}^{n}a_n)(\sum_{i=0}^{n}b_n) = \sum_{i=0}^{n}\sum_{j=0}^{n}a_nb_n \\
&= \sum_{k=0}^{2n}c_k
\end{align*}

其中 

\[
c_k = \sum_{i=0}^{k}a_ib_{k-i}
\]

注意到有二项展开

\[
\sum_{i=0}^{k}a_ib_{k-i} = \frac{x^i}{i!}\frac{y^{k-i}}{(k-i)!} = \frac{(x+y)^k}{k!}
\]

所以,令

\[
C_n = \sum_{i=0}^{n}\frac{(x+y)^i}{i!} = \sum_{k=0}^{n}c_{k}
\]

得到

\[
A_nB_n = C_{2n}
\]

两边取极限得到

\[
\lim_{n \to \infty}A_n \lim_{n \to \infty}B_n = \lim_{n \to \infty}C_{2n} = \exp(x+y)
\]

\subsection{$\exp(x) = e^x$}

先证明 $x$ 是正整数的情况

\[
\exp(n) = \exp(1+1+..+1) = (\exp(1))^n = e^n
\]

在证明 $x$ 是负整数的情况

\[
\exp(-n)\exp(n) = \exp(0) = 1
\]

由此得到

\[
\exp(-n) = \frac{1}{\exp(n)} = \frac{1}{e^n} = e^{-n}
\]

继续证明 $x=1/n$的情况

\[
(\exp(\frac{1}{n}))^n = \exp(\frac{1}{n} + .. + \frac{1}{n}) = e
\]

所以得到

\[
\exp(\frac{1}{n}) = e^{1/n}
\]

用同样的方法可以证明对有理数 $q = n/m$ 有

\[
\exp(q) = e^q
\]

对于实数 $r$ ,我们要用公理构造一个有理数列 $q_n$ 收敛到 $r$,根据指数函数的连续性得到

\[
\lim_{n \to \infty}\exp(q_n) = \lim_{n \to \infty}e^{q_n} = \exp(r) = e^r
\]

\subsection{$D \exp(x) = \exp(x)$}

幂级数的前 $n$ 项和都是连续可微的,所以幂级数在收敛半径内一定可微,而且可以用逐项微分计算。

\begin{align*}
D \exp(x) &= D(1 + x + \frac{x^2}{2!} + .. + \frac{x^n}{n!} + ..) \\
&= 1 + x + .. + \frac{x^{n-1}}{(n-1)!} + ..  \\
&= \exp(x)
\end{align*}

\subsection{解斐波那契数列}

令 $a_1 = a_2 = 1$,递推式子为 $a_{n+2}  = a_{n+1} + a_{n} $

用待定系数法,假设有

\begin{align*}
    a_{n+2} - \alpha a_{n+1} &= \beta(a_{n+1} - \alpha a_{n}) \\
    a_{n+2} &= (\alpha + \beta)a_{n+1} - \alpha \beta a_n
\end{align*}

所以得到

\begin{align*}
    \alpha + \beta &= 1 \\
    -\alpha\beta &= 1
\end{align*}

所以 $\alpha,\, \beta$ 是下面方程的两个根,思考下 vieta 定理

\[
x^2 - x - 1 = 0
\]

用求根公式计算得到

\[
\alpha = \frac{1+\sqrt{5}}{2},\, \beta = \frac{1-\sqrt{5}}{2}
\]

注意到 $\alpha$ 和 $\beta$ 具有轮换对称性,所以我们可以求解出 $a_{n+2} - \alpha a_{n+1}$ 和  $a_{n+2} - \beta a_{n+1}$ 的表达式

\begin{align*}
 a_{n+2} - \alpha a_{n+1} &= (1-\alpha) \beta^{n} = \beta^{n+1}  \\
 a_{n+2} - \beta a_{n+1} &= (1-\beta) \alpha^{n} = \alpha^{n+1}  \\
\end{align*}

然后消去 $a_{n+2}$ 得到

\[
(\alpha - \beta)a_{n+1} = \alpha^{n+1} -  \beta^{n+1}
\]

化简得到

\[
a_{n+1} = \frac{1}{\sqrt{5}}((\frac{1+\sqrt{5}}{2})^{n+1} - (\frac{1-\sqrt{5}}{2})^{n+1})
\]

把 $n+1$ 都换成 $n$

\[
a_{n} = \frac{1}{\sqrt{5}}((\frac{1+\sqrt{5}}{2})^{n} - (\frac{1-\sqrt{5}}{2})^{n})
\]

\section{行列式计算}

\subsection{三对角行列式}

计算三对角行列式,按最后一列展开

\begin{align*}
    D_n = \begin{vmatrix}
        a_1 & b_1 & 0 & 0 & .. & 0 & 0 \\
        c_1 & a_2 & b_2 & 0 & .. & 0 & 0 \\
        0 & c_2 & a_3 & b_3 & .. & 0 & 0 \\
        .. & .. & .. & .. & .. & .. & .. \\
        0 & 0 & 0 & 0 & ..& a_{n-1} & b_{n-1} \\
        0 & 0 & 0 & 0 & .. & c_{n-1} & a_n \\
    \end{vmatrix} &= a_nD_{n-1} + (-1)^nb_{n-1} \begin{vmatrix}
        a_1 & b_1 & 0 & 0 & .. & 0 & 0 \\
        c_1 & a_2 & b_2 & 0 & .. & 0 & 0 \\
        0 & c_2 & a_3 & b_3 & .. & 0 & 0 \\
        .. & .. & .. & .. & .. & .. & .. \\
        0 & 0 & 0 & 0 & ..& a_{n-2} & b_{n-2} \\
        0 & 0 & 0 & 0 & .. & 0 & c_{n-1} \\
    \end{vmatrix}  \\
    &= a_nD_{n-1} + (-1)b_{n-1}c_{n-1}D_{n-2} \\
    &= a_nD_{n-1} - b_{n-1}c_{n-1}D_{n-2}
\end{align*}

\end{document}
