%!LW recipe=latexmk (xelatex)
% a4 页面 字体大小12像素
\documentclass[12pt,a4paper]{ctexart}
% 数学公式
\usepackage{amsmath}
\usepackage{amsfonts}
\usepackage{amssymb}
\usepackage[hidelinks]{hyperref}
\usepackage{enumitem}
\usepackage{geometry}


\geometry{left=1.5cm, right=1.5cm, top=2cm, bottom=2cm}

\title{学习记录}
\author{朱英杰}
\date{2025-03-24}

\begin{document}
\zihao{-4}
\maketitle

\section{三次方程化简}

\subsection{多项式展开}

假设 $x_1, x_2, x_3$ 是方程 $(x-x_1)(x-x_2)(x-x_3)$ 的三个根。

\begin{align*}
   (x-x_1)(x-x_2)(x-x_3) = x^3 +(x_1x_2 + x_1x_3 + x_2x_3)x^2 - (x_1 + x_2 + x_3)x - x_1x_2x_3
\end{align*}

\subsection{从已知条件生成多项式}

已知 $x_1 + x_2 = a$ 并且 $x_1 x_2 = b$ 可以通过构建二次多项式的方法求 $x_1$ 和 $x_2$,因为

$(x-x_1)(x-x_2) = x^2 - (x_1 + x_2)x + x_1 x_2$ 所以 $x_1$ 和 $x_2$ 是方程

\[
x^2 - ax + b = 0
\]

的根

同理,已知 $x_1x_2x_3 = c$,$x_1 + x_2 + x_3 = b$ 并且 $x_1x_2 + x_1x_3 + x_2x_3 = a$ 后可以通过求

\[
x^3 + ax^2 - bx -c = 0
\]

的根来求解

\subsection{重根}

这里要注意的是多项式 $x^3 - 1$ 和 $(x-1)^3$ 的区别,这两个多项式都有 $x=1$ 这个根,但是 $x=1$ 是 $x^3-1$ 的一重根,
而 $x=1$ 是 $(x-1)^3$ 的三重根。

\section{实变函数习题}

\subsection{集合习题}

\begin{enumerate}
    \item $A_1 \subseteq A_2 \subseteq .. \subseteq A_n ...$,而且$B_1 \subseteq B_2 \subseteq .. \subseteq B_n ...$ 
     那么 
     \[
        \left( \bigcup_{n=1}^{\infty} A_n \right) \cap \left( \bigcup_{n=1}^{\infty} B_n \right) = \bigcup_{n=1}^{\infty} A_n \cap B_n
     \]

     证明:

     令

     \begin{align*}
        A &= \bigcup_{n=1}^{\infty} A_n \\
        B &= \bigcup_{n=1}^{\infty} B_n \\
     \end{align*}

     则

     \begin{align*}
        A \cap \bigcup_{j=1}^{\infty}B_n &= \bigcup_{j=1}^{\infty}A \cap B_j \\
        &= \bigcup_{j=1}^{\infty} \bigcup_{i=1}^{\infty} A_i \cap B_j
     \end{align*}

     所以 $x \in A \cup B$ 等价于存在 $i,j $ 满足 $x \in A_i \cap B_j$,在 $A_n$ 和 $B_n$ 都是单增德前提下, 这个和

     \[
     x \in \bigcup_{n=1}^{\infty}A_n \cap B_n
     \]

     是等价的


    \item $A_1 \supseteq A_2 \supseteq .. \supseteq A_n ...$,而且$B_1 \supseteq B_2 \supseteq .. \supseteq B_n ...$ 
     那么 
     \[
        \left( \bigcap_{n=1}^{\infty} A_n \right) \cup \left( \bigcap_{n=1}^{\infty} B_n \right) = \bigcap_{n=1}^{\infty} A_n \cup B_n
     \]

     我们对 $A_1^C, \, A_2^C, .. $ 以及 $B_1^C, B_2^C, .. $ 用到上面的结论得到

     \[
        \left( \bigcup_{n=1}^{\infty} A_n^C \right) \cap \left( \bigcup_{n=1}^{\infty} B_n^C \right) = \bigcup_{n=1}^{\infty} A_n^C \cap B_n ^C
     \]

     然后两边取补集得到

     \[
        \left( \bigcap_{n=1}^{\infty} A_n \right) \cup \left( \bigcap_{n=1}^{\infty} B_n \right) = \bigcap_{n=1}^{\infty} A_n \cup B_n 
     \]

    \item 若 $A \cup B = E \cup F,\, A \cap F = \emptyset,\, B \cap E = \emptyset$ 那么 $A = E$ 且 $B = F$

    \begin{align*}
        a + b + ab &= e + f + ef \\
        af = 0 \\
        be = 0 \\
    \end{align*}

    两边同时乘以 $e$

    \[
    ae = e
    \]

    两边同时乘以 $a$ 得到

    \[
        a = ae 
    \]

    所以 $a = ae = e$

    同理,两边同时乘以 $b,f$ 后可以得到

    \[
        b = f = bf
    \]
\end{enumerate}

\subsection{集合极限习题}

\begin{enumerate}
    \item 设 $f_n: \mathbb{R} \to \mathbb{R}$ 有

\[
\lim_{n \to \infty} f_n(x) = f(x),\, x \in \mathbb{R}
\]

那么

\[
\{ x \,\vert\, f(x) \le t \} = \bigcap_{k=1}^{\infty} \bigcup_{m=1}^{\infty} \bigcap_{n=m}^{\infty}\{ x \,\vert\, f_n(x) < t + \frac{1}{k} \}
\]

证明,如果 $f(x) \le t$ 说明

\[
\lim_{n \to \infty}f_n(x) \le t
\]

那么对所有的 $\epsilon > 0$ 一定存在  $m \in \mathbb{N}$ 满足 $\forall n \ge m,\, f_n(x) < t + \epsilon$

然后我们把 $\epsilon$ 换成 $1/k$ 得到

\[
\{ x \,\vert\, f(x) \le t \} \subseteq \bigcap_{k=1}^{\infty} \bigcup_{m=1}^{\infty} \bigcap_{n=m}^{\infty}\{ x \,\vert\, f_n(x) < t + \frac{1}{k} \}
\]

另一方面,若 

\[
x \in \bigcap_{k=1}^{\infty} \bigcup_{m=1}^{\infty} \bigcap_{n=m}^{\infty}\{ x \,\vert\, f_n(x) < t + \frac{1}{k} \}
\]


那么对所有的 $k \in \mathbb{N}^+$ 一定存在  $m \in \mathbb{N}$ 满足 $\forall n \ge m,\, f_n(x) < t + \frac{1}{k}$

然后我们对 $n$ 取极限得到

\[
\lim_{n \to \infty}f_n(x) = f(x) \le t + \frac{1}{k}
\]

再对 $k$ 取极限得到

\[
f(x) \le t
\]

所以有

\[
\bigcap_{k=1}^{\infty} \bigcup_{m=1}^{\infty} \bigcap_{n=m}^{\infty}\{ x \,\vert\, f_n(x) < t + \frac{1}{k} \} \subseteq \{ x \,\vert \, f(x) \le t \}
\]

\item 若 $\lim_{n \to \infty} a_n = a$ 有如下命题

\[
    \bigcap_{k=1}^{\infty}\bigcup_{N=1}^{\infty}\bigcap_{n=N}^{\infty}(a_n - \frac{1}{k}, a_n + \frac{1}{k}) = \{ a \}
\]

首先我们令

\[
A = \bigcap_{k=1}^{\infty}\bigcup_{N=1}^{\infty}\bigcap_{n=N}^{\infty}(a_n - \frac{1}{k}, a_n + \frac{1}{k}) 
\]

我们先证明 $a \in A$ ,因为 $\lim_{n \to \infty} a_n = a$ ,所以任取 $\epsilon > 0$,都有 $N$ 满足 $\forall n \ge N,\, \lvert a_n - a \rvert \le \epsilon$,
那么我们取 $\epsilon = \frac{1}{2k}$,可以得到

\[
a \in \bigcup_{N=1}^{\infty}\bigcap_{n=N}^{\infty}(a_n - \frac{1}{k}, a_n + \frac{1}{k})
\]

因为 $k$ 可以任意取,所以有 $a \in A$

我们继续证明只有 $a \in A$,用反证法,假设存在 $a + \epsilon \in A,\, \lvert \epsilon \rvert > 0$。然后我们取 $k$ 满足$\frac{1}{k} < \lvert \epsilon \rvert$。
带入 $a + \epsilon \in A$ 的条件,可以得到 $a_n$ 收敛到 $a + \epsilon$,这个显然和 $a_n$ 收敛到 $a$ 矛盾了。


\end{enumerate}

\section{高等代数习题}

\subsection{Vandermonde 行列式}

\begin{align*}
    \begin{vmatrix}
       1 & x_1 & x_1^2 & ... & x_1^{n-1} \\
       1 & x_2 & x_2^2 & ... & x_2^{n-1} \\
       ... & ... & ... & ... & ...\\
       1 & x_n & x_n^2 & ... & x_n^{n-1} \\
    \end{vmatrix} &= 
    \begin{vmatrix}
       1 & 0 & 0 & ... & 0 \\
       1 & x_2-x_1 & x_2^2-x_1x_2 & ... & x_2^{n-1}-x_1(x_2^{n-2}) \\
       ... & ... & ... & ... & ...\\
       1 & x_n-x_1 & x_n^2-x_1x_n & ... & x_n^{n-1}-x_1(x_n^{n-2}) \\
    \end{vmatrix}  \\
    &= (x_2-x_1)(x_3-x_2)..(x_n-x_1) 
    \begin{vmatrix}
       1 & 0 & 0 & ... & 0 \\
       1 & 1 & x_2 & ... & x_2^{n-2} \\
       ... & ... & ... & ... & ...\\
       1 & 1 & x_n & ... & x_n^{n-2} \\
    \end{vmatrix}  \\
    &= \begin{vmatrix}
         1 & x_2 & ... & x_2^{n-2} \\
         ... & ... & ... & ...\\
         1 & x_n & ... & x_n^{n-2} \\
    \end{vmatrix}\prod_{i=2}^{n}(x_i-x_1) \\
    &= \prod_{i=1}^{n-1}\prod_{j=i+1}^{n}(x_j-x_i)
\end{align*}


\subsection{线性空间}

\subsubsection{八大公理}

\begin{enumerate}
    \item $\mathbf{x} + \mathbf{y} = \mathbf{y} + \mathbf{x}$
    \item $\mathbf{x} + \mathbf{y} + \mathbf{z} = \mathbf{x} + (\mathbf{y} + \mathbf{z})$
    \item $\mathbf{x} + (-\mathbf{x}) = \mathbf{0}$
    \item $\mathbf{x} + \mathbf{0} = \mathbf{x}$
    \item $a(\mathbf{x}+\mathbf{y}) = a\mathbf{x} + a\mathbf{y}$
    \item $(a+b)\mathbf{x} = a\mathbf{x} + b\mathbf{x}$
    \item $(ab)\mathbf{x} = a(b\mathbf{x})$
    \item $1\mathbf{x} = \mathbf{x}$
\end{enumerate}

\subsection{结论}

\begin{enumerate}
    \item $-\mathbf{x}$ 唯一

    假设有两个 $-\mathbf{x}$ 分别是  $-\mathbf{x}_1$ 和$-\mathbf{x}_2$
    那么有

    \[
        \mathbf{x} + (-\mathbf{x}_1) = \mathbf{0}
    \]

    两边加上 $-\mathbf{x}_2$ 得到

    \[
        -\mathbf{x}_2 + \mathbf{x} + (-\mathbf{x}_1) = -\mathbf{x}_1 = -\mathbf{x}_2
    \]


    \item $0 \mathbf{x} = \mathbf{0}$

    证明: $0\mathbf{x} = (0+0)\mathbf{x} = 0\mathbf{x} + 0\mathbf{x}$ 
    两边同时加上 $0\mathbf{x}$ 的逆元得到 

    \[
        \mathbf{0} = 0 \mathbf{x}
    \]


    \item $(-1) \mathbf{x} = -\mathbf{x}$

    \[
        (-1)\mathbf{x} + 1 \mathbf{x} = 0\mathbf{x} = \mathbf{0}
    \]

\end{enumerate}

\section{Laplace 定理}

\subsection{Laplace 定理}

行列式任取 $k$ 行 $i_1, i_2, .. i_k \in K$,然后按照 $\binom{n}{k}$ 中方式选择列,计算选中行列交叉的行列式,再乘以交叉部分的代数余子式,相加

\[
\text{det}(A) = \sum_{J \in \binom{n}{k}} \text{det}(A_{i \in I,\, j \in J})(-1)^{\sum I + \sum J} \text{det}(A_{i \notin I,\, j \notin J})
\]

\subsection{分块矩阵行列式}

利用 Laplace 定理可以计算分块矩阵的行列式

下面按照左边 $n$ 列展开得到。

\[
\begin{vmatrix}
    A_n & M \\
    O & B_m \\
\end{vmatrix} = \text{det}(A_n) \text{det}(B_m) (-1)^{2\binom{n+1}{2}} = \text{det}(A_n)\text{det}(B_m)
\]

同理有


\[
\begin{vmatrix}
    A_n & O \\
    M & B_m \\
\end{vmatrix} = \text{det}(A_n) \text{det}(B_m)
\]

此外要注意


\[
\begin{vmatrix}
    O & A_n \\
    B_m & M \\
\end{vmatrix} = \text{det}(A_n) \text{det}(B_m)(-1)^{n\binom{n+1}{2} + \frac{n^2(2m + n+1)}{2}}
\]

注意到有

\[
n\binom{n+1}{2} + \frac{n^2(2m + n+1)}{2} = \frac{n^2}{2}(2(n+1) + 2m ) = n^2(n+1 + m)
\]

因为我们只需要考虑奇偶性,所以可以对 $2$ 取模

\[
n^2(n+1+m) \equiv n + n + nm \equiv nm \quad \text{mod} 2
\]

所以

\[
\begin{vmatrix}
    O & A_n \\
    B_m & M \\
\end{vmatrix} = \text{det}(A_n) \text{det}(B_m)(-1)^{nm}
\]

\end{document}