
%!LW recipe=latexmk (xelatex)
% a4 页面 字体大小12像素
\documentclass[12pt,a4paper]{ctexart}
\input% math 
\usepackage{amsmath,amsfonts,amssymb,amsthm}
% cross reference, use \autoref instead of \ref
\usepackage{aliascnt}
\usepackage[hidelinks]{hyperref}
\usepackage{enumitem}
\usepackage{geometry}


\geometry{left=1.5cm, right=1.5cm, top=2cm, bottom=2cm}

\newtheorem{thm}{Theorem}[section]

\newaliascnt{lem}{thm}
\newaliascnt{prop}{thm}
\newaliascnt{definition}{thm}
\newaliascnt{exercise}{thm}
\newaliascnt{corollary}{thm}

\theoremstyle{definition}
\newtheorem{lem}[lem]{Lemma}
\newtheorem{prop}[prop]{Proposition}
\newtheorem{definition}[definition]{Definition}
\newtheorem{exercise}[exercise]{Exercise}
\newtheorem{corollary}[corollary]{Corollary}

\def\lemautorefname{Lemma}
\def\thmautorefname{Theorm}
\aliascntresetthe{lem}
\aliascntresetthe{prop}
\aliascntresetthe{definition}
\aliascntresetthe{exercise}
\aliascntresetthe{corollary}


\title{学习记录}
\author{朱英杰}
\date{2025-03-31}

\begin{document}
\zihao{-4}
\maketitle
\tableofcontents

\section{高等代数}

\subsection{行列式-拆分法}

\subsubsection{证明}

请证明

\begin{align*}
    \left| A(t)\right| &= \begin{vmatrix}
        a_{11} + t & a_{12} + t & .. & a_{1n} + t \\
        a_{21} + t & a_{22} + t & .. & a_{2n} + t \\
        .. & .. & .. & .. \\
        a_{n1} + t & a_{n2} + t & .. & a_{nn} + t \\
    \end{vmatrix} \\
    &= \lvert A(0) \rvert + t\sum_{i=1}^{n}\sum_{j=1}^{n}A_{ij}
\end{align*}

证明如下

\begin{align*}
    \left| A(t)\right| &= \begin{vmatrix}
        a_{11} + t & a_{12} + t & .. & a_{1n} + t \\
        a_{21} + t & a_{22} + t & .. & a_{2n} + t \\
        .. & .. & .. & .. \\
        a_{n1} + t & a_{n2} + t & .. & a_{nn} + t \\
    \end{vmatrix} \\
    &=  \begin{vmatrix}
        a_{11}  & a_{12} + t & .. & a_{1n} + t \\
        a_{21}  & a_{22} + t & .. & a_{2n} + t \\
        .. & .. & .. & .. \\
        a_{n1} & a_{n2} + t & .. & a_{nn} + t \\
    \end{vmatrix}  + t \begin{vmatrix}
        1  & a_{12} + t & .. & a_{1n} + t \\
        1  & a_{22} + t & .. & a_{2n} + t \\
        .. & .. & .. & .. \\
        1 & a_{n2} + t & .. & a_{nn} + t \\
    \end{vmatrix}
\end{align*}

以此类推把每一列都拆分后得到

\begin{align*}
&\begin{vmatrix}
        a_{11}  & a_{12} & .. & a_{1n}  \\
        a_{21}  & a_{22}  & .. & a_{2n} \\
        .. & .. & .. & .. \\
        a_{n1} & a_{n2} & .. & a_{nn}  \\
\end{vmatrix} + t\begin{vmatrix}
        1  & a_{12} + t & .. & a_{1n} + t \\
        1  & a_{22} + t & .. & a_{2n} + t \\
        .. & .. & .. & .. \\
        1 & a_{n2} + t & .. & a_{nn} + t \\
    \end{vmatrix} \\ + t & \begin{vmatrix}
        a_{11}  & 1  & .. & a_{1n} + t \\
        a_{21}  & 1  & .. & a_{2n} + t \\
        .. & .. & .. & .. \\
        a_{n1} & 1 & .. & a_{nn} + t \\
    \end{vmatrix} + .. + t\begin{vmatrix}
        a_{11}  & a_{12}  & .. & 1 \\
        a_{21}  & a_{21}  & .. & 1 \\
        .. & .. & .. & .. \\
        a_{n1} & a_{n2} & .. & 1 \\
    \end{vmatrix}
\end{align*}

注意到那些带有 $+t$ 的列的 $+t$ 都可以用全为 $1$ 的列消去,最终得到

\begin{align*}
&\begin{vmatrix}
        a_{11}  & a_{12} & .. & a_{1n}  \\
        a_{21}  & a_{22}  & .. & a_{2n} \\
        .. & .. & .. & .. \\
        a_{n1} & a_{n2} & .. & a_{nn}  \\
\end{vmatrix} + t\begin{vmatrix}
        1  & a_{12}  & .. & a_{1n}  \\
        1  & a_{22}  & .. & a_{2n}  \\
        .. & .. & .. & .. \\
        1 & a_{n2}  & .. & a_{nn}  \\
    \end{vmatrix} \\ + t & \begin{vmatrix}
        a_{11}  & 1  & .. & a_{1n}  \\
        a_{21}  & 1  & .. & a_{2n}  \\
        .. & .. & .. & .. \\
        a_{n1} & 1 & .. & a_{nn}  \\
    \end{vmatrix} + .. + t\begin{vmatrix}
        a_{11}  & a_{12}  & .. & 1 \\
        a_{21}  & a_{21}  & .. & 1 \\
        .. & .. & .. & .. \\
        a_{n1} & a_{n2} & .. & 1 \\
    \end{vmatrix}
\end{align*}

然后各自按照全为 $1$ 的列展开得到

\[
\lvert A(0) \rvert + t\sum_{j=1}^{n}\sum_{i=1}^{n}A_{ij}
\]

\subsubsection{推论}

\begin{align*}
 \begin{vmatrix}
        a_{11} + t_1 & a_{12} + t_2 & .. & a_{1n} + t_n \\
        a_{21} + t_1 & a_{22} + t_2 & .. & a_{2n} + t_n \\
        .. & .. & .. & .. \\
        a_{n1} + t_1 & a_{n2} + t_2 & .. & a_{nn} + t_n \\
\end{vmatrix}  = \lvert A(0) \rvert  + \sum_{j=1}^{n}t_j \sum_{i=1}^{n}A_{ij}
\end{align*}

\subsubsection{推论}

上面的结论可以用来计算

\begin{align*}
    \begin{vmatrix}
        a & b & b & .. & b\\
        c & a & b & .. & b\\
        c & c & a & .. & b\\
        .. & .. & .. & .. & ..\\
        c & c & c & .. & a\\
    \end{vmatrix}
\end{align*}

令 

\begin{align*}
A(t) &= \begin{vmatrix}
        a + t& b + t& b + t& .. & b + t\\
        c + t& a + t& b + t& .. & b +t \\
        c + t& c + t & a + t & .. & b + t \\
        .. & .. & .. & .. & ..\\
        c + t& c + t& c + t& .. & a + t\\
    \end{vmatrix} \\
u &= \sum_{i=1}^{n}\sum_{j=1}^{n}A_{ij}
\end{align*}

得到

\begin{align*}
A(-b) &= \lvert  A \rvert - bu = (a-b)^n\\
A(-c) &= \lvert A \rvert - cu  = (a-c)^n
\end{align*}

若 $b \ne c$,用克莱姆法则可以求解得到

\[
\lvert A \rvert = \frac{b(a-c)^n - c(a-b)^n}{b-c}
\]

若 $b = c$,用求和法得到

\[
(a+(n-1)b) (a-b)^{n-1}
\]

\subsection{抽屉原理}

若 $f_1(x),f_2(x),..,f_n(x)$ 都是多项式,而且 $\deg (f(x)) \le n-2$,证明

\[
\begin{vmatrix}
    f_1(a_1) & f_2(a_1) & .. & f_n(a_1) \\
    f_1(a_2) & f_2(a_2) & .. & f_n(a_2) \\
    .. & .. & .. & .. \\
    f_1(a_n) & f_2(a_n) & .. & f_n(a_n) \\
\end{vmatrix} = 0
\]

也就是 $b_{ij} = f_j(a_i)$

我们把上面的行列式按照单项式的和进行拆分,至多可以拆分成 $(n-1)^n$ 项,其中每一项都是只包含单项式的行列式。
我们分析其中任意一项

\[
\begin{vmatrix}
    f_{1,d_1}(a_1) & f_{2,d_2}(a_1) & .. & f_{n,d_n}(a_1) \\
    f_{1,d_1}(a_2) & f_{2,d_2}(a_2) & .. & f_{n,d_n}(a_2) \\
    .. & .. & .. & .. \\
    f_{1,d_1}(a_n) & f_{2,d_2}(a_n) & .. & f_{n,d_n}(a_n) \\
\end{vmatrix} 
\]

这里 $d_1, d_2,..$ 表示这个单项式的次数是 $d_1,d_2, ..$,因为 $d_1,d_2,..d_n$ 都是整数,而且满足 $0 \le d_i \le n-2$,
所以一定存在如下两个次数相同的列,不妨设就是第一列和第二列

\[
\begin{vmatrix}
    c_{1d}a_1^d & c_{2d}a_1^d & .. & f_{n,d_n}(a_1) \\
    c_{1d}a_2^d & c_{2d}a_2^d & .. & f_{n,d_n}(a_2) \\
    .. & .. & .. & .. \\
    c_{1d}a_n^d & c_{2d}a_n^d & .. & f_{n,d_n}(a_n) \\
\end{vmatrix} 
\]

可以看到这两列是成比例的,所以行列式为0,所以每一项都是0

\subsubsection{拆分+递推}

\begin{align*}
    D_n &= \begin{vmatrix}
        1+a_1^2 & a_1a_2 & .. & a_1a_n \\
        a_2a_1 & 1+a_2^2 & .. & a_2a_n \\
        .. & .. & .. & .. \\
        a_na_1 & a_na_2 & .. & 1 + a_n^2  
    \end{vmatrix} \\
    &= \begin{vmatrix}
        1+a_1^2 & a_1a_2 & .. & 0 \\
        a_2a_1 & 1+a_2^2 & .. & 0 \\
        .. & .. & .. & .. \\
        a_na_1 & a_na_2 & .. & 1  
    \end{vmatrix} + \begin{vmatrix}
        1+a_1^2 & a_1a_2 & .. & a_1a_n \\
        a_2a_1 & 1+a_2^2 & .. & a_2a_n \\
        .. & .. & .. & .. \\
        a_na_1 & a_na_2 & .. & a_n^2  
    \end{vmatrix} \\
    &= D_{n-1} + a_n \begin{vmatrix}
        1 & 0 & .. & 0 \\
        0 & 1 & .. & 0 \\
        .. & .. & .. & .. \\
        a_1 & a_2 & .. & a_n
    \end{vmatrix} = D_{n-1} + a_n^2
\end{align*}

所以 

\[
D_n = 1 + \sum_{i=1}^{n}a_i^2
\]

\subsection{Vandermonde 行列式}

\subsubsection{齐次行列式}

\begin{align*}
    \begin{vmatrix}
        a_1^{n-1} & a_1^{n-2}b_1 & .. & a_1b_1^{n-2} & b_1^{n-1} \\
        a_2^{n-1} & a_2^{n-2}b_2 & .. & a_2b_2^{n-2} & b_2^{n-1} \\
        .. & .. &.. &.. &..  \\ 
        a_n^{n-1} & a_n^{n-2}b_n & .. & a_nb_n^{n-2} & b_n^{n-1} \\
    \end{vmatrix}
\end{align*}

若 $a_1a_2..a_n \ne 0$,那么每一行 $i$ 都提取因子 $a_i^{n-1}$ 得到

\begin{align*}
    & = (a_1a_2..a_n)^{n-1}\begin{vmatrix}
        1 & b_1/a_1 & .. & (b_1/a_1)^{n-2} & (b_1/a_1)^{n-1} \\
        1 & b_2/a_2 & .. & (b_2/a_2)^{n-2} & (b_2/a_2)^{n-1} \\
        .. & .. &.. &.. &..  \\ 
        1 & b_n/a_n & .. & (b_n/a_n)^{n-2} & (b_n/a_n)^{n-1} \\
    \end{vmatrix} \\
    &= (a_1a_2..a_n)^{n-1} \prod_{j=2}^{n}\prod_{i=1}^{j-1}(\frac{b_j}{a_j} - \frac{b_i}{a_i})
\end{align*}

注意到

\[
\prod_{j=2}^{n}\prod_{i=1}^{j-1}a_ja_i = (a_1a_2.a_n)^{n-1}
\]

因为 $a_i$ 的次数为它出现在 $a_j$ 的次数 加上它出现在 $a_i$ 的次数

\[
i-1 + n-(i+1) + 1 = n-1
\]

所以

\begin{align*}
& (a_1a_2..a_n)^{n-1} \prod_{j=2}^{n}\prod_{i=1}^{j-1}(\frac{b_j}{a_j} - \frac{b_i}{a_i})  \\
= & \prod_{j=2}^{n}\prod_{i=1}^{j-1}(a_ib_j - a_jb_i) 
\end{align*}

此外若存在不同的 $i,j$  满足 $a_i = a_j = 0$,那么这个行列式为 0

如果仅有一个 $i$ 满足 $a_i = 0$,那么对第 $i$ 行展开得到相同类型的行列式

\subsection{多项式}

令 $f_k(k) = x^k + a_{k1}x^{k-1} + a_{k2}x^{k-2} + .. + a_{kk} $

那么

\begin{align*}
    \begin{vmatrix}
        1 & f_1(x_1) & f_2(x_1) & .. & f_{n-1}(x_1) \\
        1 & f_1(x_2) & f_2(x_2) & .. & f_{n-1}(x_2) \\
        .. & .. &.. &.. &..  \\
        1 & f_1(x_n) & f_2(x_n) & .. & f_{n-1}(x_n) \\
    \end{vmatrix} =     \begin{vmatrix}
        1 & x_1 + a_{11} & x_1^2 + a_{21}x_1 + a_{22} & .. & f_{n-1}(x_1) \\
        1 & x_2 + a_{11} & x_2^2 + a_{21}x_2 + a_{22} & .. & f_{n-1}(x_2) \\
        .. & .. &.. &.. &..  \\
        1 & x_n + a_{11} & x_n^2 + a_{21}x_n + a_{22} & .. & f_{n-1}(x_n) \\
    \end{vmatrix}
\end{align*}

我们可以先用第一列消去后面所有列的 0 次项,然后用第二列消去后面所有列的一次项,最终所有都只包含最高次项。
所以得到

\[
\prod_{1 \le i < j \le n}(x_j -x_i)
\]

\subsubsection{三角多项式}

首先推导一个等式

\begin{align*}
    \cos{kx} + \text{i} \sin kx &= \text{e} ^{\text{i} kx} = \left( \cos x + \text{i} \sin x \right)^k \\
    &= \binom{k}{0}\cos^k x + \binom{k}{1}\text{i} \sin x \cos^{k-1} x - \binom{k}{2}\cos^{k-2} x (1- \cos^2 x)  \\
    & - \binom{k}{3}\text{i} \sin^3 x \cos^{k-3} x + \binom{k}{4}\cos^{k-4} x (1- \cos^2 x)^2  \\
    & + \binom{k}{5}\text{i} \sin^5 x \cos^{k-5} x  - \binom{k}{6}\cos^{k-6} x (1- \cos^2 x)^3 + .. \\
    & + \binom{k}{k-1}\text{i}^{k-1} \sin^{k-1} x \cos x + \binom{k}{k}\text{i}^k \sin^k x
\end{align*}


比较实部后发现, $\cos kx$ 可以表示成 $\cos x$ 的多项式,而且它的最高次为 $k$,最高次的系数为 $2^{k-1}$

注意当 $k$ 为奇数时有,考虑奇数项和偶数项的对称性
\[
    \binom{k}{0} + \binom{k}{2} .. + \binom{k}{k-1} = 2^{k-1}
\]

当 $k$ 为偶数时,

\begin{align*}
\binom{k}{2} + \binom{k}{4} + .. + \binom{k}{k-2} &= \binom{k-1}{2} + \binom{k-1}{4} .. + \binom{k-1}{k-2} \\
& + \binom{k-1}{1} + \binom{k-1}{3} .. + \binom{k-1}{k-3} \\
&= 2^{k-1} - 2
\end{align*}

两边同时加上 $\binom{k}{0}$ 和 $\binom{k}{k}$ 得到

\[
\binom{k}{0} + \binom{k}{2} + .. + \binom{k}{k} = 2^{k-1}
\]

所以下面的行列式

\begin{align*}
    \begin{vmatrix}
        1 & \cos \theta_1 & \cos 2\theta_1 & .. & \cos(n-1)\theta_1 \\
        1 & \cos \theta_2 & \cos 2\theta_2 & .. & \cos(n-1)\theta_2 \\
        .. & .. &.. &.. &.. \\
        1 & \cos \theta_n & \cos 2\theta_n & .. & \cos(n-1)\theta_n \\
    \end{vmatrix}
\end{align*}

可以逐列消去只保留最高次项,然后利用 Vandermonde 行列是计算得到

\[
2^{\binom{n-1}{2}}\prod_{1 \le i < j \le n}\left( \cos \theta_j - \cos \theta_i \right)
\]

\subsubsection{Another example}

\begin{align*}
    \begin{vmatrix}
        \sin \theta_1 & \sin 2 \theta_1 & .. & \sin n \theta_1 \\
        \sin \theta_2 & \sin 2 \theta_2 & .. & \sin n \theta_2 \\
        .. & .. & .. & .. \\
        \sin \theta_n & \sin 2 \theta_n & .. & \sin n \theta_n \\
    \end{vmatrix}
\end{align*}

利用和差化积公式有

\[
\sin kx - \sin(k-2)x = 2 \sin x \cos (k-1)x
\]

然后把第 $n-2$ 列乘以 $-1$ 加到第 $n$ 列,以此类推得到

\begin{align*}
    \begin{vmatrix}
        \sin \theta_1 & \sin 2 \theta_1 & 2\sin\theta_1 \cos 2 \theta_1 & .. & 2\sin\theta_1 \cos (n-1) \theta_1\\
        \sin \theta_2 & \sin 2 \theta_2 & 2\sin\theta_2 \cos 2 \theta_2 & .. & 2\sin\theta_2 \cos (n-1) \theta_2\\
        .. & .. & .. & .. & ..\\
        \sin \theta_n & \sin 2 \theta_n & 2\sin\theta_n \cos 2 \theta_n & .. & 2\sin\theta_n \cos (n-1) \theta_n\\
    \end{vmatrix} = 2^{\binom{n}{2}}\prod_{i=1}^{n} \sin \theta_i \prod_{1 \le i < j \le n}\left( \cos \theta_j - \cos \theta_i \right)
\end{align*}

\subsubsection{多项式命题证明}

设 $n$ 次多项式

\[
f(x) = a^nx^n + a_{n-1}x^{n-1} + .. + a_1x + a_0
\]

有 $n+1$ 个不同的根 $b_1, b_2, .. b_{n+1}$,证明 $f(x)$ 是零多项式。

根据题意可以得到 $a_0, a_1, .. a_n$ 满足是如下关于 $y$ 方程的解


\begin{align*}
y + b_1y + .. + b_1^{n-1}y^{n-1} + b_1^ny^{n} &= 0 \\
y + b_2y + .. + b_2^{n-1}y^{n-1} + b_2^ny^{n} &= 0 \\
&.. \\
y + b_ny + .. + b_n^{n-1}y^{n-1} + b_n^ny^{n} &= 0 \\
\end{align*}

这个方程对应的行列式为

\[
\begin{vmatrix}
    1 & b_1 & b_1^2 & .. &b_1^n \\
    1 & b_2 & b_2^2 & .. &b_2^n \\
    .. & .. &.. &.. &..  \\
    1 & b_n & b_n^2 & .. &b_n^n \\
\end{vmatrix} = \prod_{1 \le i < j \le n}\left( b_j - b_i\right)
\]

因为 $b_j \ne b_i$,所以行列式不为零,所以方程唯一的解是 $a_0 = a_1 = a_2 = .. = a_n = 0$

\section{三角函数公式}

\subsection{和差化积}

\begin{align*}
    \sin x + \sin y &= 2 \sin(\frac{x+y}{2}) \cos (\frac{x-y}{2}) \\
    \cos x + \cos y &= 2 \cos(\frac{x+y}{2}) \cos (\frac{x-y}{2}) \\
\end{align*}

\subsection{积化和差}

\begin{align*}
   \sin x \cos y &= \frac{\sin(x+y) + \sin(x-y)}{2}  \\
   \cos x \sin y &= \frac{\sin(x+y) - \sin(x-y)}{2}  \\
   \sin x \sin x &= \frac{\cos(x-y) - \cos(x+y)}{2} \\
   \cos x \cos x &= \frac{\cos(x-y) + \cos(x+y)}{2} \\
\end{align*}

\section{实变函数}

\subsection{点集拓扑}

证明 $f: \mathbb{R} \to \mathbb{R}$,那么 $f(x)$ 不连续但是右侧极限存在的点是可数集。

令 

\[
E_n = \{ x \,\vert\, \exists \delta > 0,\, \forall x_1,x_2 \in (x-\delta, x+\delta),\, \lvert f(x_1) - f(x_2) \rvert \le \frac{1}{n}\}
\]

显然 $f(x)$ 连续的点可以表示为

\[
\bigcap_{n=1}^{\infty}E_n
\]

记 $f(x)$ 右极限存在的点是 $S$,那么题目中描述的点集就是

\[
S \setminus \bigcap_{n=1}^{\infty}E_n = \bigcup_{n=1}^{\infty}S \setminus E_n
\]

所以我们只需要证明 $S \setminus E_n$ 是可数集。

我们固定 $n$ 取 $E_n$,然后取 $x \in E_n$,因为 $x \in S$,所以我们可以取到 $\delta$ 满足

\[
\forall x_1,\, x<x_1 < x+ \delta,\, \lvert f(x_1) - f(x+0) \rvert \le \frac{1}{2n}
\]

由此可以得到


\[
\forall x_1,x_2 \in (x, x+\delta),\, \lvert f(x_1) - f(x_2) \rvert \le \frac{1}{n}
\]

所以我们对任意 $x \in S \setminus E_n$ 都能构造出一个区间 $I_x$,满足 $I_x \subseteq E_n$从而有 $I_x \cap (S \setminus E_n) = \emptyset$

因此,若 $x_1,\, x_2 \in S \setminus E_n$ 时,$I_{x_1}$ 和 $I_{x_2}$ 一定是不相交的,如果它们相交,相交的部分一定包含 $x_1,\, x_2$ 其中一个,
这个和 $I_{x_1} \cap S \setminus E_n = \emptyset$ 矛盾。

所以 $S \setminus E_n$ 可以映射到不相交的开区间,所以 $S \setminus E_n$ 是至多可数的。

\section{初等数论}

\subsection{整除理论}

\subsubsection{LCM}

若 $a_j \vert c,\, 1 \le j \le k$ 那么必然有 $[a_1,..a_k] \vert c$

证明:

令 $L = [a_1,a_2, .. ,a_k]$,用 $c$ 对 $L$ 作带余除法得到

\[
c = qL + r
\]

由 $a_j \vert c,\, a_j \vert L$ 能得到 $a_j \vert r$,所以 $r$ 也是公倍数,但因为 $r < L$,所以 $r$ 只能是 $0$

\subsubsection{LCM 数乘}

$m[a_1,a_2,..,a_k] = [ma_1,ma_2,..,ma_k]$

证明:

令 $L= [a_1,a_2,..,a_k],\, L' =[ma_1, ma_2,..,ma_k]$,显然 $a_i \vert L$ 得到 $ma_i \vert mL$,所以 $mL$ 是 $ma_i$ 公倍数。

此外,$ma_i \vert L'$ 可以得到 $a_i \vert L'/m$,所以 $L'/m$ 是 $a_1,a_2,..,a_k$ 公倍数。

所以

\begin{align*}
    mL &\ge L' \\
    L'/m &\ge L
\end{align*}

所以 $mL = L'$

\subsubsection{GCD}

$D=(a_1,..,a_k)$ 的充分必要条件是

\begin{enumerate}
    \item $\forall 1 \le j \le k,\, D \vert a_j$
    \item $\forall d \vert a_j(1 \le j \le k),\, d \vert D$
\end{enumerate}

我们记 $a_1,a_2,..,a_k$ 所有的公约数 $d_1,d_2,..$ 的最小公倍数为 $L$,然后我们证明 $L = D$
根据之前我们证明的最小公倍数的性质,我们可以得到 $L \vert a_j,\, 1 \le j \le k$,所以 $L$ 是公约数,

因为 $D$ 是公约数,所以 $D \vert L$,又因为 $D$ 是最大公约数,所以有 $L \le D$,所以 $L = D$

\subsubsection{GCD 数乘}

$m(b_1,..,b_k) = (mb_1,..,mb_k)$

证明:

$D_1 \vert b_i$ 可以得到 $mD_1 \vert mb_i$,所以左边小于等于右边 $mD_1 \le D_2$。

同理 $D_2 \vert mb_i$ 可以得到 $D_2/m \vert b_i$,所以 $D_2/m \le D_1$,即 $D_2 \le mD_1$

结合上面得到 $mD_1 = D_2$

\subsubsection{GCD 结合律}

$(a_1,a_2,a_3,..,a_k) = ((a_1,a_2),a_3,..,a_k)$

这个用最大公约数 $D$ 是公约数的最小公倍数证明即可。

\subsection{互素 I}

$(m,a) = 1$ 那么 $(m,ab) = (m,b)$

\[
(m,b) = (m,b(m,a)) = (m,(mb,ab)) = ((m,mb),ab) = (m,ab)
\]

\subsubsection{互素 II}

$(m,a)=1$ 若 $m \vert ab$ 则 $m \vert b$

用上面的结论即可,$(m,b) = (m,ab) = m$


\subsubsection{互素 III}

$(a,b) = 1$ 若 $a \vert c$ 且 $b \vert c$,那么 $ab \vert c$

证明: 令 $c = ka$,因为 $b \vert ka$ 得到 $b \vert k$,两边乘上 $a$ 得到 $ab \vert ka$,
也就是 $ab \vert c$


\subsection{最大公约数和最小公倍数}

$[a_1,a_2](a_1,a_2) = a_1a_2$

证明:

令 $L=[a_1,a_2]$ 我们先假定 $(a_1,a_2) = 1$,接下来只要证明 $[a_1,a_2] = a_1a_2$。

根据之前的结论有 $L \vert a_1a_2$,此外根据 $(a_1, a_2) = 1$,以及 $a_1 \vert L$ 且 $a_2 \vert L$
得到 $a_1a_2 \vert L$

所以 $a_1a_2 = L$

当 $(a_1,a_2) \ne 1$ 时,有 $(a_1/D, a_2/D) = 1$,此时有

\begin{align*}
[a_1/D, a_2/D] &= a_1a_2/D^2 \\
[a_1, a_2] (a_1, a_2) &= a_1a_2
\end{align*}

\subsubsection{线性组合}

\begin{enumerate}
    \item $(a_1,a_2,..,a_k) = \min \{ s \,\vert\, s = x_1a_1 + x_2a_2 + .. + a_kx_k,\, s > 0 \}$

    最大公约数是最小的线性组合

    证明:显然左边一定能整除右边,令右边为 $D'$ 然后对 $a_i$ 作带余除法得到

    \[
        D' = q_ia_i + r_i
    \]

    显然 $r_i$ 也是线性组合,因为 $D'$ 是最小的,所以 $r_i = 0$,所以 $a_i \vert D'$,所以 $D' \vert D$

    \item 存在整数 $x_1,x_2, .., x_k$ 满足

    $(a_1,a_2,..,a_k) = x_1a_1 + x_2a_2 + .. + x_ka_k$

    证明:

    我们只要证明 $\{ s \,\vert\, s = x_1a_1 + x_2a_2 + .. + a_kx_k,\, s > 0 \}$ 不是空集即可,因为

    \[
        x_1 = a_1, x_2 = a_2, .., x_k =a_k
    \]

    时有

    \[
        s= a_1^2+ a_2^2 + .. + a_k^2 > 0
    \]

\end{enumerate}

\end{document}

