%!LW recipe=latexmk (xelatex)
% a4 页面 字体大小12像素
\documentclass[12pt,a4paper]{ctexart}
\input% 数学公式
\usepackage{amsmath}
\usepackage{amsfonts}
\usepackage{amssymb}
\usepackage[hidelinks]{hyperref}
\usepackage{enumitem}
\usepackage{geometry}


\geometry{left=1.5cm, right=1.5cm, top=2cm, bottom=2cm}

\title{学习记录}
\author{朱英杰}
\date{2025-03-26}

\begin{document}
\zihao{-4}
\maketitle
\tableofcontents


\section{行列式计算}

\subsection{求和法}

\subsubsection{根}

假设 $x_1, x_2, x_3$ 是方程 $x^3 + px + q = 0$ 的三个根,求下列行列式的值

\[
\begin{vmatrix}
    x_1 & x_2 & x_3\\
    x_2 & x_3 & x_1\\
    x_3 & x_1 & x_2\\
\end{vmatrix}
\]

相加得到

\begin{align*}
\begin{vmatrix}
    x_1 & x_2 & x_3\\
    x_2 & x_3 & x_1\\
    x_3 & x_1 & x_2\\
\end{vmatrix} = \begin{vmatrix}
    x_1 + x_2 + x_3 & x_2 & x_3\\
    x_1 + x_2 + x_3 & x_3 & x_1\\
    x_1 + x_2 + x_3 & x_1 & x_2\\
\end{vmatrix}
\end{align*}

根据 vieta 定理,有

\[
(x-x_1)(x-x_2)(x-x_3) = x^3 - (x_1 + x_2 + x_3)x^2 + (x_1x_2 + x_1x_3 + x_2x_3)x - x_1x_2x_3
\]

所以 $x_1 + x_2 + x_3 = 0$

所以行列式的值为 $0$

\subsubsection{证明}

设 $b_{ij} = (a_{i1} + a_{i2} + .. + a_{in}) - a_{ij}$,求 $\text{det}(B)$ 关于 $\text{det}(A)$ 的表达式。


把右边 $n-1$ 列加到第一列得到

令

\[
c_i = \sum_{j=1}^{n}a_{ij}
\]

\[
\begin{vmatrix}
    (n-1)c_1 & c_1 - a_{12} & c_1 - a_{13} & .. & c_1-a_{1n} \\
    (n-1)c_2 & c_2 - a_{22} & c_2 - a_{23} & .. & c_2-a_{2n} \\
    .. & .. &.. &.. &.. \\
    (n-1)c_n & c_n - a_{n2} & c_n - a_{n3} & .. & c_n-a_{nn} \\
\end{vmatrix} = (n-1) \text{det}(A)(-1)^{n-1}
\]

\subsubsection{$n$ 阶行列式}

\begin{align*}
    \begin{vmatrix}
        0 & 1 & 1 & .. & 1 & 1 \\
        1 & 0 & 1 & .. & 1 & 1 \\
        1 & 1 & 0 & .. & 1 & 1 \\
        .. & .. &.. &.. &.. & .. \\ 
        1 & 1 & 1 & .. & 1 & 0 \\
    \end{vmatrix} &=     \begin{vmatrix}
        n-1 & 1 & 1 & .. & 1 & 1 \\
        n-1 & 0 & 1 & .. & 1 & 1 \\
        n-1 & 1 & 0 & .. & 1 & 1 \\
        .. & .. &.. &.. &.. & .. \\ 
        n-1 & 1 & 1 & .. & 1 & 0 \\
    \end{vmatrix} \\
    &= (n-1)\begin{vmatrix}
        1 & 0 & 0 & .. & 0 & 0 \\
        1 & -1 & 0 & .. & 0 & 0 \\
        1 & 0 & -1 & .. & 0 & 0 \\
        .. & .. &.. &.. &.. & .. \\ 
        1 & 0 & 0 & .. & 0 & -1 \\
    \end{vmatrix} = (n-1)(-1)^{n-1}
\end{align*}

\subsubsection{计算}

\begin{align*}
\begin{vmatrix}
    a_1 + b & a_2 & a_3 & .. & a_n \\
    a_1 & a_2 + b & a_3 & .. & a_n \\
    a_1 & a_2  & a_3 + b & .. & a_n \\
    .. & .. &.. &.. &.. \\
    a_1 & a_2  & a_3 & .. & a_n + b \\
\end{vmatrix} &= \begin{vmatrix}
    s + b & a_2 & a_3 & .. & a_n \\
    s + b & a_2 + b & a_3 & .. & a_n \\
    s + b & a_2  & a_3 + b & .. & a_n \\
    .. & .. &.. &.. &.. \\
    s + b & a_2  & a_3 & .. & a_n + b \\
\end{vmatrix} \\
&= (s+b)\begin{vmatrix}
    1 & a_2 & a_3 & .. & a_n \\
    1 & a_2 + b & a_3 & .. & a_n \\
    1 & a_2  & a_3 + b & .. & a_n \\
    .. & .. &.. &.. &.. \\
    1 & a_2  & a_3 & .. & a_n + b \\
\end{vmatrix} \\
&= (s+b)\begin{vmatrix}
    1 & 0 & 0 & .. & 0 \\
    1 & b & 0 & .. & 0 \\
    1 & 0  &  b & .. & 0 \\
    .. & .. &.. &.. &.. \\
    1 & 0  & 0 & .. &   b \\
\end{vmatrix} = (\sum_{i=1}^{n}a_i + b)b^{n-1}
\end{align*}

\subsubsection{计算}

\begin{align*}
    \begin{vmatrix}
        1 & 2 & 3 & .. & n-1 & n \\
        n & 1 & 2 & .. & n-2 & n-1  \\
        n- 1& n & 1 & ..  & n-3 & n-2  \\
        .. & .. & .. & .. & .. & .. \\
        3 & 4 & 5 & .. & 1 & 2 \\
        2 & 3 & 4 & .. & n & 1  \\
   \end{vmatrix} &=     \begin{vmatrix}
        s_n & 2 & 3 & .. & n-1 & n \\
        s_n & 1 & 2 & .. & n-2 & n-1  \\
        s_n & n & 1 & ..  & n-3 & n-2  \\
        .. & .. & .. & .. & .. & .. \\
        s_n & 4 & 5 & .. & 1 & 2 \\
        s_n & 3 & 4 & .. & n & 1  \\
    \end{vmatrix} \\
    &=  s_n \begin{vmatrix}
        1 & 2 & 3 & .. & n-1 & n \\
        1 & 1 & 2 & .. & n-2 & n-1  \\
        1 & n & 1 & ..  & n-3 & n-2  \\
        .. & .. & .. & .. & .. & .. \\
        1 & 4 & 5 & .. & 1 & 2 \\
        1 & 3 & 4 & .. & n & 1  \\
    \end{vmatrix} \\
     &=  s_n \begin{vmatrix}
        1 & 2 & 3 & .. & n-1 & n \\
        0 & -1 & -1 & .. & -1 & -1  \\
        0 & n-2 & -2 & ..  & -2 & -2  \\
        .. & .. & .. & .. & .. & .. \\
        0 & 2 & 2 & .. & 2-n & 2-n \\
        0 & 1 & 1 & .. & 1 & 1-n  \\
    \end{vmatrix}  \\
    &= s_n \begin{vmatrix}
          -1 & -1 & .. & -1 & -1  \\
          n-2 & -2 & ..  & -2 & -2  \\
          .. & .. & .. & .. & .. \\
          2 & 2 & .. & 2-n & 2-n \\
          1 & 1 & .. & 1 & 1-n  \\
    \end{vmatrix} \\
    &= s_n \begin{vmatrix}
          -1 & -1 & .. & -1 & -1  \\
          n & 0 & ..  & 0 & 0  \\
          n & n & ..  & 0 & 0  \\
          .. & .. & .. & .. & .. \\
          n & n & .. & 0 & 0 \\
          n & n & .. & n & 0  \\
    \end{vmatrix} = s_n(-1)^n n^{n-2}
\end{align*}

其中 

\[
s_n = \frac{n(n+1)}{2}
\]

\section{实变函数习题}

\subsection{点集拓扑}

\begin{enumerate}
    \item $f: X \to Y$ 是满射,当且仅当对任意真子集 $B \subset Y$ 有 $f(f^{-1}(B)) = B$

    证明,若 $f$ 满射 $y \in f(f^{-1}(B))$ 等价于 $\exists x \in f^{-1}(B),\, f(x) = y$
    等价于 $\exists x,\, f(x) \in B,\, f(x) = y$ 显然这个可以和 $y \in B$ 互相推导。

    另一方面,若 $f$ 不是满射,那么取 $y \in Y,\, y \notin f(X)$,显然单点集 $\{ y \}$ 有

    \[
        f(f^{-1}(\{ y \})) = \emptyset
    \]

    \item 若 $f$ 是满射,那么如下命题等价

    \begin{enumerate}
        \item f 是双射
        \item 对任意 $A, B \subseteq X$ 有 $f(A \cap B) = f(A) \cap f(B)$

        若 $f$ 是双射,那么 $y \in f(A)$ 可以推导出 $\exists x \in A,\, x' \in B,\, f(x) = f(x') = y$,
        因为 $f$ 是双射,所以 $x' = x $,所以 $y \in f(A \cap B)$

        令一方面,若 $f$ 不是单射,那么存在 $x_1,\, x_2$ 满足 $f(x_1) = f(x_2) = y$,取 $A = \{ x_1 \}$
        $B = \{ x_2 \}$,显然有 $f(A \cap B) = \emptyset$ 且 $f(A) \cap f(B) = \{ y \}$

        \item 对任意不相交的 $A,\, B \subseteq X$ 有 $f(A) \cap f(B) = \emptyset$

        $f(A \cap B) = f(A) \cap f(B) = \emptyset$


        另一方面,若 $f$ 不是双射,可以构造两个不相交的单点集满足 $f(A) \cap f(B) \ne \emptyset$

        \item 对任意 $A \subseteq B \subseteq X$ 有 $f(B \setminus A) = f(B) \setminus f(A)$

        根据上面的结论有 $f(B \cap A^C) = f(B) \cap f(A^C)$,因为 $f$ 是双射,所以 $f(A^C) = f(A)^C$,所以

        \[
        f(B \cap A^C) = f(B) \setminus f(A)
        \]

        另一方面,若 $f$ 不是双射,构造两个单点集 $A_1, A_2$,得到

        \begin{align*}
            f(A_1) \setminus f(A_2) &= \emptyset \\
            f(A_1 \setminus A_2) = f(A_1) \ne \emptyset
        \end{align*}

        矛盾。

    \end{enumerate}
\end{enumerate}

\subsection{单调映射上的不动点}

$X$ 是非空集合,$f: 2^X \to 2^X$ 是从集合到集合的映射,若对任意 $A \subseteq B$ 有 $f(A) \subseteq f(B)$,那么一定存在 $T \in 2^X$ 满足 $T = f(T)$

证明:令

\[
S = \{ A \in 2^X \,\vert\, A \subseteq f(A) \}
\]

显然有 $\emptyset \in S$,而且若 $A_1,\, A_2 \in S$ 必然有 $A_1 \cup A_2 \in S$,于是我们令

\[
T = \bigcup_{A \in S} A
\]

显然有

\[
f(T) = \bigcup_{A \in S}f(A)
\]

所以 $T \in S$,此外 $T \subseteq f(T)$ 可以得到 $f(T) \subseteq f(f(T))$,所以有 $f(T) \in S$,所以 $f(T) \subseteq T$

\subsection{集合在映射下的分解}

\subsubsection{命题}

若 $f: X \to Y$,而且有 $g: Y \to X$,那么存在分解 $X = A \cup (X \setminus A)$ 且 $Y = B \cup (Y \setminus B)$ 满足

$f(A) = B$ 而且 $g(Y \setminus B) = X \setminus A$ 

\subsubsection{定义分离集}

$E$ 是分离集当且仅当 

\[
E \cap g(Y \setminus f(E)) = \emptyset
\]

\subsubsection{最大分离集}

定义 $\Gamma$ 是包含所有分离集的集合,定义

\[
A = \bigcup_{E \in \Gamma} E
\]

我们验证 $A$ 是否是分离集,显然有

\[
g(Y \setminus f(\bigcup_{E \in \Gamma}E)) = g(Y \setminus \bigcup_{E \in \Gamma}f(E)) \subseteq \bigcap_{E \in \Gamma} g(Y \setminus f(E))
\]

显然 $x \in A$ 可以推断出存在 $E \in \Gamma$ 满足 $x \in E$,于是有 $x \notin g(Y \setminus f(E))$ 
所以 $A$ 是分离集

\subsubsection{$A$ 满足分解条件}

令 $B = f(A)$,注意这里 $A$ 是最大的分离集。我们可以尝试计算 $ g(Y \setminus B)$,并且讨论它和 $X \setminus A$ 的关系。

首先,若 $x \in  g(Y \setminus B)$,那说明我们可以找到 $y \in Y \setminus f(A)$ 满足 $g(y) = x$然后利用分离集的性质

\[
A \cap g(Y \setminus f(A)) = \emptyset
\]

得到 $x \notin A$,所以有 $g(Y \setminus B) \subseteq X \setminus A$


另一方面,若 $x \notin A$ 我们想证明 $x \in g(Y \setminus B)$,这里我们用反证法,假设 $x \notin g(Y \setminus f(A))$

那么有

\begin{align*}
(A \cup \{x\}) \cap g(Y \setminus f(A \cup \{ x \})) =  
A \cap g(Y \setminus f(A \cup \{ x \})) \cup  \{x\} \cap g(Y \setminus f(A \cup \{ x \}))
\end{align*}

其中

\[
A \cap g(Y \setminus f(A \cup \{ x \})) \subseteq A \cap g(Y \setminus f(A)) = \emptyset
\]

并且有

\[
\{x\} \cap g(Y \setminus f(A \cup \{ x \})) \subseteq \{x\} \cap g(Y \setminus f(A)) = \emptyset
\]

所以 $A \cup \{ x \}$ 也是一个分离集,这个和 $A$ 是最大分离集矛盾了。

\end{document}