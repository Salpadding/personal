%!LW recipe=latexmk
\documentclass[11pt,a4paper]{article}
\input% math 
\usepackage{amsmath,amsfonts,amssymb,amsthm}
% cross reference, use \autoref instead of \ref
\usepackage{aliascnt}
\usepackage[hidelinks]{hyperref}
\usepackage{enumitem}
\usepackage{geometry}


\geometry{left=1.5cm, right=1.5cm, top=2cm, bottom=2cm}

\newtheorem{thm}{Theorem}[section]

\newaliascnt{lem}{thm}
\newaliascnt{prop}{thm}
\newaliascnt{definition}{thm}
\newaliascnt{exercise}{thm}
\newaliascnt{corollary}{thm}

\theoremstyle{definition}
\newtheorem{lem}[lem]{Lemma}
\newtheorem{prop}[prop]{Proposition}
\newtheorem{definition}[definition]{Definition}
\newtheorem{exercise}[exercise]{Exercise}
\newtheorem{corollary}[corollary]{Corollary}

\def\lemautorefname{Lemma}
\def\thmautorefname{Theorm}
\aliascntresetthe{lem}
\aliascntresetthe{prop}
\aliascntresetthe{definition}
\aliascntresetthe{exercise}
\aliascntresetthe{corollary}


\title{Learning Notes}
\author{Yingjie Zhu}
\date{\today}


\begin{document}

\section{Theory of Real Variable Function}
\begin{exercise}
    let $f_n: \Omega \to \mathbb{R}^*$ be a sequence of non-negative measurable functions. Then we have

    \[
        \sum_{n=1}^{\infty}\int_{\Omega}f_n \mathrm{d}x =\int_{\Omega}\sum_{n=1}^{\infty}f_n \mathrm{d}x
    \]
\end{exercise}

\begin{proof}
    let 

    \[
        F_n = \sum_{i=1}^{n}f_i
    \]

    and

    \[
        F = \lim_{n \to \infty} F_n
    \]

    Then $F_n$ is increasing and 

    \[
        \sum_{n=1}^{\infty}f_n = F
    \]

    by Beppo Levi's theorem, we have

    \[
        \lim_{n \to \infty}\int_{\Omega}F_n \mathrm{d}x = \int_{\Omega}F \mathrm{d}x
    \]

    where

    \[
\int_{\Omega}F \mathrm{d}x = \int_{\Omega}\left( \sum_{n=1}^{\infty}f_n \right) \mathrm{d}x
    \]

    and

    \begin{align*}
\lim_{n \to \infty}\int_{\Omega}F_n \mathrm{d}x &= \lim_{n \to \infty}\int_{\Omega}\left( \sum_{i=1}^{n}f_i \right) \mathrm{d}x \\
&= \lim_{n \to \infty}\sum_{i=1}^{n}\left( \int_{\Omega}f_i \mathrm{d}x \right) \\
&= \sum_{n=1}^{\infty}\left( \int_{\Omega}f_n \mathrm{d}x \right)
    \end{align*}
\end{proof}

\begin{exercise}
    let $f: \Omega \to \mathbb{R}^*$ be non-negative measurable function. and $E_n \subseteq \Omega$ be a sequence of measurable sets.
    and $\forall i \ne j,\, E_i \cap E_j = \emptyset$. also

    \[
        \bigcup_{n=1}^{\infty}E_n = \Omega
    \]

    Then we have

    \[
        \sum_{n=1}^{\infty} \int_{E_n} f \mathrm{d}x =  \int_{\Omega} f \mathrm{d}x
    \]
\end{exercise}

\begin{proof}
    \begin{align*}
        \sum_{n=1}^{\infty} \int_{E_n} f \mathrm{d}x &= \sum_{n=1}^{\infty} \int_{\Omega} f \chi_{E_n}  \mathrm{d}x \\
        &= \int_{\Omega} \left( \sum_{n=1}^{\infty} f \chi_{E_n}  \right) \mathrm{d}x \\
        &= \int_{\Omega} f \mathrm{d}x
    \end{align*} 

\end{proof}

\begin{exercise}
    let $E_1, E_2, .. , E_n \subseteq [0,1]$ and measurable. And we have

    \[
        \sum_{j=1}^{n} \chi_{E_j}(x) \ge k,\, \forall x \in [0,1]
    \]

    then we have at least $1 \le l \le n$ such that  $m(E_l) \ge k/n$
\end{exercise}

\begin{proof}
    assume $\forall 1 \le l \le n, m(E_l) < k/n$. then we have 

    \[
        \sum_{j=1}^{n} \int_{E_j} 1 \mathrm{d}x \le \sum_{j=1}^{n} k/n \le k
    \]

    however we have

    \begin{align*}
        \sum_{j=1}^{n} \int_{E_j} 1 \mathrm{d}x &= \sum_{j=1}^{n} \int_{[0,1]} \chi_{E_j} \mathrm{d}x \\
        &= \int_{[0,1]} \left( \sum_{j=1}^{n}\chi_{E_j} \right) \mathrm{d}x \\
        & \ge \int_{[0,1]} k \mathrm{d}x \ge k 
    \end{align*}
\end{proof}


\begin{exercise}
    let $f: \Omega \to \mathbb{R}^*$ and $f < \infty \quad \text{a.e.},\, m(\Omega) < \infty$. divide $[0, \infty)$ as

    $0 = y_0 < y_1 < y_2 ...$ and $\exists \delta > 0$ such that $y_{n+1} - y_n > \delta$

    let 

    \[
        E_k = \{ x: y_k \le f(x) < y_{k+1}\}
    \]

    then $f$ is integrable iff:

    \[
        \sum_{k=0}^{\infty}y_k m(E_k) < \infty
    \]
    if $f$ is integrable we have

    \[
        \int_{\Omega} f \mathrm{d}x= \lim_{\delta \to 0} \sum_{k=0}^{\infty}y_k m(E_k) 
    \]
\end{exercise}

\begin{proof}
    let 

    \[
        h = \sum_{k=0}^{\infty} y_k \chi_{E_k}
    \]

    and

    \[
        g = \sum_{k=0}^{\infty} y_{k+1} \chi_{E_k}
    \]

    then we have

    \[
        h \le f \le g
    \]

    so we have

    \[
     \int_{\Omega}h \mathrm{d}x  \le  \int_{\Omega}f \mathrm{d}x \le \int_{\Omega}g \mathrm{d}x
    \]

    which is

    \begin{align*}
        \sum_{k=0}^{\infty}y_k m(E_k) &\le \int_{\Omega}f \mathrm{d}x \le \sum_{k=0}^{\infty}y_{k+1} m(E_k) \\
            & \le \sum_{k=0}^{\infty}y_{k} + (y_{k+1} - y_k) m(E_k) \\
            & \le \sum_{k=0}^{\infty}y_{k}m(E_k) + \sum_{k=0}^{\infty}\delta m(E_k) \\
            & \le \sum_{k=0}^{\infty}y_{k}m(E_k) + \delta m(\Omega)
    \end{align*}

    so $f: \Omega \to \mathbb{R}^*$ is integrable iff 

    \[
        \sum_{k=0}^{\infty}y_k m(E_k) < \infty
    \]

\end{proof}


\begin{exercise}
    let $f: \Omega \to \mathbb{R}^*$ and $f < \infty \quad \text{a.e.},\, m(\Omega) < \infty$. 
    then $f$ is integrable iff

    \[
        \sum_{n=0}^{\infty}m(\{ x: f(x) \ge n \}) < \infty
    \]

\end{exercise}

\begin{proof}
    let
    \[
        E_n  = \{x : f(x) \ge n \}
    \]
    and

    \[
        F_n = \{x: n \le f(x) < n + 1\}
    \]

    then we have

    \[
        E_n = \bigcup_{k=n}^{\infty}F_k
    \]

    so

    \begin{align*}
        \sum_{n=0}^{\infty} m(E_n) &=  \sum_{n=0}^{\infty} \sum_{k=n}^{\infty}m(F_k) \\
        &= \sum_{k=0}^{\infty}(k+1)m(F_k)
    \end{align*}

    since 

    \[
        \Omega = \bigcup_{k=0}^{\infty}
    \]

    so $f: \Omega \to \mathbb{R}^*$ is integrable iff

    \[
        \sum_{k=0}^{\infty}k m(F_k) < \infty
    \]

    and we have

    \[
\sum_{k=0}^{\infty}k m(F_k) \le \sum_{k=0}^{\infty}(k+1) m(F_k) \le \sum_{k=0}^{\infty}k m(F_k) + m(\Omega)
    \]
\end{proof}

\begin{thm}[ lebesgue dominated convergence theorem ]
    let $f_n : \Omega \to \mathbb{R}^*$ be a sequence of measurable functions.
    and 

    \[
        \lim_{n \to \infty}f_n(x) = f(x) \: \text{a.e.}  \: x \in \Omega
    \]

    and exists integrable function $F: \Omega \to \mathbb{R}^*$ such that

    \[
        \left| f_n(x) \right| \le F(x) \: \text{a.e.} \: x \in \Omega
    \]

    then we have

    \[
        \lim_{n \to \infty}\int_{\Omega}f_n \mathrm{d}x = \int_{\Omega} f(x) \mathrm{d}x
    \]
\end{thm}

\begin{proof}
    \begin{enumerate}
        \item $f_n$ is measurable obviously at first:

            let 

            \[
                E_n = \{x \in \Omega: \lvert f_n(x)\rvert \le F(x) \}
            \]

            then we have


            \begin{align*}
                \int_{\Omega} \left| f_n(x) \right| \mathrm{d}x &= \int_{\Omega} \left| f_n(x) \right| \chi_{E_n}(x) \mathrm{d}x \\
                & \le \int_{\Omega} \left| F(x) \right| \chi_{E_n}(x) \mathrm{d}x  \\
                & \le \int_{\Omega} \left| F(x) \right| \mathrm{d}x < \infty
            \end{align*}

        \item $f$ is measurable:

        let 
        \[
            E = \bigcap_{n=1}^{\infty}E_n
        \]

        then we have

        \[
            m(\Omega \setminus E) = m(\bigcup_{n=1}^{\infty} \left(\Omega \setminus E_n \right)) \le \sum_{n=1}^{\infty}m(\left(\Omega \setminus E_n \right)) = 0
        \]

        and

        \begin{align*}
            \int_{\Omega}f(x) \mathrm{d}x &= \int_{\Omega}f(x) \chi_{E}\mathrm{d}x \\
            & \le \int_{\Omega}F(x) \chi_{E}\mathrm{d}x \\
            & \le \int_{\Omega}F(x) \mathrm{d}x < \infty
        \end{align*}

        \item define $g_n$        

        let 

        \[
            g_n = \left| f_n(x) - f(x) \right|
        \]

        \item $g_n$ is integrable

        \[
            \lvert g_n(x) \rvert \le \lvert f_n(x) \rvert + \lvert f(x) \rvert \le 2F(x) \: \text{a.e.} \: x \in \Omega
        \]

        \item apply Fatou lemma on $2F(x) - g_n(x)$:

        \[
            \int_{\Omega} \left(\lim_{n \to \infty} 2F(x) - g_n(x)\right) \mathrm{d}x \le \varliminf_{n \to \infty} \int_{\Omega} 2F(x) - g_n(x) \mathrm{d}x
        \]

        since

        \[
            2F(x) - g_n(x) \to 2F(x) \: \text{a.e.} \: x \in \Omega
        \]

        so we have

        \begin{align*}
            \int_{\Omega} 2F(x) \mathrm{d}x &\le \varliminf_{n \to \infty}\left( \int_{\Omega} 2F(x) \mathrm{d}x - \int_{\Omega} g_n(x) \mathrm{d}x\right) \\
            & \le \int_{\Omega} 2F(x) \mathrm{d}x + \varliminf_{n \to \infty} \int_{\Omega} -g_n(x) \mathrm{d}x
        \end{align*}

        so we have

        \[
            \varlimsup_{n \to \infty}\int_{\Omega} g_n(x) \mathrm{d}x \le 0
        \]

        since

        \[
            g_n(x) \ge 0
        \]

        we must have

        \[
            \lim_{n \to \infty}\int_{\Omega} g_n(x) \mathrm{d}x = 0
        \]

        also we have

        \begin{align*}
            & \left| \lim_{n \to \infty} \left(\int_{\Omega} f_n \mathrm{d}x\right) -  \int_{\Omega} f \mathrm{d}x  \right| \\
            & = \left| \lim_{n \to \infty} \left(\int_{\Omega} f_n \mathrm{d}x -  \int_{\Omega} f \mathrm{d}x \right) \right| \\ 
            & = \left| \lim_{n \to \infty} \left(\int_{\Omega} f_n(x) - f(x) \mathrm{d}x \right) \right| \\
            & \le \lim_{n \to \infty} \left(\int_{\Omega} \left| f_n(x) - f(x) \right| \mathrm{d}x \right)  \\
            & \le \varlimsup_{n \to \infty} \int_{\Omega} g_n(x) \mathrm{d}x \le 0
        \end{align*}
    \end{enumerate}
\end{proof}


\begin{exercise}
    prove that: $L_1$ convergence imply convergence in measure.
\end{exercise}

\begin{proof}
    let $f_n: \Omega \to \mathbb{R}^*$ and $f: \Omega \to \mathbb{R}^*$ be integrable and we have

    \[
        \lim_{n \to \infty}\int_{\Omega}\left| f_n(x) - f(x)\right| \mathrm{d}x = 0
    \]

    pick any $\epsilon > 0$, let

    \[
        E_{\epsilon} = \{ x \in \Omega : \left| f_n(x) - f(x)\right| > \epsilon \}
    \]

    then we have

    \[
\epsilon m(E_{\epsilon})  = \int_{E_{\epsilon}} \epsilon \mathrm{d}x \le \int_{\Omega}\left| f_n(x) - f(x)\right| \mathrm{d}x
    \]

    so we have

    \[
        \lim_{n \to \infty}m(E_{\epsilon}) = 0
    \]
\end{proof}

\begin{exercise}
    let $f_n: \Omega \to \mathbb{R}^*$ be integrable and we have

    \[
        \sum_{n=1}^{\infty} \int_{\Omega} \lvert f_n(x) \rvert \mathrm{d}x < \infty
    \]

    prove that

    \[
        \sum_{n=1}^{\infty} \int_{\Omega}  f_n(x)  \mathrm{d}x  =  \int_{\Omega}\left( \sum_{n=1}^{\infty} f_n(x) \right) \mathrm{d}x
    \]
\end{exercise}

\begin{proof}
    let 

    \[
        F(x) = \sum_{n=1}^{\infty} \lvert f_n(x) \rvert
    \]

    by Beppo's theorem we have

    \[
        \int_{\Omega}F(x)\mathrm{d}x = \sum_{n=1}^{\infty} \int_{\Omega} \lvert f_n(x)\rvert  \mathrm{d}x < \infty
    \]

    so $F$ is absolutely integrable.

    then we define

    \[
        f(x) = \sum_{n=1}^{\infty}f_n(x)
    \]

    since $f(x) \le F(x)$, so $f$ is also absolutely integrable.

    then we define

    \[
        S_N(x) = \sum_{n=1}^{N}f_n(x)
    \]

    then $S_N$ is absolutely and dominated by absolutely integrable function $F$, by Lebesgue dominated convergence theorem, we have

    \[
        \lim_{N \to \infty} \int_{\Omega}S_N \mathrm{d}x = \int_{\Omega}\left( \lim_{N \to \infty}S_N(x) \right) \mathrm{d}x
    \]

    so we have

    \[
        \sum_{n=1}^{\infty} \int_{\Omega} f_n(x) \mathrm{d}x =  \int_{\Omega} \sum_{n=1}^{\infty} f_n(x) \mathrm{d}x
    \]

\end{proof}

\end{document}














